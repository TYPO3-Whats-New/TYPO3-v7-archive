% ------------------------------------------------------------------------------
% TYPO3 CMS 7.5 - What's New - Chapter "In-Depth Changes" (German Version)
%
% @author	Patrick Lobacher <patrick@lobacher.de> and Michael Schams <schams.net>
% @license	Creative Commons BY-NC-SA 3.0
% @link		http://typo3.org/download/release-notes/whats-new/
% @language	German
% ------------------------------------------------------------------------------
% LTXE-CHAPTER-UID:		b8f1c138-2fb2f5ce-15d1ffc0-e80171e2
% LTXE-CHAPTER-NAME:	In-Depth Changes
% ------------------------------------------------------------------------------
% LTXE-SLIDE-START
% LTXE-SLIDE-UID:		03be70bc-886235ca-10f0cef5-6b33fe60
% LTXE-SLIDE-TITLE:		Fluid-based Content Elements Introduced (1)
% LTXE-SLIDE-REFERENCE:	Feature-38732-Fluid-basedContentElementsIntroduced.rst
% ------------------------------------------------------------------------------

\begin{frame}[fragile]
	\frametitle{Änderungen im System}
	\framesubtitle{Fluid-basierte Inhaltselemente (1)}

	\begin{itemize}

		\item Es wurde eine Alternative zur Extension \textit{CSS Styled Content} geschaffen:\newline
			\textbf{"Fluid-based Content Elements"}

		\item Hier werden anstelle von TypoScript Fluid-Templates für das Rendering von Inhalten verwendet

		\item Dazu müssen die folgenden beiden static-Templates eingebunden werden:

			\begin{itemize}
				\item Content Elements (\texttt{fluid\_styled\_content})
				\item Content Elements CSS (optional) (\texttt{fluid\_styled\_content})
			\end{itemize}

	\end{itemize}

\end{frame}

% ------------------------------------------------------------------------------
% LTXE-SLIDE-START
% LTXE-SLIDE-UID:		bb7124ec-4658414b-757ca8ca-f0addee3
% LTXE-SLIDE-TITLE:		Fluid-based Content Elements Introduced (2)
% LTXE-SLIDE-REFERENCE:	Feature-38732-Fluid-basedContentElementsIntroduced.rst
% ------------------------------------------------------------------------------

\begin{frame}[fragile]
	\frametitle{Änderungen im System}
	\framesubtitle{Fluid-basierte Inhaltselemente (2)}

	% decrease font size for code listing
	\lstset{basicstyle=\tiny\ttfamily}

	\begin{itemize}

		\item Zusätzlich muss das PageTSconfig Template \texttt{Fluid-based Content Elements fluid\_styled\_content}
			in den Seiteneigenschaften eingebunden werden, damit der New-Content-Element Wizard entsprechend
			angepasst wird

		\item Eigene Fluid-Templates können wie folgt festgelegt werden:

			\begin{lstlisting}
				lib.fluidContent.templateRootPaths.50 = EXT:site_example/Resources/Private/Templates/
				lib.fluidContent.partialRootPaths.50 = EXT:site_example/Resources/Private/Partials/
				lib.fluidContent.layoutRootPaths.50 = EXT:site_example/Resources/Private/Layouts/
			\end{lstlisting}

	\end{itemize}

\end{frame}

% ------------------------------------------------------------------------------
% LTXE-SLIDE-START
% LTXE-SLIDE-UID:		04cd65d3-e013813d-d2eac450-7d3a79a8
% LTXE-SLIDE-TITLE:		Fluid-based Content Elements Introduced (3)
% LTXE-SLIDE-REFERENCE:	Feature-38732-Fluid-basedContentElementsIntroduced.rst
% LTXE-SLIDE-REFERENCE:	Important-67954-MigrateCTypesTextImageAndTextpicToTextmedia.rst
% ------------------------------------------------------------------------------

\begin{frame}[fragile]
	\frametitle{Änderungen im System}
	\framesubtitle{Fluid-basierte Inhaltselemente (3)}

	\begin{itemize}

		\item Um eine Installation auf die neue Struktur zu migrieren, kann man wie folgt vorgehen:

			\begin{itemize}

				\item Deinstallieren der Extension \texttt{css\_styled\_content}

				\item Installieren der Extension \texttt{fluid\_styled\_content}

				\item Nun ist ein "Upgrade Wizard" im Install Tool verfügbar,
					der die Migration der Inhaltselemente \texttt{text}, \texttt{image}
					und \texttt{textpic} in \texttt{textmedia}, durchführt

			\end{itemize}
	\end{itemize}

\end{frame}


% ------------------------------------------------------------------------------
% LTXE-SLIDE-START
% LTXE-SLIDE-UID:		8807e1a0-2b3fadac-ac842970-b85e5130
% LTXE-SLIDE-TITLE:		Add SELECTmmQuery method to DatabaseConnection
% LTXE-SLIDE-REFERENCE:	Feature-19494-AddSELECTmmQueryMethodToDatabaseConnection.rst
% ------------------------------------------------------------------------------

\begin{frame}[fragile]
	\frametitle{Änderungen im System}
	\framesubtitle{SELECTmmQuery Methode für Datenbank-Zugang}

	% decrease font size for code listing
	\lstset{basicstyle=\tiny\ttfamily}

	\begin{itemize}

		\item Bislang enthielt die Datenbank-Klasse die Methode \texttt{exec\_SELECT\_mm\_query},
			die die Datenbank-Abfrage direkt ausführte

		\item Nun wurde die Generierung des Queries (\textit{Query-Building}) und Ausführung getrennt,
			indem die Methode \texttt{SELECT\_mm\_query} hinzugefügt wurde

			\begin{lstlisting}
				$query = SELECT_mm_query('*', 'table1', 'table1_table2_mm', 'table2', 'AND table1.uid = 1',
				'', 'table1.title DESC');
			\end{lstlisting}

	\end{itemize}

\end{frame}

% ------------------------------------------------------------------------------
% LTXE-SLIDE-START
% LTXE-SLIDE-UID:		398fdcb2-1cc9a862-42e59a09-e4ed65ba
% LTXE-SLIDE-TITLE:		Scheduler task to optimize database tables
% LTXE-SLIDE-REFERENCE:	Feature-25341-SchedulerTaskToOptimizeDatabaseTables.rst
% ------------------------------------------------------------------------------

\begin{frame}[fragile]
	\frametitle{Änderungen im System}
	\framesubtitle{Scheduler Task zur Datenbank-Optimierung}

	\begin{itemize}

		\item Es wurde ein Scheduler Task implementiert, der die Datenbank via MySQL-Kommando
			\texttt{OPTIMIZE TABLE} optimiert

		\item Optimiert werden können lediglich Tabellen vom Typ\newline
			\texttt{MyISAM}, \texttt{InnoDB} und \texttt{ARCHIVE}

		\item DBAL wird nicht unterstützt

	\end{itemize}

\end{frame}

% ------------------------------------------------------------------------------
% LTXE-SLIDE-START
% LTXE-SLIDE-UID:		f9cbeb69-d08efdaf-a8c00d48-353f667e
% LTXE-SLIDE-TITLE:		Improved handling of online media (1)
% LTXE-SLIDE-REFERENCE:	Feature-61799-ImprovedHandlingOfOnlineMedia.rst
% ------------------------------------------------------------------------------

\begin{frame}[fragile]
	\frametitle{Änderungen im System}
	\framesubtitle{Online Medien Unterstützung (1)}

	\begin{itemize}

		\item Der Core wurde um eine externe Medien-Unterstützung erweitert\newline
			\small
				(exemplarisch für YouTube- und Vimeo-Videos)
			\normalsize

		\item Diese kann (z.B. im Inhaltselement \textbf{"Text \& Media"}) als URL
			eingeben werden. Anschließend wird die Resource wie eine interne Datei integriert.

	\end{itemize}

\end{frame}

% ------------------------------------------------------------------------------
% LTXE-SLIDE-START
% LTXE-SLIDE-UID:		76e87d6a-cd1ebbf9-9a4d49ec-49cc569f
% LTXE-SLIDE-TITLE:		Improved handling of online media (2)
% LTXE-SLIDE-REFERENCE:	Feature-61799-ImprovedHandlingOfOnlineMedia.rst
% ------------------------------------------------------------------------------

\begin{frame}[fragile]
	\frametitle{Änderungen im System}
	\framesubtitle{Online Medien Unterstützung (2)}

	Folgende YouTube/Vimeo URLs sind möglich:
	\vspace{0.4cm}

	\begin{columns}[T]
		\begin{column}{.52\textwidth}
			\smaller
				\tabto{0.2cm}\texttt{youtu.be/<code>}\newline
				\tabto{0.2cm}\texttt{www.youtube.com/watch?v=<code>}\newline
				\tabto{0.2cm}\texttt{www.youtube.com/v/<code>}\newline
				\tabto{0.2cm}\texttt{www.youtube-nocookie.com/v/<code>}\newline
				\tabto{0.2cm}\texttt{www.youtube.com/embed/<code>}\newline
		\end{column}
		\begin{column}{.48\textwidth}
			\vspace{-0.25cm}\smaller
				\texttt{vimeo.com/<code>}\newline
				\texttt{player.vimeo.com/video/<code>}\newline
		\end{column}
	\end{columns}

\end{frame}

% ------------------------------------------------------------------------------
% LTXE-SLIDE-START
% LTXE-SLIDE-UID:		aef8746a-e1babbc9-4dd74eec-cf379dae
% LTXE-SLIDE-TITLE:		Improved handling of online media (3)
% LTXE-SLIDE-REFERENCE:	Feature-61799-ImprovedHandlingOfOnlineMedia.rst
% ------------------------------------------------------------------------------

\begin{frame}[fragile]
	\frametitle{Änderungen im System}
	\framesubtitle{Online Medien Unterstützung (3)}

	% decrease font size for code listing
	\lstset{basicstyle=\tiny\ttfamily}

	\begin{itemize}

		\item Der Zugriff per Fluid kann z.B. wie folgt durchgeführt werden:

		\begin{lstlisting}
			<!-- enable js api and set no-cookie support for YouTube videos -->
			<f:media file="{file}" additionalConfig="{enablejsapi:1, 'no-cookie': true}" ></f:media>

			<!-- show title and uploader for YouTube and Vimeo before video starts playing -->
			<f:media file="{file}" additionalConfig="{showinfo:1}" ></f:media>
		\end{lstlisting}

		\item Für YouTube existieren folgende Optionen:\newline
			\small
				\texttt{autoplay}, \texttt{controls}, \texttt{loop}, \texttt{enablejsapi}, \texttt{showinfo}, \texttt{no-cookie}
			\normalsize
		\item Für Vimeo existieren folgende Optionen:\newline
			\small
				\texttt{autoplay}, \texttt{loop}, \texttt{showinfo}
			\normalsize
	\end{itemize}

\end{frame}

% ------------------------------------------------------------------------------
% LTXE-SLIDE-START
% LTXE-SLIDE-UID:		9ed71aaa-a7615392-78f0ac0c-f9bd7400
% LTXE-SLIDE-TITLE:		Improved handling of online media (4)
% LTXE-SLIDE-REFERENCE:	Feature-61799-ImprovedHandlingOfOnlineMedia.rst
% ------------------------------------------------------------------------------

\begin{frame}[fragile]
	\frametitle{Änderungen im System}
	\framesubtitle{Online Medien Unterstützung (4)}

	% decrease font size for code listing
	\lstset{basicstyle=\tiny\ttfamily}

	\begin{itemize}

		\item Für einen eigenen Media-Service benötigt man eine
			\small\texttt{OnlineMediaHelper}\normalsize\space Klasse, welche das
			\small\texttt{OnlineMediaHelperInterface}\normalsize\space implementiert, sowie eine
			\small\texttt{FileRenderer}\normalsize\space Klasse, die das
			\small\texttt{FileRendererInterface}\normalsize\space implementiert

			\begin{lstlisting}
				// Registrierung eines eigenen Online-Video-Services
				$GLOBALS['TYPO3_CONF_VARS']['SYS']['OnlineMediaHelpers']['myvideo'] =
				  \MyCompany\Myextension\Helpers\MyVideoHelper::class;

				$rendererRegistry = \TYPO3\CMS\Core\Resource\Rendering\RendererRegistry::getInstance();
				$rendererRegistry->registerRendererClass(
				  \MyCompany\Myextension\Rendering\MyVideoRenderer::class
				);

				// Registrierung eines eigenen Mime-Types
				$GLOBALS['TYPO3_CONF_VARS']['SYS']['FileInfo']['fileExtensionToMimeType']['myvideo'] =
				  'video/myvideo';

				// Registrierung einer eigenen Datei-Extension
				$GLOBALS['TYPO3_CONF_VARS']['SYS']['mediafile_ext'] .= ',myvideo';
			\end{lstlisting}

	\end{itemize}

\end{frame}

% ------------------------------------------------------------------------------
% LTXE-SLIDE-START
% LTXE-SLIDE-UID:		c675e3a8-9a429b0b-61a4a603-d5b0740f
% LTXE-SLIDE-TITLE:		Unified Backend Routing
% LTXE-SLIDE-REFERENCE:	Feature-65493-BackendRouting.rst
% ------------------------------------------------------------------------------

\begin{frame}[fragile]
	\frametitle{Änderungen im System}
	\framesubtitle{Backend Routing}

	% decrease font size for code listing
	\lstset{basicstyle=\tiny\ttfamily}

	\begin{itemize}

		\item Es wurde eine neue Routing Komponente zum TYPO3-Backend hinzugefügt,
			welche verschiedene Aufrufe handhaben kann\newline
			\small
				(z.B. \texttt{http://www.example.com/typo3/document/edit})
			\normalsize

		\item Die Routen werden in folgender Datei definiert:\newline
			\small
				\texttt{Configuration/Backend/Routes.php}
			\normalsize

			\begin{lstlisting}
				return [
				  'myRouteIdentifier' => [
				    'path' => '/document/edit',
				    'controller' => Acme\MyExtension\Controller\MyExampleController::class . '::methodToCall'
				  ]
				];
			\end{lstlisting}

		\item Die Methode erhält das Response- und Request-Objekt:

			\begin{lstlisting}
				public function methodToCall(ServerRequestInterface $request, ResponseInterface $response) {
				  ...
				}
			\end{lstlisting}

	\end{itemize}

\end{frame}

% ------------------------------------------------------------------------------
% LTXE-SLIDE-START
% LTXE-SLIDE-UID:		df56086a-498ae937-9b2bf393-a95828f3
% LTXE-SLIDE-TITLE:		Autoload definition can be provided in ext_emconf.php
% LTXE-SLIDE-REFERENCE:	Feature-68700-AutoloadDefinitionCanBeProvidedInExt_emconfphp.rst
% ------------------------------------------------------------------------------
\begin{frame}[fragile]
	\frametitle{Änderungen im System}
	\framesubtitle{Autoload Definition in \texttt{ext\_emconf.php}}

	% decrease font size for code listing
	\lstset{basicstyle=\tiny\ttfamily}

	\begin{itemize}

		\item Zusätzlich zur Datei \texttt{composer.json} können nun Autoload-Definitionen
			in der Datei \texttt{ext\_emconf.php} hinterlegt werden

		\item Das hat den Vorteil, dass nicht die gesamte Extension nach Klassen gescannt wird

			\begin{lstlisting}
				$EM_CONF[$_EXTKEY] = array (
				  'title' => 'Extension Skeleton for TYPO3 CMS 7',
				  ...
				'autoload' =>
				  array(
				    'psr-4' => array(
				      'Helhum\\ExtScaffold\\' => 'Classes'
				    )
				  )
				);
			\end{lstlisting}

	\end{itemize}

\end{frame}

% ------------------------------------------------------------------------------
% LTXE-SLIDE-START
% LTXE-SLIDE-UID:		137a9fca-b6558571-346141d2-9304a223
% LTXE-SLIDE-TITLE:		Icon-Factory (1)
% LTXE-SLIDE-REFERENCE:	Feature-68741-IntroduceNewIconFactoryAsBaseForReplaceTheIconSkinningAPI.rst
% LTXE-SLIDE-REFERENCE:	Feature-69095-IntroduceIconStateForIconFactory.rst
% ------------------------------------------------------------------------------
\begin{frame}[fragile]
	\frametitle{Änderungen im System}
	\framesubtitle{Neue Icon-Factory (1)}

	% decrease font size for code listing
	\lstset{basicstyle=\tiny\ttfamily}

	\begin{itemize}

		\item Die Logik, um mit Icons, Größen und Overlays zu arbeiten, wurde in die neue \texttt{IconFactory} ausgelagert

		\item Es gibt drei "IconProvider": \texttt{BitmapIconProvider}, \texttt{FontawesomeIconProvider}
			und \texttt{SvgIconProvider}

		\item Die Registrierung eines Icons erfolgt folgendermaßen:

			\begin{lstlisting}
				IconRegistry::registerIcon($identifier, $iconProviderClassName, array $options = array());
			\end{lstlisting}

	\end{itemize}

\end{frame}

% ------------------------------------------------------------------------------
% LTXE-SLIDE-START
% LTXE-SLIDE-UID:		6b59b4c7-21a7ff5c-4e67b53d-e4509355
% LTXE-SLIDE-TITLE:		Icon-Factory (2)
% LTXE-SLIDE-REFERENCE:	Feature-68741-IntroduceNewIconFactoryAsBaseForReplaceTheIconSkinningAPI.rst
% LTXE-SLIDE-REFERENCE:	Feature-69095-IntroduceIconStateForIconFactory.rst
% ------------------------------------------------------------------------------
\begin{frame}[fragile]
	\frametitle{Änderungen im System}
	\framesubtitle{Neue Icon-Factory (2)}

	% decrease font size for code listing
	\lstset{basicstyle=\tiny\ttfamily}

	\begin{itemize}

		\item Anwendung:

			\begin{lstlisting}
				$iconFactory = GeneralUtility::makeInstance(IconFactory::class);
				$iconFactory->getIcon(
				  $identifier,
				  Icon::SIZE_SMALL,
				  $overlay,
				  IconState::cast(IconState::STATE_DEFAULT)
				)->render();
			\end{lstlisting}

		\item Zulässige Werte für \texttt{Icon::SIZE\_...} sind:\newline
			\small\texttt{SIZE\_SMALL}, \texttt{SIZE\_DEFAULT} und \texttt{SIZE\_LARGE}\normalsize
			\vspace{0.4cm}

		\item Zulässige Werte für \texttt{Icon::STATE\_...} sind:\newline
			\small\texttt{STATE\_DEFAULT} und \texttt{STATE\_DISABLED}\normalsize

	\end{itemize}

\end{frame}

% ------------------------------------------------------------------------------
% LTXE-SLIDE-START
% LTXE-SLIDE-UID:		21a7eeb5-b4c7ff5c-b53de450-4e679355
% LTXE-SLIDE-TITLE:		Icon-Factory (3)
% LTXE-SLIDE-REFERENCE:	Feature-68741-IntroduceNewIconFactoryAsBaseForReplaceTheIconSkinningAPI.rst
% LTXE-SLIDE-REFERENCE:	Feature-69095-IntroduceIconStateForIconFactory.rst
% ------------------------------------------------------------------------------
\begin{frame}[fragile]
	\frametitle{Änderungen im System}
	\framesubtitle{Neue Icon-Factory (3)}

	% decrease font size for code listing
	\lstset{basicstyle=\tiny\ttfamily}

	\begin{itemize}

		\item Der Core stellt einen eigenen ViewHelper zur Verfügung, um Icons anzuzeigen:

			\begin{lstlisting}
				{namespace core = TYPO3\CMS\Core\ViewHelpers}

				<core:icon identifier="my-icon-identifier"></core:icon>

				<!-- use the "small" size if none given ->
				<core:icon identifier="my-icon-identifier"></core:icon>
				<core:icon identifier="my-icon-identifier" size="large"></core:icon>
				<core:icon identifier="my-icon-identifier" overlay="overlay-identifier"></core:icon>

				<core:icon identifier="my-icon-identifier" size="default" overlay="overlay-identifier">
				</core:icon>

				<core:icon identifier="my-icon-identifier" size="large" overlay="overlay-identifier">
				</core:icon>
			\end{lstlisting}

	\end{itemize}

\end{frame}

% ------------------------------------------------------------------------------
% LTXE-SLIDE-START
% LTXE-SLIDE-UID:		d9de708f-474f2f36-c104b01c-0d23a352
% LTXE-SLIDE-TITLE:		Signal for pre processing linkvalidator records
% LTXE-SLIDE-REFERENCE:	Feature-52217-SignalForPreProcessingLinkvalidatorRecords.rst
% ------------------------------------------------------------------------------
\begin{frame}[fragile]
	\frametitle{Änderungen im System}
	\framesubtitle{Hooks und Signals (1)}

	% decrease font size for code listing
	\lstset{basicstyle=\tiny\ttfamily}

	\begin{itemize}

		\item Es wurde ein Signal im LinkValidator zugefügt, welches die zusätzliche
			Verarbeitung eines Eintrages möglich macht\newline
			\small
				(z.B. um Daten aus der Plugin-Konfiguration zu ermitteln o.ä.).
			\normalsize

		\item Der Hook kann wie folgt in der Datei \texttt{ext\_localconf.php} registriert werden:

			\begin{lstlisting}
				$signalSlotDispatcher = \TYPO3\CMS\Core\Utility\GeneralUtility::makeInstance(
				  \TYPO3\CMS\Extbase\SignalSlot\Dispatcher::class
				);
				$signalSlotDispatcher->connect(
				  \TYPO3\CMS\Linkvalidator\LinkAnalyzer::class,
				  'beforeAnalyzeRecord',
				  \Vendor\Package\Slots\RecordAnalyzerSlot::class,
				  'beforeAnalyzeRecord'
				);
			\end{lstlisting}

	\end{itemize}

\end{frame}

% ------------------------------------------------------------------------------
% LTXE-SLIDE-START
% LTXE-SLIDE-UID:		bcb8ef73-b8669a17-102da209-5f5c9adc
% LTXE-SLIDE-TITLE:		Replaced JumpURL features with hooks (1)
% LTXE-SLIDE-REFERENCE:	Breaking-52156-ReplaceJumpUrlWithHooks.rst
% ------------------------------------------------------------------------------
\begin{frame}[fragile]
	\frametitle{Änderungen im System}
	\framesubtitle{JumpUrl als System-Extension (1)}

	% decrease font size for code listing
	\lstset{basicstyle=\tiny\ttfamily}

	\begin{itemize}

		\item Die Erzeugung und das Handling von JumpURLs wurde aus der Frontend-Extension
			entfernt und zur neuen System-Extension \texttt{jumpurl} verschoben

		\item Hook zur Manipulation von \textbf{URLs} in der Datei \texttt{ext\_localconf.php}:

			\begin{lstlisting}
				$GLOBALS['TYPO3_CONF_VARS']['SC_OPTIONS']['urlProcessing']['urlHandlers']
				  ['myext_myidentifier']['handler'] = \Company\MyExt\MyUrlHandler::class;

				// Die Klasse muss das UrlHandlerInterface implementieren
				class MyUrlHandler implements \TYPO3\CMS\Frontend\Http\UrlHandlerInterface {
				  ...
				}
			\end{lstlisting}

	\end{itemize}

	\breakingchange

\end{frame}

% ------------------------------------------------------------------------------
% LTXE-SLIDE-START
% LTXE-SLIDE-UID:		ecb8fef3-69b86a17-02da2109-9adc5f5c
% LTXE-SLIDE-TITLE:		Replaced JumpURL features with hooks (2)
% LTXE-SLIDE-REFERENCE:	Breaking-52156-ReplaceJumpUrlWithHooks.rst
% ------------------------------------------------------------------------------
\begin{frame}[fragile]
	\frametitle{Änderungen im System}
	\framesubtitle{JumpUrl als System-Extension (2)}

	% decrease font size for code listing
	\lstset{basicstyle=\tiny\ttfamily}

	\begin{itemize}

		\item Handling von \textbf{Links} in der Datei \texttt{ext\_localconf.php}:
			\normalsize

			\begin{lstlisting}
				$GLOBALS['TYPO3_CONF_VARS']['SC_OPTIONS']['urlProcessing']['urlProcessors']
				  ['myext_myidentifier']['processor'] = \Company\MyExt\MyUrlProcessor::class;

				// Die Klasse muss das UrlProcessorInterface implementieren
				class MyUrlProcessor implements \TYPO3\CMS\Frontend\Http\UrlProcessorInterface {
				  ...
				}
			\end{lstlisting}

	\end{itemize}

	\breakingchange

\end{frame}

% ------------------------------------------------------------------------------
% LTXE-SLIDE-START
% LTXE-SLIDE-UID:		08a70f0f-528daf11-ee485c29-d9707f25
% LTXE-SLIDE-TITLE:		Kommandozeilenaufruf (CLI)
% LTXE-SLIDE-REFERENCE:	Feature-67056-AddOptionToDisableMoveButtonsTCAGroupType.rst
% LTXE-SLIDE-REFERENCE:	Feature-67875-OverrideCategoryRegistryEntry.rst
% LTXE-SLIDE-REFERENCE:	Feature-68804-ColoredOutputForCLI-relevantErrorMessages.rst
% ------------------------------------------------------------------------------
\begin{frame}[fragile]
	\frametitle{Änderungen im System}
	\framesubtitle{Kommandozeilenaufruf (CLI)}

	% decrease font size for code listing
	\lstset{basicstyle=\tiny\ttfamily}

	\begin{itemize}

		\item Sollte es beim Aufruf von \texttt{typo3/cli\_dispatch.phpsh} zu Fehlern kommen,
			so werden diese farbig dargestellt

		\item CommandController können nun auch in Unterordnern liegen

		\item Beispiel:\newline

			Controller in Datei:\newline
			\smaller\texttt{my\_ext/Classes/Command/Hello/WorldCommandController.php}\normalsize\newline
			...kann im CLI wie folgt aufgerufen werden:\newline
			\smaller\texttt{typo3/cli\_dispatch.sh extbase my\_ext:hello:world <arguments>}\normalsize

	\end{itemize}

\end{frame}

% ------------------------------------------------------------------------------
% LTXE-SLIDE-START
% LTXE-SLIDE-UID:		a708daf1-e48f525c-7010fe7f-2529d9e4
% LTXE-SLIDE-TITLE:		Diverse Änderungen (1)
% LTXE-SLIDE-REFERENCE:	Feature-68804-ColoredOutputForCLI-relevantErrorMessages.rst
% LTXE-SLIDE-REFERENCE:	Feature-69512-SupportTyposcriptFilesAsTextFileType.rst
% LTXE-SLIDE-REFERENCE:	Feature-69543-IntroducedGLOBALSTYPO3_CONF_VARSSYSmediafile_ext.rst
% ------------------------------------------------------------------------------
\begin{frame}[fragile]
	\frametitle{Änderungen im System}
	\framesubtitle{Diverse Änderungen (1)}

	\begin{itemize}

		\item Die Verschieben-Buttons beim TCA-Type \texttt{group} können
			nun mit der TCA-Option \texttt{hideMoveIcons = TRUE} deaktiviert werden

		\item Die Funktion \texttt{makeCategorizable()} kann nun überschrieben werden,
			sofern diese vorher bereits aufgerufen wurde (z.B. für \texttt{tt\_content}).

		\item Beispiel:

			\begin{lstlisting}
				\TYPO3\CMS\Core\Utility\ExtensionManagementUtility::makeCategorizable(
				  'css_styled_content', 'tt_content', 'categories', array(), TRUE
				);
			\end{lstlisting}

			\small
				Der letzte Parameter steuert das Überschreiben (hier: \texttt{TRUE}).\newline
				Standardmäßig ist der Wert \texttt{FALSE}.
			\normalsize

	\end{itemize}

\end{frame}

% ------------------------------------------------------------------------------
% LTXE-SLIDE-START
% LTXE-SLIDE-UID:		d52d6418-554d801f-352b85cb-54046d28
% LTXE-SLIDE-TITLE:		Diverse Änderungen (2)
% LTXE-SLIDE-REFERENCE:	Feature-69730-IntroduceUniqueIdGenerator.rst
% LTXE-SLIDE-REFERENCE:	Important-68758-CommandControllersAllowedInSubfolders.rst
% ------------------------------------------------------------------------------

\begin{frame}[fragile]
	\frametitle{Änderungen im System}
	\framesubtitle{Diverse Änderungen (2)}

	% decrease font size for code listing
	%\lstset{basicstyle=\tiny\ttfamily}

	\begin{itemize}

		\item Es gibt nun eine Funktion, um eine Unique-ID zu erzeugen

			\begin{lstlisting}
				$uniqueId = \TYPO3\CMS\Core\Utility\StringUtility::getUniqueId('Prefix');
			\end{lstlisting}

		\item Als Plaintext Dateiendung wurde \texttt{typoscript} hinzugefügt

		\item Es gibt nun eine neue Konfigurations-Option, die regelt, welche Dateiendungen
			als Media-Dateien interpretiert werden:

			\begin{lstlisting}
				$GLOBALS['TYPO3_CONF_VARS']['SYS']['mediafile_ext'] =
				  'gif,jpg,jpeg,bmp,png,pdf,svg,ai,mov,avi';
			\end{lstlisting}

	\end{itemize}

	\breakingchange

\end{frame}

% ------------------------------------------------------------------------------
