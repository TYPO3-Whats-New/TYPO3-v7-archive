% ------------------------------------------------------------------------------
% TYPO3 CMS 7.5 - What's New - Chapter "TypoScript" (German Version)
%
% @author	Patrick Lobacher <patrick@lobacher.de> and Michael Schams <schams.net>
% @license	Creative Commons BY-NC-SA 3.0
% @link		http://typo3.org/download/release-notes/whats-new/
% @language	German
% ------------------------------------------------------------------------------
% LTXE-CHAPTER-UID:		f607fb32-7f7b68d8-68d0e69a-16738006
% LTXE-CHAPTER-NAME:	TypoScript
% ------------------------------------------------------------------------------
% LTXE-SLIDE-START
% LTXE-SLIDE-UID:		c76757e2-fc02b94b-166a8168-d68774c9
% LTXE-SLIDE-TITLE:		Feature #16525: Add conditions to INCLUDE_TYPOSCRIPT
% LTXE-SLIDE-REFERENCE:	Feature-16525-AddConditionsToINCLUDE_TYPOSCRIPT.rst
% ------------------------------------------------------------------------------
\begin{frame}[fragile]
	\frametitle{TSconfig \& TypoScript}
	\framesubtitle{Conditions für TypoScript-Include}

	% decrease font size for code listing
	\lstset{basicstyle=\tiny\ttfamily}

	\begin{itemize}
		\item Der \texttt{INCLUDE\_TYPOSCRIPT} Tag besitzt nun das optionale Attribut "condition",
			welches es ermöglicht, die Datei (oder das Verzeichnis) nur dann zu inkludieren,
			wenn die Condition erfüllt ist:

		\begin{lstlisting}
			// TypoScript nur laden, wenn Benutzer eingeloggt ist:
			<INCLUDE_TYPOSCRIPT: source="FILE:EXT:my_extension/Configuration/TypoScript/feuser.ts"
			  condition="[loginUser = *]">

			// TypoScript nur laden, wenn ApplicationContext gesetzt ist:
			<INCLUDE_TYPOSCRIPT: source="FILE:EXT:my_extension/Configuration/TypoScript/staging.ts"
			  condition="applicationContext = /^Production\\/Staging\\/Server\\d+$/">
		\end{lstlisting}

	\end{itemize}

\end{frame}

% ------------------------------------------------------------------------------
% LTXE-SLIDE-START
% LTXE-SLIDE-UID:		1bde9e8a-82352f02-c5c9f3db-b194cfb1
% LTXE-SLIDE-TITLE:		Feature #28243: Introduce TCA option to disable age display of dates per field
% LTXE-SLIDE-REFERENCE:	Feature-28243-IntroduceTcaOptionToDisableAgeDisplay.rst
% ------------------------------------------------------------------------------
\begin{frame}[fragile]
	\frametitle{TSconfig \& TypoScript}
	\framesubtitle{TCA-Option, um Datum Feldweise auszublenden}

	% decrease font size for code listing
	\lstset{basicstyle=\tiny\ttfamily}

	\begin{itemize}

		\item Es gibt nun eine TCA-Option \texttt{disableAgeDisplay}, um die Anzeige des Datums auszublenden

			Voraussetzung hierfür ist, dass der Typ des Feldes \texttt{input} ist, und \texttt{eval} auf
			\texttt{date} gesetzt ist

		\begin{lstlisting}
			$GLOBALS['TCA']['tt_content']['columns']['date']['config']['disableAgeDisplay'] = true;
		\end{lstlisting}

	\end{itemize}

\end{frame}

% ------------------------------------------------------------------------------
% LTXE-SLIDE-START
% LTXE-SLIDE-UID:		c3aa0981-411fcabb-bd9dca10-16f42898
% LTXE-SLIDE-TITLE:		Feature #57632: Include inline language label files with TypoScript (1)
% LTXE-SLIDE-REFERENCE:	Feature-57632-AddInlineLanguageLabelFilesWithTypoScript.rst
% ------------------------------------------------------------------------------
\begin{frame}[fragile]
	\frametitle{TSconfig \& TypoScript}
	\framesubtitle{Inline Sprachlabels mit TypoScript (1)}

	% decrease font size for code listing
	\lstset{basicstyle=\tiny\ttfamily}

	\begin{itemize}

		\item Man kann nun Sprachdateien mittels TypoScript auslesen und als Inline-Array
			in den Quelltext schreiben, um z.B. per JavaScript darauf zuzugreifen

		\item Folgende Optionen sind möglich:

			\begin{itemize}
				\item \texttt{selectionPrefix}:\newline
					nur Schlüssel, die mit diesem Prefix anfangen, werden ermittelt
				\item \texttt{stripFromSelectionName}:\newline
					String, der von jedem Schlüssel entfernt wird
				\item \texttt{errorMode}:\newline
					Mode, wenn die Sprachdatei nicht gefunden wird\newline
					(0: Eintrag im Syslog vornehmen, 1: ignorieren, 3: Exception generieren)
			\end{itemize}

	\end{itemize}

\end{frame}

% ------------------------------------------------------------------------------
% LTXE-SLIDE-START
% LTXE-SLIDE-UID:		289816f4-411fcabb-0981cabb-bd9dca10
% LTXE-SLIDE-TITLE:		Feature #57632: Include inline language label files with TypoScript (2)
% LTXE-SLIDE-REFERENCE:	Feature-57632-AddInlineLanguageLabelFilesWithTypoScript.rst
% ------------------------------------------------------------------------------
\begin{frame}[fragile]
	\frametitle{TSconfig \& TypoScript}
	\framesubtitle{Inline Sprachlabels mit TypoScript (2)}

	% decrease font size for code listing
	\lstset{basicstyle=\tiny\ttfamily}

	\begin{itemize}

		\item Beispiel:

			\begin{lstlisting}
				page = PAGE
				page.inlineLanguageLabelFiles {
				  someLabels = EXT:myExt/Resources/Private/Language/locallang.xlf
				  someLabels.selectionPrefix = idPrefix
				  someLabels.stripFromSelectionName = strip_me
				  someLabels.errorMode = 2
				}
			\end{lstlisting}

		\item Ausgabe:

			\begin{lstlisting}
				<script type="text/javascript">
				/*<![CDATA[*/
				  var TYPO3 = TYPO3 || {};
				  TYPO3.lang = {"firstLabel":[{"source":"first Label","target":"erstes Label"}],
				  "secondLabel":[{"source":"second Label","target":"zweites Label"}]};
				/*]]>*/
				</script>
			\end{lstlisting}

	\end{itemize}

\end{frame}

% ------------------------------------------------------------------------------
% LTXE-SLIDE-START
% LTXE-SLIDE-UID:		f14c90e2-bf35bf40-e38d74a7-f4a2e973
% LTXE-SLIDE-TITLE:		Feature #59144: Previewing workspace records using Page TSconfig
% LTXE-SLIDE-REFERENCE:	Feature-59144-PageTSconfigWorkspacePreview.rst
% ------------------------------------------------------------------------------
\begin{frame}[fragile]
	\frametitle{TSconfig \& TypoScript}
	\framesubtitle{Workspace Preview per TSconfig}

	% decrease font size for code listing
	\lstset{basicstyle=\tiny\ttfamily}

	\begin{itemize}

		\item Standardmäßig erzeugt TYPO3 lediglich Vorschau-Links für die Tabellen \texttt{tt\_content},
			\texttt{pages} und \texttt{pages\_language\_overlay}

		\item Dies kann nun per PageTSconfig angepasst werden:

			\begin{lstlisting}
				# Verwendung der Seite 123 fuer Workspace Preview (fuer alle Tabellen)
				options.workspaces.previewPageId = 123

				# Verwendung des Feldes pid (fuer alle Tabellen)
				options.workspaces.previewPageId = field:pid

				# Verwendung der Seite 123 fuer Workspace Preview (fuer die Tabelle tx_myext_table)
				options.workspaces.previewPageId.tx_myext_table = 123

				# Verwendung des Feldes pid fuer Workspace Preview (fuer die Tabelle tx_myext_table)
				options.workspaces.previewPageId.tx_myext_table = field:pid
			\end{lstlisting}

	\end{itemize}

\end{frame}

% ------------------------------------------------------------------------------
% LTXE-SLIDE-START
% LTXE-SLIDE-UID:		faeca558-5a675884-a1b77fae-36e316eb
% LTXE-SLIDE-TITLE:		Feature #59591: Image quality definable per sourceCollection
% LTXE-SLIDE-REFERENCE:	Feature-59591-ImageQualityDefinablePerSourceCollection.rst
% ------------------------------------------------------------------------------
\begin{frame}[fragile]
	\frametitle{TSconfig \& TypoScript}
	\framesubtitle{Bildqualität kann per SourceCollection gesetzt werden}

	% decrease font size for code listing
	\lstset{basicstyle=\tiny\ttfamily}

	\begin{itemize}

		\item Die Bildqualität jeder \text{sourceCollection} kann nun konfiguriert werden

		\item Dies überschreibt die Einstellungen, die im Install Tool gemacht wurden und
			in der Datei \texttt{LocalConfiguration.php} gespeichert sind

			\begin{lstlisting}
				# fuer kleine Retina Bilder
				tt_content.image.20.1.sourceCollection.smallRetina.quality = 80

				# fuer groessere Retina Bilder
				tt_content.image.20.1.sourceCollection.largeRetina.quality = 65
			\end{lstlisting}
	\end{itemize}

\end{frame}

% ------------------------------------------------------------------------------
% LTXE-SLIDE-START
% LTXE-SLIDE-UID:		0073160e-6c95f70d-fcaf008f-3fd84819
% LTXE-SLIDE-TITLE:		Feature #67880: Added count to split
% LTXE-SLIDE-REFERENCE:	Feature-67880-AddedCountToSplit.rst
% ------------------------------------------------------------------------------
\begin{frame}[fragile]
	\frametitle{TSconfig \& TypoScript}
	\framesubtitle{Count für Split hinzugefügt}

	% decrease font size for code listing
	\lstset{basicstyle=\tiny\ttfamily}

	\begin{itemize}

		\item Es wurde eine neue Eigenschaft \texttt{returnCount} zur stdWrap-Funktion
			\texttt{split} hinzugefügt, die die Anzahl der Elemente nach dem Split enthält

			\begin{lstlisting}
				1 = TEXT
				1 {
				  value = x,y,z,1,2,3,a,b,c
				  split.token = ,
				  split.returnCount = 1
				}

				# result: 9
			\end{lstlisting}

	\end{itemize}

\end{frame}

% ------------------------------------------------------------------------------
% LTXE-SLIDE-START
% LTXE-SLIDE-UID:		da2aa521-48abf012-e8b4b95f-0e3f3556
% LTXE-SLIDE-TITLE:		Feature #69602: Simplify handling of backend layouts in frontend (1)
% LTXE-SLIDE-REFERENCE:	Feature-69602-SimplifyHandlingOfBackendLayoutsInFrontend.rst
% ------------------------------------------------------------------------------

\begin{frame}[fragile]
	\frametitle{TSconfig \& TypoScript}
	\framesubtitle{Handling von Backend-Layouts vereinfacht (1)}

	% decrease font size for code listing
	\lstset{basicstyle=\tiny\ttfamily}

	\begin{itemize}

		\item Das Handling, um Backend-Layouts mit Templates für die Frontend-Ausgabe zu versehen,
			wurde vereinfacht, indem die Option \texttt{pagelayout} eingeführt wurde

		\item Beispiel:

			\begin{lstlisting}
				page.10 = FLUIDTEMPLATE
				page.10 {
				  file.stdWrap.cObject = CASE
				  file.stdWrap.cObject {
				    key.data = pagelayout
				    default = TEXT
				    default.value = EXT:sitepackage/Resources/Private/Templates/Home.html
				    3 = TEXT
				    3.value = EXT:sitepackage/Resources/Private/Templates/1-col.html
				    4 = TEXT
				    4.value = EXT:sitepackage/Resources/Private/Templates/2-col.html
				  }
				}
			\end{lstlisting}

			\smaller
				(Fortsetzung auf nächster Seite)
			\normalsize

	\end{itemize}

\end{frame}

% ------------------------------------------------------------------------------
% LTXE-SLIDE-START
% LTXE-SLIDE-UID:		b95fe8b4-a521da2a-f01248ab-35560e3f
% LTXE-SLIDE-TITLE:		Feature #69602: Simplify handling of backend layouts in frontend (2)
% LTXE-SLIDE-REFERENCE:	Feature-69602-SimplifyHandlingOfBackendLayoutsInFrontend.rst
% ------------------------------------------------------------------------------

\begin{frame}[fragile]
	\frametitle{TSconfig \& TypoScript}
	\framesubtitle{Handling von Backend-Layouts vereinfacht (2)}

	% decrease font size for code listing
	\lstset{basicstyle=\tiny\ttfamily}

	\begin{itemize}

		\item \texttt{pagelayout} ersetzt dabei den folgenden Code:

			\begin{lstlisting}
				field = backend_layout
				ifEmpty.data = levelfield:-2,backend_layout_next_level,slide
				ifEmpty.ifEmpty = default
			\end{lstlisting}

	\end{itemize}

\end{frame}

% ------------------------------------------------------------------------------
% LTXE-SLIDE-START
% LTXE-SLIDE-UID:		6f1f7bb6-b50c019e-29c9be36-7d30c87d
% LTXE-SLIDE-TITLE:		Feature #68756: Add config "base" to stdWrap
% LTXE-SLIDE-REFERENCE:	Feature-68756-AddConfigBaseToStdWrap.rst
% ------------------------------------------------------------------------------
\begin{frame}[fragile]
	\frametitle{TSconfig \& TypoScript}
	\framesubtitle{Diverse}

	\begin{itemize}

		\item Für die mit TYPO3 CMS 7.4 eingeführte stdWrap-Funktion \texttt{bytes}
			kann nun die Basis (z.B. 1000 oder 1024) gesetzt werden:\newline
			\texttt{bytes.base = 1000}

	\end{itemize}

\end{frame}

% ------------------------------------------------------------------------------
