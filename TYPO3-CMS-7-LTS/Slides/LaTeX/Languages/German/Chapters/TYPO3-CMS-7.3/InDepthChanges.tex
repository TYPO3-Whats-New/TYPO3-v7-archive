% ------------------------------------------------------------------------------
% TYPO3 CMS 7.3 - What's New - Chapter "In-Depth Changes" (German Version)
%
% @author	Patrick Lobacher <patrick@lobacher.de> and Michael Schams <schams.net>
% @license	Creative Commons BY-NC-SA 3.0
% @link		http://typo3.org/download/release-notes/whats-new/
% @language	German
% ------------------------------------------------------------------------------
% LTXE-CHAPTER-UID:		e8cc33e1-d7bdb74f-0421b75f-7be06b19
% LTXE-CHAPTER-NAME:	In-Depth Changes
% ------------------------------------------------------------------------------
% LTXE-SLIDE-START
% LTXE-SLIDE-UID:		bd03e141-3842b75e-8a44920f-a8c139e5
% LTXE-SLIDE-TITLE:		Feature #59606: Integrate Symfony/Console into CommandController (1)
% LTXE-SLIDE-REFERENCE:	Feature-59606-IntegrateSymfonyConsoleIntoCommandController.rst
% ------------------------------------------------------------------------------

\begin{frame}[fragile]
	\frametitle{Änderungen im System}
	\framesubtitle{Symfony/Console im CommandController (1)}

	% decrease font size for code listing
	\lstset{basicstyle=\tiny\ttfamily}

	Der CommandController verwendet nun intern Symfony/Console und stellt damit
	verschiedene Methoden zur Verfügung:

	\begin{itemize}

		\item \smaller TableHelper
			\begin{itemize}
				\item\smaller\texttt{outputTable(\$rows, \$headers = NULL)}
			\end{itemize}

		\item DialogHelper
			\begin{itemize}
				\item\smaller\texttt{select(\$question, \$choices, \$default = NULL, \$multiSelect = false, \$attempts = FALSE)}
				\item\texttt{ask(\$question, \$default = NULL, array \$autocomplete = array())}
				\item\texttt{askConfirmation(\$question, \$default = TRUE)}
				\item\texttt{askHiddenResponse(\$question, \$fallback = TRUE)}
				\item\texttt{askAndValidate(\$question, \$validator, \$attempts = FALSE, \$default = NULL, array \$autocomplete = NULL)}
				\item\texttt{askHiddenResponseAndValidate(\$question, \$validator, \$attempts = FALSE, \$fallback = TRUE)}
			\end{itemize}

	\end{itemize}

\end{frame}

% ------------------------------------------------------------------------------
% LTXE-SLIDE-START
% LTXE-SLIDE-UID:		3842b75e-8a44920f-a8c139e5-bd03e141
% LTXE-SLIDE-TITLE:		Feature #59606: Integrate Symfony/Console into CommandController (2)
% LTXE-SLIDE-REFERENCE:	Feature-59606-IntegrateSymfonyConsoleIntoCommandController.rst
% ------------------------------------------------------------------------------

\begin{frame}[fragile]
	\frametitle{Änderungen im System}
	\framesubtitle{Symfony/Console im CommandController (2)}

	% decrease font size for code listing
	\lstset{basicstyle=\tiny\ttfamily}

	\begin{itemize}
		\item \smaller ProgressHelper
			\begin{itemize}
				\item \smaller\texttt{progressStart(\$max = NULL)}
				\item \texttt{progressSet(\$current)}
				\item \texttt{progressAdvance(\$step = 1)}
				\item \texttt{progressFinish()}
			\end{itemize}
	\end{itemize}

	\smaller
		(siehe Code-Beispiel auf den folgenden Slides)
	\normalsize

\end{frame}

% ------------------------------------------------------------------------------
% LTXE-SLIDE-START
% LTXE-SLIDE-UID:		a8c139e5-8a44920f-3842b75e-bd03e141
% LTXE-SLIDE-TITLE:		Feature #59606: Integrate Symfony/Console into CommandController (3)
% LTXE-SLIDE-REFERENCE:	Feature-59606-IntegrateSymfonyConsoleIntoCommandController.rst
% ------------------------------------------------------------------------------

\begin{frame}[fragile]
	\frametitle{Änderungen im System}
	\framesubtitle{Symfony/Console im CommandController (3)}

	% decrease font size for code listing
	\lstset{basicstyle=\tiny\ttfamily}

		\begin{lstlisting}
			<?php
			namespace Acme\Demo\Command;
			use TYPO3\CMS\Extbase\Mvc\Controller\CommandController;

			class MyCommandController extends CommandController {
			  public function myCommand() {

			    // Table rendern
			    $this->output->outputTable(array(
			      array('Bob', 34, 'm'),
			      array('Sally', 21, 'f'),
			      array('Blake', 56, 'm')
			    ),
			    array('Name', 'Age', 'Gender'));

			    // Selektieren
			    $colors = array('red', 'blue', 'yellow');
			    $selectedColorIndex = $this->output->select('Please select one color', $colors, 'red');
			    $this->outputLine('You choose the color %s.', array($colors[$selectedColorIndex]));

			    [...]
		\end{lstlisting}

\end{frame}

% ------------------------------------------------------------------------------
% LTXE-SLIDE-START
% LTXE-SLIDE-UID:		a8c139e5-8a44920f-3842b75e-bd03e141
% LTXE-SLIDE-TITLE:		Feature #59606: Integrate Symfony/Console into CommandController (4)
% LTXE-SLIDE-REFERENCE:	Feature-59606-IntegrateSymfonyConsoleIntoCommandController.rst
% ------------------------------------------------------------------------------

\begin{frame}[fragile]
	\frametitle{Änderungen im System}
	\framesubtitle{Symfony/Console im CommandController (4)}

	% decrease font size for code listing
	\lstset{basicstyle=\tiny\ttfamily}

		\begin{lstlisting}
			    [...]
			    // Abfrage
			    $name = $this->output->ask('What is your name?' . PHP_EOL, 'Bob', array('Bob', 'Sally', 'Blake'));
			    $this->outputLine('Hello %s.', array($name));

			    // Prompt
			    $likesDogs = $this->output->askConfirmation('Do you like dogs?');
			    if ($likesDogs) {
			      $this->outputLine('You do like dogs!');
			    }

			    // Progress
			    $this->output->progressStart(600);
			    for ($i = 0; $i < 300; $i ++) {
			      $this->output->progressAdvance();
			      usleep(5000);
			    }
			    $this->output->progressFinish();
			  }
			}
			?>
		\end{lstlisting}

\end{frame}

% ------------------------------------------------------------------------------
% LTXE-SLIDE-START
% LTXE-SLIDE-UID:		9d56ca19-04a98896-ba0a973c-8a16e877
% LTXE-SLIDE-TITLE:		Feature #66669: BE login form API (1)
% LTXE-SLIDE-REFERENCE:	Feature-66669-BeLoginFormAPI.rst
% ------------------------------------------------------------------------------

\begin{frame}[fragile]
	\frametitle{Änderungen im System}
	\framesubtitle{Backend Login API (1)}

	% decrease font size for code listing
	\lstset{basicstyle=\tiny\ttfamily}

	\begin{itemize}

		\item Das Backend-Login wurde nun komplett als API realisiert und lässt
			sich damit selbst per Programmierung ansprechen

		\item Grundsätzlich muss man dazu einen sogenannten Login-Provider in
			der Datei \texttt{ext\_localconf.php} registrieren:

			\begin{lstlisting}
				$GLOBALS['TYPO3_CONF_VARS']['EXTCONF']['backend']['loginProviders'][1433416020] = [
				  'provider' => \TYPO3\CMS\Backend\LoginProvider\UsernamePasswordLoginProvider::class,
				  'sorting' => 50,
				  'icon-class' => 'fa-key',
				  'label' => 'LLL:EXT:backend/Resources/Private/Language/locallang.xlf:login.link'
				];
			\end{lstlisting}

	\end{itemize}

\end{frame}

% ------------------------------------------------------------------------------
% LTXE-SLIDE-START
% LTXE-SLIDE-UID:		9889604a-ca199d56-73cba0a9-e8778a16
% LTXE-SLIDE-TITLE:		Feature #66669: BE login form API (2)
% LTXE-SLIDE-REFERENCE:	Feature-66669-BeLoginFormAPI.rst
% ------------------------------------------------------------------------------

\begin{frame}[fragile]
	\frametitle{Änderungen im System}
	\framesubtitle{Backend Login API (2)}

	\begin{itemize}

		\item Die Optionen sind wie folgt definiert:

			\begin{itemize}

				\item \texttt{provider}:\newline
					Name der Login-Provider Klasse, die das Interface
						\texttt{TYPO3\textbackslash
							CMS\textbackslash
							Backend\textbackslash
							LoginProvider\textbackslash
							LoginProviderInterface}
						implementiert

				\item \texttt{sorting}:\newline
					Sortierung, der die Reihenfolge der Links zum Login-Provider
					festlegt

				\item \texttt{icon-class}:\newline
					Font-Awesome Icon-Name für den Link auf dem Login-Screen

				\item \texttt{label}:\newline
					Label für den Link auf dem Login-Screen
			\end{itemize}

	\end{itemize}

\end{frame}

% ------------------------------------------------------------------------------
% LTXE-SLIDE-START
% LTXE-SLIDE-UID:		04a98896-ba0a973c-8a16e877-9d56ca19
% LTXE-SLIDE-TITLE:		Feature #66669: BE login form API (3)
% LTXE-SLIDE-REFERENCE:	Feature-66669-BeLoginFormAPI.rst
% ------------------------------------------------------------------------------

\begin{frame}[fragile]
	\frametitle{Änderungen im System}
	\framesubtitle{Backend Login API (3)}

	\begin{itemize}

		\item Das \texttt{LoginProviderInterface} enthält lediglich die Methode\newline
			\smaller\texttt{public function render(StandaloneView \$view, PageRenderer \$pageRenderer, LoginController \$loginController);}\normalsize

		\item Die Parameter sind wie folgt definiert:

			\begin{itemize}
				\item \texttt{\$view}:\newline
					Der Fluid-StandaloneView, welcher den Login-Screen rendert.
					Hier muss das Template-File gesetzt und die Variablen
					übergeben werden, die man benötigt.
				\item \texttt{\$pageRenderer}:\newline
					PageRenderer-Instanz, welche z.B. JavaScript-Ressourcen
					einbringen kann.
				\item \texttt{\$loginController}:\newline
					Instanz des Login-Controllers.
			\end{itemize}

	\end{itemize}

\end{frame}

% ------------------------------------------------------------------------------
% LTXE-SLIDE-START
% LTXE-SLIDE-UID:		04a98896-ba0a973c-8a16e877-9d56ca19
% LTXE-SLIDE-TITLE:		Feature #66669: BE login form API (4)
% LTXE-SLIDE-REFERENCE:	Feature-66669-BeLoginFormAPI.rst
% ------------------------------------------------------------------------------

\begin{frame}[fragile]
	\frametitle{Änderungen im System}
	\framesubtitle{Backend Login API (4)}

	% decrease font size for code listing
	\lstset{basicstyle=\tiny\ttfamily}

	\begin{itemize}

		\item Das Template muss
			\texttt{<f:layout name="Login">} und\newline
			\texttt{<f:section name="loginFormFields">}
			(für die Formular-Felder) enthalten:

			\begin{lstlisting}
				<f:layout name="Login" />
				<f:section name="loginFormFields">
				  <div class="form-group t3js-login-openid-section" id="t3-login-openid_url-section">
				    <div class="input-group">
				      <input type="text" id="openid_url"
				        name="openid_url"
				        value="{presetOpenId}"
				        autofocus="autofocus"
				        placeholder="{f:translate(key: 'openId', extensionName: 'openid')}"
				        class="form-control input-login t3js-clearable t3js-login-openid-field" />
				      <div class="input-group-addon">
				        <span class="fa fa-openid"></span>
				      </div>
				    </div>
				  </div>
				</f:section>
			\end{lstlisting}

	\end{itemize}

\end{frame}

% ------------------------------------------------------------------------------
% LTXE-SLIDE-START
% LTXE-SLIDE-UID:		cc63cd98-524ee6df-38430963-f7c7067c
% LTXE-SLIDE-TITLE:		Feature #66681: CategoryRegistry: add options to set l10n_mode and l10n_display
% LTXE-SLIDE-REFERENCE:	Feature-66681-CategoryRegistryAddOptionsToSetL10n_modeAndL10n_display.rst
% ------------------------------------------------------------------------------

\begin{frame}[fragile]
	\frametitle{Änderungen im System}
	\framesubtitle{\texttt{CategoryRegistry} mit neue Optionen}

	% decrease font size for code listing
	\lstset{basicstyle=\tiny\ttfamily}

	\begin{itemize}

		\item Die Methode \texttt{CategoryRegistry->addTcaColumn} hat Optionen
			erhalten, um sowohl \texttt{l10n\_mode} als auch \texttt{l10n\_display}
			zu setzen:

			\begin{lstlisting}
				\TYPO3\CMS\Core\Utility\ExtensionManagementUtility::makeCategorizable(
				  $extensionKey,
				  $tableName,
				  'categories',
				  array(
				    'l10n_mode' => 'string (keyword)',
				    'l10n_display' => 'list of keywords'
				  )
				);
			\end{lstlisting}

	\end{itemize}

\end{frame}

% ------------------------------------------------------------------------------
% LTXE-SLIDE-START
% LTXE-SLIDE-UID:		c9a5288c-f45aac1e-34606612-fdf6ec18
% LTXE-SLIDE-TITLE:		Feature #66822: Allow Sprites For Backend Modules
% LTXE-SLIDE-REFERENCE:	Feature-66822-SpriteIconsInBackendModules.rst
% ------------------------------------------------------------------------------

\begin{frame}[fragile]
	\frametitle{Änderungen im System}
	\framesubtitle{Sprites in Backend Modulen}

	% decrease font size for code listing
	\lstset{basicstyle=\tiny\ttfamily}

	\begin{itemize}

		\item Backend Module (z.B. Hauptmodule wie "Web" sowie Untermodule wie
			"Filelist") können nun Sprites als Icons verwenden\newline
			\small
				(nur Sprites, die TYPO3 bekannt sind!)
			\normalsize

		\item Beispiel:

			\begin{lstlisting}
				\TYPO3\CMS\Core\Utility\ExtensionManagementUtility::addModule(
				  'web',
				  'layout',
				  'top',
				  \TYPO3\CMS\Core\Utility\ExtensionManagementUtility::extPath($_EXTKEY) . 'Modules/Layout/',
				  array(
				    'script' => '_DISPATCH',
				    'access' => 'user,group',
				    'name' => 'web_layout',
				    'configuration' => array('icon' => 'module-web'),
				    'labels' => array(
				      'll_ref' => 'LLL:EXT:cms/layout/locallang_mod.xlf',
				    ),
				  )
				);
			\end{lstlisting}

	\end{itemize}

\end{frame}

% ------------------------------------------------------------------------------
% LTXE-SLIDE-START
% LTXE-SLIDE-UID:		d9657b5d-5cd56c65-d7266a27-401dd233
% LTXE-SLIDE-TITLE:		Feature #67229: FormEngine NodeFactory API
% LTXE-SLIDE-REFERENCE:	Feature-67229-FormEngineNodeFactoryApi.rst
% ------------------------------------------------------------------------------

\begin{frame}[fragile]
	\frametitle{Änderungen im System}
	\framesubtitle{FormEngine NodeFactory API (1)}

	% decrease font size for code listing
	\lstset{basicstyle=\tiny\ttfamily}

	\begin{itemize}

		\item Es ist nun möglich, neue Nodes zu registrieren und bestehende zu
			überschreiben

		\begin{lstlisting}
			$GLOBALS['TYPO3_CONF_VARS']['SYS']['formEngine']['nodeRegistry'][1433196792] = array(
			  'nodeName' => 'input',
			  'priority' => 40,
			  'class' => \MyVendor\MyExtension\Form\Element\T3editorElement::class
			);
		\end{lstlisting}

		\item Dies registriert eine neue Klasse, die den TCA-Type \texttt{input}
			rendert und das \texttt{NodeInterface} implementieren muss

		\item Als Array-Key wird der Unix-Timestamp des Zeitpunkts verwendet,
			wenn man das Element zufügt

	\end{itemize}

\end{frame}

% ------------------------------------------------------------------------------
% LTXE-SLIDE-START
% LTXE-SLIDE-UID:		5cd56c65-d7266a27-401dd233-d9657b5d
% LTXE-SLIDE-TITLE:		Feature #67229: FormEngine NodeFactory API
% LTXE-SLIDE-REFERENCE:	Feature-67229-FormEngineNodeFactoryApi.rst
% ------------------------------------------------------------------------------

\begin{frame}[fragile]
	\frametitle{Änderungen im System}
	\framesubtitle{FormEngine NodeFactory API (2)}

	% decrease font size for code listing
	\lstset{basicstyle=\tiny\ttfamily}

	\begin{itemize}

		\item Wenn mehrere Registry-Elemente für den selben Typ registriert
			werden, gewinnt das, mit der höchsten Priorität (0-100)

		\item Ein neuer TCA-Type wird wie folgt registriert:

		\smaller\textbf{\texttt{TCA}}
		\begin{lstlisting}
			'columns' => array(
			  'bodytext' => array(
			    'config' => array(
			      'type' => 'text',
			      'renderType' => '3dCloud',
			    ),
			  ),
			)
		\end{lstlisting}

		\smaller\textbf{\texttt{ext\_localconf.php}}
		\begin{lstlisting}
			$GLOBALS['TYPO3_CONF_VARS']['SYS']['formEngine']['nodeRegistry'][1433197759] = array(
			  'nodeName' => '3dCloud',
			  'priority' => 40,
			  'class' => \MyVendor\MyExtension\Form\Element\ShowTextAs3dCloudElement::class
			);
		\end{lstlisting}

	\end{itemize}

\end{frame}

% ------------------------------------------------------------------------------
% LTXE-SLIDE-START
% LTXE-SLIDE-UID:		d7266a27-401dd233-d9657b5d-5cd56c65
% LTXE-SLIDE-TITLE:		Breaking #62983: postProcessMirrorUrl signal has moved
% LTXE-SLIDE-REFERENCE:	Breaking-62983-PostProcessMirrorUrlSignalHasMoved.rst
% ------------------------------------------------------------------------------

\begin{frame}[fragile]
	\frametitle{Änderungen im System}
	\framesubtitle{Signal \texttt{postProcessMirrorUrl} wurde verschoben}

	% decrease font size for code listing
	\lstset{basicstyle=\tiny\ttfamily}

	\begin{itemize}

		\item Die Klasse für das Signal \texttt{postProcessMirrorUrl} wurde geändert

		\breakingchange

		\item Mit folgendem Code kann man die Klasse je nach TYPO3-Version ansprechen:

		\begin{lstlisting}
			$signalSlotDispatcher->connect(
			  version_compare(TYPO3_version, '7.0', '<')
			    ? 'TYPO3\\CMS\\Lang\\Service\\UpdateTranslationService'
			    : 'TYPO3\\CMS\\Lang\\Service\\TranslationService',
			  'postProcessMirrorUrl',
			  'Vendor\\Extension\\Slots\\CustomMirror',
			  'postProcessMirrorUrl'
			);
		\end{lstlisting}

	\end{itemize}

\end{frame}

% ------------------------------------------------------------------------------
