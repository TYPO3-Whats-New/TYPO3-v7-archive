% ------------------------------------------------------------------------------
% TYPO3 CMS 7.3 - What's New - Chapter "TypoScript" (German Version)
%
% @author	Patrick Lobacher <patrick@lobacher.de> and Michael Schams <schams.net>
% @license	Creative Commons BY-NC-SA 3.0
% @link		http://typo3.org/download/release-notes/whats-new/
% @language	German
% ------------------------------------------------------------------------------
% LTXE-CHAPTER-UID:		e8cc33e1-d7bdb74f-0421b75f-7be06b19
% LTXE-CHAPTER-NAME:	TypoScript
% ------------------------------------------------------------------------------
% LTXE-SLIDE-START
% LTXE-SLIDE-UID:		56b025e6-cf380b11-c9d1c009-0b548d6a
% LTXE-SLIDE-TITLE:		Feature #63561: Add TypoScript stdWrap strtotime
% LTXE-SLIDE-REFERENCE:	Feature-63561-AddTypoScriptStdWrapStrtotime.rst
% ------------------------------------------------------------------------------
\begin{frame}[fragile]
	\frametitle{TSconfig \& TypoScript}
	\framesubtitle{stdWrap Funktion \texttt{strtotime}}

	\begin{itemize}
		\item Es gibt nun eine stdWrap Funktion \texttt{strtotime}, welche es
			ermöglicht, formatierte Datum-Angaben in einen Timestamp umzuwandeln

			\begin{lstlisting}
				date_as_timestamp = TEXT
				date_as_timestamp {
				  value = 2015-04-15
				  strtotime = 1
				}

				next_weekday = TEXT
				next_weekday {
				  data = GP:selected_date
				  strtotime = + 2 weekdays
				  strftime = %Y-%m-%d
				}
			\end{lstlisting}

	\end{itemize}

\end{frame}

% ------------------------------------------------------------------------------
% LTXE-SLIDE-START
% LTXE-SLIDE-UID:		c5dbf3c7-655fc6e4-227a3bf6-f5d89fb1
% LTXE-SLIDE-TITLE:		Feature #65250: TypoScript condition add GPmerged
% LTXE-SLIDE-REFERENCE:	Feature-65250-TypoScriptConditionAddGPmerged.rst
% ------------------------------------------------------------------------------

\begin{frame}[fragile]
	\frametitle{TSconfig \& TypoScript}
	\framesubtitle{\texttt{GPmerged} in Conditions}

	\begin{itemize}

		\item Prüft man in Conditions nur mittels \text{GP} so wird beim
			gleichzeitigen Vorhandensein von \text{POST}- und \text{GET}-Variablen
			(z.B. \texttt{tx\_demo\_demo[...]=...}), lediglich die
			\text{POST}-Variable zurückgegeben

		\item Mit der neuen Option \texttt{GPmerged} werden beide Variablen
			zusammengeführt und dann zurückgegeben

			\begin{lstlisting}
				[globalVar = GPmerged:tx_demo|foo = 1]
				  page.90 = TEXT
				  page.90.value = DEMO
				[global]
			\end{lstlisting}

	\end{itemize}

\end{frame}

% ------------------------------------------------------------------------------
% LTXE-SLIDE-START
% LTXE-SLIDE-UID:		05b2f338-7bcb651f-a4cebabd-2c677a20
% LTXE-SLIDE-TITLE:		Feature #66697: Add uppercamelcase and lowercamelcase to stdWrap.case
% LTXE-SLIDE-REFERENCE:	Feature-66697-AddUppercamelcaseAndLowercamelcaseToStdWrap.case.rst
% ------------------------------------------------------------------------------

\begin{frame}[fragile]
	\frametitle{TSconfig \& TypoScript}
	\framesubtitle{Weitere Werte für die Funktion \texttt{stdWrap.case}}

	% decrease font size for code listing
	\lstset{basicstyle=\tiny\ttfamily}

	\begin{itemize}

		\item Die stdWrap Funktion \texttt{case} ist um die beiden Werte
			\texttt{uppercamelcase} und \texttt{lowercamelcase} ergänzt worden

		\item Beispiel:

			\begin{lstlisting}
				tt_content = CASE
				tt_content {
				  key.field = CType
				  my_custom_ctype =< lib.userContent
				  my_custom_ctype {
				    file = EXT:site_base/Resources/Private/Templates/SomeOtherTemplate.html
				    settings.extraParam = 1
				  }
				  default =< lib.userContent
				  default {
				    file = TEXT
				    file.field = CType
				    file.stdWrap.case = uppercamelcase
				    file.wrap = EXT:site_base/Resources/Private/Templates/|.html
				  }
				}
			\end{lstlisting}

	\end{itemize}

\end{frame}

% ------------------------------------------------------------------------------
% LTXE-SLIDE-START
% LTXE-SLIDE-UID:		cd745fc5-b8bd6483-ba1cf2a5-0ebf4310
% LTXE-SLIDE-TITLE:		Feature #66698: Add integrity property to JavaScript files (1)
% LTXE-SLIDE-REFERENCE:	Feature-66698-AddIntegrityPropertyToJavaScriptFiles.rst
% ------------------------------------------------------------------------------

\begin{frame}[fragile]
	\frametitle{TSconfig \& TypoScript}
	\framesubtitle{Eigenschaft \texttt{integrity} für JavaScript-Dateien (1)}

	% decrease font size for code listing
	\lstset{basicstyle=\tiny\ttfamily}

	\begin{itemize}

		\item Es wurde die Eigenschaft \texttt{integrity} zugefügt, um einen
			SRI Hash zum JavaScript-Markup hinzuzufügen, mit dem die Quelle
			verifiziert werden kann (SRI: Sub-Resource Integrity)

		\item Dies betrifft die Eigenschaften \texttt{page.includeJSLibs},
			\texttt{page.includeJSFooterlibs}, \texttt{ includeJS} und
			\texttt{includeJSFooter}

		\item Beispiel:

			\begin{lstlisting}
				page {
				  includeJS {
				    jQuery = https://code.jquery.com/jquery-1.11.3.min.js
				    jquery.external = 1
				    jQuery.disableCompression = 1
				    jQuery.excludeFromConcatenation = 1
				    jQuery.integrity = sha256-7LkWEzqTdpEfELxcZZlS6wAx5Ff13zZ83lYO2/ujj7g=
				  }
				}
			\end{lstlisting}

	\end{itemize}

\end{frame}

% ------------------------------------------------------------------------------
% LTXE-SLIDE-START
% LTXE-SLIDE-UID:		b8bd6483-cd745fc5-ba1cf2a5-0ebf4310
% LTXE-SLIDE-TITLE:		Feature #66698: Add integrity property to JavaScript files (2)
% LTXE-SLIDE-REFERENCE:	Feature-66698-AddIntegrityPropertyToJavaScriptFiles.rst
% ------------------------------------------------------------------------------

\begin{frame}[fragile]
	\frametitle{TSconfig \& TypoScript}
	\framesubtitle{Eigenschaft \texttt{integrity} für JavaScript-Dateien (2)}

	% decrease font size for code listing
	%\lstset{basicstyle=\tiny\ttfamily}

	\begin{itemize}

		\item SRI ist eine Spezifikation des W3C, die es ermöglicht zu verifizieren,
			ob extern gehostete Dateien manipuliert worden sind

		\item Erstellen von Integrity Hashes:

			\begin{itemize}
				\item Option 1: \url{https://srihash.org}
				\item Option 2: durch folgende Kommandos
			\end{itemize}

			\begin{lstlisting}
				cat FILENAME.js | openssl dgst -sha256 -binary | openssl enc -base64 -A
			\end{lstlisting}

		\item Weitere Informationen:

			\begin{itemize}
				\item \url{http://www.w3.org/TR/SRI/}
			\end{itemize}

	\end{itemize}

\end{frame}

% ------------------------------------------------------------------------------
