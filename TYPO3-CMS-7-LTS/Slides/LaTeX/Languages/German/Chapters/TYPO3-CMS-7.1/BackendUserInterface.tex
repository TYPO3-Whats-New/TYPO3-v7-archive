% ------------------------------------------------------------------------------
% TYPO3 CMS 7.1 - What's New (German Version)
%
% @author	Patrick Lobacher <patrick@lobacher.de> and Michael Schams <schams.net>
% @license	Creative Commons BY-NC-SA 3.0
% @link		http://typo3.org/download/release-notes/whats-new/
% @language	German
% ------------------------------------------------------------------------------
% LTXE-CHAPTER-UID:		e7264f0e-3f82290d-94c50cda-fb2d8e66
% LTXE-CHAPTER-NAME:	Backend User Interface
% ------------------------------------------------------------------------------
% LTXE-SLIDE-START
% LTXE-SLIDE-UID:		b4dc1576-57b4854f-26a32d85-11f7c52b
% LTXE-SLIDE-ORIGIN:	d5fddde9-b3ee31c0-f0509300-40a2928e English
% LTXE-SLIDE-TITLE:		Date/Time Picker
% LTXE-SLIDE-REFERENCE:	Breaking-62925-RemoveExtJsDateTimePicker.rst
% ------------------------------------------------------------------------------

\begin{frame}[fragile]
	\frametitle{Backend User Interface}
	\framesubtitle{Look \& Feel: Date/Time Picker}

	Date/Time Picker im Backend wurde auf Twitter Bootstrap umgestellt
	\begin{figure}
		\includegraphics[width=0.75\linewidth]{BackendUserInterface/be-datepicker.png}
	\end{figure}

\end{frame}

% ------------------------------------------------------------------------------
% LTXE-SLIDE-START
% LTXE-SLIDE-UID:		168b1424-1ebb2552-ed5bac3e-8a9ac737
% LTXE-SLIDE-ORIGIN:	1c391eec-dfb1dfa6-f783ae7a-d0b214ae English
% LTXE-SLIDE-TITLE:		Functions Module
% LTXE-SLIDE-REFERENCE:	Breaking-63310-Wizard-Modules-Moved.rst
% ------------------------------------------------------------------------------

\begin{frame}[fragile]
	\frametitle{Backend User Interface}
	\framesubtitle{Look \& Feel: Funktionen-Modul}

	Die Funktionen "Seiten sortieren" und "mehrere Seiten erzeugen" sind jetzt unter \texttt{WEB => Funktionen} zu finden\newline
	\smaller(in TYPO3 CMS < 7.1 waren diese unter "\texttt{WEB => Funktionen => Wizards}" zu finden)

	\begin{figure}
		\includegraphics[width=0.80\linewidth]{BackendUserInterface/be-functions.png}
	\end{figure}


\end{frame}

% ------------------------------------------------------------------------------
% LTXE-SLIDE-START
% LTXE-SLIDE-UID:		daa83c1e-08d2716b-de74cbda-42361551
% LTXE-SLIDE-ORIGIN:	dd127630-5ccc729a-835e5836-e8796962 English
% LTXE-SLIDE-TITLE:		Access Module: Leave Unchaged
% LTXE-SLIDE-REFERENCE:	Feature-15619-LeaveUnchagedInAccessModule.rst
% ------------------------------------------------------------------------------

\begin{frame}[fragile]
	\frametitle{Backend User Interface}
	\framesubtitle{Look \& Feel: Access-Modul}

	Im Modul \texttt{WEB => Access} kann man Benutzer und/oder Gruppen unverändert
	lassen, wenn man nur die Berechtigungen ändern möchte

	\begin{figure}
		\includegraphics[width=0.75\linewidth]{BackendUserInterface/be-access.png}
	\end{figure}

\end{frame}

% ------------------------------------------------------------------------------
% LTXE-SLIDE-START
% LTXE-SLIDE-UID:		9012ee0f-70c911f5-35110c51-fbd8aef2
% LTXE-SLIDE-ORIGIN:	eb6cc867-e4f5d0d3-ea06672c-7ccdf227 English
% LTXE-SLIDE-TITLE:		Icons in List Module
% LTXE-SLIDE-REFERENCE:	Feature-63207-SplitActionButtonsIntoGroups.rst
% ------------------------------------------------------------------------------

\begin{frame}[fragile]
	\frametitle{Backend User Interface}
	\framesubtitle{Look \& Feel: Icons im List-Modul}

	Icons ("Action Buttons") im List-Modul sind nun in zwei Gruppen angeordnet\newline
	\smaller(primäre Aktionen (lesen, speichern, löschen) zuerst, anschließend die sekundären Icons)

	\begin{figure}
		\includegraphics[width=0.75\linewidth]{BackendUserInterface/be-icons.png}
	\end{figure}

\end{frame}

% ------------------------------------------------------------------------------
