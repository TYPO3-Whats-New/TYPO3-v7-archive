% ------------------------------------------------------------------------------
% TYPO3 CMS 7.1 - What's New - Chapter "TypoScript" (German Version)
%
% @author	Patrick Lobacher <patrick@lobacher.de> and Michael Schams <schams.net>
% @license	Creative Commons BY-NC-SA 3.0
% @link		http://typo3.org/download/release-notes/whats-new/
% @language	German
% ------------------------------------------------------------------------------
% LTXE-CHAPTER-UID:		d92ab5b9-9ffa4925-30a2e34f-0d6939e5
% LTXE-CHAPTER-NAME:	TypoScript
% ------------------------------------------------------------------------------
% LTXE-SLIDE-START
% LTXE-SLIDE-UID:		56b025e6-cf380b11-c9d1c009-0b548d6a
% LTXE-SLIDE-ORIGIN:	a55771e2-74f3a0fc-5be51f93-8b40a5d8 English
% LTXE-SLIDE-TITLE:		StdWrap for page.headTag
% LTXE-SLIDE-REFERENCE:	Feature-22086-StdWrapForHeadTag.rst
% ------------------------------------------------------------------------------

\begin{frame}[fragile]
	\frametitle{TSconfig \& TypoScript}
	\framesubtitle{StdWrap für \texttt{page.headTag}}

	\begin{itemize}
		\item TypoScript Option \texttt{page.headTag} hat nun \texttt{stdWrap}-Funktionalität

			\begin{lstlisting}
				page = PAGE
				page.headTag = <head>
				page.headTag.override = <head class="special">
				page.headTag.override.if {
		  		  isInList.field = uid
		  		  value = 24
				}
			\end{lstlisting}

	\end{itemize}

\end{frame}

% ------------------------------------------------------------------------------
% LTXE-SLIDE-START
% LTXE-SLIDE-UID:		c5dbf3c7-655fc6e4-227a3bf6-f5d89fb1
% LTXE-SLIDE-ORIGIN:	618eb9ea-56d5a4eb-d5d69d66-cc62f953 English
% LTXE-SLIDE-TITLE:		Include JavaScript files asynchronously
% LTXE-SLIDE-REFERENCE:	Feature-28382-AddAsyncPropertyToJavaScriptFiles.rst
% ------------------------------------------------------------------------------

\begin{frame}[fragile]
	\frametitle{TSconfig \& TypoScript}
	\framesubtitle{JavaScript-Dateien asynchron laden}

	\begin{itemize}
		\item JavaScript-Dateien könen nun asynchron geladen werden

			\begin{lstlisting}
				page {
				  includeJS {
				    jsFile = /path/to/file.js
				    jsFile.async = 1
				  }
				}
			\end{lstlisting}

		\item Das gilt für:

			\begin{itemize}
				\item \texttt{includeJSlibs} / \texttt{includeJSLibs}
				\item \texttt{includeJSFooterlibs}
				\item \texttt{includeJS}
				\item \texttt{includeJSFooter}
			\end{itemize}

	\end{itemize}

\end{frame}

% ------------------------------------------------------------------------------
% LTXE-SLIDE-START
% LTXE-SLIDE-UID:		05b2f338-7bcb651f-a4cebabd-2c677a20
% LTXE-SLIDE-ORIGIN:	8a6ca08e-cf9ba481-18fd9411-30dd0a0f English
% LTXE-SLIDE-TITLE:		HMENU item selection via additionalWhere
% LTXE-SLIDE-REFERENCE:	Feature-46624-AdditionalWhereForMenu.rst
% ------------------------------------------------------------------------------

\begin{frame}[fragile]
	\frametitle{TSconfig \& TypoScript}
	\framesubtitle{HMENU Eigenschaft mit \texttt{additionalWhere}}

	\begin{itemize}

		\item TypoScript cObject \texttt{HMENU} erhält eine neue Eigenschaft \texttt{additionalWhere}
		\item Jenes erlaubt eine spezifischere DB Abfrage (z.B. Filterung)

		\item Beispiel:

			\begin{lstlisting}
				lib.authormenu = HMENU
				lib.authormenu.1 = TMENU
				lib.authormenu.1.additionalWhere = AND author!=""
			\end{lstlisting}

	\end{itemize}

\end{frame}

% ------------------------------------------------------------------------------
% LTXE-SLIDE-START
% LTXE-SLIDE-UID:		b8bd6483-cd745fc5-ba1cf2a5-0ebf4310
% LTXE-SLIDE-ORIGIN:	e07d1984-117d1212-a1f7fab8-8ea21ebc English
% LTXE-SLIDE-TITLE:		Additional properties for HMENU browse menus
% LTXE-SLIDE-REFERENCE:	Feature-57178-SpecialHmenuExcludeSpecialParameters.rst
% ------------------------------------------------------------------------------

\begin{frame}[fragile]
	\frametitle{TSconfig \& TypoScript}
	\framesubtitle{Zusätzliche Eigenschaften für \texttt{HMENU} Browse-Menü}

	\begin{itemize}
		\item Zwei neue Eigenschaften für das cObject \texttt{HMENU} (Option
			"\texttt{special=browse"}), um detaillierter definieren zu können,
			welche Seiten im Menü erscheinen sollen:

			\begin{itemize}
				\item \texttt{excludeNoSearchPages}
				\item \texttt{includeNotInMenu}
			\end{itemize}

		\item Beispiel:

			\begin{lstlisting}
				lib.browsemenu = HMENU
				lib.browsemenu.special = browse
				lib.browsemenu.special.excludeNoSearchPages = 1
				lib.browsemenu.includeNotInMenu = 1
			\end{lstlisting}

	\end{itemize}

\end{frame}

% ------------------------------------------------------------------------------
% LTXE-SLIDE-START
% LTXE-SLIDE-UID:		fdeb8bc8-4055b503-351a4aed-bf9201f8
% LTXE-SLIDE-ORIGIN:	a286a87a-161365ed-5560e66f-e9f1a2ee English
% LTXE-SLIDE-TITLE:		Multiple HTTP headers
% LTXE-SLIDE-REFERENCE:	Feature-56236-Multiple-HTTP-Headers-In-Frontend.rst
% ------------------------------------------------------------------------------

\begin{frame}[fragile]
	\frametitle{TSconfig \& TypoScript}
	\framesubtitle{Mehrere HTTP-Header}

	\begin{itemize}

		\item HTTP Header können nun mittels \small\texttt{config.additionalHeaders}\normalsize\newline
			als Array gesetzt werden
		\item Das ermöglicht es, mehreren Header-Zeilen auf einmal zu konfigurieren

			\begin{lstlisting}
				config.additionalHeaders {
				  10 {
				    # header string
				    header = WWW-Authenticate: Negotiate
				    # (optional) replace previous headers with the same name (default: 1)
				    replace = 0
				    # (optional) force HTTP response code
				    httpResponseCode = 401
				  }
				  # set second additional HTTP header
				  20.header = Cache-control: Private
				}
			\end{lstlisting}

	\end{itemize}

\end{frame}

% ------------------------------------------------------------------------------
% LTXE-SLIDE-START
% LTXE-SLIDE-UID:		411ab08a-82c6a003-5286e8aa-b00085bf
% LTXE-SLIDE-ORIGIN:	a296b9b8-9403ef17-30f80f28-2a8a94ce English
% LTXE-SLIDE-TITLE:		Option "auto" added for config.absRefPrefix
% LTXE-SLIDE-REFERENCE:	Feature-58366-AutomaticAbsRefPrefix.rst
% ------------------------------------------------------------------------------

\begin{frame}[fragile]
	\frametitle{TSconfig \& TypoScript}
	\framesubtitle{Option "auto" für \texttt{config.absRefPrefix}}

	\begin{itemize}
		\item TypoScript Konfiguration \texttt{config.absRefPrefix} kann verwendet werden,
			um der URL einen Prefix bei relativen Pfaden zu geben. Als Alternative zu
			\texttt{config.baseURL} (um eine bestimmte Domain zu spezifizieren),
			erkennt \texttt{absRefPrefix} die Site-Root automatisch:

			\begin{lstlisting}
				config.absRefPrefix = auto

				# anstelle von:
				[ApplicationContext = Production]
				config.absRefPrefix = /

				[ApplicationContext = Testing]
				config.absRefPrefix = /my_site_root/
			\end{lstlisting}

		\smaller
			Hinweis: diese Option ist "Multi-Domain"-sicher und mehrfaches
			Caching der selben Daten wird verhindern.
		\normalsize

	\end{itemize}

\end{frame}

% ------------------------------------------------------------------------------
% LTXE-SLIDE-START
% LTXE-SLIDE-UID:		2ce73c58-ac19be87-8f293f4d-0da8858b
% LTXE-SLIDE-ORIGIN:	8210d253-d630446f-ba747ba6-5568843d English
% LTXE-SLIDE-TITLE:		Two-letter ISO code for sys_language (1)
% LTXE-SLIDE-REFERENCE:	Feature-61542-AddIsoLanguageKeys.rst
% ------------------------------------------------------------------------------

\begin{frame}[fragile]
	\frametitle{TSconfig \& TypoScript}
	\framesubtitle{Zwei-Zeichen ISO Code für \texttt{sys\_language} (1)}

	\begin{itemize}
		\item Die Behandlung von Sprachen wird durch Einträge in DB Tabelle
			\texttt{sys\_language} vorgenommen, die durch \texttt{sys\_language\_uid}
			referenziert werden
		\item In TYPO3 CMS 7.1 wurden ISO 639-1 Zwei-Zeichen Codes implementiert:

			\begin{itemize}
				\item Neues DB Feld: \texttt{sys\_language.language\_isocode}
				\item Neue TypoScript-Option: \texttt{sys\_language\_isocode}
			\end{itemize}

		\vspace{0.2cm}

		\small
			Hinweis: bei \textbf{ISO 639} handelt es sich um eine Sammlung von
			Standards der "International Organization for Standardization".
			Eine List der ISO 639-1 Codes ist hier abrufbar:\newline
			\url{http://en.wikipedia.org/wiki/List_of_ISO_639-1_codes}
		\normalsize

	\end{itemize}

\end{frame}

% ------------------------------------------------------------------------------
% LTXE-SLIDE-START
% LTXE-SLIDE-UID:		a19035b5-8da878ef-dfbf3c01-4a374d63
% LTXE-SLIDE-ORIGIN:	d630446f-ba747ba6-5568843d-8210d253 English
% LTXE-SLIDE-TITLE:		Two-letter ISO code for sys_language (2)
% LTXE-SLIDE-REFERENCE:	Feature-61542-AddIsoLanguageKeys.rst
% ------------------------------------------------------------------------------

\begin{frame}[fragile]
	\frametitle{TSconfig \& TypoScript}
	\framesubtitle{Zwei-Zeichen ISO Code für \texttt{sys\_language} (2)}

	\begin{itemize}
		\item Beispiel:

			\begin{lstlisting}
				# Danish by default
				config.sys_language_uid = 0
				config.sys_language_isocode_default = da

				[globalVar = GP:L = 1]
				  # ISO code stored in table sys_language (uid 1)
				  config.sys_language_uid = 1
				  # overwrite ISO code as required
				  config.sys_language_isocode = fr
				[GLOBAL]

				page.10 = TEXT
				page.10.data = TSFE:sys_language_isocode
				page.10.wrap = <div class="main" data-language="|">
			\end{lstlisting}

	\end{itemize}

\end{frame}

% ------------------------------------------------------------------------------
% LTXE-SLIDE-START
% LTXE-SLIDE-UID:		79465740-a44dd50b-e7e43403-a785cd3a
% LTXE-SLIDE-ORIGIN:	df2f7fec-fb8bc3b9-446fb779-c409ef27 English
% LTXE-SLIDE-TITLE:		Custom TypoScript Conditions in Backend
% LTXE-SLIDE-REFERENCE:	Feature-63600-CustomTypoScriptConditionsInBackend.rst
% ------------------------------------------------------------------------------

\begin{frame}[fragile]
	\frametitle{TSconfig \& TypoScript}
	\framesubtitle{Eigene Conditions im Backend}

	% decrease font size for code listing
	\lstset{basicstyle=\tiny\ttfamily}

	\begin{itemize}

		\item Eigene Conditions für das \textbf{Frontend} wurden bereits mit TYPO3 CMS 7.0 eingeführt
		\item Seit TYPO3 CMS 7.1 ist es nun auch möglich, eigene Conditions für das \textbf{Backend} zu implementieren
		\item Die Condition muss von \texttt{AbstractCondition} ableiten und die Methode \texttt{matchCondition} bereitstellen
		\item Anwendungsbeispiel in TypoScript:

			\begin{lstlisting}
				[BigCompanyName\TypoScriptLovePackage\MyCustomTypoScriptCondition]

				[BigCompanyName\TypoScriptLovePackage\MyCustomTypoScriptCondition = 7]

				[BigCompanyName\TypoScriptLovePackage\MyCustomTypoScriptCondition = 7, != 6]

				[BigCompanyName\TypoScriptLovePackage\MyCustomTypoScriptCondition = {$mysite.myconstant}]
			\end{lstlisting}

	\end{itemize}

\end{frame}

% ------------------------------------------------------------------------------
% LTXE-SLIDE-START
% LTXE-SLIDE-UID:		d11d0bd2-62aaef44-5692dfb8-e0e1f750
% LTXE-SLIDE-ORIGIN:	b9687dbb-c0502234-7ca16b06-1879fe80 English
% LTXE-SLIDE-TITLE:		FormEngine: customize icons via PageTSconfig
% LTXE-SLIDE-REFERENCE:	Feature-35891-AddTCAItemsWithIconsViaPageTSConfig.rst
% ------------------------------------------------------------------------------

\begin{frame}[fragile]
	\frametitle{TSconfig \& TypoScript}
	\framesubtitle{Zufügen von Icons in TCEFORM via PageTSconfig}

	\begin{itemize}
		\item Eigene Werte und Labels von Select-Feldern können bereits mit der PageTSconfig Option \texttt{addItems} vergeben werden
		\item Nun können auch Icons für diese Felder definiert werden

			\begin{itemize}
				\item Option 1: mittels \texttt{addItems} und der Eigenschaft \texttt{.icon}
				\item Option 2: mittels \texttt{altIcons} (generell für alle Felder)
			\end{itemize}

		\item Beispiel:

			\begin{lstlisting}
				TCEFORM.pages.doktype.addItems {
				  10 = My Label
				  10.icon = EXT:t3skin/icons/gfx/i/pages.gif
				}
				TCEFORM.pages.doktype.altIcons {
				  10 = EXT:myext/icon.gif
				}
			\end{lstlisting}

	\end{itemize}

\end{frame}

% ------------------------------------------------------------------------------
% LTXE-SLIDE-START
% LTXE-SLIDE-UID:		5b06c114-b7c90f6f-b48bcb23-a19b6f86
% LTXE-SLIDE-ORIGIN:	b8794774-4912764b-cd1cc91d-de4396fe English
% LTXE-SLIDE-TITLE:		Extend element browser with mount points
% LTXE-SLIDE-REFERENCE:	Feature-50780-AppendElementBrowserMountPoints.rst
% ------------------------------------------------------------------------------

\begin{frame}[fragile]
	\frametitle{TSconfig \& TypoScript}
	\framesubtitle{Element Browser: Mountpoints hinzufügen}

	\begin{itemize}
		\item Neue UserTSconfig Option \texttt{.append} erlaubt es Administratoren Mountpoints
			\textbf{hinzuzufügen}, anstatt die Liste der konfigurierten DB Mountpoints eines
			Benutzers neu zu schreiben

		\item Beispiel:

			\begin{lstlisting}
				options.pageTree.altElementBrowserMountPoints = 20,31
				options.pageTree.altElementBrowserMountPoints.append = 1
			\end{lstlisting}

	\end{itemize}

\end{frame}

% ------------------------------------------------------------------------------
% LTXE-SLIDE-START
% LTXE-SLIDE-UID:		e9daea46-551d29d0-de19267e-b9eca1b5
% LTXE-SLIDE-ORIGIN:	04b7f7c0-af900c25-5f16f178-8a94fcca English
% LTXE-SLIDE-TITLE:		Label override of checkboxes and radio buttons by TSconfig
% LTXE-SLIDE-REFERENCE:	Feature-58033-AltLabelsForFormEngineCheckboxAndRadioButtons.rst
% ------------------------------------------------------------------------------

\begin{frame}[fragile]
	\frametitle{TSconfig \& TypoScript}
	\framesubtitle{Überschreiben der Labels von Radio-Buttons und Checkboxen}

	% decrease font size for code listing
	\lstset{basicstyle=\tiny\ttfamily}

	\begin{itemize}

		\item Labels von Radio-Buttons und Checkboxen können nun überschrieben werden
		\item Beispiel:

			\begin{lstlisting}
				// field with a single checkbox (use ".default")
				TCEFORM.pages.hidden.altLabels.default = new label
				TCEFORM.pages.hidden.altLabels.default = LLL:path/to/languagefile.xlf:individualLabel

				// field with multiple checkboxes (0, 1, 2, 3...)
				TCEFORM.pages.l18n_cfg.altLabels.0 = new label of first checkbox
				TCEFORM.pages.l18n_cfg.altLabels.1 = new label of second checkbox
				TCEFORM.pages.l18n_cfg.altLabels.2 = new label of third checkbox
				...
			\end{lstlisting}

	\end{itemize}

\end{frame}

% ------------------------------------------------------------------------------
% LTXE-SLIDE-START
% LTXE-SLIDE-UID:		75a81722-7cefd7eb-713b75a3-1c7cde59
% LTXE-SLIDE-ORIGIN:	6cdd5010-f7d7d7bd-0a3c74ea-05424fb0 English
% LTXE-SLIDE-TITLE:		Miscellaneous (1)
% LTXE-SLIDE-REFERENCE: Feature-xxxxx ... UserTSconfig-width-and-height-of-element-browser
% LTXE-SLIDE-REFERENCE: Feature-59646-AddRteConfigurationPropertyButtonsLinkTypePropertiesTargetDefault.rst
% ------------------------------------------------------------------------------

\begin{frame}[fragile]
	\frametitle{TSconfig \& TypoScript}
	\framesubtitle{Diverses (1)}

	\begin{itemize}
		\item Breite und Höhe des Element-Browsers können nun per UserTSconfig festgelegt werden

			\begin{lstlisting}
				options.popupWindowSize = 400x900
				options.RTE.popupWindowSize = 200x200
			\end{lstlisting}


		\item PageTSconfig: mit einer neue RTE-Konfiguration kann das Standard-Ziel von Links beeinflusst werden

			\begin{lstlisting}
				buttons.link.[type].properties.target.default
			\end{lstlisting}

			Wobei \texttt{[type]} zum Beispiel \texttt{page}, \texttt{file}, \texttt{url}, \texttt{mail} or \texttt{spec} sein kann\newline
			(Extensions können weitere Typen zur Verfügung stellen)

	\end{itemize}

\end{frame}

% ------------------------------------------------------------------------------
% LTXE-SLIDE-START
% LTXE-SLIDE-UID:		66d808a9-f6ec6e6c-38e272be-b7b8fe6a
% LTXE-SLIDE-ORIGIN:	80595308-128c5961-8ae4cf89-8611ce66 English
% LTXE-SLIDE-TITLE:		Miscellaneous (2)
% LTXE-SLIDE-REFERENCE:	Feature-16794-MakeSectionLinkingForIndexedSearchResultsConfigurable.rst
% LTXE-SLIDE-REFERENCE: Feature-20767-getDataByNestedKey.rst
% LTXE-SLIDE-REFERENCE: Feature-33491-StdWrapForTitleTag.rst
% ------------------------------------------------------------------------------

\begin{frame}[fragile]
	\frametitle{TSconfig \& TypoScript}
	\framesubtitle{Diverses (2)}

	\begin{itemize}

		\item Standardmäßig sind Section-Headlines der Indexed-Search Resultate verlinkt.
			Das kann nun mittels TypoScript deaktiviert werden

			\begin{lstlisting}
				plugin.tx_indexedsearch.linkSectionTitles = 0
			\end{lstlisting}

		\item \texttt{getData} kann jetzt auch \texttt{field}-Daten abfragen (nicht
			nur Arrays, wie beispielsweise \texttt{GPVar} und \texttt{TSFE})

			\begin{lstlisting}
				10 = TEXT
				10.data = field:fieldname|level1|level2
			\end{lstlisting}

		\item TypoScript Konfiguration \texttt{config.pageTitle} hat jetzt \texttt{stdWrap}-Funktionalität

			\begin{lstlisting}
				# make value of <title> upper case
				page = PAGE
				page.config.pageTitle.case = upper
			\end{lstlisting}

	\end{itemize}

\end{frame}

% ------------------------------------------------------------------------------
