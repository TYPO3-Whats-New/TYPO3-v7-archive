% ------------------------------------------------------------------------------
% TYPO3 CMS 7.2 - What's New - Chapter "TypoScript" (English Version)
%
% @author	Michael Schams <schams.net>
% @license	Creative Commons BY-NC-SA 3.0
% @link		http://typo3.org/download/release-notes/whats-new/
% @language	English
% ------------------------------------------------------------------------------
% LTXE-CHAPTER-UID:		12518a77-2b90a173-4e3f8420-485e0497
% LTXE-CHAPTER-NAME:	TypoScript
% ------------------------------------------------------------------------------

\section{TSconfig \& TypoScript}

% ------------------------------------------------------------------------------
% LTXE-SLIDE-START
% LTXE-SLIDE-UID:		ad5f66f0-592f3af2-c3b43ad4-9ccc5d9d
% LTXE-SLIDE-ORIGIN:	56b025e6-cf380b11-c9d1c009-0b548d6a English
% LTXE-SLIDE-TITLE:		Add flexible preview URL configuration (1)
% LTXE-SLIDE-REFERENCE:	Feature-66370-AddFlexiblePreviewUrlConfiguration.rst
% ------------------------------------------------------------------------------
\begin{frame}[fragile]
	\frametitle{TSconfig \& TypoScript}
	\framesubtitle{Flexibele URL voor voorvertoning (1)}

	% decrease font size for code listing
	\lstset{basicstyle=\tiny\ttfamily}

	\begin{itemize}

		\item De voorvertoningslink van de knop "opslaan \& bekijken" in de backend kan nu\newline
			geconfigureerd worden.

		\item Dit wordt vaak gebruikt om blog- of nieuwsrecords vooraf te bekijken maar het
			is ook mogelijk om andere pagina's te definiëren om normale contentelementen
			vooraf te bekijken.

			\begin{lstlisting}
				TCEMAIN.preview {
				  <tabelnaam> {
				    previewPageId = 123
				    useDefaultLanguageRecord = 0
				    fieldToParameterMap {
				      uid = tx_myext_pi1[showUid]
				    }
				    additionalGetParameters {
				      tx_myext_pi1[special] = HELLO
				    }
				  }
				}
			\end{lstlisting}

	\end{itemize}

\end{frame}

% ------------------------------------------------------------------------------
% LTXE-SLIDE-START
% LTXE-SLIDE-UID:		c98d6857-043784e9-21166aed-dfbd3656
% LTXE-SLIDE-ORIGIN:	6822c376-37649cd8-0acc836a-b856fc61 English
% LTXE-SLIDE-TITLE:		Add flexible preview URL configuration (2)
% LTXE-SLIDE-REFERENCE:	Feature-66370-AddFlexiblePreviewUrlConfiguration.rst
% ------------------------------------------------------------------------------
\begin{frame}[fragile]
	\frametitle{TSconfig \& TypoScript}
	\framesubtitle{Flexibele URL voor voorvertoning (2)}

	\begin{itemize}
		\item \texttt{previewPageId}:\newline
			\smaller
				UID van de pagina om het item vooraf te bekijken
				(bij het ontbreken hiervan wordt de huidige pagina gebruikt)
			\normalsize
		\item \texttt{useDefaultLanguageRecord}:\newline
			\smaller
				of vertaalde records de UID van het standaard record moeten gebruiken\newline
				(standaard geactiveerd, waarde: 1)
			\normalsize
		\item \texttt{fieldToParameterMap}:\newline
			\smaller
				mapping om velden te selecteren van het record dat in de GET-parameters
				opgenomen moet worden
			\normalsize
		\item \texttt{additionalGetParameters}:\newline
			\smaller
				toevoegen of overschrijven van alle mogelijke GET-parameters
			\normalsize
	\end{itemize}

\end{frame}

% ------------------------------------------------------------------------------
% LTXE-SLIDE-START
% LTXE-SLIDE-UID:		2f4860bb-971fe754-88f82ab2-2134d81c
% LTXE-SLIDE-ORIGIN:	c5dbf3c7-655fc6e4-227a3bf6-f5d89fb1 English
% LTXE-SLIDE-TITLE:
% LTXE-SLIDE-REFERENCE:	Feature-59646-AddRteConfigurationPropertyButtonsLinkTypePropertiesTargetDefault.rst
% ------------------------------------------------------------------------------
\begin{frame}[fragile]
	\frametitle{TSconfig \& TypoScript}
	\framesubtitle{RTE Configuratie: Standaard doel}

	\begin{itemize}

		\item eigenschap van RTE-configuratie voor PageTSconfig om een standaard doel in te
			stellen voor links van een bepaald type\newline

			\small
				\texttt{buttons.link.[}
				\textit{type}
				\texttt{].properties.target.default = ...}
			\normalsize\newline

		\item Mogelijke linktypes zijn:\newline
			\small
				(extra types kunnen door extensies aangevuld worden)
			\normalsize

			\begin{itemize}
				\item \texttt{page}
				\item \texttt{file}
				\item \texttt{url}
				\item \texttt{mail}
				\item \texttt{spec}
			\end{itemize}
	\end{itemize}

\end{frame}

% ------------------------------------------------------------------------------
% LTXE-SLIDE-START
% LTXE-SLIDE-UID:		7131736c-eafdb039-3ae40019-34c2c179
% LTXE-SLIDE-ORIGIN:	fd957580-301e4a0a-ad5325f5-ffa024c3 English
% LTXE-SLIDE-TITLE:		Strip empty HTML tags in HtmlParser
% LTXE-SLIDE-REFERENCE:	Feature-20555-StripEmptyHtmlTags.rst
% ------------------------------------------------------------------------------
\begin{frame}[fragile]
	\frametitle{TSconfig \& TypoScript}
	\framesubtitle{Lege HTML tags verwijderen in HTMLparser}

	% decrease font size for code listing
	\lstset{basicstyle=\tiny\ttfamily}

	\begin{itemize}
		\item Nieuwe functionaliteit in de HTMLparser om lege tags te verwijderen

			\begin{lstlisting}
				stdWrap {
				   // alle lege HTML tags verwijderen
				   HTMLparser.stripEmptyTags = 1
				   // alleen lege h2 en h3 tags verwijderen
				   HTMLparser.stripEmptyTags.tags = h2, h3
				}

				RTE.default.proc.entryHTMLparser_db {
				   stripEmptyTags = 1
				   stripEmptyTags.tags = p
				   stripEmptyTags.treatNonBreakingSpaceAsEmpty = 1
				}
			\end{lstlisting}

			\underline{\textbf{Note:}}
				Let op: HTMLparser verwijdert standaard alle onbekende tags.\newline
				Daarom is het misschien nuttig om deze te bewaren:\newline
				\texttt{HTMLparser.keepNonMatchedTags = 1}

	\end{itemize}

\end{frame}

% ------------------------------------------------------------------------------
% LTXE-SLIDE-START
% LTXE-SLIDE-UID:		0479f858-6de96b03-13389faa-7da9d8a3
% LTXE-SLIDE-ORIGIN:	62960e76-c3dca953-fe5a1070-831633fc English
% LTXE-SLIDE-TITLE:		Miscellaneous
% LTXE-SLIDE-REFERENCE:	Feature-63040-AddRteConfigurationPropertyButtonsAbbreviationRemoveFieldsets.rst
% LTXE-SLIDE-REFERENCE:	commit 31c62f9311ee3d33bf792d548cc4a5fac83aa3d0
% ------------------------------------------------------------------------------
\begin{frame}[fragile]
	\frametitle{TSconfig \& TypoScript}
	\framesubtitle{Allerlei}

	\begin{itemize}
		\item Nieuwe eigenschap \texttt{buttons.abbreviation.removeFieldsets} in PageTSconfig om het
			schermpje voor afkortingen te configureren

			\begin{lstlisting}
				# Mogelijke waarden zijn:
				# acronym, definedAcronym, abbreviation, definedAbbreviation
				buttons.abbreviation.removeFieldsets = acronym,definedAcronym
			\end{lstlisting}

		\item Eigenschap \texttt{inlineLanguageLabel} van het object \texttt{PAGE} kan nu \newline
			\texttt{LLL:} verwijzingen afhandelen

	\end{itemize}

\end{frame}

% ------------------------------------------------------------------------------
