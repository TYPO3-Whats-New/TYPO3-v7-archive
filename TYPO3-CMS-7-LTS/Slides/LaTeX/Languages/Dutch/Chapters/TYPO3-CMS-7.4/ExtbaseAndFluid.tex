% ------------------------------------------------------------------------------
% TYPO3 CMS 7.4 - What's New - Chapter "Extbase & Fluid" (English Version)
%
% @author	Michael Schams <schams.net>
% @license	Creative Commons BY-NC-SA 3.0
% @link		http://typo3.org/download/release-notes/whats-new/
% @language	English
% ------------------------------------------------------------------------------
% LTXE-CHAPTER-UID:		d1a9ab65-989c4bc7-f84807fd-d96febe7
% LTXE-CHAPTER-NAME:	Extbase & Fluid
% ------------------------------------------------------------------------------

\section{Extbase \& Fluid}

% ------------------------------------------------------------------------------
% LTXE-SLIDE-START
% LTXE-SLIDE-UID:		4d05f7ba-6c9ab170-cb02121f-0b4b7744
% LTXE-SLIDE-ORIGIN:	235ca4ed-fafe968a-814d4bd4-a8f936d0 English
% LTXE-SLIDE-ORIGIN:	7b7e03da-55cc30d6-8f375c39-c82ffc4f German
% LTXE-SLIDE-TITLE:		Feature #66070: Configure anchor for pagination widget
% LTXE-SLIDE-REFERENCE:	Feature-66070-ConfigureSectionForPaginationWidget.rst
% ------------------------------------------------------------------------------

\begin{frame}[fragile]
	\frametitle{Extbase \& Fluid}
	\framesubtitle{Fragment voor Paginatie-widget}

	% decrease font size for code listing
	\lstset{basicstyle=\tiny\ttfamily}

	\begin{itemize}

		\item Met deze feature kan een item \texttt{section} aan de configuratie van een Fluid paginatiewidget toegevoegd worden

		\item Het fragment wordt toegevoegd aan elke link in de paginering

		\item De volgende code voegt het fragment \texttt{\#archive} toe:

			\begin{lstlisting}
				<f:widget.paginate objects="{plantpestWarnings}" as="paginatedWarnings"
				  configuration="{section: 'archive', itemsPerPage: 10, insertAbove: 0, insertBelow: 1,
				  maximumNumberOfLinks: 10}">

				   [...]

				</f:widget.paginate>
			\end{lstlisting}

	\end{itemize}

\end{frame}

% ------------------------------------------------------------------------------
% LTXE-SLIDE-START
% LTXE-SLIDE-UID:		f2f45738-707ed068-52657fb5-e38ad4f2
% LTXE-SLIDE-ORIGIN:	db9ef18b-234cf92c-ef6869a5-248692f4 English
% LTXE-SLIDE-ORIGIN:	3356cd43-6aa02ac1-54fd276d-0c509286 German
% LTXE-SLIDE-TITLE:		Feature #68022: Added base date attribute to DateViewHelper
% LTXE-SLIDE-REFERENCE:	Feature-68022-AddedBaseDateAttributeToDateViewHelper.rst
% ------------------------------------------------------------------------------

\begin{frame}[fragile]
	\frametitle{Extbase \& Fluid}
	\framesubtitle{Attribuut \texttt{base} voor DateViewHelper}

	% decrease font size for code listing
	%\lstset{basicstyle=\tiny\ttfamily}

	\begin{itemize}

		\item De DateViewHelper is uitgebreid met een optioneel attribuut \texttt{base}
		\item Het attribuut definieert de basis voor relatieve datumaanduidingen
		\item Als de datum een DateTime object is wordt \texttt{base} genegeerd
		\item Het volgende voorbeeld geeft "2016", als \texttt{dateObject} een datum is in 2017:

			\begin{lstlisting}
				<f:format.date format="Y" base="{dateObject}">-1 year</f:format.date>
			\end{lstlisting}

		\small
			(zie \href{http://www.php.net/manual/en/datetime.formats.relative.php}{PHP documentatie} voor geldige waarden)
		\normalsize

	\end{itemize}

\end{frame}

% ------------------------------------------------------------------------------
% LTXE-SLIDE-START
% LTXE-SLIDE-UID:		5b076652-d7c4b54f-4fffa609-8e540659
% LTXE-SLIDE-ORIGIN:	aa476541-0e34625d-a79996ee-ecbf2e62 English
% LTXE-SLIDE-ORIGIN:	097a17c3-6c43ed7b-3f2db1ff-35b37660 German
% LTXE-SLIDE-TITLE:		Breaking #67890: Redesign FluidTemplateDataProcessorInterface to DataProcessorInterface
% LTXE-SLIDE-REFERENCE:	Breaking-67890-RedesignFluidTemplateDataProcessorInterfaceToDataProcessorInterface.rst
% ------------------------------------------------------------------------------

\begin{frame}[fragile]
	\frametitle{Extbase \& Fluid}
	\framesubtitle{Optie \texttt{dataProcessing} voor FLUIDTEMPLATE}

	% decrease font size for code listing
	\lstset{basicstyle=\tiny\ttfamily}

	\begin{itemize}

		\item In TYPO3 CMS 7.3 was de optie \texttt{dataProcessing} voor cObject \texttt{FLUIDTEMPLATE}  toegevoegd

		\item De \texttt{FluidTemplateDataProcessorInterface} is herbouwd naar \texttt{DataProcessorInterface},
			wat ook van invloed is op de functie \texttt{process()}

			\begin{lstlisting}
				public function process(
				  ContentObjectRenderer $cObj,
				  array $contentObjectConfiguration,
				  array $processorConfiguration,
				  array $processedData
				);
			\end{lstlisting}

	\end{itemize}

	\breakingchange

\end{frame}

% ------------------------------------------------------------------------------
