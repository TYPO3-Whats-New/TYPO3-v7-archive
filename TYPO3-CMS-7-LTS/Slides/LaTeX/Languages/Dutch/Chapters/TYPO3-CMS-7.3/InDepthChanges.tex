% ------------------------------------------------------------------------------
% TYPO3 CMS 7.3 - What's New - Chapter "In-Depth Changes" (English Version)
%
% @author	Michael Schams <schams.net>
% @license	Creative Commons BY-NC-SA 3.0
% @link		http://typo3.org/download/release-notes/whats-new/
% @language	English
% ------------------------------------------------------------------------------
% LTXE-CHAPTER-UID:		e8cc33e1-d7bdb74f-0421b75f-7be06b19
% LTXE-CHAPTER-NAME:	In-Depth Changes
% ------------------------------------------------------------------------------

\section{Wijzigingen nader bekeken}

% ------------------------------------------------------------------------------
% LTXE-SLIDE-START
% LTXE-SLIDE-UID:		6973100e-e3c0d679-943fcb2a-ba342736
% LTXE-SLIDE-ORIGIN:	17eaaa35-2a21b728-4c3488a0-30a7ac7b English
% LTXE-SLIDE-ORIGIN:	bd03e141-3842b75e-8a44920f-a8c139e5 German
% LTXE-SLIDE-TITLE:		Feature #59606: Integrate Symfony/Console into CommandController (1)
% LTXE-SLIDE-REFERENCE:	Feature-59606-IntegrateSymfonyConsoleIntoCommandController.rst
% ------------------------------------------------------------------------------

\begin{frame}[fragile]
	\frametitle{Wijzigingen nader bekeken}\frametitle{Wijzigingen nader bekeken}
	\framesubtitle{Symfony/Console integratie in de CommandController (1)}

	% decrease font size for code listing
	\lstset{basicstyle=\tiny\ttfamily}

	De CommandController maakt nu intern gebruik van Symfony/Console en biedt diverse functies:

	\begin{itemize}

		\item \smaller TableHelper
			\begin{itemize}
				\item\smaller\texttt{outputTable(\$rows, \$headers = NULL)}
			\end{itemize}

		\item DialogHelper
			\begin{itemize}
				\item\smaller\texttt{select(\$question, \$choices, \$default = NULL, \$multiSelect = false, \$attempts = FALSE)}
				\item\texttt{ask(\$question, \$default = NULL, array \$autocomplete = array())}
				\item\texttt{askConfirmation(\$question, \$default = TRUE)}
				\item\texttt{askHiddenResponse(\$question, \$fallback = TRUE)}
				\item\texttt{askAndValidate(\$question, \$validator, \$attempts = FALSE, \$default = NULL, array \$autocomplete = NULL)}
				\item\texttt{askHiddenResponseAndValidate(\$question, \$validator, \$attempts = FALSE, \$fallback = TRUE)}
			\end{itemize}

	\end{itemize}

\end{frame}

% ------------------------------------------------------------------------------
% LTXE-SLIDE-START
% LTXE-SLIDE-UID:		fce509fe-13bac530-67eb1eb9-798998e9
% LTXE-SLIDE-ORIGIN:	a24edba5-e325709b-54cabffa-5ba42de7 English
% LTXE-SLIDE-ORIGIN:	3842b75e-8a44920f-a8c139e5-bd03e141 German
% LTXE-SLIDE-TITLE:		Feature #59606: Integrate Symfony/Console into CommandController (2)
% LTXE-SLIDE-REFERENCE:	Feature-59606-IntegrateSymfonyConsoleIntoCommandController.rst
% ------------------------------------------------------------------------------

\begin{frame}[fragile]
	\frametitle{In-Depth Changes}
	\framesubtitle{Symfony/Console integratie in de CommandController (2)}

	% decrease font size for code listing
	\lstset{basicstyle=\tiny\ttfamily}

	\begin{itemize}
		\item \smaller ProgressHelper
			\begin{itemize}
				\item \smaller\texttt{progressStart(\$max = NULL)}
				\item \texttt{progressSet(\$current)}
				\item \texttt{progressAdvance(\$step = 1)}
				\item \texttt{progressFinish()}
			\end{itemize}
	\end{itemize}

	\smaller
		(zie volgende pagina's voor voobeelden)
	\normalsize

\end{frame}

% ------------------------------------------------------------------------------
% LTXE-SLIDE-START
% LTXE-SLIDE-UID:		21924700-b71a977d-8399a719-d95330d9
% LTXE-SLIDE-ORIGIN:	26ee64a3-a8f3ef0d-247faf3b-7df91473 English
% LTXE-SLIDE-ORIGIN:	a8c139e5-8a44920f-3842b75e-bd03e141 German
% LTXE-SLIDE-TITLE:		Feature #59606: Integrate Symfony/Console into CommandController (3)
% LTXE-SLIDE-REFERENCE:	Feature-59606-IntegrateSymfonyConsoleIntoCommandController.rst
% ------------------------------------------------------------------------------

\begin{frame}[fragile]
	\frametitle{Wijzigingen nader bekeken}
	\framesubtitle{Symfony/Console integratie in de CommandController (3)}

	% decrease font size for code listing
	\lstset{basicstyle=\tiny\ttfamily}

		\begin{lstlisting}
			<?php
			namespace Acme\Demo\Command;
			use TYPO3\CMS\Extbase\Mvc\Controller\CommandController;

			class MyCommandController extends CommandController {
			  public function myCommand() {

			    // render a table
			    $this->output->outputTable(array(
			      array('Bob', 34, 'm'),
			      array('Sally', 21, 'v'),
			      array('Blake', 56, 'm')
			    ),
			    array('Naam', 'Leeftijd', 'Geslacht'));

			    // select
			    $colors = array('rood', 'blauw', 'geel');
			    $selectedColorIndex = $this->output->select('Kies een kleur', $colors, 'red');
			    $this->outputLine('Gekozen kleur: %s.', array($colors[$selectedColorIndex]));

			    [...]
		\end{lstlisting}

\end{frame}

% ------------------------------------------------------------------------------
% LTXE-SLIDE-START
% LTXE-SLIDE-UID:		040c4142-4fb997e8-f2605baf-6944c278
% LTXE-SLIDE-ORIGIN:	e7f1a1e9-cabd20ae-78d335fc-3868ca27 English
% LTXE-SLIDE-ORIGIN:	a8c139e5-8a44920f-3842b75e-bd03e141 German
% LTXE-SLIDE-TITLE:		Feature #59606: Integrate Symfony/Console into CommandController (4)
% LTXE-SLIDE-REFERENCE:	Feature-59606-IntegrateSymfonyConsoleIntoCommandController.rst
% ------------------------------------------------------------------------------

\begin{frame}[fragile]
	\frametitle{Wijzigingen nader bekeken}
	\framesubtitle{Symfony/Console integratie in de CommandController (4)}

	% decrease font size for code listing
	\lstset{basicstyle=\tiny\ttfamily}

		\begin{lstlisting}
			    [...]
			    // vraag
			    $name = $this->output->ask('Wat is je naam?' . PHP_EOL, 'Bob', array('Bob', 'Sally', 'Blake'));
			    $this->outputLine('Hallo %s.', array($name));

			    // bevestiging
			    $likesDogs = $this->output->askConfirmation('Hou je van honden?');
			    if ($likesDogs) {
			      $this->outputLine('Je houdt van honden!');
			    }

			    // progress
			    $this->output->progressStart(600);
			    for ($i = 0; $i < 300; $i ++) {
			      $this->output->progressAdvance();
			      usleep(5000);
			    }
			    $this->output->progressFinish();
			  }
			}
			?>
		\end{lstlisting}

\end{frame}

% ------------------------------------------------------------------------------
% LTXE-SLIDE-START
% LTXE-SLIDE-UID:		4d50ee9a-bf4ba9ae-475d08f3-f94d53fd
% LTXE-SLIDE-ORIGIN:	b9b53d17-7e2c0c83-dabec46a-1a756e39 English
% LTXE-SLIDE-ORIGIN:	9d56ca19-04a98896-ba0a973c-8a16e877 German
% LTXE-SLIDE-TITLE:		Feature #66669: BE login form API (1)
% LTXE-SLIDE-REFERENCE:	Feature-66669-BeLoginFormAPI.rst
% ------------------------------------------------------------------------------

\begin{frame}[fragile]
	\frametitle{Wijzigingen nader bekeken}
	\framesubtitle{Backend aanmelding API (1)}

	% decrease font size for code listing
	\lstset{basicstyle=\tiny\ttfamily}

	\begin{itemize}

		\item De aanmelding voor de Backend is compleet verbouwd en heeft een nieuwe API

		\item Het OpenID-formulier is eruit gehaald en gebruikt de nieuwe API
			(nu onafhankelijk van de klassen uit de core)

		\item Het concept van de backend aanmelding is gebaseerd op "login providers", die
			als volgt geregistreerd worden in \texttt{ext\_localconf.php}:

			\begin{lstlisting}
				$GLOBALS['TYPO3_CONF_VARS']['EXTCONF']['backend']['loginProviders'][1433416020] = [
				  'provider' => \TYPO3\CMS\Backend\LoginProvider\UsernamePasswordLoginProvider::class,
				  'sorting' => 50,
				  'icon-class' => 'fa-key',
				  'label' => 'LLL:EXT:backend/Resources/Private/Language/locallang.xlf:login.link'
				];
			\end{lstlisting}

	\end{itemize}

\end{frame}

% ------------------------------------------------------------------------------
% LTXE-SLIDE-START
% LTXE-SLIDE-UID:		d2f09582-5f32a9d0-74a0f1f9-71cf30e4
% LTXE-SLIDE-ORIGIN:	c443db74-ed3267f9-a113f486-d586081d English
% LTXE-SLIDE-ORIGIN:	9889604a-ca199d56-73cba0a9-e8778a16 German
% LTXE-SLIDE-TITLE:		Feature #66669: BE login form API (2)
% LTXE-SLIDE-REFERENCE:	Feature-66669-BeLoginFormAPI.rst
% ------------------------------------------------------------------------------

\begin{frame}[fragile]
	\frametitle{Wijzigingen nader bekeken}
	\framesubtitle{Backend aanmelding API (2)}

	\begin{itemize}

		\item Opties zijn als volgt:

			\begin{itemize}

				\item \texttt{provider}:\newline
					login provider klassenaam, die moet implementeren:
						\texttt{TYPO3\textbackslash
							CMS\textbackslash
							Backend\textbackslash
							LoginProvider\textbackslash
							LoginProviderInterface}

				\item \texttt{sorting}:\newline
					volgorde van de links naar de beschikbare login providers op het aanmeldscherm

				\item \texttt{icon-class}:\newline
					naam van het font-awesome icoon voor de link naar het aanmeldscherm

				\item \texttt{label}:\newline
					label voor de link naar de login provider op het aanmeldscherm
			\end{itemize}

	\end{itemize}

\end{frame}

% ------------------------------------------------------------------------------
% LTXE-SLIDE-START
% LTXE-SLIDE-UID:		cc5c310a-219a63bf-3524b6a7-994dbba6
% LTXE-SLIDE-ORIGIN:	66005bf1-6706e6ba-98feeb3f-ac972c3d English
% LTXE-SLIDE-ORIGIN:	04a98896-ba0a973c-8a16e877-9d56ca19 German
% LTXE-SLIDE-TITLE:		Feature #66669: BE login form API (3)
% LTXE-SLIDE-REFERENCE:	Feature-66669-BeLoginFormAPI.rst
% ------------------------------------------------------------------------------

\begin{frame}[fragile]
	\frametitle{Wijzigingen nader bekeken}
	\framesubtitle{Backend aanmelding API (3)}

	\begin{itemize}

		\item De \texttt{LoginProviderInterface} bevat slechts de functie\newline
			\smaller\texttt{public function render(StandaloneView \$view, PageRenderer \$pageRenderer, LoginController \$loginController);}\normalsize

		\item Parameters zijn als volgt gedefinieerd:

			\begin{itemize}

				\item \texttt{\$view}:\newline
					Fluid StandaloneView die het aanmeldscherm rendert. Het sjabloonbestand moet ingesteld worden
					en optioneel kunnen variabelen toegevoegd worden.

				\item \texttt{\$pageRenderer}:\newline
					PageRenderer instantie geeft mogelijkeheid om JavaScript toe te voegen.

				\item \texttt{\$loginController}:\newline
					LoginController instantie.

			\end{itemize}

	\end{itemize}

\end{frame}

% ------------------------------------------------------------------------------
% LTXE-SLIDE-START
% LTXE-SLIDE-UID:		9e2eb058-58e55982-6ac07cf8-6b1273f8
% LTXE-SLIDE-ORIGIN:	dd0c782c-161abebc-ba03f841-6d108444 English
% LTXE-SLIDE-ORIGIN:	04a98896-ba0a973c-8a16e877-9d56ca19 German
% LTXE-SLIDE-TITLE:		Feature #66669: BE login form API (4)
% LTXE-SLIDE-REFERENCE:	Feature-66669-BeLoginFormAPI.rst
% ------------------------------------------------------------------------------

\begin{frame}[fragile]
	\frametitle{Wijzigingen nader bekeken}
	\framesubtitle{Backend aanmelding API (4)}

	% decrease font size for code listing
	\lstset{basicstyle=\tiny\ttfamily}

	\begin{itemize}

		\item Sjabloon moet \texttt{<f:layout name="Login">} en
			\texttt{<f:section name="loginFormFields">} (voor formuliervelden) bevatten:

			\begin{lstlisting}
				<f:layout name="Login" />
				<f:section name="loginFormFields">
				  <div class="form-group t3js-login-openid-section" id="t3-login-openid_url-section">
				    <div class="input-group">
				      <input type="text" id="openid_url"
				        name="openid_url"
				        value="{presetOpenId}"
				        autofocus="autofocus"
				        placeholder="{f:translate(key: 'openId', extensionName: 'openid')}"
				        class="form-control input-login t3js-clearable t3js-login-openid-field" />
				      <div class="input-group-addon">
				        <span class="fa fa-openid"></span>
				      </div>
				    </div>
				  </div>
				</f:section>
			\end{lstlisting}

	\end{itemize}

\end{frame}

% ------------------------------------------------------------------------------
% LTXE-SLIDE-START
% LTXE-SLIDE-UID:		617cfd37-cb1fd34a-0a34c8a2-acf1ca02
% LTXE-SLIDE-ORIGIN:	2051c5f1-bd00622f-dfcbba05-c329dd60 English
% LTXE-SLIDE-ORIGIN:	cc63cd98-524ee6df-38430963-f7c7067c German
% LTXE-SLIDE-TITLE:		Feature #66681: CategoryRegistry: add options to set l10n_mode and l10n_display
% LTXE-SLIDE-REFERENCE:	Feature-66681-CategoryRegistryAddOptionsToSetL10n_modeAndL10n_display.rst
% ------------------------------------------------------------------------------

\begin{frame}[fragile]
	\frametitle{Wijzigingen nader bekeken}
	\framesubtitle{\texttt{CategoryRegistry} met nieuwe opties}

	% decrease font size for code listing
	\lstset{basicstyle=\tiny\ttfamily}

	\begin{itemize}

		\item Functie \texttt{CategoryRegistry->addTcaColumn} heeft extra opties om
			\texttt{l10n\_mode} en \texttt{l10n\_display} in te stellen:

			\begin{lstlisting}
				\TYPO3\CMS\Core\Utility\ExtensionManagementUtility::makeCategorizable(
				  $extensionKey,
				  $tableName,
				  'categories',
				  array(
				    'l10n_mode' => 'string (keyword)',
				    'l10n_display' => 'list of keywords'
				  )
				);
			\end{lstlisting}

	\end{itemize}

\end{frame}

% ------------------------------------------------------------------------------
% LTXE-SLIDE-START
% LTXE-SLIDE-UID:		9dd9f684-c45c074b-1ea2e517-ac4968d2
% LTXE-SLIDE-ORIGIN:	52d9c9a1-37301644-b52c1658-372774a8 English
% LTXE-SLIDE-ORIGIN:	c9a5288c-f45aac1e-34606612-fdf6ec18 German
% LTXE-SLIDE-TITLE:		Feature #66822: Allow Sprites For Backend Modules
% LTXE-SLIDE-REFERENCE:	Feature-66822-SpriteIconsInBackendModules.rst
% ------------------------------------------------------------------------------

\begin{frame}[fragile]
	\frametitle{Wijzigingen nader bekeken}
	\framesubtitle{Sprites in Backend modules}

	% decrease font size for code listing
	\lstset{basicstyle=\tiny\ttfamily}

	\begin{itemize}

		\item Backend modules (hoofdmodules zoals "Web" maar ook submodules zoals
			"Bestandslijst") kunnen nu sprites gebruiken als iconen\newline
			\small
				(alleen sprite-iconen die bekend zijn bij TYPO3 zijn beschikbaar!)
			\normalsize

		\item Example:

			\begin{lstlisting}
				\TYPO3\CMS\Core\Utility\ExtensionManagementUtility::addModule(
				  'web',
				  'layout',
				  'top',
				  \TYPO3\CMS\Core\Utility\ExtensionManagementUtility::extPath($_EXTKEY) . 'Modules/Layout/',
				  array(
				    'script' => '_DISPATCH',
				    'access' => 'user,group',
				    'name' => 'web_layout',
				    'configuration' => array('icon' => 'module-web'),
				    'labels' => array(
				      'll_ref' => 'LLL:EXT:cms/layout/locallang_mod.xlf',
				    ),
				  )
				);
			\end{lstlisting}

	\end{itemize}

\end{frame}

% ------------------------------------------------------------------------------
% LTXE-SLIDE-START
% LTXE-SLIDE-UID:		7b77b88b-269a8053-9d01674a-98422735
% LTXE-SLIDE-ORIGIN:	8d50a473-76b35cf0-4414be81-8efd94f3 English
% LTXE-SLIDE-ORIGIN:	d9657b5d-5cd56c65-d7266a27-401dd233 German
% LTXE-SLIDE-TITLE:		Feature #67229: FormEngine NodeFactory API
% LTXE-SLIDE-REFERENCE:	Feature-67229-FormEngineNodeFactoryApi.rst
% ------------------------------------------------------------------------------

\begin{frame}[fragile]
	\frametitle{Wijzigingen nader bekeken}
	\framesubtitle{FormEngine NodeFactory API (1)}

	% decrease font size for code listing
	\lstset{basicstyle=\tiny\ttfamily}

	\begin{itemize}

		\item Het is nu mogelijk om nieuwe knooppunten te registreren en bestaande knooppunten te overschrijven

		\begin{lstlisting}
			$GLOBALS['TYPO3_CONF_VARS']['SYS']['formEngine']['nodeRegistry'][1433196792] = array(
			  'nodeName' => 'input',
			  'priority' => 40,
			  'class' => \MyVendor\MyExtension\Form\Element\T3editorElement::class
			);
		\end{lstlisting}

		\item Voorbeeld hierboven registreert een nieuwe klasse
			\texttt{MyVendor\textbackslash
				MyExtension\textbackslash
				Form\textbackslash
				Element\textbackslash
				T3editorElement}
			voor het renderen van TCA-type \texttt{input}, die de interface
			\texttt{TYPO3\textbackslash
				CMS\textbackslash
				Backend\textbackslash
				Form\textbackslash
				NodeInterface}
			moet implementeren

		\item De index van de array is de Unix timestamp van het moment dat het element is toegevoegd

	\end{itemize}

\end{frame}

% ------------------------------------------------------------------------------
% LTXE-SLIDE-START
% LTXE-SLIDE-UID:		b675d39a-65dc88b8-f58cba7c-40b1362d
% LTXE-SLIDE-ORIGIN:	94bf7d97-b60d2c04-a1693909-e7397a81 English
% LTXE-SLIDE-ORIGIN:	5cd56c65-d7266a27-401dd233-d9657b5d German
% LTXE-SLIDE-TITLE:		Feature #67229: FormEngine NodeFactory API
% LTXE-SLIDE-REFERENCE:	Feature-67229-FormEngineNodeFactoryApi.rst
% ------------------------------------------------------------------------------

\begin{frame}[fragile]
	\frametitle{Wijzigingen nader bekeken}
	\framesubtitle{FormEngine NodeFactory API (2)}

	% decrease font size for code listing
	\lstset{basicstyle=\tiny\ttfamily}

	\begin{itemize}

		\item Als er meerdere registry-elementen zijn geregisteerd voor hetzelfde
			type, wordt degene met de hoogste prioriteit (0 to 100) gebruikt

		\item A nieuw TCA type kan geregistreerd worden met:

		\smaller\textbf{\texttt{TCA}}
		\begin{lstlisting}
			'columns' => array(
			  'bodytext' => array(
			    'config' => array(
			      'type' => 'text',
			      'renderType' => '3dCloud',
			    ),
			  ),
			)
		\end{lstlisting}

		\smaller\textbf{\texttt{ext\_localconf.php}}
		\begin{lstlisting}
			$GLOBALS['TYPO3_CONF_VARS']['SYS']['formEngine']['nodeRegistry'][1433197759] = array(
			  'nodeName' => '3dCloud',
			  'priority' => 40,
			  'class' => \MyVendor\MyExtension\Form\Element\ShowTextAs3dCloudElement::class
			);
		\end{lstlisting}

	\end{itemize}

\end{frame}

% ------------------------------------------------------------------------------
% LTXE-SLIDE-START
% LTXE-SLIDE-UID:		8ce2b66d-f55b844e-df9c3151-5e6084f3
% LTXE-SLIDE-ORIGIN:	818411e6-dd2a7aba-3a7afb94-4e970169 English
% LTXE-SLIDE-ORIGIN:	d7266a27-401dd233-d9657b5d-5cd56c65 German
% LTXE-SLIDE-TITLE:		Breaking #62983: postProcessMirrorUrl signal has moved
% LTXE-SLIDE-REFERENCE:	Breaking-62983-PostProcessMirrorUrlSignalHasMoved.rst
% ------------------------------------------------------------------------------

\begin{frame}[fragile]
	\frametitle{Wijzigingen nader bekeken}
	\framesubtitle{Signaal \texttt{postProcessMirrorUrl}}

	% decrease font size for code listing
	\lstset{basicstyle=\tiny\ttfamily}

	\begin{itemize}

		\item Het Signaal \texttt{postProcessMirrorUrl} is naar een nieuwe klasse verplaatst

		\breakingchange

		\item Dit voorbeeld houdt rekening met de TYPO3-versie:

		\begin{lstlisting}
			$signalSlotDispatcher->connect(
			  version_compare(TYPO3_version, '7.0', '<')
			    ? 'TYPO3\\CMS\\Lang\\Service\\UpdateTranslationService'
			    : 'TYPO3\\CMS\\Lang\\Service\\TranslationService',
			  'postProcessMirrorUrl',
			  'Vendor\\Extension\\Slots\\CustomMirror',
			  'postProcessMirrorUrl'
			);
		\end{lstlisting}

	\end{itemize}

\end{frame}

% ------------------------------------------------------------------------------
