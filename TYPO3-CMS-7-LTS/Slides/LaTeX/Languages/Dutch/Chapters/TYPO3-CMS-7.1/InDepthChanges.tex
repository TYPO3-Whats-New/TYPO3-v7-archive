% ------------------------------------------------------------------------------
% TYPO3 CMS 7.1 - What's New - Chapter "In-Depth Changes" (Dutch Version)
%
% @author	Michael Schams <schams.net>
% @license	Creative Commons BY-NC-SA 3.0
% @link		http://typo3.org/download/release-notes/whats-new/
% @language	Dutch
% ------------------------------------------------------------------------------
% LTXE-CHAPTER-UID:		7cb2d402-105890ae-a1632623-719adbb6
% LTXE-CHAPTER-NAME:	In-Depth Changes
% ------------------------------------------------------------------------------

\section{Systeemwijzigingen}

% ------------------------------------------------------------------------------
% LTXE-SLIDE-START
% LTXE-SLIDE-UID:		28eed0ef-8dd7e92a-36275259-58d07f16
% LTXE-SLIDE-ORIGIN:	505c827e-aab9a39a-612d0288-bb873878 English
% LTXE-SLIDE-TITLE:		Maximum chars in text element
% LTXE-SLIDE-REFERENCE:	Feature-24906-MaxForTextElement.rst
% ------------------------------------------------------------------------------

\begin{frame}[fragile]
	\frametitle{Systeemwijzigingen}
	\framesubtitle{TCA: Maximaal aantal karakters in tekstelement}

	\begin{itemize}
		\item TCA-type \texttt{text} biedt nu ondersteuning voor het HTML5 attribuut \texttt{maxlength}
			om de lengte van een tekst te beperken (opmerking: regeleinden tellen gewoonlijk als twee karakters)

			\begin{lstlisting}
				'teaser' => array(
				  'label' => 'Teaser',
				  'config' => array(
				    'type' => 'text',
				    'cols' => 60,
				    'rows' => 2,
				    'max' => '30' // <-- maxlength
				  )
				),
			\end{lstlisting}

			Merk op dat niet elke browser dit attribuut ondersteunt.\newline
			Zie \href{http://www.w3schools.com/tags/att_textarea_maxlength.asp}{Browser Support List} voor details.

	\end{itemize}

\end{frame}

% ------------------------------------------------------------------------------
% LTXE-SLIDE-START
% LTXE-SLIDE-UID:		ba9b9878-8eb40f7e-37ac0605-37617bcf
% LTXE-SLIDE-ORIGIN:	a3d28e8d-6d40bc8f-f60bb05f-d9e31bba English
% LTXE-SLIDE-TITLE:		New SplFileInfo implementation
% LTXE-SLIDE-REFERENCE:	Feature-60019-SplFileInfo-MimeTypeGuesser-hook.rst
% ------------------------------------------------------------------------------

\begin{frame}[fragile]
	\frametitle{Systeemwijzigingen}
	\framesubtitle{Nieuwe implementatie van SplFileInfo}

	% decrease font size for code listing
	\lstset{basicstyle=\smaller\ttfamily}

	\begin{itemize}

		\item Nieuwe klasse:
			\texttt{TYPO3\textbackslash
				CMS\textbackslash
				Core\textbackslash
				Type\textbackslash
				File\textbackslash
				FileInfo}

		\item Deze klasse is een uitbreiding van klasse \texttt{SplFileInfo} die het mogelijk 
			maakt om meta-informatie van bestanden op te halen

			\begin{lstlisting}
				$fileIdentifier = '/tmp/foo.html';
				$fileInfo = GeneralUtility::makeInstance(
				  \TYPO3\CMS\Core\Type\File\FileInfo::class,
				  $fileIdentifier
				);
				echo $fileInfo->getMimeType(); // output: text/html
			\end{lstlisting}

		\item Custom implementaties kunnen de volgende hook gebruiken:

			\begin{lstlisting}
				$GLOBALS['TYPO3_CONF_VARS']['SC_OPTIONS']
				  [\TYPO3\CMS\Core\Type\File\FileInfo::class]['mimeTypeGuessers']
			\end{lstlisting}

	\end{itemize}

\end{frame}

% ------------------------------------------------------------------------------
% LTXE-SLIDE-START
% LTXE-SLIDE-UID:		4d3e8aa3-c75342d3-89914c05-b3137ec9
% LTXE-SLIDE-ORIGIN:	5351c635-145cc752-9c930534-3a8b4e50 English
% LTXE-SLIDE-TITLE:		UserFunc in TCA Display Condition
% LTXE-SLIDE-REFERENCE:	Feature-62944-UserFuncAsDisplayCond.rst
% ------------------------------------------------------------------------------

\begin{frame}[fragile]
	\frametitle{Systeemwijzigingen}
	\framesubtitle{userFunc in de TCA Display Condition}

	\begin{itemize}
		\item Een userFunc in de \texttt{displayCondition} maakt het mogelijk om op elke denkbare conditie of toestand te controleren 
		\item Wanneer een situatie niet kan worden afgevangen met één van de bestaande controles, 
			kunnen ontwikkelaars hun eigen gebruikerfuncties ontwikkelen 
			(return \texttt{TRUE/FALSE} om het daarvoor bestemde veld te tonen/verbergen)

			\begin{lstlisting}
				$GLOBALS['TCA']['tt_content']['columns']['bodytext']['displayCond'] =
				  'USER:Vendor\\Example\\User\\ElementConditionMatcher->
				    checkHeaderGiven:any:more:information';
			\end{lstlisting}

	\end{itemize}

\end{frame}

% ------------------------------------------------------------------------------
% LTXE-SLIDE-START
% LTXE-SLIDE-UID:		fd7a3794-cdeb6841-54df2b60-dceb028c
% LTXE-SLIDE-ORIGIN:	3ba78d86-ca95152a-5b7e29c9-7e6593da English
% LTXE-SLIDE-TITLE:		API for Twitter Bootstrap modals (1)
% LTXE-SLIDE-REFERENCE:	Feature-63729-ApiForBootstrapModals.rst
% ------------------------------------------------------------------------------

\begin{frame}[fragile]
	\frametitle{Systeemwijzigingen}
	\framesubtitle{API voor Twitter Bootstrap Modals (1)}

	% decrease font size for code listing
	\lstset{basicstyle=\smaller\ttfamily}

	\begin{itemize}

		\item Twee nieuwe API-methoden om popup modals te creëeren/verbergen:
			\begin{itemize}
				\item \texttt{TYPO3.Modal.confirm(title, content, severity, buttons)}
				\item \texttt{TYPO3.Modal.dismiss()}
			\end{itemize}

		\item De opties \texttt{title} en \texttt{content} zijn verplicht
		\item Wanneer \texttt{buttons} wordt gebruikt, zijn ook de opties \texttt{buttons.text} en \texttt{buttons.trigger} verplicht

		\item Voorbeeld 1:

			\begin{lstlisting}
				TYPO3.Modal.confirm(
				  'The title of the modal',         // title
				  'This the the body of the modal', // content
				  TYPO3.Severity.warning            // severity
				);
			\end{lstlisting}

	\end{itemize}

\end{frame}

% ------------------------------------------------------------------------------
% LTXE-SLIDE-START
% LTXE-SLIDE-UID:		42491203-de18b6e5-8106b2c8-1e37d678
% LTXE-SLIDE-ORIGIN:	ca95152a-5b7e29c9-7e6593da-3ba78d86 English
% LTXE-SLIDE-TITLE:		API for Twitter Bootstrap modals (2)
% LTXE-SLIDE-REFERENCE:	Feature-63729-ApiForBootstrapModals.rst
% ------------------------------------------------------------------------------

\begin{frame}[fragile]
	\frametitle{Systeemwijzigingen}
	\framesubtitle{API voor Twitter Bootstrap Modals (2)}

	\begin{itemize}

		\item Voorbeeld 2:

		\begin{lstlisting}
			TYPO3.Modal.confirm('Warning', 'You may break the internet!',
			  TYPO3.Severity.warning,
			  [
			    {
			      text: 'Break it',
			      active: true,
			      trigger: function() { ... }
			    },
			    {
			      text: 'Abort!',
			      trigger: function() {
			        TYPO3.Modal.dismiss();
			      }
			    }
			  ]
			);
		\end{lstlisting}

	\end{itemize}

\end{frame}

% ------------------------------------------------------------------------------
% LTXE-SLIDE-START
% LTXE-SLIDE-UID:		9b6f0551-fa428235-92b27306-e9516760
% LTXE-SLIDE-ORIGIN:	4e7489f8-bced2d21-6e6b87d3-2700d161 English
% LTXE-SLIDE-TITLE:		JavaScript Storage API (1)
% LTXE-SLIDE-REFERENCE:	Feature-64031-JavaScript-Storage-API.rst
% ------------------------------------------------------------------------------

\begin{frame}[fragile]
	\frametitle{Systeemwijzigingen}
	\framesubtitle{API voor JavaScript Storage (1)}

	\begin{itemize}
		\item Toegang tot de BE-user configuratie (\texttt{\$BE\_USER->uc}) kan worden
			afgehandeld in JavaScript door simpele sleutel/waarde-paren te gebruiken
		\item Bovendien kan HTML5's \href{http://www.w3.org/TR/webstorage/}{localStorage}
			worden gebruikt om gegevens (client-side) op te slaan in de browser van de gebruiker

		\item Twee nieuwe globale TYPO3 objecten:
			\begin{itemize}
				\item \texttt{top.TYPO3.Storage.Client}
				\item \texttt{top.TYPO3.Storage.Persistent}
			\end{itemize}

		\item Elk object heeft de volgende API-methoden:
			\begin{itemize}
				\item \texttt{get(key)}: haal gegevens op
				\item \texttt{set(key,value)}: schrijf gegevens weg
				\item \texttt{isset(key)}: controleer of de sleutel bestaat
				\item \texttt{clear()}: leeg de gehele storage
			\end{itemize}

	\end{itemize}

\end{frame}

% ------------------------------------------------------------------------------
% LTXE-SLIDE-START
% LTXE-SLIDE-UID:		8850a5bf-f10cc695-25dbcd72-a6cb341d
% LTXE-SLIDE-ORIGIN:	bced2d21-6e6b87d3-2700d161-4e7489f8 English
% LTXE-SLIDE-TITLE:		JavaScript Storage API (2)
% LTXE-SLIDE-REFERENCE:	Feature-64031-JavaScript-Storage-API.rst
% ------------------------------------------------------------------------------

\begin{frame}[fragile]
	\frametitle{Systeemwijzigingen}
	\framesubtitle{API voor JavaScript Storage  (2)}

	\begin{itemize}
		\item Voorbeeld:
			\begin{lstlisting}
				// get value of key 'startModule'
				var value = top.TYPO3.Storage.Persistent.get('startModule');

				// write value 'web_info' as key 'start_module'
				top.TYPO3.Storage.Persistent.set('startModule', 'web_info');
			\end{lstlisting}

	\end{itemize}

\end{frame}

% ------------------------------------------------------------------------------
% LTXE-SLIDE-START
% LTXE-SLIDE-UID:		3ac30b18-e6c4062e-c8f6a439-d5d22548
% LTXE-SLIDE-ORIGIN:	62879554-fac9feac-89f11d0e-afeefc65 English
% LTXE-SLIDE-TITLE:		Inline Rendering of Checkboxes
% LTXE-SLIDE-REFERENCE:	Feature-64190-FormEngineCheckboxElement.rst
% ------------------------------------------------------------------------------

\begin{frame}[fragile]
	\frametitle{Systeemwijzigingen}
	\framesubtitle{Gealigneerde weergave van selectievakjes}

	% decrease font size for code listing
	\lstset{basicstyle=\tiny\ttfamily}

	\begin{itemize}

		\item Bij selectievakjes kan bij 'cols' de instelling \texttt{inline} worden gebruikt 
			om selectievakjes naast elkaar weer te geven om de gebruikte ruimte te beperken

		\begin{lstlisting}
			'weekdays' => array(
			  'label' => 'Weekdays',
			  'config' => array(
			    'type' => 'check',
			    'items' => array(
			      array('Mo', ''),
			      array('Tu', ''),
			      array('We', ''),
			      array('Th', ''),
			      array('Fr', ''),
			      array('Sa', ''),
			      array('Su', '')
			    ),
			    'cols' => 'inline'
			  )
			),
			...
		\end{lstlisting}

	\end{itemize}

\end{frame}

% ------------------------------------------------------------------------------
% LTXE-SLIDE-START
% LTXE-SLIDE-UID:		88a2f341-bcfa41e8-9b92e0a4-06002141
% LTXE-SLIDE-ORIGIN:	75519e56-b1721347-754a017e-ac5d5e03 English
% LTXE-SLIDE-TITLE:		Content Object Registration
% LTXE-SLIDE-REFERENCE:	Feature-64386-ContentObjectRegistration.rst
% ------------------------------------------------------------------------------

\begin{frame}[fragile]
	\frametitle{Systeemwijzigingen}
	\framesubtitle{Registratie van Content Object}

	\begin{itemize}

		\item Er is een nieuwe globale optie geïntroduceerd om cObjecten zoals \texttt{TEXT} 
			te registeren, uit te breiden en/of te overschrijven

		\item Een lijst van alle beschikbare cObjecten is beschikbaar middels:

			\begin{lstlisting}
				$GLOBALS['TYPO3_CONF_VARS']['FE']['ContentObjects']
			\end{lstlisting}

		\item Voorbeeld: registreer een nieuw cObject \texttt{EXAMPLE}

			\begin{lstlisting}
				$GLOBALS['TYPO3_CONF_VARS']['FE']['ContentObjects']['EXAMPLE'] =
				  Vendor\MyExtension\ContentObject\ExampleContentObject::class;
			\end{lstlisting}

		\item De geregistreerde klasse moet een subklasse zijn van
			\small
				\texttt{TYPO3\textbackslash
					CMS\textbackslash
					Frontend\textbackslash
					ContentObject\textbackslash
					AbstractContentObject}
			\normalsize

		\item Bewaar je klasse in map\newline
			\small
				\texttt{typo3conf/myextension/Classes/ContentObject/}
			\normalsize\newline
			om voorbereid te zijn op toekomstige autoloadmechanismen

	\end{itemize}

\end{frame}

% ------------------------------------------------------------------------------
% LTXE-SLIDE-START
% LTXE-SLIDE-UID:		8080b402-b919b12c-d90ed8cf-dc7b1bc9
% LTXE-SLIDE-ORIGIN:	2a4acfc9-da01c536-7ff147af-7cd94754 English
% LTXE-SLIDE-TITLE:		Hooks and Signals (1)
% LTXE-SLIDE-REFERENCE:	Feature-52131-HookForPageRepositoryInit.rst
% ------------------------------------------------------------------------------

\begin{frame}[fragile]
	\frametitle{Systeemwijzigingen}
	\framesubtitle{Hooks en Signals (1)}

	\begin{itemize}

		\item Er is een nieuwe hook toegevoerd aan het einde van \texttt{PageRepository->init()},
			die het mogelijk maakt om de zichtbaarheid van pagina's te beïnvloeden

		\item Registreer de hook als volgt:
			\begin{lstlisting}
				$GLOBALS['TYPO3_CONF_VARS']['SC_OPTIONS']
				   [\TYPO3\CMS\Frontend\Page\PageRepository::class]['init']
			\end{lstlisting}

		\item De hookklasse moet de volgende interface implementeren:
			\begin{lstlisting}
				\TYPO3\CMS\Frontend\Page\PageRepositoryInitHookInterface
			\end{lstlisting}

	\end{itemize}

\end{frame}

% ------------------------------------------------------------------------------
% LTXE-SLIDE-START
% LTXE-SLIDE-UID:		859fa916-16b62cde-699173d5-8e08e859
% LTXE-SLIDE-ORIGIN:	da01c536-7ff147af-7cd94754-2a4acfc9 English
% LTXE-SLIDE-TITLE:		Hooks and Signals (2)
% LTXE-SLIDE-REFERENCE: Feature-58929-FooterHookInPageLayoutView.rst
% ------------------------------------------------------------------------------

\begin{frame}[fragile]
	\frametitle{Systeemwijzigingen}
	\framesubtitle{Hooks en Signals (2)}

	\begin{itemize}

		\item Er is een nieuwe hook toegevoegd aan de PageLayoutView om de weergave 
			van de onderkant van contentelementen te bewerken.

		\item Voorbeeld:
			\begin{lstlisting}
				$GLOBALS['TYPO3_CONF_VARS']['SC_OPTIONS']
				  ['cms/layout/class.tx_cms_layout.php']['tt_content_drawFooter'];
			\end{lstlisting}

		\item De hookklasse moet de volgende interface implementeren:
			\begin{lstlisting}
				\TYPO3\CMS\Backend\View\PageLayoutViewDrawFooterHookInterface
			\end{lstlisting}

	\end{itemize}

\end{frame}

% ------------------------------------------------------------------------------
% LTXE-SLIDE-START
% LTXE-SLIDE-UID:		d004387a-2e2a76e5-70a5a1f4-01450fe6
% LTXE-SLIDE-ORIGIN:	babf7cab-2216fff5-9bf5848c-f76386df English
% LTXE-SLIDE-TITLE:		Hooks and Signals (3)
% LTXE-SLIDE-REFERENCE: Feature-61725-AddHookToBackendUtilityCountVersionsOfRecordsOnPage.rst
% ------------------------------------------------------------------------------

\begin{frame}[fragile]
	\frametitle{Systeemwijzigingen}
	\framesubtitle{Hooks en Signals (3)}

	\begin{itemize}

		\item Er is een nieuwe hook toegevoegd als een post-processor van
			\small
				\texttt{BackendUtility::countVersionsOfRecordsOnPage}
			\normalsize

		\item Dit kan bijvoorbeeld worden gebruikt om de toestand van de workspace in de paginaboom zichtbaar te maken
		\item Registreer de hook als volgt:

			\begin{lstlisting}
				$GLOBALS['TYPO3_CONF_VARS']['SC_OPTIONS']
				  ['t3lib/class.t3lib_befunc.php']['countVersionsOfRecordsOnPage'][] =
				  'My\Package\HookClass->hookMethod';
			\end{lstlisting}

	\end{itemize}

\end{frame}

% ------------------------------------------------------------------------------
% LTXE-SLIDE-START
% LTXE-SLIDE-UID:		aa69b12a-efd12a9d-c9480264-5ad2be4e
% LTXE-SLIDE-ORIGIN:	2216fff5-9bf5848c-f76386df-babf7cab English
% LTXE-SLIDE-TITLE:		Hooks and Signals (4)
% LTXE-SLIDE-REFERENCE:	Feature-61711-SignalAtVeryEndOfDataPreprocessorFetchRecord.rst
% ------------------------------------------------------------------------------

\begin{frame}[fragile]
	\frametitle{Systeemwijzigingen}
	\framesubtitle{Hooks en Signals (4)}

	\begin{itemize}

		\item Een nieuw signal is toegevoegd aan het eind van de methode \texttt{DataPreprocessor::fetchRecord()}
		\item Dit kan bijvoorbeeld worden gebruikt om de array \texttt{regTableItems\_data} te bewerken,
			teneinde gemanipuleerde data in TCEForms weer te geven

			\begin{lstlisting}
				$this->getSignalSlotDispatcher()->dispatch(
				  \TYPO3\CMS\Backend\Form\DataPreprocessor::class,
				  'fetchRecordPostProcessing',
				  array($this)
				);
			\end{lstlisting}

	\end{itemize}

\end{frame}

% ------------------------------------------------------------------------------
% LTXE-SLIDE-START
% LTXE-SLIDE-UID:		e4bfa753-5833bef7-7cbb4d1e-26b75d2e
% LTXE-SLIDE-ORIGIN:	016bf36a-e935e00f-54bfa8f7-d0d2c16d English
% LTXE-SLIDE-TITLE:		Hooks and Signals (5)
% LTXE-SLIDE-REFERENCE:	Feature-62960-SignalForMailerInitialization.rst
% ------------------------------------------------------------------------------

\begin{frame}[fragile]
	\frametitle{Systeemwijzigingen}
	\framesubtitle{Hooks en Signals (5)}

	% decrease font size for code listing
	\lstset{basicstyle=\tiny\ttfamily}

	\begin{itemize}

		\item Er is een nieuw signal toegevoegd die extra bewerking mogelijk maakt na initialisatie 
			van een mailer object, bijvoorbeeld de registratie van een Swift Mailer-plugin

			\begin{lstlisting}
				$signalSlotDispatcher = \TYPO3\CMS\Core\Utility\GeneralUtility::makeInstance(
				  \TYPO3\CMS\Extbase\SignalSlot\Dispatcher::class
				);

				$signalSlotDispatcher->connect(
				  \TYPO3\CMS\Core\Mail\Mailer::class,
				  'postInitializeMailer',
				  \Vendor\Package\Slots\MailerSlot::class,
				  'registerPlugin'
				);
		\end{lstlisting}

	\end{itemize}

\end{frame}

% ------------------------------------------------------------------------------
% LTXE-SLIDE-START
% LTXE-SLIDE-UID:		721062de-d269b48a-7def9506-2f393a87
% LTXE-SLIDE-ORIGIN:	e935e00f-54bfa8f7-d0d2c16d-016bf36a English
% LTXE-SLIDE-TITLE:		Multiple UID in PageRepository::getMenu()
% LTXE-SLIDE-REFERENCE: Feature-64257-MultipleUidInPageRepositoryGetMenu.rst
% ------------------------------------------------------------------------------

\begin{frame}[fragile]
	\frametitle{Systeemwijzigingen}
	\framesubtitle{Meerdere UID's in \texttt{PageRepository::getMenu()}}

	\begin{itemize}


		\item De methode \texttt{PageRepository::getMenu()} accepteert nu arrays 
			om meerdere rootpagina's te kunnen definiëren

			\begin{lstlisting}
				$pageRepository = new \TYPO3\CMS\Frontend\Page\PageRepository();
				$pageRepository->init(FALSE);
				$rows = $pageRepository->getMenu(array(2, 3));
			\end{lstlisting}

	\end{itemize}

\end{frame}

% ------------------------------------------------------------------------------
