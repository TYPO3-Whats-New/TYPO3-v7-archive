% ------------------------------------------------------------------------------
% TYPO3 CMS 7.6 - What's New - Chapter "In-Depth Changes" (English Version)
%
% @author	Patrick Lobacher <patrick@lobacher.de> and Michael Schams <schams.net>
% @license	Creative Commons BY-NC-SA 3.0
% @link		http://typo3.org/download/release-notes/whats-new/
% @language	English
% ------------------------------------------------------------------------------
% LTXE-CHAPTER-UID:		7229f1b9-b481e9bc-09c46183-a86b6a7e
% LTXE-CHAPTER-NAME:	In-Depth Changes
% ------------------------------------------------------------------------------

\section{Cambios en Profundidad}
\begin{frame}[fragile]
	\frametitle{Cambios en Profundidad}

	\begin{center}\huge{Capítulo 3:}\end{center}
	\begin{center}\huge{\color{typo3darkgrey}\textbf{Cambios en Profundidad}}\end{center}

\end{frame}

% ------------------------------------------------------------------------------
% LTXE-SLIDE-START
% LTXE-SLIDE-UID:		44a702f2-796f16e5-09095d43-e2fc7d10
% LTXE-SLIDE-ORIGIN:	159a2d7f-b679989c-30cf1736-6693f827 English
% LTXE-SLIDE-TITLE:		Bootstrap for Install Tool (1)
% ------------------------------------------------------------------------------

\begin{frame}[fragile]
	\frametitle{Cambios en Profundidad}
	\framesubtitle{Bootstrap para la Herramienta de Instalación (1)}

	\begin{itemize}

		\item La Herramienta de Instalación está ahora basada en Bootstrap - para la parte de instalación:

			\begin{figure}
				\includegraphics[width=0.7\linewidth]{InDepthChanges/InstallToolBootstrap01.jpg}
			\end{figure}

	\end{itemize}

\end{frame}

% ------------------------------------------------------------------------------
% LTXE-SLIDE-START
% LTXE-SLIDE-UID:	7e4ed4d5-1f6086ca-b9576ccb-e9b68c95
% LTXE-SLIDE-ORIGIN:	66a71f73-91b6c90e-e548a037-e2d47f94 English
% LTXE-SLIDE-TITLE:		Bootstrap for Install Tool (2)
% ------------------------------------------------------------------------------

\begin{frame}[fragile]
	\frametitle{Cambios en Profundidad}
	\framesubtitle{Bootstrap para la Herramienta de Instalación (2)}

	\begin{itemize}

		\item La Herramienta de Instalación está ahora basada en Bootstrap - para la configuración:

			\begin{figure}
				\includegraphics[width=0.7\linewidth]{InDepthChanges/InstallToolBootstrap02.png}
			\end{figure}

	\end{itemize}

\end{frame}

% ------------------------------------------------------------------------------
% LTXE-SLIDE-START
% LTXE-SLIDE-UID:		4095e219-5ea4d9fa-4b4b08f4-de904493
% LTXE-SLIDE-ORIGIN:	f901074a-4080140f-dd93e505-1e8c2178 English
% LTXE-SLIDE-ORIGIN:	d5acb2d2-159a2d7f-6693f827-30cf1736 German
% LTXE-SLIDE-TITLE:		Form protection API for frontend usage
% LTXE-SLIDE-REFERENCE:	Feature-56633-FormProtectionAPIForFrontEndUsage.rst
% ------------------------------------------------------------------------------

\begin{frame}[fragile]
	\frametitle{Cambios en Profundidad}
	\framesubtitle{Protección CSRF para Plugins del Frontend}

	% decrease font size for code listing
	\lstset{basicstyle=\tiny\ttfamily}

	\begin{itemize}

		\item Nueva clase permite el uso de la API FormProtection API en el frontend

		\item Esto implementa una protección CSRF (Cross-Site Request Forgery)

			\begin{lstlisting}
				$formToken = \TYPO3\CMS\Core\FormProtection\FormProtectionFactory::get()->getFormProtection()->generateToken('news', 'edit', $uid);
				if (
				  $dataHasBeenSubmitted
				  && \TYPO3\CMS\Core\FormProtection\FormProtectionFactory::get()->validateToken(
				    \TYPO3\CMS\Core\Utility\GeneralUtility::_POST('formToken'), 'User setup', 'edit')) {
				  // processes the data
				}
				else {
				  // invalid token!
				}
			\end{lstlisting}

	\end{itemize}

\end{frame}

% ------------------------------------------------------------------------------
% LTXE-SLIDE-START
% LTXE-SLIDE-UID:		194ec828-acc302ad-fa8d7e42-44629e21
% LTXE-SLIDE-ORIGIN:	0198b067-68da7843-6a6d3f6d-94cee9b7 English
% LTXE-SLIDE-ORIGIN:	388c7243-b679989c-e2b30dbb-78f1aea4 German
% LTXE-SLIDE-TITLE:		Added LinkBrowser APIs (1)
% LTXE-SLIDE-REFERENCE:	Feature-66369-AddedLinkBrowserAPIs.rst
% ------------------------------------------------------------------------------

\begin{frame}[fragile]
	\frametitle{Cambios en Profundidad}
	\framesubtitle{Tabs para LinkBrowser (1)}

	% decrease font size for code listing
	\lstset{basicstyle=\tiny\ttfamily}

	\begin{itemize}

		\item Esta nueva característica permite extender el LinkBrowser con nuevos tabs

		\item Cada tab es manejado por el denominado "LinkHandler", que tiene que implementar la
			siguiente Interfaz:\newline
			\small
				\texttt{\textbackslash TYPO3\textbackslash CMS\textbackslash Recordlist\textbackslash LinkHandler\textbackslash LinkHandlerInterface}
			\normalsize

		\item LinkHandlers se registran en PageTSconfig como sigue:

			\begin{lstlisting}
				file {
				  handler = TYPO3\\CMS\\Recordlist\\LinkHandler\\FileLinkHandler
				  label = LLL:EXT:lang/locallang_browse_links.xlf:file
				  displayAfter = page
				  scanAfter = page
				  configuration {
				    customConfig = passed to the handler
				  }
				}
			\end{lstlisting}

	\end{itemize}

\end{frame}

% ------------------------------------------------------------------------------
% LTXE-SLIDE-START
% LTXE-SLIDE-UID:		263acfa3-8fa759fa-15c88123-92f9b964
% LTXE-SLIDE-ORIGIN:	7ff322bc-2ace574c-34ddebb5-559fd90c English
% LTXE-SLIDE-ORIGIN:	dfd89b4b-7b4b816c-e5904ba8-2527339b German
% LTXE-SLIDE-TITLE:		Added LinkBrowser APIs (2)
% LTXE-SLIDE-REFERENCE:	Feature-66369-AddedLinkBrowserAPIs.rst
% ------------------------------------------------------------------------------

\begin{frame}[fragile]
	\frametitle{Cambios en Profundidad}
	\framesubtitle{Tabs para LinkBrowser (2)}

	% decrease font size for code listing
	\lstset{basicstyle=\tiny\ttfamily}

	\begin{itemize}

		\item Opciones \texttt{displayBefore} y \texttt{displayAfter} definen las posiciones de las tabs

		\item Opciones \texttt{scanBefore} y \texttt{scanAfter} definen el orden en que los manejadores son
			ejecutados cuando se escanean enlaces existentes

			\begin{lstlisting}
				$GLOBALS['TYPO3_CONF_VARS']['SC_OPTIONS']['LinkBrowser']['hooks'][1444048118] = [
				  'handler' => \Vendor\Ext\MyClass::class,
				  'before' => [], // optional
				  'after' => [] // optional
				];
			\end{lstlisting}

	\end{itemize}

\end{frame}

% ------------------------------------------------------------------------------
% LTXE-SLIDE-START
% LTXE-SLIDE-UID:		bdcf51d6-efe9141a-6367198f-663d3729
% LTXE-SLIDE-ORIGIN:	1ef90646-4f4342dd-3bf0112b-95acba1b English
% LTXE-SLIDE-ORIGIN:	687f24a3-032124a3-7ac41f15-26329962 German
% LTXE-SLIDE-TITLE:		Module Template API (1)
% LTXE-SLIDE-REFERENCE:	Feature-69814-ModuleTemplateAPI.rst
% ------------------------------------------------------------------------------

\begin{frame}[fragile]
	\frametitle{Cambios en Profundidad}
	\framesubtitle{API del Módulo Template (1)}

	% decrease font size for code listing
	\lstset{basicstyle=\tiny\ttfamily}

	\begin{itemize}

		\item Una nueva API del Módulo Template pretende normalizar la implementación de DocHeaders

		\item Ejemplo 1: añadir un botón

			\begin{lstlisting}
				$openInNewWindowButton = $this->moduleTemplate->getDocHeaderComponent()->getButtonBar()
				  ->makeLinkButton()
				  ->setHref('#')
				  ->setTitle($this->getLanguageService()->sL(
				    'LLL:EXT:lang/locallang_core.xlf:labels.openInNewWindow', TRUE
				    ))
				  ->setIcon($this->iconFactory->getIcon('actions-window-open', Icon::SIZE_SMALL))
				  ->setOnClick($aOnClick);

				$this->moduleTemplate->getDocHeaderComponent()->getButtonBar()
				  ->addButton($openInNewWindowButton, ButtonBar::BUTTON_POSITION_RIGHT);
			\end{lstlisting}
	\end{itemize}

\end{frame}

% ------------------------------------------------------------------------------
% LTXE-SLIDE-START
% LTXE-SLIDE-UID:		4493fee5-cbcbffe2-5d0a29c9-915964e4
% LTXE-SLIDE-ORIGIN:	44c8e88e-5668ccb6-3cf424ea-c6a40ecf English
% LTXE-SLIDE-ORIGIN:	1570c8c9-2dfa48e9-058d2d04-bff5465d German
% LTXE-SLIDE-TITLE:		Module Template API (2)
% LTXE-SLIDE-REFERENCE:	Feature-69814-ModuleTemplateAPI.rst
% ------------------------------------------------------------------------------

\begin{frame}[fragile]
	\frametitle{Cambios en Profundidad}
	\framesubtitle{API del Módulo Template (2)}

	% decrease font size for code listing
	\lstset{basicstyle=\tiny\ttfamily}

	\begin{itemize}
		\item Ejemplo 2: añadir un menú con items de menú

			\begin{lstlisting}
				$languageMenu = $this->moduleTemplate->getDocHeaderComponent()
				  ->getModuleMenuRegistry()->makeMenu()
				  ->setIdentifier('_langSelector')
				  ->setLabel($this->getLanguageService()->sL(
				    'LLL:EXT:lang/locallang_general.xlf:LGL.language', TRUE
				  ));

				$menuItem = $languageMenu->makeMenuItem()
				  ->setTitle($lang['title'] . $newTranslation)
				  ->setHref($href);

				if((int)$lang['uid'] === $currentLanguage) {
				  $menuItem->setActive(TRUE);
				}

				$languageMenu->addMenuItem($menuItem);
				$this->moduleTemplate->getDocHeaderComponent()->getModuleMenuRegistry()->addMenu($languageMenu);
			\end{lstlisting}
	\end{itemize}

\end{frame}


% ------------------------------------------------------------------------------
% LTXE-SLIDE-START
% LTXE-SLIDE-UID:		a8cf1763-80663298-951ccb63-6efce633
% LTXE-SLIDE-ORIGIN:	2ffbc623-117a44dc-923610ec-78a3afc2 English
% LTXE-SLIDE-ORIGIN:	df3ea848-d5406e98-edbd6684-485ff477 German
% LTXE-SLIDE-TITLE:		PSR-7-based Routing for Backend AJAX Requests
% LTXE-SLIDE-REFERENCE:	Feature-69916-PSR-7-basedRoutingForBackendAJAXRequests.rst
% ------------------------------------------------------------------------------

\begin{frame}[fragile]
	\frametitle{Cambios en Profundidad}
	\framesubtitle{Ruteado PSR-7 para Peticiones AJAX del Backend}

	% decrease font size for code listing
	\lstset{basicstyle=\tiny\ttfamily}

	\begin{itemize}

		\item Para añadir una ruta para una petición AJAX, el fichero
			\texttt{Configuration/Backend/AjaxRoutes.php}\newline
			puede crearse con el siguiente contenido:

			\begin{lstlisting}
				return [
				  // do something
				  'unique_route_name' => [
				    'path' => '/toolcollection/some-action',
				    'target' => \Vendor\Controller\SomeController::class . '::myAction',
				  ]
				];
			\end{lstlisting}

	\end{itemize}

\end{frame}

% ------------------------------------------------------------------------------
% LTXE-SLIDE-START
% LTXE-SLIDE-UID:		449cfb05-4aeff6d5-26bed7be-74fcccce
% LTXE-SLIDE-ORIGIN:	b68a9d26-30e6d4a0-7bccecf2-ad1f93c6 English
% LTXE-SLIDE-ORIGIN:	cb01cca6-58e91977-d18fa0b3-cba021ab German
% LTXE-SLIDE-TITLE:		Introduced two new Hooks for OpenID (getUserRecord)
% LTXE-SLIDE-REFERENCE:	Feature-44127-HooksForOpenIdToAutomaticallyCreateUserAccounts.rst
% ------------------------------------------------------------------------------
\begin{frame}[fragile]
	\frametitle{Cambios en Profundidad}
	\framesubtitle{Hook OpenID \texttt{getUserRecord}}

	% decrease font size for code listing
	\lstset{basicstyle=\tiny\ttfamily}

	Se han añadido dos hooks al servicio OpenID (1/2)

		\begin{itemize}

			\item Hook 1:\newline
				\smaller\smaller
					\texttt{\$GLOBALS['TYPO3\_CONF\_VARS']['SC\_OPTIONS']['openid']['getUserRecord']}
				\normalsize

				\begin{itemize}
					\item Modifica el registro de usuario después de que ha sido recogido, o:
					\item Crea un nuevo registro si no se ha encontrado ninguno
					\item Parámetros \texttt{record}, \texttt{response} y \texttt{authInfo} se pasan al hook
				\end{itemize}

		\end{itemize}

\end{frame}

% ------------------------------------------------------------------------------
% LTXE-SLIDE-START
% LTXE-SLIDE-UID:		b0cdb3fe-35741024-a15d9697-991b36e0
% LTXE-SLIDE-ORIGIN:	b9b259e4-9d35ace7-edff1014-7390d1d1 English
% LTXE-SLIDE-ORIGIN:	ea0bfed8-831666d7-1ba40bb9-7a95723a German
% LTXE-SLIDE-TITLE:		Introduced two new Hooks for OpenID (authRequest)
% LTXE-SLIDE-REFERENCE:	Feature-44127-HooksForOpenIdToAutomaticallyCreateUserAccounts.rst
% ------------------------------------------------------------------------------
\begin{frame}[fragile]
	\frametitle{Cambios en Profundidad}
	\framesubtitle{OpenID Hook \texttt{authRequest}}

	% decrease font size for code listing
	\lstset{basicstyle=\tiny\ttfamily}

	Se han añadido dos hooks al servicio OpenID (2/2)

		\begin{itemize}

			\item Hook 2:\newline
				\smaller\smaller
					\texttt{\$GLOBALS['TYPO3\_CONF\_VARS']['SC\_OPTIONS']['openid']['authRequest']}
				\normalsize

				\begin{itemize}
					\item Modifies the Authentication Request, before it is sent
					\item Can be used to request additional attributes such as a nickname from the OpenID Server for example
					\item Parameters \texttt{authRequest} and \texttt{authInfo} are passed to the hook
				\end{itemize}

		\end{itemize}

\end{frame}

% ------------------------------------------------------------------------------
% LTXE-SLIDE-START
% LTXE-SLIDE-UID:		4bbf6ec5-298250ee-ce7ca1c2-850d9911
% LTXE-SLIDE-ORIGIN:	1626edc6-8d93ff84-f64178bf-8e7650df English
% LTXE-SLIDE-ORIGIN:	115a4459-3653bb99-1680d6ab-d8c3c69b German
% LTXE-SLIDE-TITLE:		Hook in BackendUserAuthentication::getDefaultUploadFolder (1)
% LTXE-SLIDE-REFERENCE:	Feature-68895-IntroducedHookInBackendUserAuthenticationgetDefaultUploadFolder.rst
% ------------------------------------------------------------------------------
\begin{frame}[fragile]
	\frametitle{Cambios en Profundidad}
	\framesubtitle{Hooks y Señales (1)}

	% decrease font size for code listing
	\lstset{basicstyle=\tiny\ttfamily}

	\begin{itemize}

		\item Ahora es posible cambiar la carpeta de subida devuelta por
			\texttt{BackendUserAuthentication::getDefaultUploadFolder()}

		\item Registre el hook en el fichero \texttt{ext\_localconf.php} como sigue:

			\begin{lstlisting}
				$GLOBALS['TYPO3_CONF_VARS']['SC_OPTIONS']['t3lib/class.t3lib_userauthgroup.php']
				  ['getDefaultUploadFolder'][] =
				  \Vendor\MyExtension\Hooks\DefaultUploadFolder::class . '->getDefaultUploadFolder';
			\end{lstlisting}

	\end{itemize}

\end{frame}

% ------------------------------------------------------------------------------
% LTXE-SLIDE-START
% LTXE-SLIDE-UID:		286a9299-7031fce8-da8411d0-3afaa639
% LTXE-SLIDE-ORIGIN:	9c150d50-ee3ba19b-120ef653-e714135c English
% LTXE-SLIDE-ORIGIN:	d1041967-32814ee3-2172e570-4a6d21bd German
% LTXE-SLIDE-TITLE:		Hook in BackendUserAuthentication::getDefaultUploadFolder (2)
% LTXE-SLIDE-REFERENCE:	Feature-68895-IntroducedHookInBackendUserAuthenticationgetDefaultUploadFolder.rst
% ------------------------------------------------------------------------------
\begin{frame}[fragile]
	\frametitle{Cambios en Profundidad}
	\framesubtitle{Hooks y Señales (2)}

	% decrease font size for code listing
	\lstset{basicstyle=\tiny\ttfamily}

	\small Ejemplo:\normalsize

		\begin{lstlisting}
			<?php
			namespace Vendor\MyExtension\Hooks;
			use TYPO3\CMS\Core\Authentication\BackendUserAuthentication;
			use TYPO3\CMS\Core\Resource\Folder;

			/**
			 * Class DefaultUploadFolder
			 */
			class DefaultUploadFolder {

			  /**
			   * Get default upload folder
			   * If there is a folder present with the same name as the last part of the table name use that folder.
			   * @param array $params
			   * @param BackendUserAuthentication $backendUserAuthentication
			   * @return Folder
			   */
			   public function getDefaultUploadFolder($params, BackendUserAuthentication $backendUserAuthentication) {
			    [...]
		\end{lstlisting}

\end{frame}

% ------------------------------------------------------------------------------
% LTXE-SLIDE-START
% LTXE-SLIDE-UID:		b4c5419d-298b7149-cd49f75e-8086dc16
% LTXE-SLIDE-ORIGIN:	a4d7bec0-6c19d8b2-6a6be951-5d5aa0c0 English
% LTXE-SLIDE-ORIGIN:	2f855107-2262065c-c865f953-91b4f0ea German
% LTXE-SLIDE-TITLE:		Hook in BackendUserAuthentication::getDefaultUploadFolder (3)
% LTXE-SLIDE-REFERENCE:	Feature-68895-IntroducedHookInBackendUserAuthenticationgetDefaultUploadFolder.rst
% ------------------------------------------------------------------------------
\begin{frame}[fragile]
	\frametitle{Cambios en Profundidad}
	\framesubtitle{Hooks y Señales (3)}

	% decrease font size for code listing
	\lstset{basicstyle=\tiny\ttfamily}

	\small Ejemplo (continuación):\normalsize

		\begin{lstlisting}
			    [...]

			    /** @var Folder $uploadFolder */
			    $uploadFolder = $params['uploadFolder'];
			    $pid = $params['pid'];
			    $table = $params['table'];
			    $field = $params['field'];

			    $matches = [];
			    if (!empty($uploadFolder) && preg_match('/_([a-z]+)$/', $table, $matches)) {
			      $folderName = $matches[1];
			      if ($uploadFolder->hasFolder($folderName)) {
			        $uploadFolder = $uploadFolder->getSubfolder($folderName);
			      }
			    }
			    return $uploadFolder;
			  }
			}
		\end{lstlisting}

\end{frame}

% ------------------------------------------------------------------------------
% LTXE-SLIDE-START
% LTXE-SLIDE-UID:		78ebd19e-d960bc65-5453d934-6ce36719
% LTXE-SLIDE-ORIGIN:	2c656d9e-38b9be7a-d590cb6e-21eccd4d English
% LTXE-SLIDE-ORIGIN:	cd6ba83d-9f068dee-b8887ffa-4a2eff29 German
% LTXE-SLIDE-TITLE:		Diverse Änderungen
% LTXE-SLIDE-REFERENCE:	Deprecation-69822-DeprecateSelectFieldTca.rst
% ------------------------------------------------------------------------------
\begin{frame}[fragile]
	\frametitle{Cambios en Profundidad}
	\framesubtitle{Miscelánea}

	\begin{itemize}

		\item Usar el tipo de campo TCA \texttt{select} requiere especificar una opción \texttt{renderType}

		\item Valores válidos son:

			\begin{lstlisting}
				'renderType' => 'selectMultipleSideBySide',
				'renderType' => 'selectCheckBox',
				'renderType' => 'selectSingle',
				'renderType' => 'selectSingleBox',
				'renderType' => 'selectTree',
			\end{lstlisting}

	\end{itemize}

\end{frame}

% ------------------------------------------------------------------------------
