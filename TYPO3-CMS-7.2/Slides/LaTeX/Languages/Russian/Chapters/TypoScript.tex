% ------------------------------------------------------------------------------
% TYPO3 CMS 7.2 - What's New - Chapter "TypoScript" (English Version)
%
% @author	Michael Schams <schams.net>
% @license	Creative Commons BY-NC-SA 3.0
% @link		http://typo3.org/download/release-notes/whats-new/
% @language	English
% ------------------------------------------------------------------------------
% LTXE-CHAPTER-UID:		12518a77-2b90a173-4e3f8420-485e0497
% LTXE-CHAPTER-NAME:	TypoScript
% ------------------------------------------------------------------------------

\section{TSconfig и TypoScript}
\begin{frame}[fragile]
	\frametitle{TSconfig и TypoScript}

	\begin{center}\huge{Глава 2:}\end{center}
	\begin{center}\huge{\color{typo3darkgrey}\textbf{TSconfig и TypoScript}}\end{center}

\end{frame}

% ------------------------------------------------------------------------------
% LTXE-SLIDE-START
% LTXE-SLIDE-UID:		0175de40-eb71eeb5-04a28bc0-a849fbb6
% LTXE-SLIDE-ORIGIN:	56b025e6-cf380b11-c9d1c009-0b548d6a English
% LTXE-SLIDE-TITLE:		Add flexible preview URL configuration (1)
% LTXE-SLIDE-REFERENCE:	Feature-66370-AddFlexiblePreviewUrlConfiguration.rst
% ------------------------------------------------------------------------------
\begin{frame}[fragile]
	\frametitle{TSconfig и TypoScript}
	\framesubtitle{Гибкая настройка просмотра URL (1)}

	% decrease font size for code listing
	\lstset{basicstyle=\tiny\ttfamily}

	\begin{itemize}

		\item Теперь возможно настроить просмотр ссылок, формируемых для кнопки\newline
			"сохранить и просмотреть / save \& view" внутреннего интерфейса.

		\item Разрабатывалось, как возможность предпросмотра для записей новостей или блога,
		 но можно определить разные страницы для предпросмотра обычных элементов содержимого.

			\begin{lstlisting}
				TCEMAIN.preview {
				  <table name> {
				    previewPageId = 123
				    useDefaultLanguageRecord = 0
				    fieldToParameterMap {
				      uid = tx_myext_pi1[showUid]
				    }
				    additionalGetParameters {
				      tx_myext_pi1[special] = HELLO
				    }
				  }
				}
			\end{lstlisting}

	\end{itemize}

\end{frame}

% ------------------------------------------------------------------------------
% LTXE-SLIDE-START
% LTXE-SLIDE-UID:		c8d9a8d1-0aea9eec-d3407d0e-9ad1604e
% LTXE-SLIDE-ORIGIN:	6822c376-37649cd8-0acc836a-b856fc61 English
% LTXE-SLIDE-TITLE:		Add flexible preview URL configuration (2)
% LTXE-SLIDE-REFERENCE:	Feature-66370-AddFlexiblePreviewUrlConfiguration.rst
% ------------------------------------------------------------------------------
\begin{frame}[fragile]
	\frametitle{TSconfig и TypoScript}
	\framesubtitle{Гибкая настройка просмотра URL (2)}

	\begin{itemize}
		\item \texttt{previewPageId}:\newline
			\smaller
				UID страницы, используемой для предпросмотра\newline
				(если не указано, используется текущая страница)
			\normalsize
		\item \texttt{useDefaultLanguageRecord}:\newline
			\smaller
				указывает, что запись перевода будет использовать UID записи по умолчанию\newline
				(по умолчанию задействовано, значение: 1)
			\normalsize
		\item \texttt{fieldToParameterMap}:\newline
			\smaller
				разметка, позволяющая выбрать поля записи, включаемые в виде параметров GET
			\normalsize
		\item \texttt{additionalGetParameters}:\newline
			\smaller
				позволяет добавить произвольные параметры GET и даже их переназначить
			\normalsize
	\end{itemize}

\end{frame}

% ------------------------------------------------------------------------------
% LTXE-SLIDE-START
% LTXE-SLIDE-UID:		854573ac-ed1a16bc-b7b345e4-c9c22a55
% LTXE-SLIDE-ORIGIN:	c5dbf3c7-655fc6e4-227a3bf6-f5d89fb1 English
% LTXE-SLIDE-TITLE:
% LTXE-SLIDE-REFERENCE:	Feature-59646-AddRteConfigurationPropertyButtonsLinkTypePropertiesTargetDefault.rst
% ------------------------------------------------------------------------------
\begin{frame}[fragile]
	\frametitle{TSconfig и TypoScript}
	\framesubtitle{Настройка RTE: цель по умолчанию}

	\begin{itemize}

		\item Свойство настроек RTE для PageTSconfig позволяет настроить цель по умолчанию для
		ссылок различного типа\newline

			\small
				\texttt{buttons.link.[}
				\textit{type}
				\texttt{].properties.target.default = ...}
			\normalsize\newline

		\item Возможные типы ссылок:\newline
			\small
				(другие могут быть добавлены через расширения)
			\normalsize

			\begin{itemize}
				\item \texttt{page}
				\item \texttt{file}
				\item \texttt{url}
				\item \texttt{mail}
				\item \texttt{spec}
			\end{itemize}
	\end{itemize}

\end{frame}

% ------------------------------------------------------------------------------
% LTXE-SLIDE-START
% LTXE-SLIDE-UID:		2a7ee256-93ecd0e3-eef8e0d8-6f38fae2
% LTXE-SLIDE-ORIGIN:	fd957580-301e4a0a-ad5325f5-ffa024c3 English
% LTXE-SLIDE-TITLE:		Strip empty HTML tags in HtmlParser
% LTXE-SLIDE-REFERENCE:	Feature-20555-StripEmptyHtmlTags.rst
% ------------------------------------------------------------------------------
\begin{frame}[fragile]
	\frametitle{TSconfig и TypoScript}
	\framesubtitle{Удаление пустых тегов HTML в HTMLparser}

	% decrease font size for code listing
	\lstset{basicstyle=\tiny\ttfamily}

	\begin{itemize}
		\item В HTMLparser применяется новый функционал, позволяющий удалять пустые
		  теги HTML

			\begin{lstlisting}
				stdWrap {
				   // this removes all empty HTML tags
				   HTMLparser.stripEmptyTags = 1
				   // this removes empty h2 and h3 tags only
				   HTMLparser.stripEmptyTags.tags = h2, h3
				}

				RTE.default.proc.entryHTMLparser_db {
				   stripEmptyTags = 1
				   stripEmptyTags.tags = p
				   stripEmptyTags.treatNonBreakingSpaceAsEmpty = 1
				}
			\end{lstlisting}

			\underline{\textbf{Note:}}
				HTMLparser по умолчанию удаляет все известные теги.\newline
				Поэтому иногда полезно будет оставять все неизвестные:\newline
				\texttt{HTMLparser.keepNonMatchedTags = 1}

	\end{itemize}

\end{frame}

% ------------------------------------------------------------------------------
% LTXE-SLIDE-START
% LTXE-SLIDE-UID:		216188b6-184f346a-f4f8af17-5895573a
% LTXE-SLIDE-ORIGIN:	62960e76-c3dca953-fe5a1070-831633fc English
% LTXE-SLIDE-TITLE:		Miscellaneous
% LTXE-SLIDE-REFERENCE:	Feature-63040-AddRteConfigurationPropertyButtonsAbbreviationRemoveFieldsets.rst
% LTXE-SLIDE-REFERENCE:	commit 31c62f9311ee3d33bf792d548cc4a5fac83aa3d0
% ------------------------------------------------------------------------------
\begin{frame}[fragile]
	\frametitle{TSconfig и TypoScript}
	\framesubtitle{Другое}

	\begin{itemize}
		\item Для настройки диалога аббревиатур в PageTSconfig можно воспользоваться новым свойством
		 \texttt{buttons.abbreviation.removeFieldsets}

			\begin{lstlisting}
				# Possible values are:
				# acronym, definedAcronym, abbreviation, definedAbbreviation
				buttons.abbreviation.removeFieldsets = acronym,definedAcronym
			\end{lstlisting}

		\item Свойство \texttt{inlineLanguageLabel} объекта \texttt{PAGE} теперь может обрабатывать\newline
			\texttt{LLL:} ссылки

	\end{itemize}

\end{frame}

% ------------------------------------------------------------------------------
