% ------------------------------------------------------------------------------
% TYPO3 CMS 7.2 - What's New - Chapter "In-Depth Changes" (English Version)
%
% @author	Michael Schams <schams.net>
% @license	Creative Commons BY-NC-SA 3.0
% @link		http://typo3.org/download/release-notes/whats-new/
% @language	English
% ------------------------------------------------------------------------------
% LTXE-CHAPTER-UID:		e28087f2-4bffcc7b-3c685cca-579f18a8
% LTXE-CHAPTER-NAME:	In-Depth Changes
% ------------------------------------------------------------------------------

\section{Глубинные изменения}
\begin{frame}[fragile]
	\frametitle{Глубинные изменения}

	\begin{center}\huge{Глава 3:}\end{center}
	\begin{center}\huge{\color{typo3darkgrey}\textbf{Глубинные изменения}}\end{center}

\end{frame}

% ------------------------------------------------------------------------------
% LTXE-SLIDE-START
% LTXE-SLIDE-UID:		ac2623f5-f1628164-ab6e8f2b-23285906
% LTXE-SLIDE-ORIGIN:	547708aa-92282fd1-6c27d618-4f13f86b English
% LTXE-SLIDE-TITLE:		Add SVG support
% LTXE-SLIDE-REFERENCE:	Feature-50136-AddSVGSupport.rst
% ------------------------------------------------------------------------------
\begin{frame}[fragile]
	\frametitle{Глубинные изменения}
	\framesubtitle{Поддержка SVG в ядре}

	\begin{itemize}
		\item Ядро TYPO3 CMS теперь поддерживает изображения SVG ("Scalable Vector Graphics")

		\item При масштабировании изображения SVG, вместо создания изменённого файла,
		запись с вычисленными новыми размерами сохраняется в \texttt{sys\_file\_processedfile}\newline
			\small(если только изображение не обрабатывается далее, например, обрезается)\normalsize.

		\item Добавлена возможность указания размеров SVG для случая, когда ImageMagick/GraphicsMagick
			не может определить их самостоятельно. При этом читается содержимое файла XML.

		\item SVG также был добавлен к списку расширений файлов для изображений:\newline
			\texttt{\$GLOBALS['TYPO3\_CONF\_VARS']['GFX']['imagefile\_ext']}

	\end{itemize}

\end{frame}

% ------------------------------------------------------------------------------
% LTXE-SLIDE-START
% LTXE-SLIDE-UID:		e1eadefd-8ca79ef5-af33ea9a-25ad7fd7
% LTXE-SLIDE-ORIGIN:	bd03e141-3842b75e-8a44920f-a8c139e5 English
% LTXE-SLIDE-TITLE:		Add count methods and sort functionality to FAL drivers
% LTXE-SLIDE-REFERENCE:	Breaking-56746-AddCountMethodsAndSortFunctionalityToFalDrivers.rst
% ------------------------------------------------------------------------------
\begin{frame}[fragile]
	\frametitle{Глубинные изменения}
	\framesubtitle{Расширенный драйвер FAL}

	% decrease font size for code listing
	\lstset{basicstyle=\tiny\ttfamily}

	\begin{itemize}
		\item Для исправления производительности при выводе списка (удалённых) хранилищ, драйвер
		FAL должен позаботиться об упорядочивании, сортировке и подсчёте файлов/папок. Были
		добавлены два новых параметра \texttt{sort} и \texttt{sortRev} позволяющих:

			\begin{lstlisting}
				public function getFilesInFolder($folderIdentifier, $start = 0, $numberOfItems = 0,
				  $recursive = FALSE, array $filenameFilterCallbacks = array(), $sort = '', $sortRev = FALSE);

				public function getFoldersInFolder($folderIdentifier, $start = 0, $numberOfItems = 0,
				  $recursive = FALSE, array $folderNameFilterCallbacks = array(), $sort = '', $sortRev = FALSE);
			\end{lstlisting}

		\item Были также реализованы два новых метода:

			\begin{lstlisting}
				public function getFilesInFolderCount($folderIdentifier, $recursive = FALSE,
				  array $filenameFilterCallbacks = array());

				public function getFoldersInFolderCount($folderIdentifier, $recursive = FALSE,
				  array $folderNameFilterCallbacks = array());
			\end{lstlisting}

	\end{itemize}

\end{frame}

% ------------------------------------------------------------------------------
% LTXE-SLIDE-START
% LTXE-SLIDE-UID:		ce266cf1-f8f7be3a-ecdb0573-00b133b2
% LTXE-SLIDE-ORIGIN:	d0df6060-4b8d33a9-69adad59-422e9ada English
% LTXE-SLIDE-TITLE:		Introduce Backend Routing (1)
% LTXE-SLIDE-REFERENCE:	commit a08ce7238e583d1962edfe998e04c7d9c3d7c2ae
% ------------------------------------------------------------------------------
\begin{frame}[fragile]
	\frametitle{Глубинные изменения}
	\framesubtitle{Backend Routing API (1)}

	\begin{itemize}
		\item Была создана Backend Routing API, управляющая точками внутренн точками входа / Entry Points
		\item В основе взята Symfony Routing Framework, и этот API в значительной степени совместима с ним\newline
			\small(хотя сейчас TYPO3 использует примерно её 20\%)\normalsize

		\item Основные функции реализуют три класса:
			\begin{itemize}
				\item \texttt{class Route}:\tabto{3.6cm}содержит данные о путях и параметрах
				\item \texttt{class Router}:\tabto{3.6cm}API для сопоставления маршрута
				\item \texttt{class UrlGenerator}:\tabto{3.6cm}формирует URL
			\end{itemize}
	\end{itemize}

\end{frame}

% ------------------------------------------------------------------------------
% LTXE-SLIDE-START
% LTXE-SLIDE-UID:		1e8f6f1b-2daba4fa-018b2f6e-90a51ba7
% LTXE-SLIDE-ORIGIN:	fe1eec67-bbf31ed9-57b8e83a-3a922f0e English
% LTXE-SLIDE-TITLE:		Introduce Backend Routing (2)
% LTXE-SLIDE-REFERENCE:	commit a08ce7238e583d1962edfe998e04c7d9c3d7c2ae
% ------------------------------------------------------------------------------
\begin{frame}[fragile]
	\frametitle{Глубинные изменения}
	\framesubtitle{Backend Routing API (2)}

	\begin{itemize}

		\item Маршруты определяются в следующем файле расширения:
			\texttt{Configuration/Backend/Routes.php}\newline
			(в качестве примера смотрите системное расширение \texttt{backend})

		\item Дополнительные сведения об Backend Routing API:\newline
			\small\url{http://wiki.typo3.org/Blueprints/BackendRouting}\normalsize

	\end{itemize}

\end{frame}

% ------------------------------------------------------------------------------
% LTXE-SLIDE-START
% LTXE-SLIDE-UID:		fefc5aea-0b4dfbef-b5cc1132-ea34a660
% LTXE-SLIDE-ORIGIN:	9d56ca19-04a98896-ba0a973c-8a16e877 English
% LTXE-SLIDE-TITLE:		New system extension: mediace
% LTXE-SLIDE-REFERENCE:	Breaking-64719-MediaContentMovedToSystemExtension.rst
% LTXE-SLIDE-REFERENCE:	Breaking-65778-MediaWizardProviderMovedToSystemExtension.rst
% ------------------------------------------------------------------------------
\begin{frame}[fragile]
	\frametitle{Глубинные изменения}
	\framesubtitle{Новое системное расширение для медиа элементов содержимого}

	\begin{itemize}

		\item Новое системное расширение "\texttt{mediace}" содержит следующие cObjects:

			\begin{itemize}
				\item \texttt{MULTIMEDIA}
				\item \texttt{MEDIA}
				\item \texttt{SWFOBJECT}
				\item \texttt{FLOWPLAYER}
				\item \texttt{QTOBJECT}
			\end{itemize}

		\item Элементы содержимого \texttt{media} и \texttt{multimedia} также были перенесены в
			это системное расширение наряду с "Media Wizard Provider"

		\item Это расширение по умолчанию \underline{\textbf{не}} устанавливается!

	\end{itemize}

\end{frame}

% ------------------------------------------------------------------------------
% LTXE-SLIDE-START
% LTXE-SLIDE-UID:		3833af5e-b2959c1d-f450fdfc-63ffe511
% LTXE-SLIDE-ORIGIN:	cc63cd98-524ee6df-38430963-f7c7067c English
% LTXE-SLIDE-TITLE:		Composer vendor directory
% LTXE-SLIDE-REFERENCE:	Breaking-66001-ComposerVendorDirectoryChanged.rst
% ------------------------------------------------------------------------------
\begin{frame}[fragile]
	\frametitle{Глубинные изменения}
	\framesubtitle{Местоположение сторонних библиотек}

	\begin{itemize}

		\item Теперь устанавливаемые через Composer сторонние библиотеки располагаются под \texttt{typo3/contrib/vendor}\newline
			\small
				(TYPO3 CMS < 7.2: в папке \texttt{Packages/Libraries})
			\normalsize

		\item Таким образом, процесс упаковки для выпуска TYPO3 CMS в виде tarball или zip можно переключить на
			полностью рабочую установку, без необходимости загрузки \texttt{пакетов/} для сторонних библиотек

		\item При установке могут произойти проблемы, если настройки делались через composer с использованием \texttt{phpunit},
			несмотря на то, что зависимости были полностью исправлены. Для исправления выполните:

			\begin{lstlisting}
				# cd htdocs/
				# rm -rf typo3/contrib/vendor/ bin/ Packages/Libraries/ composer.lock
				# composer install
			\end{lstlisting}
	\end{itemize}

\end{frame}

% ------------------------------------------------------------------------------
% LTXE-SLIDE-START
% LTXE-SLIDE-UID:		62f787e7-84062247-a984df6a-f233081d
% LTXE-SLIDE-ORIGIN:	c9a5288c-f45aac1e-34606612-fdf6ec18 English
% LTXE-SLIDE-TITLE:		Introduce JavaScript notification API
% LTXE-SLIDE-REFERENCE:	Feature-66047-IntroduceJavascriptNotificationApi.rst
% ------------------------------------------------------------------------------
\begin{frame}[fragile]
	\frametitle{Глубинные изменения}
	\framesubtitle{Уведомления JavaScript}

	\begin{itemize}

		\item Используется новый API JavaScript уведомлений:
			\begin{lstlisting}
				// old and deprecated:
				top.TYPO3.Flashmessages.display(TYPO3.Severity.notice)

				// new and the only correct way since TYPO3 CMS 7.2:
				top.TYPO3.Notification.notice(title, message)
    		\end{lstlisting}

		\item Имеются следующие функции API:\newline
			\small(параметр \texttt{duration} необязателен, значение по умолчанию 5 секунд)\normalsize
			\begin{itemize}
				\item \normalsize\smaller\texttt{top.TYPO3.Notification.notice(title, message, duration)}\normalsize
				\item \smaller\texttt{top.TYPO3.Notification.info(title, message, duration)}\normalsize
				\item \smaller\texttt{top.TYPO3.Notification.success(title, message, duration)}\normalsize
				\item \smaller\texttt{top.TYPO3.Notification.warning(title, message, duration)}\normalsize
				\item \smaller\texttt{top.TYPO3.Notification.error(title, message, duration)}\normalsize
			\end{itemize}

	\end{itemize}

\end{frame}

% ------------------------------------------------------------------------------
% LTXE-SLIDE-START
% LTXE-SLIDE-UID:		34ddbbc1-075ecf5f-bd5b4e3e-dc7cac3f
% LTXE-SLIDE-ORIGIN:	b2acb5df-60b9a7dd-35cf697d-2a50d5db English
% LTXE-SLIDE-TITLE:		System Information Dropdown (1)
% LTXE-SLIDE-REFERENCE:	Feature-65767-SystemInformationDropdown.rst
% ------------------------------------------------------------------------------
\begin{frame}[fragile]
	\frametitle{Глубинные изменения}
	\framesubtitle{Выпадающий список системной информации (1)}

	% decrease font size for code listing
	\lstset{basicstyle=\tiny\ttfamily}

	\begin{itemize}
		\item Возможно добавить элементы системной информации к выпадающему списку, путём создания слота

		\item Слот необходимо зарегистрировать в файле \texttt{ext\_localconf.php}:

			\begin{lstlisting}
				$signalSlotDispatcher = \TYPO3\CMS\Core\Utility\GeneralUtility::makeInstance(
				  \TYPO3\CMS\Extbase\SignalSlot\Dispatcher::class);

				$signalSlotDispatcher->connect(
				  \TYPO3\CMS\Backend\Backend\ToolbarItems\SystemInformationToolbarItem::class,
				  'getSystemInformation',
				  \Vendor\Extension\SystemInformation\Item::class,
				  'getItem'
				);
			\end{lstlisting}

	\end{itemize}

\end{frame}

% ------------------------------------------------------------------------------
% LTXE-SLIDE-START
% LTXE-SLIDE-UID:		4f61c221-dcd18dbc-3f0ce7c6-d4051be1
% LTXE-SLIDE-ORIGIN:	e0f13209-22e0ff8c-7ff23bf9-31b1bb52 English
% LTXE-SLIDE-TITLE:		System Information Dropdown (2)
% LTXE-SLIDE-REFERENCE:	Feature-65767-SystemInformationDropdown.rst
% ------------------------------------------------------------------------------
\begin{frame}[fragile]
	\frametitle{Глубинные изменения}
	\framesubtitle{Выпадающий список системной информации (2)}

	% decrease font size for code listing
	\lstset{basicstyle=\tiny\ttfamily}

	\begin{itemize}

		\item Возможно добавить элементы системной информации к выпадающему списку, путём создания слота

		\item Необходим класс \texttt{Item} с методами \texttt{getItem()} в файле
			\small
				\texttt{EXT:extension\textbackslash
					Classes\textbackslash
					SystemInformation\textbackslash
					Item.php}:
			\normalsize

		\begin{lstlisting}
			class Item {
			  public function getItem() {
			    return array(array(
			      'title' => 'The title shown on hover',
			      'value' => 'Description shown in the list',
			      'status' => SystemInformationHookInterface::STATUS_OK,
			      'count' => 4,
			      'icon' => \TYPO3\CMS\Backend\Utility\IconUtility::getSpriteIcon(
				    'extensions-example-information-icon')
			    ));
			  }
			}
		\end{lstlisting}

	\end{itemize}

\end{frame}

% ------------------------------------------------------------------------------
% LTXE-SLIDE-START
% LTXE-SLIDE-UID:		161fbec1-53f4758c-cb928469-7ee5a2a4
% LTXE-SLIDE-ORIGIN:	7d27df49-260500ef-0fc78a89-6e80069b English
% LTXE-SLIDE-TITLE:		System Information Dropdown (3)
% LTXE-SLIDE-REFERENCE:	Feature-65767-SystemInformationDropdown.rst
% ------------------------------------------------------------------------------
\begin{frame}[fragile]
	\frametitle{Глубинные изменения}
	\framesubtitle{Выпадающий список системной информации (3)}

	% decrease font size for code listing
	\lstset{basicstyle=\tiny\ttfamily}

	\begin{itemize}
		\item Значок \texttt{extensions-example-information-icon} должен быть зарегистрирован в \texttt{ext\_localconf.php}:
		\begin{lstlisting}
			\TYPO3\CMS\Backend\Sprite\SpriteManager::addSingleIcons(
			  array(
			    'information-icon' => \TYPO3\CMS\Core\Utility\ExtensionManagementUtility::extRelPath(
			      $_EXTKEY) . 'Resources/Public/Images/Icons/information-icon.png'
			    ),
			   $_EXTKEY
			);
		\end{lstlisting}

	\end{itemize}

\end{frame}


% ------------------------------------------------------------------------------
% LTXE-SLIDE-START
% LTXE-SLIDE-UID:		8b7a552e-2dbc8fbd-d489d26f-14443500
% LTXE-SLIDE-ORIGIN:	d7449c27-fd9676cf-d12be80b-c24a4536 English
% LTXE-SLIDE-TITLE:		System Information Dropdown (4)
% LTXE-SLIDE-REFERENCE:	Feature-65767-SystemInformationDropdown.rst
% ------------------------------------------------------------------------------
\begin{frame}[fragile]
	\frametitle{Глубинные изменения}
	\framesubtitle{Выпадающий список системной информации (4)}

	% decrease font size for code listing
	\lstset{basicstyle=\tiny\ttfamily}

	\begin{itemize}

		\item Сообщения выводятся в нижней части выпадающего списка

		\item Расширения могут добавлять свои слоты с информацией:

		\begin{lstlisting}
			$signalSlotDispatcher = \TYPO3\CMS\Core\Utility\GeneralUtility::makeInstance(
			  \TYPO3\CMS\Extbase\SignalSlot\Dispatcher::class);

			$signalSlotDispatcher->connect(
			  \TYPO3\CMS\Backend\Backend\ToolbarItems\SystemInformationToolbarItem::class,
			  'loadMessages',
			  \Vendor\Extension\SystemInformation\Message::class,
			  'getMessage'
			);
		\end{lstlisting}

	\end{itemize}

\end{frame}

% ------------------------------------------------------------------------------
% LTXE-SLIDE-START
% LTXE-SLIDE-UID:		c0c1c94f-f883bf9f-4006ca5f-5bfd55e2
% LTXE-SLIDE-ORIGIN:	ab3b46ce-c7d7a228-7f7888af-4312fcdd English
% LTXE-SLIDE-TITLE:		System Information Dropdown (5)
% LTXE-SLIDE-REFERENCE:	Feature-65767-SystemInformationDropdown.rst
% ------------------------------------------------------------------------------
\begin{frame}[fragile]
	\frametitle{Глубинные изменения}
	\framesubtitle{Выпадающий список системной информации (5)}

	% decrease font size for code listing
	\lstset{basicstyle=\tiny\ttfamily}

	\begin{itemize}

		\item Сообщения выводятся в нижней части выпадающего списка

		\item Необходим класс \texttt{Message} с методом \texttt{getMessage()} в файле
			\small
				\texttt{EXT:extension\textbackslash
					Classes\textbackslash
					SystemInformation\textbackslash
					Message.php}:
			\normalsize

		\begin{lstlisting}
			class Message {
			  public function getMessage() {
			    return array(array(
			      'status' => SystemInformationHookInterface::STATUS_OK,
			      'text' => 'Something went wrong. Take a look at the reports module.'
			    ));
			  }
			}
		\end{lstlisting}

	\end{itemize}

\end{frame}


% ------------------------------------------------------------------------------
% LTXE-SLIDE-START
% LTXE-SLIDE-UID:		66ffd4fb-863a6072-48e1c381-b6fd383a
% LTXE-SLIDE-ORIGIN:	98e2056a-d165b638-b1a1ee08-206bc15a English
% LTXE-SLIDE-TITLE:		Add image cropping
% LTXE-SLIDE-REFERENCE:	Feature-65584-AddImageCropping.rst
% ------------------------------------------------------------------------------
\begin{frame}[fragile]
	\frametitle{Глубинные изменения}
	\framesubtitle{Параметры настойки для изменения изображений (1)}

	\begin{itemize}
		\item Доступны следующие параметры настройки TypoScript:
			\begin{lstlisting}
				# disable cropping for all images
				tt_content.image.20.1.file.crop =

				# override or set cropping for all images
				# offsetX,offsetY,width,height
				tt_content.image.20.1.file.crop = 50,50,100,100
			\end{lstlisting}

		\item Fluid также поддерживает функции обрезки:
			\begin{lstlisting}
				# disable cropping for all images
				<f:image image="{imageObject}" crop="" ></f:image>

				# override or set cropping for all images
				# offsetX,offsetY,width,height
				<f:image image="{imageObject}" crop="50,50,100,100" ></f:image>
			\end{lstlisting}

	\end{itemize}

\end{frame}

% ------------------------------------------------------------------------------
% LTXE-SLIDE-START
% LTXE-SLIDE-UID:		333ebdb0-4668641e-1c2fffd6-76577f05
% LTXE-SLIDE-ORIGIN:	e9cce085-6c300f1f-010de4ee-2c744704 English
% LTXE-SLIDE-TITLE:		Add image cropping (2)
% LTXE-SLIDE-REFERENCE:	Feature-65584-AddImageCropping.rst
% ------------------------------------------------------------------------------
\begin{frame}[fragile]
	\frametitle{Глубинные изменения}
	\framesubtitle{Параметры настойки для изменения изображений (2)}

	% decrease font size for code listing
	\lstset{basicstyle=\smaller\ttfamily}

	\begin{itemize}
		\item TCA также поддерживает функции обрезки изображений:

			\begin{itemize}
				\item Column Type: \texttt{image\_manipulation}
				\item Config \texttt{file\_field}: string	\tabto{5.6cm}(default: \texttt{uid\_local})
				\item Config \texttt{enableZoom}: boolean	\tabto{5.6cm}(default: \texttt{FALSE})
				\item Config \texttt{allowedExtensions}: string\newline
					(default: \smaller\texttt{\$GLOBALS['TYPO3\_CONF\_VARS']['GFX']['imagefile\_ext']}\small)
				\item Config \texttt{ratios}: array, default:

					\begin{lstlisting}
						array(
						  '1.7777777777777777' => '16:9',
						  '1.3333333333333333' => '4:3',
						  '1' => '1:1',
						  'NaN' => 'Free'
						)
					\end{lstlisting}
			\end{itemize}

	\end{itemize}

\end{frame}

% ------------------------------------------------------------------------------
% LTXE-SLIDE-START
% LTXE-SLIDE-UID:		7753aaee-bfa87061-533a0f38-dfa60d8a
% LTXE-SLIDE-ORIGIN:	ed863269-574e4061-d69f855e-6494c062 English
% LTXE-SLIDE-TITLE:		Additional params for HTMLparser userFunc
% LTXE-SLIDE-REFERENCE:	Feature-59712-HtmlParserAdditionalUserFuncParams.rst
% ------------------------------------------------------------------------------
\begin{frame}[fragile]
	\frametitle{Глубинные изменения}
	\framesubtitle{Дополнительные парамерты для HTMLparser userFunc}

	% decrease font size for code listing
	\lstset{basicstyle=\tiny\ttfamily}

	\begin{itemize}

		\item В userFunc для HTMLparser могут быть переданы дополнительные параметры:

			\begin{lstlisting}
				myobj = TEXT
				myobj.value = <a href="/" class="myclass">MyText</a>
				myobj.HTMLparser.tags.a.fixAttrib.class {
				  userFunc = Tx\MyExt\Myclass->htmlUserFunc
				  userFunc.myparam = test
				}
			\end{lstlisting}

		\item Доступ к этим параметрам в расширениях происходит так:

			\begin{lstlisting}
				function htmlUserFunc(array $params, HtmlParser $htmlParser) {
				  // $params['attributeValue'] contains the attribute value "myclass"
				  // $params['myparam'] is set to "test" in this example
				  ...
				}
			\end{lstlisting}

	\end{itemize}

\end{frame}

% ------------------------------------------------------------------------------
% LTXE-SLIDE-START
% LTXE-SLIDE-UID:		f2cc132c-81967ad2-8a5278f7-50211b69
% LTXE-SLIDE-ORIGIN:	b04f8d44-fd695c73-b768f3cb-fd6d8643 English
% LTXE-SLIDE-TITLE:		New Locking API (1)
% LTXE-SLIDE-REFERENCE:	Feature-47712-NewLockingAPI.rst
% ------------------------------------------------------------------------------
\begin{frame}[fragile]
	\frametitle{Глубинные изменения}
	\framesubtitle{Locking API (1)}

	\begin{itemize}

		\item Был представлен новый API блокировки, предоставляющий разные методы блокировки (SimpleFile, Semaphore, ...)
		\item Метод блокировки должен реализовывать \small\texttt{LockingStrategyInterface}\normalsize:
		\begin{lstlisting}
			$lockFactory = GeneralUtility::makeInstance(LockFactory::class);
			$locker = $lockFactory->createLocker('someId');
			$locker->acquire() || die('Could not acquire lock.');
			...
			$locker->release();
		\end{lstlisting}

	\end{itemize}

\end{frame}

% ------------------------------------------------------------------------------
% LTXE-SLIDE-START
% LTXE-SLIDE-UID:		fc08ba6c-5782ea31-e3aa4f15-62e2ebdc
% LTXE-SLIDE-ORIGIN:	79a00412-c72401e6-4d9667c9-0e15d3f4 English
% LTXE-SLIDE-TITLE:		New Locking API (2)
% LTXE-SLIDE-REFERENCE:	Feature-47712-NewLockingAPI.rst
% ------------------------------------------------------------------------------
\begin{frame}[fragile]
	\frametitle{Глубинные изменения}
	\framesubtitle{Locking API (2)}

	% decrease font size for code listing
	\lstset{basicstyle=\tiny\ttfamily}

	\begin{itemize}

		\item Некоторые методы также поддерживают не блокирующие остановки:

		\begin{lstlisting}
			$lockFactory = GeneralUtility::makeInstance(LockFactory::class);
			$locker = $lockFactory->createLocker(
			  'someId',
			  LockingStrategyInterface::LOCK_CAPABILITY_SHARED |
			    LockingStrategyInterface::LOCK_CAPABILITY_NOBLOCK
			);
			try {
			  $result = $locker->acquire(LockingStrategyInterface::LOCK_CAPABILITY_SHARED | LockingStrategyInterface::LOCK_CAPABILITY_NOBLOCK);
			  catch (\RuntimeException $e) {
			  if ($e->getCode() === 1428700748) {
			    // some process owns the lock
			    // let's do something else meanwhile
			    ...
			  }
			}
			if ($result) {
			  $locker->release();
			}
		\end{lstlisting}

	\end{itemize}

\end{frame}

% ------------------------------------------------------------------------------
% LTXE-SLIDE-START
% LTXE-SLIDE-UID:		63e40b7e-c4656c75-0a9eecfd-d4193b76
% LTXE-SLIDE-ORIGIN:	108cb4a7-6e006507-c25a91e7-3a867f09 English
% LTXE-SLIDE-TITLE:		Add signal after extension installation
% LTXE-SLIDE-REFERENCE:	commit 39e0fd0a2c919da270cb49223433bcf95febbb10
% ------------------------------------------------------------------------------
\begin{frame}[fragile]
	\frametitle{Глубинные изменения}
	\framesubtitle{Сигнал после установки расширения}

	% decrease font size for code listing
	\lstset{basicstyle=\tiny\ttfamily}

	\begin{itemize}

		\item Был реализован новый сигнал в методе
			\smaller
				\texttt{\textbackslash
					TYPO3\textbackslash
					CMS\textbackslash
					Extensionmanager\textbackslash
					Utility\textbackslash
					InstallUtility::install()}
			\normalsize
			возникающий после установки расширения и завершения всего импорта/обновления

		\begin{lstlisting}
			// execution
			$this->emitAfterExtensionInstallSignal($extensionKey);

			// methode
			protected function emitAfterExtensionInstallSignal($extensionKey) {
			  $this->signalSlotDispatcher->dispatch(
			    __CLASS__,
			    'afterExtensionInstall',
			    array($extensionKey, $this)
			  );
			}
		\end{lstlisting}

	\end{itemize}

\end{frame}

% ------------------------------------------------------------------------------
% LTXE-SLIDE-START
% LTXE-SLIDE-UID:		c7461feb-5b5103d0-9cbff7d7-30b1a920
% LTXE-SLIDE-ORIGIN:	acb1ba95-1a401bef-549072a1-f32263a9 English
% LTXE-SLIDE-TITLE:		Registry for adding text extractor classes (1)
% LTXE-SLIDE-REFERENCE:	Feature-36743-FAL-TextExtractorRegistry.rst
% ------------------------------------------------------------------------------
\begin{frame}[fragile]
	\frametitle{Глубинные изменения}
	\framesubtitle{Реестр для извлечения текста (1)}

	% decrease font size for code listing
	\lstset{basicstyle=\tiny\ttfamily}

	\begin{itemize}

		\item Возможна регистрация нескольких добытчиков текстов для работы с разными типами
			файлов (например, Office, PDF и т. п.)

		\item Ядро TYPO3 умеет извлекать тексты из обычных текстовых файлов

		\item Каждый регистрируемый класс добытчика должен реализовывать \texttt{TextExtractorInterface}

		\item ... со следующими методами:\newline
			\texttt{canExtractText()}\newline
			\small
				проверяет возможность извлечения текста из указанного файла
			\normalsize
			\newline
			\texttt{extractText()}\newline
			\small
				возвращает содержимое текста файла в виде строки
			\normalsize

	\end{itemize}

\end{frame}

% ------------------------------------------------------------------------------
% LTXE-SLIDE-START
% LTXE-SLIDE-UID:		3e961382-e37567cc-92e86dd2-f939409a
% LTXE-SLIDE-ORIGIN:	acb1ba95-1a401bef-549072a1-f32263a9 English
% LTXE-SLIDE-TITLE:		Registry for adding text extractor classes (2)
% LTXE-SLIDE-REFERENCE:	Feature-36743-FAL-TextExtractorRegistry.rst
% ------------------------------------------------------------------------------
\begin{frame}[fragile]
	\frametitle{Глубинные изменения}
	\framesubtitle{Реестр для извлечения текста (2)}

	% decrease font size for code listing
	\lstset{basicstyle=\tiny\ttfamily}

	\begin{itemize}

		\item Добытчик тектса регистрируется в файле \texttt{ext\_localconf.php}:

			\begin{lstlisting}
				$textExtractorRegistry = \TYPO3\CMS\Core\Resource\TextExtraction\TextExtractorRegistry::getInstance();
				$textExtractorRegistry->registerTextExtractor(
				  \TYPO3\CMS\Core\Resource\TextExtraction\PlainTextExtractor::class
				);
			\end{lstlisting}

		\item И используется следующим образом:

			\begin{lstlisting}
				$textExtractorRegistry = \TYPO3\CMS\Core\Resource\TextExtraction\TextExtractorRegistry::getInstance();
				$extractor = $textExtractorRegistry->getTextExtractor($file);
				if($extractor !== NULL) {
				  $content = $extractor->extractText($file);
				}
			\end{lstlisting}
	\end{itemize}

\end{frame}

% ------------------------------------------------------------------------------
% LTXE-SLIDE-START
% LTXE-SLIDE-UID:		fa687ed0-dff5a502-085e63ce-d0cdf8ac
% LTXE-SLIDE-ORIGIN:	d9657b5d-5cd56c65-d7266a27-401dd233 English
% LTXE-SLIDE-TITLE:		Miscellaneous
% LTXE-SLIDE-REFERENCE:	Feature-66042-WebLibrariesLoadedViaBower.rst
% LTXE-SLIDE-REFERENCE:	Feature-63703-AddOptionToStopTask.rst
% LTXE-SLIDE-REFERENCE:	Feature-61463-AllowProcessedFoldersInDifferentStorage.rst
% LTXE-SLIDE-REFERENCE:	Feature-52693-TSFE-RequestedId.rst
% ------------------------------------------------------------------------------

\begin{frame}[fragile]
	\frametitle{Глубинные изменения}
	\framesubtitle{Разное}

	\begin{itemize}

		\item Сетевые библиотеки (вроде Twitter Bootstrap, jQuery, Font Awesome и т. д.) используют
			"Bower" (\url{http://bower.io}) и более не являются частью Git репозитория ядра TYPO3\newline
			\small
				\texttt{\# bower install}	\tabto{3.4cm}запуск установки\newline
				\texttt{\# bower update}		\tabto{3.4cm}запуск обновления\newline
			\normalsize
			(file \texttt{bower.json} располагается в директории \texttt{Build/})

		\item Scheduler CLI имеет параметр "\texttt{-s}" для остановки работающей задачи

		\item Обрабатываемая папка (дистанционного) хранилища может находиться за пределами хранилища
			(полезно для, например, хранилища с правами лишь на чтение)

		\item Теперь возможно получить ID изначально запрашиваемой страницы:
			\texttt{\$TSFE->getRequestedId()}

	\end{itemize}

\end{frame}

% ------------------------------------------------------------------------------
