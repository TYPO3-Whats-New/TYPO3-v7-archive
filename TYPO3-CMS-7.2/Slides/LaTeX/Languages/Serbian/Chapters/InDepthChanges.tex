% ------------------------------------------------------------------------------
% TYPO3 CMS 7.2 - What's New - Chapter "In-Depth Changes" (English Version)
%
% @author	Michael Schams <schams.net>
% @license	Creative Commons BY-NC-SA 3.0
% @link		http://typo3.org/download/release-notes/whats-new/
% @language	English
% ------------------------------------------------------------------------------
% LTXE-CHAPTER-UID:		e28087f2-4bffcc7b-3c685cca-579f18a8
% LTXE-CHAPTER-NAME:	In-Depth Changes
% ------------------------------------------------------------------------------

\section{Korenite promene}
\begin{frame}[fragile]
	\frametitle{Korenite promene}

	\begin{center}\huge{Poglavlje 3:}\end{center}
	\begin{center}\huge{\color{typo3darkgrey}\textbf{Korenite promene}}\end{center}

\end{frame}

% ------------------------------------------------------------------------------
% LTXE-SLIDE-START
% LTXE-SLIDE-UID:		8b4938da-94d79196-2536596b-7e7e746c
% LTXE-SLIDE-ORIGIN:	547708aa-92282fd1-6c27d618-4f13f86b English
% LTXE-SLIDE-TITLE:		Add SVG support
% LTXE-SLIDE-REFERENCE:	Feature-50136-AddSVGSupport.rst
% ------------------------------------------------------------------------------
\begin{frame}[fragile]
	\frametitle{Korenite promene}
	\framesubtitle{SVG podrska u osnovi (Core)}

	\begin{itemize}
		\item TYPO3 CMS osnova sada podrzava SVG slike ("Scalable Vector Graphics")

		\item Nakon menjanja velicine SVG slike rekord se sa izracunatim novim dimenzijama cuva se u
		\texttt{sys\_file\_processedfile} umesto da se pravi novi preradjeni fajl \newline
			\small(izuzetak ukoliko je slika dalje obradjivana npr. secenje slike).\normalsize.

		\item Uveden je alternativni nacin za utvrdjivanje velicina SVG-a ukoliko ImageMagick/GraphicsMagick ne moze da utvrdi velicinu slike. U ovom slucaju cita se sadrzaj XML fajla.

		\item SVG je takodje dodat na listu validnih formata slika:\newline
			\texttt{\$GLOBALS['TYPO3\_CONF\_VARS']['GFX']['imagefile\_ext']}

	\end{itemize}

\end{frame}

% ------------------------------------------------------------------------------
% LTXE-SLIDE-START
% LTXE-SLIDE-UID:		8d5d6e00-28dbeff0-e3799f2f-ff173fcd
% LTXE-SLIDE-ORIGIN:	bd03e141-3842b75e-8a44920f-a8c139e5 English
% LTXE-SLIDE-TITLE:		Add count methods and sort functionality to FAL drivers
% LTXE-SLIDE-REFERENCE:	Breaking-56746-AddCountMethodsAndSortFunctionalityToFalDrivers.rst
% ------------------------------------------------------------------------------
\begin{frame}[fragile]
	\frametitle{Korenite promene}
	\framesubtitle{Prosirenje FAL Driver-a}

	% decrease font size for code listing
	\lstset{basicstyle=\tiny\ttfamily}

	\begin{itemize}
		\item Kako bi se poboljsao rad fajl liste kada prikazuje (udaljena) "skladista" 
			FAL driver bi trebalo da bude u mogucnosti da sortira, premesta redosled kao i da odredi broj fajlova/foldera.
			Dodata su dva nova parametra \texttt{sort} i \texttt{sortRev} koja ce ovo omoguciti:

			\begin{lstlisting}
				public function getFilesInFolder($folderIdentifier, $start = 0, $numberOfItems = 0,
				  $recursive = FALSE, array $filenameFilterCallbacks = array(), $sort = '', $sortRev = FALSE);

				public function getFoldersInFolder($folderIdentifier, $start = 0, $numberOfItems = 0,
				  $recursive = FALSE, array $folderNameFilterCallbacks = array(), $sort = '', $sortRev = FALSE);
			\end{lstlisting}

		\item Dodatno, implementirana su dva dodatna metoda:

			\begin{lstlisting}
				public function getFilesInFolderCount($folderIdentifier, $recursive = FALSE,
				  array $filenameFilterCallbacks = array());

				public function getFoldersInFolderCount($folderIdentifier, $recursive = FALSE,
				  array $folderNameFilterCallbacks = array());
			\end{lstlisting}

	\end{itemize}

\end{frame}

% ------------------------------------------------------------------------------
% LTXE-SLIDE-START
% LTXE-SLIDE-UID:		5ff9eeeb-06b02c7b-18779612-9fc0d176
% LTXE-SLIDE-ORIGIN:	d0df6060-4b8d33a9-69adad59-422e9ada English
% LTXE-SLIDE-TITLE:		Introduce Backend Routing (1)
% LTXE-SLIDE-REFERENCE:	commit a08ce7238e583d1962edfe998e04c7d9c3d7c2ae
% ------------------------------------------------------------------------------
\begin{frame}[fragile]
	\frametitle{Korenite promene}
	\framesubtitle{Routing API za administratorski interfejs(1)}

	\begin{itemize}
		\item Implementiran je Routing API za administratorski interfejs koji upravlja ulaznim tackama
		\item Inspirisan Symfony Routing Framework-om, ovaj API veoma kompatibilan\newline
			\small(dok TYPO3 koristi samo oko 20\% u ovom momentu)\normalsize

		\item Prakticno, tri klase implementiraju citavu funkcionalnost:
			\begin{itemize}
				\item \texttt{class Route}:\tabto{3.6cm}sadrzi detalje o putanjama i opcijama
				\item \texttt{class Router}:\tabto{3.6cm}API za pravljenje putanja
				\item \texttt{class UrlGenerator}:\tabto{3.6cm}generise URL
			\end{itemize}
	\end{itemize}

\end{frame}

% ------------------------------------------------------------------------------
% LTXE-SLIDE-START
% LTXE-SLIDE-UID:		b0657bce-2a183b54-90c92657-a672392f
% LTXE-SLIDE-ORIGIN:	fe1eec67-bbf31ed9-57b8e83a-3a922f0e English
% LTXE-SLIDE-TITLE:		Introduce Backend Routing (2)
% LTXE-SLIDE-REFERENCE:	commit a08ce7238e583d1962edfe998e04c7d9c3d7c2ae
% ------------------------------------------------------------------------------
\begin{frame}[fragile]
	\frametitle{Korenite promene}
	\framesubtitle{Routing API za administratorski interfejs (2)}

	\begin{itemize}

		\item Putanje su definisane u sledecim fajlovima prosirenja:
			\texttt{Configuration/Backend/Routes.php}\newline
			(pogledati sistemsko prosirenje \texttt{backend} kao primer)

		\item Dodatni detalji o Routing API za administratorski interfejs mogu se naci na::\newline
			\small\url{http://wiki.typo3.org/Blueprints/BackendRouting}\normalsize

	\end{itemize}

\end{frame}

% ------------------------------------------------------------------------------
% LTXE-SLIDE-START
% LTXE-SLIDE-UID:		b800afd4-7f34bf87-a1794226-390cd086
% LTXE-SLIDE-ORIGIN:	9d56ca19-04a98896-ba0a973c-8a16e877 English
% LTXE-SLIDE-TITLE:		New system extension: mediace
% LTXE-SLIDE-REFERENCE:	Breaking-64719-MediaContentMovedToSystemExtension.rst
% LTXE-SLIDE-REFERENCE:	Breaking-65778-MediaWizardProviderMovedToSystemExtension.rst
% ------------------------------------------------------------------------------
\begin{frame}[fragile]
	\frametitle{Korenite promene}
	\framesubtitle{Novo sistemsko prosirenje za media kontent elemente}

	\begin{itemize}

		\item Novo sistemsko prosirenje "\texttt{mediace}" sadrzi sledece cObjects:

			\begin{itemize}
				\item \texttt{MULTIMEDIA}
				\item \texttt{MEDIA}
				\item \texttt{SWFOBJECT}
				\item \texttt{FLOWPLAYER}
				\item \texttt{QTOBJECT}
			\end{itemize}

		\item Elementi sadrzaja \texttt{media} i \texttt{multimedia} takodje su prebaceni u ovo sistemsko prosirenje, kao i "Media Wizard Provider"

		\item Ovo prosirenje \underline{\textbf{nije}} instalirano kao podrazumevano!

	\end{itemize}

\end{frame}

% ------------------------------------------------------------------------------
% LTXE-SLIDE-START
% LTXE-SLIDE-UID:		bf86856a-da4f5645-3e5f2d2d-06b6e7dd
% LTXE-SLIDE-ORIGIN:	cc63cd98-524ee6df-38430963-f7c7067c English
% LTXE-SLIDE-TITLE:		Composer vendor directory
% LTXE-SLIDE-REFERENCE:	Breaking-66001-ComposerVendorDirectoryChanged.rst
% ------------------------------------------------------------------------------
\begin{frame}[fragile]
	\frametitle{Korenite promene}
	\framesubtitle{Lokacija dodatih biblioteka}

	\begin{itemize}

		\item Dodatne biblioteke instalirane preko Composer-a sada se snimaju u folder \texttt{typo3/contrib/vendor} \newline
			\small
				(TYPO3 CMS < 7.2: u folder \texttt{Packages/Libraries})
			\normalsize

		\item Na ovaj nacin TYPO3 CMS se moze zapakovati za objavu bez potrebe da se pakuju \texttt{Packages/} za dodatne biblioteke

		\item Problemi mogu nastati sa instalacijama koje su napravljene preko Composer-a i koriste \texttt{phpunit}
			ukoliko Composer dependencies nisu obnovljene. Da se ovo ispravi, pozvati:

			\begin{lstlisting}
				# cd htdocs/
				# rm -rf typo3/contrib/vendor/ bin/ Packages/Libraries/ composer.lock
				# composer install
			\end{lstlisting}
	\end{itemize}

\end{frame}

% ------------------------------------------------------------------------------
% LTXE-SLIDE-START
% LTXE-SLIDE-UID:		043a219b-41cfe224-94558a2e-1310b69c
% LTXE-SLIDE-ORIGIN:	c9a5288c-f45aac1e-34606612-fdf6ec18 English
% LTXE-SLIDE-TITLE:		Introduce JavaScript notification API
% LTXE-SLIDE-REFERENCE:	Feature-66047-IntroduceJavascriptNotificationApi.rst
% ------------------------------------------------------------------------------
\begin{frame}[fragile]
	\frametitle{Korenite promene}
	\framesubtitle{JavaScript obavestenja}

	\begin{itemize}

		\item Implementiran je nov API za JavaScript obavestenja:
			\begin{lstlisting}
				// stari i zastareli:
				top.TYPO3.Flashmessages.display(TYPO3.Severity.notice)

				// novi i jedini ispravan nacin od TYPO3 CMS 7.2:
				top.TYPO3.Notification.notice(title, message)
    		\end{lstlisting}

		\item Postoje sledece API funkcije:\newline
			\small(parametar \texttt{duzina trajanja} je opcionalan i ima podrazumevanu vrednost od 5 sekundi)\normalsize
			\begin{itemize}
				\item \normalsize\smaller\texttt{top.TYPO3.Notification.notice(title, message, duration)}\normalsize
				\item \smaller\texttt{top.TYPO3.Notification.info(title, message, duration)}\normalsize
				\item \smaller\texttt{top.TYPO3.Notification.success(title, message, duration)}\normalsize
				\item \smaller\texttt{top.TYPO3.Notification.warning(title, message, duration)}\normalsize
				\item \smaller\texttt{top.TYPO3.Notification.error(title, message, duration)}\normalsize
			\end{itemize}

	\end{itemize}

\end{frame}

% ------------------------------------------------------------------------------
% LTXE-SLIDE-START
% LTXE-SLIDE-UID:		d6f30d52-84d7b090-a32a820b-dc08675b
% LTXE-SLIDE-ORIGIN:	b2acb5df-60b9a7dd-35cf697d-2a50d5db English
% LTXE-SLIDE-TITLE:		System Information Dropdown (1)
% LTXE-SLIDE-REFERENCE:	Feature-65767-SystemInformationDropdown.rst
% ------------------------------------------------------------------------------
\begin{frame}[fragile]
	\frametitle{Korenite promene}
	\framesubtitle{Sistemske informacije u padajucem meniju (1)}

	% decrease font size for code listing
	\lstset{basicstyle=\tiny\ttfamily}

	\begin{itemize}
		\item Prilagodjene informacije sistema mogu da se dodaju u padajuci meni kreiranjem slot-a

		\item Slot se mora registrovati u fajlu \texttt{ext\_localconf.php}:

			\begin{lstlisting}
				$signalSlotDispatcher = \TYPO3\CMS\Core\Utility\GeneralUtility::makeInstance(
				  \TYPO3\CMS\Extbase\SignalSlot\Dispatcher::class);

				$signalSlotDispatcher->connect(
				  \TYPO3\CMS\Backend\Backend\ToolbarItems\SystemInformationToolbarItem::class,
				  'getSystemInformation',
				  \Vendor\Extension\SystemInformation\Item::class,
				  'getItem'
				);
			\end{lstlisting}

	\end{itemize}

\end{frame}

% ------------------------------------------------------------------------------
% LTXE-SLIDE-START
% LTXE-SLIDE-UID:		ce1d6388-425fbbc5-8db47c74-3efbf743
% LTXE-SLIDE-ORIGIN:	e0f13209-22e0ff8c-7ff23bf9-31b1bb52 English
% LTXE-SLIDE-TITLE:		System Information Dropdown (2)
% LTXE-SLIDE-REFERENCE:	Feature-65767-SystemInformationDropdown.rst
% ------------------------------------------------------------------------------
\begin{frame}[fragile]
	\frametitle{Korenite promene}
	\framesubtitle{Sistemske informacije u padajucem meniju (2)}

	% decrease font size for code listing
	\lstset{basicstyle=\tiny\ttfamily}

	\begin{itemize}

		\item Prilagodjene informacije sistema mogu da se dodaju u padajuci meni kreiranjem slot-a

		\item Ovo zahteva klasu \texttt{Item} i njen metod \texttt{getItem()} u fajlu
			\small
				\texttt{EXT:extension\textbackslash
					Classes\textbackslash
					SystemInformation\textbackslash
					Item.php}:
			\normalsize

		\begin{lstlisting}
			class Item {
			  public function getItem() {
			    return array(array(
			      'title' => 'The title shown on hover',
			      'value' => 'Description shown in the list',
			      'status' => SystemInformationHookInterface::STATUS_OK,
			      'count' => 4,
			      'icon' => \TYPO3\CMS\Backend\Utility\IconUtility::getSpriteIcon(
				    'extensions-example-information-icon')
			    ));
			  }
			}
		\end{lstlisting}

	\end{itemize}

\end{frame}

% ------------------------------------------------------------------------------
% LTXE-SLIDE-START
% LTXE-SLIDE-UID:		6a7aa92f-24626386-fdcf34ea-853be28d
% LTXE-SLIDE-ORIGIN:	7d27df49-260500ef-0fc78a89-6e80069b English
% LTXE-SLIDE-TITLE:		System Information Dropdown (3)
% LTXE-SLIDE-REFERENCE:	Feature-65767-SystemInformationDropdown.rst
% ------------------------------------------------------------------------------
\begin{frame}[fragile]
	\frametitle{Korenite promene}
	\framesubtitle{Sistemske informacije u padajucem meniju (3)}

	% decrease font size for code listing
	\lstset{basicstyle=\tiny\ttfamily}

	\begin{itemize}
		\item Ikonica \texttt{extensions-example-information-icon} se mora registrovati u fajlu \texttt{ext\_localconf.php}:
		\begin{lstlisting}
			\TYPO3\CMS\Backend\Sprite\SpriteManager::addSingleIcons(
			  array(
			    'information-icon' => \TYPO3\CMS\Core\Utility\ExtensionManagementUtility::extRelPath(
			      $_EXTKEY) . 'Resources/Public/Images/Icons/information-icon.png'
			    ),
			   $_EXTKEY
			);
		\end{lstlisting}

	\end{itemize}

\end{frame}


% ------------------------------------------------------------------------------
% LTXE-SLIDE-START
% LTXE-SLIDE-UID:		084ab2e7-c8e3ca24-dfba77dd-f11456f5
% LTXE-SLIDE-ORIGIN:	d7449c27-fd9676cf-d12be80b-c24a4536 English
% LTXE-SLIDE-TITLE:		System Information Dropdown (4)
% LTXE-SLIDE-REFERENCE:	Feature-65767-SystemInformationDropdown.rst
% ------------------------------------------------------------------------------
\begin{frame}[fragile]
	\frametitle{Korenite promene}
	\framesubtitle{Sistemske informacije u padajucem meniju (4)}

	% decrease font size for code listing
	\lstset{basicstyle=\tiny\ttfamily}

	\begin{itemize}

		\item Poruke su prikazane na dnu dropdown menija

		\item Prosirenja mogu obezbediti sopstvene slot-ove da ispisu poruke

		\begin{lstlisting}
			$signalSlotDispatcher = \TYPO3\CMS\Core\Utility\GeneralUtility::makeInstance(
			  \TYPO3\CMS\Extbase\SignalSlot\Dispatcher::class);

			$signalSlotDispatcher->connect(
			  \TYPO3\CMS\Backend\Backend\ToolbarItems\SystemInformationToolbarItem::class,
			  'loadMessages',
			  \Vendor\Extension\SystemInformation\Message::class,
			  'getMessage'
			);
		\end{lstlisting}

	\end{itemize}

\end{frame}

% ------------------------------------------------------------------------------
% LTXE-SLIDE-START
% LTXE-SLIDE-UID:		c39cbd99-a64dd8af-7d9e7e08-64015158
% LTXE-SLIDE-ORIGIN:	ab3b46ce-c7d7a228-7f7888af-4312fcdd English
% LTXE-SLIDE-TITLE:		System Information Dropdown (5)
% LTXE-SLIDE-REFERENCE:	Feature-65767-SystemInformationDropdown.rst
% ------------------------------------------------------------------------------
\begin{frame}[fragile]
	\frametitle{Korenite promene}
	\framesubtitle{Sistemske informacije u padajucem meniju (5)}

	% decrease font size for code listing
	\lstset{basicstyle=\tiny\ttfamily}

	\begin{itemize}

		\item Poruke su prikazane na dnu dropdown menija
		
		\item Ovo zahteva klasu \texttt{Message} i njen metod \texttt{getMessage()} u fajlu
			\small
				\texttt{EXT:extension\textbackslash
					Classes\textbackslash
					SystemInformation\textbackslash
					Message.php}:
			\normalsize

		\begin{lstlisting}
			class Message {
			  public function getMessage() {
			    return array(array(
			      'status' => SystemInformationHookInterface::STATUS_OK,
			      'text' => 'Something went wrong. Take a look at the reports module.'
			    ));
			  }
			}
		\end{lstlisting}

	\end{itemize}

\end{frame}


% ------------------------------------------------------------------------------
% LTXE-SLIDE-START
% LTXE-SLIDE-UID:		b4f5a435-db4f0614-b4036b29-fe672189
% LTXE-SLIDE-ORIGIN:	98e2056a-d165b638-b1a1ee08-206bc15a English
% LTXE-SLIDE-TITLE:		Add image cropping
% LTXE-SLIDE-REFERENCE:	Feature-65584-AddImageCropping.rst
% ------------------------------------------------------------------------------
\begin{frame}[fragile]
	\frametitle{Korenite promene}
	\framesubtitle{Opcije konfiguracije za manipulisanje sa slikama (1)}

	\begin{itemize}
		\item Sledece TypoScript opcije konfiguracije su dostupne:
			\begin{lstlisting}
				# disable cropping for all images
				tt_content.image.20.1.file.crop =

				# override or set cropping for all images
				# offsetX,offsetY,width,height
				tt_content.image.20.1.file.crop = 50,50,100,100
			\end{lstlisting}

		\item Fluid takodje podrzava funkcije secenja:
			\begin{lstlisting}
				# disable cropping for all images
				<f:image image="{imageObject}" crop="" ></f:image>

				# override or set cropping for all images
				# offsetX,offsetY,width,height
				<f:image image="{imageObject}" crop="50,50,100,100" ></f:image>
			\end{lstlisting}

	\end{itemize}

\end{frame}

% ------------------------------------------------------------------------------
% LTXE-SLIDE-START
% LTXE-SLIDE-UID:		279e853b-d5e7d754-36802df0-a3cf6ff0
% LTXE-SLIDE-ORIGIN:	e9cce085-6c300f1f-010de4ee-2c744704 English
% LTXE-SLIDE-TITLE:		Add image cropping (2)
% LTXE-SLIDE-REFERENCE:	Feature-65584-AddImageCropping.rst
% ------------------------------------------------------------------------------
\begin{frame}[fragile]
	\frametitle{Korenite promene}
	\framesubtitle{Opcije konfiguracije za manipulisanje sa slikama (2)}

	% decrease font size for code listing
	\lstset{basicstyle=\smaller\ttfamily}

	\begin{itemize}
		\item TCA takodje ima funkcionalnost secenja slike:

			\begin{itemize}
				\item Column Type: \texttt{image\_manipulation}
				\item Config \texttt{file\_field}: string	\tabto{5.6cm}(default: \texttt{uid\_local})
				\item Config \texttt{enableZoom}: boolean	\tabto{5.6cm}(default: \texttt{FALSE})
				\item Config \texttt{allowedExtensions}: string\newline
					(default: \smaller\texttt{\$GLOBALS['TYPO3\_CONF\_VARS']['GFX']['imagefile\_ext']}\small)
				\item Config \texttt{ratios}: array, default:

					\begin{lstlisting}
						array(
						  '1.7777777777777777' => '16:9',
						  '1.3333333333333333' => '4:3',
						  '1' => '1:1',
						  'NaN' => 'Free'
						)
					\end{lstlisting}
			\end{itemize}

	\end{itemize}

\end{frame}

% ------------------------------------------------------------------------------
% LTXE-SLIDE-START
% LTXE-SLIDE-UID:		3fc7d2fb-1993d27d-e2bd7c60-e35b236f
% LTXE-SLIDE-ORIGIN:	ed863269-574e4061-d69f855e-6494c062 English
% LTXE-SLIDE-TITLE:		Additional params for HTMLparser userFunc
% LTXE-SLIDE-REFERENCE:	Feature-59712-HtmlParserAdditionalUserFuncParams.rst
% ------------------------------------------------------------------------------
\begin{frame}[fragile]
	\frametitle{Korenite promene}
	\framesubtitle{Dodatni parametri za HTMLparser userFunc}

	% decrease font size for code listing
	\lstset{basicstyle=\tiny\ttfamily}

	\begin{itemize}

		\item Dodatni parametri mogu se proslediti metodi userFunc od HTMLparser:

			\begin{lstlisting}
				myobj = TEXT
				myobj.value = <a href="/" class="myclass">MyText</a>
				myobj.HTMLparser.tags.a.fixAttrib.class {
				  userFunc = Tx\MyExt\Myclass->htmlUserFunc
				  userFunc.myparam = test
				}
			\end{lstlisting}

		\item Pristup ovim parametrima u prosirenju radi se na sledeci nacin:

			\begin{lstlisting}
				function htmlUserFunc(array $params, HtmlParser $htmlParser) {
				  // $params['attributeValue'] contains the attribute value "myclass"
				  // $params['myparam'] is set to "test" in this example
				  ...
				}
			\end{lstlisting}

	\end{itemize}

\end{frame}

% ------------------------------------------------------------------------------
% LTXE-SLIDE-START
% LTXE-SLIDE-UID:		456d2d85-6e6f0111-6aa98b48-2dd82b0c
% LTXE-SLIDE-ORIGIN:	b04f8d44-fd695c73-b768f3cb-fd6d8643 English
% LTXE-SLIDE-TITLE:		New Locking API (1)
% LTXE-SLIDE-REFERENCE:	Feature-47712-NewLockingAPI.rst
% ------------------------------------------------------------------------------
\begin{frame}[fragile]
	\frametitle{Korenite promene}
	\framesubtitle{Locking API (1)}

	\begin{itemize}

		\item Predstavljen je novi Locking API, koji obezbedjuje razlicite locking metode (SimpleFile, Semaphore, ...)
		\item Locking metoda mora implementirati \small\texttt{LockingStrategyInterface}\normalsize:
		\begin{lstlisting}
			$lockFactory = GeneralUtility::makeInstance(LockFactory::class);
			$locker = $lockFactory->createLocker('someId');
			$locker->acquire() || die('Could not acquire lock.');
			...
			$locker->release();
		\end{lstlisting}

	\end{itemize}

\end{frame}

% ------------------------------------------------------------------------------
% LTXE-SLIDE-START
% LTXE-SLIDE-UID:		80bae2d0-c8b4c2ab-b19d48a3-195ef4dd
% LTXE-SLIDE-ORIGIN:	79a00412-c72401e6-4d9667c9-0e15d3f4 English
% LTXE-SLIDE-TITLE:		New Locking API (2)
% LTXE-SLIDE-REFERENCE:	Feature-47712-NewLockingAPI.rst
% ------------------------------------------------------------------------------
\begin{frame}[fragile]
	\frametitle{Korenite promene}
	\framesubtitle{Locking API (2)}

	% decrease font size for code listing
	\lstset{basicstyle=\tiny\ttfamily}

	\begin{itemize}

		\item Neke metode takodje podrzavaju non-blocking locks:

		\begin{lstlisting}
			$lockFactory = GeneralUtility::makeInstance(LockFactory::class);
			$locker = $lockFactory->createLocker(
			  'someId',
			  LockingStrategyInterface::LOCK_CAPABILITY_SHARED |
			    LockingStrategyInterface::LOCK_CAPABILITY_NOBLOCK
			);
			try {
			  $result = $locker->acquire(LockingStrategyInterface::LOCK_CAPABILITY_SHARED | LockingStrategyInterface::LOCK_CAPABILITY_NOBLOCK);
			  catch (\RuntimeException $e) {
			  if ($e->getCode() === 1428700748) {
			    // some process owns the lock
			    // let's do something else meanwhile
			    ...
			  }
			}
			if ($result) {
			  $locker->release();
			}
		\end{lstlisting}

	\end{itemize}

\end{frame}

% ------------------------------------------------------------------------------
% LTXE-SLIDE-START
% LTXE-SLIDE-UID:		7dccb39f-56826fae-21bbebfe-c572e1ae
% LTXE-SLIDE-ORIGIN:	108cb4a7-6e006507-c25a91e7-3a867f09 English
% LTXE-SLIDE-TITLE:		Add signal after extension installation
% LTXE-SLIDE-REFERENCE:	commit 39e0fd0a2c919da270cb49223433bcf95febbb10
% ------------------------------------------------------------------------------
\begin{frame}[fragile]
	\frametitle{Korenite promene}
	\framesubtitle{Signal nakon instaliranja prosirenja}

	% decrease font size for code listing
	\lstset{basicstyle=\tiny\ttfamily}

	\begin{itemize}

		\item Implementiran je novi signal u metodi
			\smaller
				\texttt{\textbackslash
					TYPO3\textbackslash
					CMS\textbackslash
					Extensionmanager\textbackslash
					Utility\textbackslash
					InstallUtility::install()}
			\normalsize
			koji se emituje cim se prosirenje instalira i zavrseni su svi uvozi i unapredjenja

		\begin{lstlisting}
			// execution
			$this->emitAfterExtensionInstallSignal($extensionKey);

			// methode
			protected function emitAfterExtensionInstallSignal($extensionKey) {
			  $this->signalSlotDispatcher->dispatch(
			    __CLASS__,
			    'afterExtensionInstall',
			    array($extensionKey, $this)
			  );
			}
		\end{lstlisting}

	\end{itemize}

\end{frame}

% ------------------------------------------------------------------------------
% LTXE-SLIDE-START
% LTXE-SLIDE-UID:		dfa45ff7-d352c814-74bf600e-5e36d064
% LTXE-SLIDE-ORIGIN:	acb1ba95-1a401bef-549072a1-f32263a9 English
% LTXE-SLIDE-TITLE:		Registry for adding text extractor classes (1)
% LTXE-SLIDE-REFERENCE:	Feature-36743-FAL-TextExtractorRegistry.rst
% ------------------------------------------------------------------------------
\begin{frame}[fragile]
	\frametitle{Korenite promene}
	\framesubtitle{Registar za izvlacenje teksta (1)}

	% decrease font size for code listing
	\lstset{basicstyle=\tiny\ttfamily}

	\begin{itemize}

		\item Visestruko izvlacenje teksta se moze registrovati da omoguci rad sa razlicitim tipovima fajlova (na primer Office, PDF files, itd.)

		\item TYPO3 osnova ima ekstraktor za plain text fajlove

		\item Svaka registrovana klasa za izvlacenje teksta mora da implementira \texttt{TextExtractorInterface}

		\item ...i sledece metode:\newline
			\texttt{canExtractText()}\newline
			\small
				proverava da li moze da se izvuce tekst iz izabranog fajla 
			\normalsize
			\newline
			\texttt{extractText()}\newline
			\small
				vraca sadrzaj teksta iz fajla kao string
			\normalsize

	\end{itemize}

\end{frame}

% ------------------------------------------------------------------------------
% LTXE-SLIDE-START
% LTXE-SLIDE-UID:		a00711c4-2918cdda-ea44d7cd-310357c0
% LTXE-SLIDE-ORIGIN:	acb1ba95-1a401bef-549072a1-f32263a9 English
% LTXE-SLIDE-TITLE:		Registry for adding text extractor classes (2)
% LTXE-SLIDE-REFERENCE:	Feature-36743-FAL-TextExtractorRegistry.rst
% ------------------------------------------------------------------------------
\begin{frame}[fragile]
	\frametitle{Korenite promene}
	\framesubtitle{Registar za izvlacenje teksta (2)}

	% decrease font size for code listing
	\lstset{basicstyle=\tiny\ttfamily}

	\begin{itemize}

		\item Izvlacenje teksta se registruje u \texttt{ext\_localconf.php}:

			\begin{lstlisting}
				$textExtractorRegistry = \TYPO3\CMS\Core\Resource\TextExtraction\TextExtractorRegistry::getInstance();
				$textExtractorRegistry->registerTextExtractor(
				  \TYPO3\CMS\Core\Resource\TextExtraction\PlainTextExtractor::class
				);
			\end{lstlisting}

		\item Koristi se na sledeci nacin:

			\begin{lstlisting}
				$textExtractorRegistry = \TYPO3\CMS\Core\Resource\TextExtraction\TextExtractorRegistry::getInstance();
				$extractor = $textExtractorRegistry->getTextExtractor($file);
				if($extractor !== NULL) {
				  $content = $extractor->extractText($file);
				}
			\end{lstlisting}
	\end{itemize}

\end{frame}

% ------------------------------------------------------------------------------
% LTXE-SLIDE-START
% LTXE-SLIDE-UID:		76ef8242-dea86411-649577de-72ef9a42
% LTXE-SLIDE-ORIGIN:	d9657b5d-5cd56c65-d7266a27-401dd233 English
% LTXE-SLIDE-TITLE:		Miscellaneous
% LTXE-SLIDE-REFERENCE:	Feature-66042-WebLibrariesLoadedViaBower.rst
% LTXE-SLIDE-REFERENCE:	Feature-63703-AddOptionToStopTask.rst
% LTXE-SLIDE-REFERENCE:	Feature-61463-AllowProcessedFoldersInDifferentStorage.rst
% LTXE-SLIDE-REFERENCE:	Feature-52693-TSFE-RequestedId.rst
% ------------------------------------------------------------------------------

\begin{frame}[fragile]
	\frametitle{Korenite promene}
	\framesubtitle{Razno}

	\begin{itemize}

		\item Web biblioteke (kao sto su Twitter Bootstrap, jQuery, Font Awesome, itd.) koriste
			"Bower" (\url{http://bower.io}) i vise nisu deo TYPO3 osnove u Git repository-ju\newline
			\small
				\texttt{\# bower install}	\tabto{3.4cm}izvrsava instalaciju\newline
				\texttt{\# bower update}		\tabto{3.4cm}izvrsava unapredjenje\newline
			\normalsize
			(fajl \texttt{bower.json} se nalazi u folderu \texttt{Build/})

		\item Scheduler CLI dobio je opciju "\texttt{-s}" da zaustavi pokrenuti task

		\item Folder sa procesuiranim fajlovima od (udaljenog) skladista mzoe biti van skladista (korisno u slucaju read-only skladista)

		\item Sada je moguce vratiti ID originalno trazene strane:
			\texttt{\$TSFE->getRequestedId()}

	\end{itemize}

\end{frame}

% ------------------------------------------------------------------------------
