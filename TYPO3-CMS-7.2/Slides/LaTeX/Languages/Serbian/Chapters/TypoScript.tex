% ------------------------------------------------------------------------------
% TYPO3 CMS 7.2 - What's New - Chapter "TypoScript" (English Version)
%
% @author	Michael Schams <schams.net>
% @license	Creative Commons BY-NC-SA 3.0
% @link		http://typo3.org/download/release-notes/whats-new/
% @language	English
% ------------------------------------------------------------------------------
% LTXE-CHAPTER-UID:		12518a77-2b90a173-4e3f8420-485e0497
% LTXE-CHAPTER-NAME:	TypoScript
% ------------------------------------------------------------------------------

\section{TSconfig i TypoScript}
\begin{frame}[fragile]
	\frametitle{TSconfig i TypoScript}

	\begin{center}\huge{Poglavlje 2:}\end{center}
	\begin{center}\huge{\color{typo3darkgrey}\textbf{TSconfig i TypoScript}}\end{center}

\end{frame}

% ------------------------------------------------------------------------------
% LTXE-SLIDE-START
% LTXE-SLIDE-UID:		c42f4349-7386307b-f934df8f-82888f91
% LTXE-SLIDE-ORIGIN:	56b025e6-cf380b11-c9d1c009-0b548d6a English
% LTXE-SLIDE-TITLE:		Add flexible preview URL configuration (1)
% LTXE-SLIDE-REFERENCE:	Feature-66370-AddFlexiblePreviewUrlConfiguration.rst
% ------------------------------------------------------------------------------
\begin{frame}[fragile]
	\frametitle{TSconfig i TypoScript}
	\framesubtitle{Fleksibilna konfiguracija linka za pregled  (1)}

	% decrease font size for code listing
	\lstset{basicstyle=\tiny\ttfamily}

	\begin{itemize}

		\item Sada je moguce konfigurisati link za pregled koji se generise nakon klika na\newline
			"save \& view" dugme u administratorskom interfejsu.

		\item Tipicna upotreba je dozvoliti pregled blog rekorda ili rekorda vesti, ali je takodje moguce definisati razlicite preview strane za obicne kontent elemente.

			\begin{lstlisting}
				TCEMAIN.preview {
				  <table name> {
				    previewPageId = 123
				    useDefaultLanguageRecord = 0
				    fieldToParameterMap {
				      uid = tx_myext_pi1[showUid]
				    }
				    additionalGetParameters {
				      tx_myext_pi1[special] = HELLO
				    }
				  }
				}
			\end{lstlisting}

	\end{itemize}

\end{frame}

% ------------------------------------------------------------------------------
% LTXE-SLIDE-START
% LTXE-SLIDE-UID:		e7f04442-d634c3c6-966d7689-a776df9f
% LTXE-SLIDE-ORIGIN:	6822c376-37649cd8-0acc836a-b856fc61 English
% LTXE-SLIDE-TITLE:		Add flexible preview URL configuration (2)
% LTXE-SLIDE-REFERENCE:	Feature-66370-AddFlexiblePreviewUrlConfiguration.rst
% ------------------------------------------------------------------------------
\begin{frame}[fragile]
	\frametitle{TSconfig i TypoScript}
	\framesubtitle{Fleksibilna konfiguracija linka za pregled (2)}

	\begin{itemize}
		\item \texttt{previewPageId}:\newline
			\smaller
				UID strane koja ce posluziti kao strana za pregled\newline
				(ukoliko se ovo podesavanje zanemari koristice se trenutna strana)
			\normalsize
		\item \texttt{useDefaultLanguageRecord}:\newline
			\smaller
				bice definisano ukoliko ce prevedeni rekordi koristiti UID standardnih rekorda\newline
				(ovo je podrazumevano podesavanje, value: 1)
			\normalsize
		\item \texttt{fieldToParameterMap}:\newline
			\smaller
				mapiranje koje dozvoljava odabir polja koja ce biti ukljucena u GET-parametre
			\normalsize
		\item \texttt{additionalGetParameters}:\newline
			\smaller
				dozvoljava dodavanje proizvodnih GET-parametara, cak i premoscavanje drugih
			\normalsize
	\end{itemize}

\end{frame}

% ------------------------------------------------------------------------------
% LTXE-SLIDE-START
% LTXE-SLIDE-UID:		b959dcf1-44aa0ee6-46cd8b2f-f4213d24
% LTXE-SLIDE-ORIGIN:	c5dbf3c7-655fc6e4-227a3bf6-f5d89fb1 English
% LTXE-SLIDE-TITLE:
% LTXE-SLIDE-REFERENCE:	Feature-59646-AddRteConfigurationPropertyButtonsLinkTypePropertiesTargetDefault.rst
% ------------------------------------------------------------------------------
\begin{frame}[fragile]
	\frametitle{TSconfig i TypoScript}
	\framesubtitle{RTE Konfiguracija: odrediste linkova (target)}

	\begin{itemize}

		\item Nova RTE osobina za konfigurisanje moze se koristiti u PageTSconfig kako bi se konfigurisalo podrazumevano odrediste (target) za linkove datog tipa\newline

			\small
				\texttt{buttons.link.[}
				\textit{type}
				\texttt{].properties.target.default = ...}
			\normalsize\newline

		\item Moguci tipovi liknova su:\newline
			\small
				(prosirenja mogu proizvesti dodatne tipove)
			\normalsize

			\begin{itemize}
				\item \texttt{page}
				\item \texttt{file}
				\item \texttt{url}
				\item \texttt{mail}
				\item \texttt{spec}
			\end{itemize}
	\end{itemize}

\end{frame}

% ------------------------------------------------------------------------------
% LTXE-SLIDE-START
% LTXE-SLIDE-UID:		d5c2de37-78643b91-f636a214-0d5e4000
% LTXE-SLIDE-ORIGIN:	fd957580-301e4a0a-ad5325f5-ffa024c3 English
% LTXE-SLIDE-TITLE:		Strip empty HTML tags in HtmlParser
% LTXE-SLIDE-REFERENCE:	Feature-20555-StripEmptyHtmlTags.rst
% ------------------------------------------------------------------------------
\begin{frame}[fragile]
	\frametitle{TSconfig i TypoScript}
	\framesubtitle{Uklanjanje praznih HTML tagova pomocu HTMLparser-a}

	% decrease font size for code listing
	\lstset{basicstyle=\tiny\ttfamily}

	\begin{itemize}
		\item Implementirana je nova funkcionalnost u HTMLparser koja dozvoljava uklanjanje praznih HTML tagova

			\begin{lstlisting}
				stdWrap {
				   // this removes all empty HTML tags
				   HTMLparser.stripEmptyTags = 1
				   // this removes empty h2 and h3 tags only
				   HTMLparser.stripEmptyTags.tags = h2, h3
				}

				RTE.default.proc.entryHTMLparser_db {
				   stripEmptyTags = 1
				   stripEmptyTags.tags = p
				   stripEmptyTags.treatNonBreakingSpaceAsEmpty = 1
				}
			\end{lstlisting}

			\small
				\underline{\textbf{Napomena:}}
				podrazumevano podesavanje HTMLparser-a je da uklanja sve nepoznate tagove.
				Stoga moze biti korisno zadrzati ih sledecom instrukcijom:\newline
				\texttt{HTMLparser.keepNonMatchedTags = 1}
			\normalsize

	\end{itemize}

\end{frame}

% ------------------------------------------------------------------------------
% LTXE-SLIDE-START
% LTXE-SLIDE-UID:		6bc9d3bd-4c19e434-57d6bc8c-3c0dc67d
% LTXE-SLIDE-ORIGIN:	62960e76-c3dca953-fe5a1070-831633fc English
% LTXE-SLIDE-TITLE:		Miscellaneous
% LTXE-SLIDE-REFERENCE:	Feature-63040-AddRteConfigurationPropertyButtonsAbbreviationRemoveFieldsets.rst
% LTXE-SLIDE-REFERENCE:	commit 31c62f9311ee3d33bf792d548cc4a5fac83aa3d0
% ------------------------------------------------------------------------------
\begin{frame}[fragile]
	\frametitle{TSconfig i TypoScript}
	\framesubtitle{Razno}

	\begin{itemize}
		\item Nova osobina \texttt{buttons.abbreviation.removeFieldsets} se moze koristiti u
			PageTSconfig da konfigurise skracenice

			\begin{lstlisting}
				# Moguce vrednosti su:
				# acronym, definedAcronym, abbreviation, definedAbbreviation
				buttons.abbreviation.removeFieldsets = acronym,definedAcronym
			\end{lstlisting}

		\item Osobina \texttt{inlineLanguageLabel} objekta \texttt{PAGE} sada moze da koristi\newline
			\texttt{LLL:} reference

	\end{itemize}

\end{frame}

% ------------------------------------------------------------------------------
