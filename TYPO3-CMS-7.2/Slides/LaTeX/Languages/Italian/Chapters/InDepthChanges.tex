% ------------------------------------------------------------------------------
% TYPO3 CMS 7.2 - What's New - Chapter "In-Depth Changes" (English Version)
%
% @author	Michael Schams <schams.net>
% @license	Creative Commons BY-NC-SA 3.0
% @link		http://typo3.org/download/release-notes/whats-new/
% @language	English	
% ------------------------------------------------------------------------------
% LTXE-CHAPTER-UID:		e28087f2-4bffcc7b-3c685cca-579f18a8
% LTXE-CHAPTER-NAME:	In-Depth Changes
% ------------------------------------------------------------------------------

\section{In-Depth Changes}
\begin{frame}[fragile]
	\frametitle{Modifiche rilevanti}

	\begin{center}\huge{Capitolo 3:}\end{center}
	\begin{center}\huge{\color{typo3darkgrey}\textbf{Modifiche rilevanti}}\end{center}

\end{frame}

% ------------------------------------------------------------------------------
% LTXE-SLIDE-START
% LTXE-SLIDE-UID:		adeca8be-bc77dd64-0ab16a0e-4a31a69b
% LTXE-SLIDE-ORIGIN:	547708aa-92282fd1-6c27d618-4f13f86b English
% LTXE-SLIDE-TITLE:		Add SVG support
% LTXE-SLIDE-REFERENCE:	Feature-50136-AddSVGSupport.rst
% ------------------------------------------------------------------------------
\begin{frame}[fragile]
	\frametitle{Modifiche rilevanti}
	\framesubtitle{Supporto SVG nel Core}

	\begin{itemize}
		\item Il core di TYPO3 CMS ora gestisce le immagini SVG ("Scalable Vector Graphics")

		\item Quando un immagine SVG è scalata, un record con le nuove dimensioni calcolate
			è registrato in \texttt{sys\_file\_processedfile} piuttosto che creare
			un nuovo file elaborato\newline
			\small(eccetto nel caso in cui l'immagine è elaborata ulterioramente, es. cropping)\normalsize.

		\item Un fallback è aggiunto per determinare le dimensioni dell'immagine SVG se ImageMagick/GraphicsMagick
			non può determinare le dimensioni. In questo caso, viene letto il contenuto del file XML.

		\item SVG è stata aggiunta alla lista dei file validi:\newline
			\texttt{\$GLOBALS['TYPO3\_CONF\_VARS']['GFX']['imagefile\_ext']}

	\end{itemize}

\end{frame}

% ------------------------------------------------------------------------------
% LTXE-SLIDE-START
% LTXE-SLIDE-UID:		367a45ce-3d55c263-c00261a4-b65c7669
% LTXE-SLIDE-ORIGIN:	bd03e141-3842b75e-8a44920f-a8c139e5 English
% LTXE-SLIDE-TITLE:		Add count methods and sort functionality to FAL drivers
% LTXE-SLIDE-REFERENCE:	Breaking-56746-AddCountMethodsAndSortFunctionalityToFalDrivers.rst
% ------------------------------------------------------------------------------
\begin{frame}[fragile]
	\frametitle{Modifiche rilevanti}
	\framesubtitle{Estensione del Driver FAL}

	% decrease font size for code listing
	\lstset{basicstyle=\tiny\ttfamily}

	\begin{itemize}
		\item Al fine di migliorare le prestazioni della lista dei file quando mostra storage (remoti)
			il driver FAL driver deve occuparsi del'ordinamento e determinare il numero di file/directory. 
			Due nuovi parametri \texttt{sort} e \texttt{sortRev} sono stati aggiunti
			per permettere che:

			\begin{lstlisting}
				public function getFilesInFolder($folderIdentifier, $start = 0, $numberOfItems = 0,
				  $recursive = FALSE, array $filenameFilterCallbacks = array(), $sort = '', $sortRev = FALSE);

				public function getFoldersInFolder($folderIdentifier, $start = 0, $numberOfItems = 0,
				  $recursive = FALSE, array $folderNameFilterCallbacks = array(), $sort = '', $sortRev = FALSE);
			\end{lstlisting}

		\item In aggiunta, due nuovi metodi sono stati implementati:

			\begin{lstlisting}
				public function getFilesInFolderCount($folderIdentifier, $recursive = FALSE,
				  array $filenameFilterCallbacks = array());

				public function getFoldersInFolderCount($folderIdentifier, $recursive = FALSE,
				  array $folderNameFilterCallbacks = array());
			\end{lstlisting}

	\end{itemize}

\end{frame}

% ------------------------------------------------------------------------------
% LTXE-SLIDE-START
% LTXE-SLIDE-UID:		0f1885f4-1f7bee44-e5ac0f93-72e81664
% LTXE-SLIDE-ORIGIN:	d0df6060-4b8d33a9-69adad59-422e9ada English
% LTXE-SLIDE-TITLE:		Introduce Backend Routing (1)
% LTXE-SLIDE-REFERENCE:	commit a08ce7238e583d1962edfe998e04c7d9c3d7c2ae
% ------------------------------------------------------------------------------
\begin{frame}[fragile]
	\frametitle{Modifiche rilevanti}
	\framesubtitle{Backend Routing API (1)}

	\begin{itemize}
		\item Un Backend Routing API è stato implementato, per gestire gli Entry Points
		\item Ispirato al Symfony Routing Framework, queste API sono compatibili in larga misura\newline
			\small(tuttavia TYPO3 usa solamente il 20\% circa in questo punto)\normalsize

		\item Fondamentalmente tre classi implementano la funzionalità:
			\begin{itemize}
				\item \texttt{class Route}:\tabto{3.6cm}contiene dettagli sul percorso e opzioni
				\item \texttt{class Router}:\tabto{3.6cm}API per abbinare il percorso
				\item \texttt{class UrlGenerator}:\tabto{3.6cm}crea l'URL
			\end{itemize}
	\end{itemize}

\end{frame}

% ------------------------------------------------------------------------------
% LTXE-SLIDE-START
% LTXE-SLIDE-UID:		f9628236-bf9fb2c8-3d64979f-fbcfbde1
% LTXE-SLIDE-ORIGIN:	fe1eec67-bbf31ed9-57b8e83a-3a922f0e English
% LTXE-SLIDE-TITLE:		Introduce Backend Routing (2)
% LTXE-SLIDE-REFERENCE:	commit a08ce7238e583d1962edfe998e04c7d9c3d7c2ae
% ------------------------------------------------------------------------------
\begin{frame}[fragile]
	\frametitle{Modifiche rilevanti}
	\framesubtitle{Backend Routing API (2)}

	\begin{itemize}

		\item I percorsi sono definiti nel seguente file di un estensione:
			\texttt{Configuration/Backend/Routes.php}\newline
			(vedi l'estensioni di sistema \texttt{backend} ad esempio)

		\item Maggiori informazioni al riguardo di Backend Routing API:\newline
			\small\url{http://wiki.typo3.org/Blueprints/BackendRouting}\normalsize

	\end{itemize}

\end{frame}

% ------------------------------------------------------------------------------
% LTXE-SLIDE-START
% LTXE-SLIDE-UID:		7023dcc3-fca97524-a288949f-e44fad65
% LTXE-SLIDE-ORIGIN:	9d56ca19-04a98896-ba0a973c-8a16e877 English
% LTXE-SLIDE-TITLE:		New system extension: mediace
% LTXE-SLIDE-REFERENCE:	Breaking-64719-MediaContentMovedToSystemExtension.rst
% LTXE-SLIDE-REFERENCE:	Breaking-65778-MediaWizardProviderMovedToSystemExtension.rst
% ------------------------------------------------------------------------------
\begin{frame}[fragile]
	\frametitle{Modifiche rilevanti}
	\framesubtitle{Nuova estensione di sistema per gli elementi di contenuti multimediali}

	\begin{itemize}

		\item La nuova estensione di sistema "\texttt{mediace}" contiene i seguenti cObjects:

			\begin{itemize}
				\item \texttt{MULTIMEDIA}
				\item \texttt{MEDIA}
				\item \texttt{SWFOBJECT}
				\item \texttt{FLOWPLAYER}
				\item \texttt{QTOBJECT}
			\end{itemize}

		\item Gli elementi di contenuto \texttt{media} e \texttt{multimedia} sono stati spostati 
			nell'estensione di sistema, come anche il "Media Wizard Provider"

		\item Questa estensione \underline{\textbf{non}} è installata di default!

	\end{itemize}

\end{frame}

% ------------------------------------------------------------------------------
% LTXE-SLIDE-START
% LTXE-SLIDE-UID:		1f70ab8e-fad593fe-a257a4de-ab4bd819
% LTXE-SLIDE-ORIGIN:	cc63cd98-524ee6df-38430963-f7c7067c English
% LTXE-SLIDE-TITLE:		Composer vendor directory
% LTXE-SLIDE-REFERENCE:	Breaking-66001-ComposerVendorDirectoryChanged.rst
% ------------------------------------------------------------------------------
\begin{frame}[fragile]
	\frametitle{Modifiche rilevanti}
	\framesubtitle{Posizione delle librerie di terze parti}

	\begin{itemize}

		\item Le installazioni con Composer di librerie di terze parti sono ora posizionate sotto \texttt{typo3/contrib/vendor}\newline
			\small
				(TYPO3 CMS < 7.2: nella directory \texttt{Packages/Libraries})
			\normalsize

		\item In questo modo il processo di creazione per il rilascio di TYPO3 CMS come archivio zip o tar 
				può generare un installazione completamente funzionante, senza dover dipendere da \texttt{Packages/} di librerie di terze parti

		\item I problemi possono verificarsi con installazioni fatte via composer e l'uso di \texttt{phpunit}
			senza dipendenze del composer che è stato completamente rivisto. Per fissare questo, eseguire:

			\begin{lstlisting}
				# cd htdocs/
				# rm -rf typo3/contrib/vendor/ bin/ Packages/Libraries/ composer.lock
				# composer install
			\end{lstlisting}
	\end{itemize}

\end{frame}

% ------------------------------------------------------------------------------
% LTXE-SLIDE-START
% LTXE-SLIDE-UID:		d561f2f1-9b76bd0d-1e901ba6-7f3991b2
% LTXE-SLIDE-ORIGIN:	c9a5288c-f45aac1e-34606612-fdf6ec18 English
% LTXE-SLIDE-TITLE:		Introduce JavaScript notification API
% LTXE-SLIDE-REFERENCE:	Feature-66047-IntroduceJavascriptNotificationApi.rst
% ------------------------------------------------------------------------------
\begin{frame}[fragile]
	\frametitle{Modifiche rilevanti}
	\framesubtitle{Notifiche JavaScript}

	\begin{itemize}

		\item Sono state implementate delle nuove API per le notifiche via JavaScript:
			\begin{lstlisting}
				// vecchio e deprecato:
				top.TYPO3.Flashmessages.display(TYPO3.Severity.notice)

				// nuovo e unico modo corretto da TYPO3 CMS 7.2:
				top.TYPO3.Notification.notice(title, message)
    		\end{lstlisting}

		\item Sono presenti le seguenti funzioni delle API:\newline
			\small(parameter \texttt{duration} is optional and features a default value of 5 seconds)\normalsize
			\begin{itemize}
				\item \normalsize\smaller\texttt{top.TYPO3.Notification.notice(title, message, duration)}\normalsize
				\item \smaller\texttt{top.TYPO3.Notification.info(title, message, duration)}\normalsize
				\item \smaller\texttt{top.TYPO3.Notification.success(title, message, duration)}\normalsize
				\item \smaller\texttt{top.TYPO3.Notification.warning(title, message, duration)}\normalsize
				\item \smaller\texttt{top.TYPO3.Notification.error(title, message, duration)}\normalsize
			\end{itemize}

	\end{itemize}

\end{frame}

% ------------------------------------------------------------------------------
% LTXE-SLIDE-START
% LTXE-SLIDE-UID:		811f4903-e66d727b-539dacc6-0fd8dd79
% LTXE-SLIDE-ORIGIN:	b2acb5df-60b9a7dd-35cf697d-2a50d5db English
% LTXE-SLIDE-TITLE:		System Information Dropdown (1)
% LTXE-SLIDE-REFERENCE:	Feature-65767-SystemInformationDropdown.rst
% ------------------------------------------------------------------------------
\begin{frame}[fragile]
	\frametitle{Modifiche rilevanti}
	\framesubtitle{Tendina per informazioni di sistema (1)}

	% decrease font size for code listing
	\lstset{basicstyle=\tiny\ttfamily}

	\begin{itemize}
		\item Nuove voci su informazioni di sistema possono essere aggiunte alla tendina creando una nuova sezione

		\item La sezione deve essere registrata nel file \texttt{ext\_localconf.php}:

			\begin{lstlisting}
				$signalSlotDispatcher = \TYPO3\CMS\Core\Utility\GeneralUtility::makeInstance(
				  \TYPO3\CMS\Extbase\SignalSlot\Dispatcher::class);

				$signalSlotDispatcher->connect(
				  \TYPO3\CMS\Backend\Backend\ToolbarItems\SystemInformationToolbarItem::class,
				  'getSystemInformation',
				  \Vendor\Extension\SystemInformation\Item::class,
				  'getItem'
				);
			\end{lstlisting}

	\end{itemize}

\end{frame}

% ------------------------------------------------------------------------------
% LTXE-SLIDE-START
% LTXE-SLIDE-UID:		73f1cc80-da9fc8a2-5f044263-a90ba71f
% LTXE-SLIDE-ORIGIN:	e0f13209-22e0ff8c-7ff23bf9-31b1bb52 English
% LTXE-SLIDE-TITLE:		System Information Dropdown (2)
% LTXE-SLIDE-REFERENCE:	Feature-65767-SystemInformationDropdown.rst
% ------------------------------------------------------------------------------
\begin{frame}[fragile]
	\frametitle{Modifiche rilevanti}
	\framesubtitle{Tendina per informazioni di sistema (2)}

	% decrease font size for code listing
	\lstset{basicstyle=\tiny\ttfamily}

	\begin{itemize}

		\item Nuove voci su informazioni di sistema possono essere aggiunte alla tendina creando una nuova sezione

		\item Richiede la classe \texttt{Item} e i metodi \texttt{getItem()} nel file
			\small
				\texttt{EXT:extension\textbackslash
					Classes\textbackslash
					SystemInformation\textbackslash
					Item.php}:
			\normalsize

		\begin{lstlisting}
			class Item {
			  public function getItem() {
			    return array(array(
			      'title' => 'The title shown on hover',
			      'value' => 'Description shown in the list',
			      'status' => SystemInformationHookInterface::STATUS_OK,
			      'count' => 4,
			      'icon' => \TYPO3\CMS\Backend\Utility\IconUtility::getSpriteIcon(
				    'extensions-example-information-icon')
			    ));
			  }
			}
		\end{lstlisting}

	\end{itemize}

\end{frame}

% ------------------------------------------------------------------------------
% LTXE-SLIDE-START
% LTXE-SLIDE-UID:		1d54e32b-48c6a0ba-efbec558-3b2d446f
% LTXE-SLIDE-ORIGIN:	7d27df49-260500ef-0fc78a89-6e80069b English
% LTXE-SLIDE-TITLE:		System Information Dropdown (3)
% LTXE-SLIDE-REFERENCE:	Feature-65767-SystemInformationDropdown.rst
% ------------------------------------------------------------------------------
\begin{frame}[fragile]
	\frametitle{Modifiche rilevanti}
	\framesubtitle{Tendina per informazioni di sistema (3)}

	% decrease font size for code listing
	\lstset{basicstyle=\tiny\ttfamily}

	\begin{itemize}
		\item L'icona \texttt{extensions-example-information-icon} deve essere registrata in \texttt{ext\_localconf.php}:
		\begin{lstlisting}
			\TYPO3\CMS\Backend\Sprite\SpriteManager::addSingleIcons(
			  array(
			    'information-icon' => \TYPO3\CMS\Core\Utility\ExtensionManagementUtility::extRelPath(
			      $_EXTKEY) . 'Resources/Public/Images/Icons/information-icon.png'
			    ),
			   $_EXTKEY
			);
		\end{lstlisting}

	\end{itemize}

\end{frame}


% ------------------------------------------------------------------------------
% LTXE-SLIDE-START
% LTXE-SLIDE-UID:		a7eec3de-505933e1-288c26bd-542d2bce
% LTXE-SLIDE-ORIGIN:	d7449c27-fd9676cf-d12be80b-c24a4536 English
% LTXE-SLIDE-TITLE:		System Information Dropdown (4)
% LTXE-SLIDE-REFERENCE:	Feature-65767-SystemInformationDropdown.rst
% ------------------------------------------------------------------------------
\begin{frame}[fragile]
	\frametitle{Modifiche rilevanti}
	\framesubtitle{Tendina per informazioni di sistema (4)}

	% decrease font size for code listing
	\lstset{basicstyle=\tiny\ttfamily}

	\begin{itemize}

		\item I messaggi sono mostrati nella parte bassa della tendina

		\item Le estensioni possono gestire la propria sezione per visualizzare i messaggi:

		\begin{lstlisting}
			$signalSlotDispatcher = \TYPO3\CMS\Core\Utility\GeneralUtility::makeInstance(
			  \TYPO3\CMS\Extbase\SignalSlot\Dispatcher::class);

			$signalSlotDispatcher->connect(
			  \TYPO3\CMS\Backend\Backend\ToolbarItems\SystemInformationToolbarItem::class,
			  'loadMessages',
			  \Vendor\Extension\SystemInformation\Message::class,
			  'getMessage'
			);
		\end{lstlisting}

	\end{itemize}

\end{frame}

% ------------------------------------------------------------------------------
% LTXE-SLIDE-START
% LTXE-SLIDE-UID:		3d41164c-bd67609d-3c7bb065-f724b9b5
% LTXE-SLIDE-ORIGIN:	ab3b46ce-c7d7a228-7f7888af-4312fcdd English
% LTXE-SLIDE-TITLE:		System Information Dropdown (5)
% LTXE-SLIDE-REFERENCE:	Feature-65767-SystemInformationDropdown.rst
% ------------------------------------------------------------------------------
\begin{frame}[fragile]
	\frametitle{Modifiche rilevanti}
	\framesubtitle{Tendina per informazioni di sistema (5)}

	% decrease font size for code listing
	\lstset{basicstyle=\tiny\ttfamily}

	\begin{itemize}

		\item I messaggi sono mostrati nella parte bassa della tendina

		\item E' richiesta la classe \texttt{Message} e i suoi metodi \texttt{getMessage()} in file
			\small
				\texttt{EXT:extension\textbackslash
					Classes\textbackslash
					SystemInformation\textbackslash
					Message.php}:
			\normalsize

		\begin{lstlisting}
			class Message {
			  public function getMessage() {
			    return array(array(
			      'status' => SystemInformationHookInterface::STATUS_OK,
			      'text' => 'Something went wrong. Take a look at the reports module.'
			    ));
			  }
			}
		\end{lstlisting}

	\end{itemize}

\end{frame}


% ------------------------------------------------------------------------------
% LTXE-SLIDE-START
% LTXE-SLIDE-UID:		f24c780e-3c79324a-5804b099-b8cf24ad
% LTXE-SLIDE-ORIGIN:	98e2056a-d165b638-b1a1ee08-206bc15a English
% LTXE-SLIDE-TITLE:		Add image cropping
% LTXE-SLIDE-REFERENCE:	Feature-65584-AddImageCropping.rst
% ------------------------------------------------------------------------------
\begin{frame}[fragile]
	\frametitle{Modifiche rilevanti}
	\framesubtitle{Opzioni di configurazione per la manipolazione di immagini (1)}

	\begin{itemize}
		\item Le seguenti opzioni per la configurazione TypoScript sono disponibili:
			\begin{lstlisting}
				# disabilita il cropping per tutte le immagini
				tt_content.image.20.1.file.crop =

				# sovrascrivi o imposta il cropping per tutte le immagini
				# offsetX,offsetY,width,height
				tt_content.image.20.1.file.crop = 50,50,100,100
			\end{lstlisting}

		\item Anche Fluid gestisce la funzione di cropping:
			\begin{lstlisting}
				# disabilita il cropping per tutte le immagini
				<f:image image="{imageObject}" crop="" ></f:image>

				# sovrascrivi o imposta il cropping per tutte le immagini
				# offsetX,offsetY,width,height
				<f:image image="{imageObject}" crop="50,50,100,100" ></f:image>
			\end{lstlisting}

	\end{itemize}

\end{frame}

% ------------------------------------------------------------------------------
% LTXE-SLIDE-START
% LTXE-SLIDE-UID:		01744aff-e4eacd9d-feacd6ea-2f039103
% LTXE-SLIDE-ORIGIN:	e9cce085-6c300f1f-010de4ee-2c744704 English
% LTXE-SLIDE-TITLE:		Add image cropping (2)
% LTXE-SLIDE-REFERENCE:	Feature-65584-AddImageCropping.rst
% ------------------------------------------------------------------------------
\begin{frame}[fragile]
	\frametitle{Modifiche rilevanti}
	\framesubtitle{Opzioni di configurazione per la manipolazione di immagini (2)}

	% decrease font size for code listing
	\lstset{basicstyle=\smaller\ttfamily}

	\begin{itemize}
		\item Anche le funzionalità TCA delle immagini gestiscono il cropping:

			\begin{itemize}
				\item Column Type: \texttt{image\_manipulation}
				\item Config \texttt{file\_field}: string	\tabto{5.6cm}(default: \texttt{uid\_local})
				\item Config \texttt{enableZoom}: boolean	\tabto{5.6cm}(default: \texttt{FALSE})
				\item Config \texttt{allowedExtensions}: string\newline
					(default: \smaller\texttt{\$GLOBALS['TYPO3\_CONF\_VARS']['GFX']['imagefile\_ext']}\small)
				\item Config \texttt{ratios}: array, default:

					\begin{lstlisting}
						array(
						  '1.7777777777777777' => '16:9',
						  '1.3333333333333333' => '4:3',
						  '1' => '1:1',
						  'NaN' => 'Free'
						)
					\end{lstlisting}
			\end{itemize}

	\end{itemize}

\end{frame}

% ------------------------------------------------------------------------------
% LTXE-SLIDE-START
% LTXE-SLIDE-UID:		b42ad257-5700308d-58cec09f-86332758
% LTXE-SLIDE-ORIGIN:	ed863269-574e4061-d69f855e-6494c062 English
% LTXE-SLIDE-TITLE:		Additional params for HTMLparser userFunc
% LTXE-SLIDE-REFERENCE:	Feature-59712-HtmlParserAdditionalUserFuncParams.rst
% ------------------------------------------------------------------------------
\begin{frame}[fragile]
	\frametitle{Modifiche rilevanti}
	\framesubtitle{Parametri aggiuntivi per la userFunc HTMLparser}

	% decrease font size for code listing
	\lstset{basicstyle=\tiny\ttfamily}

	\begin{itemize}

		\item I parametri aggiuntivi che possono essere gestiti nella userFunc di HTMLparser:

			\begin{lstlisting}
				myobj = TEXT
				myobj.value = <a href="/" class="myclass">MyText</a>
				myobj.HTMLparser.tags.a.fixAttrib.class {
				  userFunc = Tx\MyExt\Myclass->htmlUserFunc
				  userFunc.myparam = test
				}
			\end{lstlisting}

		\item Accedi a questi parametri da un estensione nel seguente modo:

			\begin{lstlisting}
				function htmlUserFunc(array $params, HtmlParser $htmlParser) {
				  // $params['attributeValue'] contains the attribute value "myclass"
				  // $params['myparam'] is set to "test" in this example
				  ...
				}
			\end{lstlisting}

	\end{itemize}

\end{frame}

% ------------------------------------------------------------------------------
% LTXE-SLIDE-START
% LTXE-SLIDE-UID:		0f769274-a3682920-bd405a24-c881fa60
% LTXE-SLIDE-ORIGIN:	b04f8d44-fd695c73-b768f3cb-fd6d8643 English
% LTXE-SLIDE-TITLE:		New Locking API (1)
% LTXE-SLIDE-REFERENCE:	Feature-47712-NewLockingAPI.rst
% ------------------------------------------------------------------------------
\begin{frame}[fragile]
	\frametitle{Modifiche rilevanti}
	\framesubtitle{Locking API (1)}

	\begin{itemize}

		\item Sono state introdotte delle nuove Locking API, che permettono vari metodi di blocco (SimpleFile, Semaphore, ...)
		\item Un metodo di blocco deve implementare \small\texttt{LockingStrategyInterface}\normalsize:
		\begin{lstlisting}
			$lockFactory = GeneralUtility::makeInstance(LockFactory::class);
			$locker = $lockFactory->createLocker('someId');
			$locker->acquire() || die('Could not acquire lock.');
			...
			$locker->release();
		\end{lstlisting}

	\end{itemize}

\end{frame}

% ------------------------------------------------------------------------------
% LTXE-SLIDE-START
% LTXE-SLIDE-UID:		84067030-2d8129d6-9db26d3e-7aeea16c
% LTXE-SLIDE-ORIGIN:	79a00412-c72401e6-4d9667c9-0e15d3f4 English
% LTXE-SLIDE-TITLE:		New Locking API (2)
% LTXE-SLIDE-REFERENCE:	Feature-47712-NewLockingAPI.rst
% ------------------------------------------------------------------------------
\begin{frame}[fragile]
	\frametitle{Modifiche rilevanti}
	\framesubtitle{Locking API (2)}

	% decrease font size for code listing
	\lstset{basicstyle=\tiny\ttfamily}

	\begin{itemize}

		\item Alcuni metodi gestiscono non-blocking locks:

		\begin{lstlisting}
			$lockFactory = GeneralUtility::makeInstance(LockFactory::class);
			$locker = $lockFactory->createLocker(
			  'someId',
			  LockingStrategyInterface::LOCK_CAPABILITY_SHARED |
			    LockingStrategyInterface::LOCK_CAPABILITY_NOBLOCK
			);
			try {
			  $result = $locker->acquire(LockingStrategyInterface::LOCK_CAPABILITY_SHARED | LockingStrategyInterface::LOCK_CAPABILITY_NOBLOCK);
			  catch (\RuntimeException $e) {
			  if ($e->getCode() === 1428700748) {
			    // some process owns the lock
			    // let's do something else meanwhile
			    ...
			  }
			}
			if ($result) {
			  $locker->release();
			}
		\end{lstlisting}

	\end{itemize}

\end{frame}

% ------------------------------------------------------------------------------
% LTXE-SLIDE-START
% LTXE-SLIDE-UID:		a811f378-59400e77-2ed3cc5f-e70d2734
% LTXE-SLIDE-ORIGIN:	108cb4a7-6e006507-c25a91e7-3a867f09 English
% LTXE-SLIDE-TITLE:		Add signal after extension installation
% LTXE-SLIDE-REFERENCE:	commit 39e0fd0a2c919da270cb49223433bcf95febbb10
% ------------------------------------------------------------------------------
\begin{frame}[fragile]
	\frametitle{Modifiche rilevanti}
	\framesubtitle{Signal dopo Extension Installation}

	% decrease font size for code listing
	\lstset{basicstyle=\tiny\ttfamily}

	\begin{itemize}

		\item Sono stati implementati nuovi signal nei metodi
			\smaller
				\texttt{\textbackslash
					TYPO3\textbackslash
					CMS\textbackslash
					Extensionmanager\textbackslash
					Utility\textbackslash
					InstallUtility::install()}
			\normalsize
			richiamato appena una estensione è stata installata e tutti gli imports/updates
			eseguiti

		\begin{lstlisting}
			// execution
			$this->emitAfterExtensionInstallSignal($extensionKey);

			// methode
			protected function emitAfterExtensionInstallSignal($extensionKey) {
			  $this->signalSlotDispatcher->dispatch(
			    __CLASS__,
			    'afterExtensionInstall',
			    array($extensionKey, $this)
			  );
			}
		\end{lstlisting}

	\end{itemize}

\end{frame}

% ------------------------------------------------------------------------------
% LTXE-SLIDE-START
% LTXE-SLIDE-UID:		150601e1-797c26a6-fc446612-93a2640d
% LTXE-SLIDE-ORIGIN:	acb1ba95-1a401bef-549072a1-f32263a9 English
% LTXE-SLIDE-TITLE:		Registry for adding text extractor classes (1)
% LTXE-SLIDE-REFERENCE:	Feature-36743-FAL-TextExtractorRegistry.rst
% ------------------------------------------------------------------------------
\begin{frame}[fragile]
	\frametitle{Modifiche rilevanti}
	\framesubtitle{Registri per estrazione di testo (1)}

	% decrease font size for code listing
	\lstset{basicstyle=\tiny\ttfamily}

	\begin{itemize}

		\item Vari estrattori di testo possono essere registrati per permettere la gestione
			di differenti tipi di file (e.g. Office, file PDF, etc.)

		\item Il core di TYPO3 dispone di un estrattore per file di testo

		\item Ogni classe registrata per estrarre testo deve implementare \texttt{TextExtractorInterface}

		\item ...e i seguenti metodi:\newline
			\texttt{canExtractText()}\newline
			\small
				checks if text extraction from the given file is possible
			\normalsize
			\newline
			\texttt{extractText()}\newline
			\small
				returns the file's text content as a string
			\normalsize

	\end{itemize}

\end{frame}

% ------------------------------------------------------------------------------
% LTXE-SLIDE-START
% LTXE-SLIDE-UID:		f7fe7a8f-f7cb5a52-ca3c73cc-9a775a14
% LTXE-SLIDE-ORIGIN:	acb1ba95-1a401bef-549072a1-f32263a9 English
% LTXE-SLIDE-TITLE:		Registry for adding text extractor classes (2)
% LTXE-SLIDE-REFERENCE:	Feature-36743-FAL-TextExtractorRegistry.rst
% ------------------------------------------------------------------------------
\begin{frame}[fragile]
	\frametitle{Modifiche rilevanti}
	\framesubtitle{Registri per estrazione di testo (2)}

	% decrease font size for code listing
	\lstset{basicstyle=\tiny\ttfamily}

	\begin{itemize}

		\item Gli estrattori di testo vanno registrati nel file \texttt{ext\_localconf.php}:

			\begin{lstlisting}
				$textExtractorRegistry = \TYPO3\CMS\Core\Resource\TextExtraction\TextExtractorRegistry::getInstance();
				$textExtractorRegistry->registerTextExtractor(
				  \TYPO3\CMS\Core\Resource\TextExtraction\PlainTextExtractor::class
				);
			\end{lstlisting}

		\item Usa come di seguito:

			\begin{lstlisting}
				$textExtractorRegistry = \TYPO3\CMS\Core\Resource\TextExtraction\TextExtractorRegistry::getInstance();
				$extractor = $textExtractorRegistry->getTextExtractor($file);
				if($extractor !== NULL) {
				  $content = $extractor->extractText($file);
				}
			\end{lstlisting}
	\end{itemize}

\end{frame}

% ------------------------------------------------------------------------------
% LTXE-SLIDE-START
% LTXE-SLIDE-UID:		c55e7853-0a711829-aa05950c-c01958c9
% LTXE-SLIDE-ORIGIN:	d9657b5d-5cd56c65-d7266a27-401dd233 English
% LTXE-SLIDE-TITLE:		Miscellaneous
% LTXE-SLIDE-REFERENCE:	Feature-66042-WebLibrariesLoadedViaBower.rst
% LTXE-SLIDE-REFERENCE:	Feature-63703-AddOptionToStopTask.rst
% LTXE-SLIDE-REFERENCE:	Feature-61463-AllowProcessedFoldersInDifferentStorage.rst
% LTXE-SLIDE-REFERENCE:	Feature-52693-TSFE-RequestedId.rst
% ------------------------------------------------------------------------------

\begin{frame}[fragile]
	\frametitle{Modifiche rilevanti}
	\framesubtitle{Varie}

	\begin{itemize}

		\item Le librerie web (ad esempio Twitter Bootstrap, jQuery, Font Awesome, etc.) usano
			"Bower" (\url{http://bower.io}) e non sono più parti del repository git del core TYPO3\newline
			\small
				\texttt{\# bower install}	\tabto{3.4cm}esegue un installazione\newline
				\texttt{\# bower update}		\tabto{3.4cm}esegue un aggiornamento\newline
			\normalsize
			(il file \texttt{bower.json} è posizinato nella directory \texttt{Build/})

		\item Scheduler CLI dispone dell'opzione "\texttt{-s}" per bloccare un task in esecuzione

		\item La gestione di un archivio di directory (remoto) può essere fuori dall'archivio
			(utile per archivi di sola lettura)

		\item E' possibile disporre dell'ID di pagina delle pagina originariamente richiesta:
			\texttt{\$TSFE->getRequestedId()}

	\end{itemize}

\end{frame}

% ------------------------------------------------------------------------------
