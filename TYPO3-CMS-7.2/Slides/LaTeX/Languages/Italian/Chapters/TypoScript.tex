% ------------------------------------------------------------------------------
% TYPO3 CMS 7.2 - What's New - Chapter "TypoScript" (English Version)
%
% @author	Michael Schams <schams.net>
% @license	Creative Commons BY-NC-SA 3.0
% @link		http://typo3.org/download/release-notes/whats-new/
% @language	English
% ------------------------------------------------------------------------------
% LTXE-CHAPTER-UID:		12518a77-2b90a173-4e3f8420-485e0497
% LTXE-CHAPTER-NAME:	TypoScript
% ------------------------------------------------------------------------------

\section{TSconfig \& TypoScript}
\begin{frame}[fragile]
	\frametitle{TSconfig \& TypoScript}

	\begin{center}\huge{Capitolo 2:}\end{center}
	\begin{center}\huge{\color{typo3darkgrey}\textbf{TSconfig \& TypoScript}}\end{center}

\end{frame}

% ------------------------------------------------------------------------------
% LTXE-SLIDE-START
% LTXE-SLIDE-UID:		b3a111c8-076f8427-87f28805-3e932d48
% LTXE-SLIDE-ORIGIN:	56b025e6-cf380b11-c9d1c009-0b548d6a English
% LTXE-SLIDE-TITLE:		Add flexible preview URL configuration (1)
% LTXE-SLIDE-REFERENCE:	Feature-66370-AddFlexiblePreviewUrlConfiguration.rst
% ------------------------------------------------------------------------------
\begin{frame}[fragile]
	\frametitle{TSconfig \& TypoScript}
	\framesubtitle{Configuratore dell'url di anteprima (1)}

	% decrease font size for code listing
	\lstset{basicstyle=\tiny\ttfamily}

	\begin{itemize}

		\item E' ora possibile configurare il generatore di link di anteprima per\newline
			il bottone di backend "salva \& vedi".

		\item Un caso frequente è quello di avere anteprime per blog o news, ma si
			possono definire anche differenti anteprima di pagina per elementi di contenuto classici.

			\begin{lstlisting}
				TCEMAIN.preview {
				  <table name> {
				    previewPageId = 123
				    useDefaultLanguageRecord = 0
				    fieldToParameterMap {
				      uid = tx_myext_pi1[showUid]
				    }
				    additionalGetParameters {
				      tx_myext_pi1[special] = HELLO
				    }
				  }
				}
			\end{lstlisting}

	\end{itemize}

\end{frame}

% ------------------------------------------------------------------------------
% LTXE-SLIDE-START
% LTXE-SLIDE-UID:		a66afe78-82a1951b-7e406d9e-08f55a98
% LTXE-SLIDE-ORIGIN:	6822c376-37649cd8-0acc836a-b856fc61 English
% LTXE-SLIDE-TITLE:		Add flexible preview URL configuration (2)
% LTXE-SLIDE-REFERENCE:	Feature-66370-AddFlexiblePreviewUrlConfiguration.rst
% ------------------------------------------------------------------------------
\begin{frame}[fragile]
	\frametitle{TSconfig \& TypoScript}
	\framesubtitle{Configuratore dell'url di anteprima (2)}

	\begin{itemize}
		\item \texttt{previewPageId}:\newline
			\smaller
				UID della pagina da usare per l'anteprima\newline
				(se questa impostazione non è presente viene usata la pagina corrente)
			\normalsize
		\item \texttt{useDefaultLanguageRecord}:\newline
			\smaller
				definisce che se i record sono tradotti, sarà utilizzato l'UID del record di default\newline
				(questa è attivata di default, valore: 1)
			\normalsize
		\item \texttt{fieldToParameterMap}:\newline
			\smaller
				una mappatura che consente di selezionare i campi del record da inserire come parametro GET
			\normalsize
		\item \texttt{additionalGetParameters}:\newline
			\smaller
				permette di aggiungere parametri GET personalizzati e di ignorarne altri
			\normalsize
	\end{itemize}

\end{frame}

% ------------------------------------------------------------------------------
% LTXE-SLIDE-START
% LTXE-SLIDE-UID:		b9fbc711-8a3cf570-52a59c85-5e03f894
% LTXE-SLIDE-ORIGIN:	c5dbf3c7-655fc6e4-227a3bf6-f5d89fb1 English
% LTXE-SLIDE-TITLE:
% LTXE-SLIDE-REFERENCE:	Feature-59646-AddRteConfigurationPropertyButtonsLinkTypePropertiesTargetDefault.rst
% ------------------------------------------------------------------------------
\begin{frame}[fragile]
	\frametitle{TSconfig \& TypoScript}
	\framesubtitle{Configurazione RTE: Target di Default}

	\begin{itemize}

		\item Le proprietà di configurazione RTE possono essere usate in PageTSconfig per configurare un target
			di default per i link di un determinato tipo\newline

			\small
				\texttt{buttons.link.[}
				\textit{type}
				\texttt{].properties.target.default = ...}
			\normalsize\newline

		\item I tipi possibili di link sono:\newline
			\small
				(altri tipi possono essere forniti dalle estensioni)
			\normalsize

			\begin{itemize}
				\item \texttt{page}
				\item \texttt{file}
				\item \texttt{url}
				\item \texttt{mail}
				\item \texttt{spec}
			\end{itemize}
	\end{itemize}

\end{frame}

% ------------------------------------------------------------------------------
% LTXE-SLIDE-START
% LTXE-SLIDE-UID:		0009fd8f-a2585c96-5a8c99b6-e53222df
% LTXE-SLIDE-ORIGIN:	fd957580-301e4a0a-ad5325f5-ffa024c3 English
% LTXE-SLIDE-TITLE:		Strip empty HTML tags in HtmlParser
% LTXE-SLIDE-REFERENCE:	Feature-20555-StripEmptyHtmlTags.rst
% ------------------------------------------------------------------------------
\begin{frame}[fragile]
	\frametitle{TSconfig \& TypoScript}
	\framesubtitle{Cancella tag HTML vuoti nell'HTMLparser}

	% decrease font size for code listing
	\lstset{basicstyle=\tiny\ttfamily}

	\begin{itemize}
		\item Una nuova funzionalità è stata implementata nell'HTMLparser che permette di
			cancellare tag HTML vuoti.

			\begin{lstlisting}
				stdWrap {
				   // rimuove tutti i tag HTML vuoti
				   HTMLparser.stripEmptyTags = 1
				   // rimuove solo i tag h2 e h3 vuoti
				   HTMLparser.stripEmptyTags.tags = h2, h3
				}

				RTE.default.proc.entryHTMLparser_db {
				   stripEmptyTags = 1
				   stripEmptyTags.tags = p
				   stripEmptyTags.treatNonBreakingSpaceAsEmpty = 1
				}
			\end{lstlisting}

			\underline{\textbf{Nota:}}
				L'HTMLparser cancella tutti i tag sconosciuti di default.\newline
				Pertanto potrebbe essere utile impostare questo:\newline
				\texttt{HTMLparser.keepNonMatchedTags = 1}

	\end{itemize}

\end{frame}

% ------------------------------------------------------------------------------
% LTXE-SLIDE-START
% LTXE-SLIDE-UID:		fa68e416-7cd68361-67471075-dbb35920
% LTXE-SLIDE-ORIGIN:	62960e76-c3dca953-fe5a1070-831633fc English
% LTXE-SLIDE-TITLE:		Miscellaneous
% LTXE-SLIDE-REFERENCE:	Feature-63040-AddRteConfigurationPropertyButtonsAbbreviationRemoveFieldsets.rst
% LTXE-SLIDE-REFERENCE:	commit 31c62f9311ee3d33bf792d548cc4a5fac83aa3d0
% ------------------------------------------------------------------------------
\begin{frame}[fragile]
	\frametitle{TSconfig \& TypoScript}
	\framesubtitle{Varie}

	\begin{itemize}
		\item Una nuova proprietà \texttt{buttons.abbreviation.removeFieldsets} può essere usata in
			PageTSconfig per configurare la finestra di dialogo con le sigle

			\begin{lstlisting}
				# Valori possibili sono:
				# acronym, definedAcronym, abbreviation, definedAbbreviation
				buttons.abbreviation.removeFieldsets = acronym,definedAcronym
			\end{lstlisting}

		\item La proprietà \texttt{inlineLanguageLabel} dell'oggetto \texttt{PAGE} è in grado\newline
			di gestire i riferimenti \texttt{LLL:}

	\end{itemize}

\end{frame}

% ------------------------------------------------------------------------------
