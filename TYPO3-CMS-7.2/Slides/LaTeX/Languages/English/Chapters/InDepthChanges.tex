% ------------------------------------------------------------------------------
% TYPO3 CMS 7.2 - What's New - Chapter "In-Depth Changes" (English Version)
%
% @author	Michael Schams <schams.net>
% @license	Creative Commons BY-NC-SA 3.0
% @link		http://typo3.org/download/release-notes/whats-new/
% @language	English
% ------------------------------------------------------------------------------
% LTXE-CHAPTER-UID:		e28087f2-4bffcc7b-3c685cca-579f18a8
% LTXE-CHAPTER-NAME:	In-Depth Changes
% ------------------------------------------------------------------------------

\section{In-Depth Changes}
\begin{frame}[fragile]
	\frametitle{In-Depth Changes}

	\begin{center}\huge{Chapter 3:}\end{center}
	\begin{center}\huge{\color{typo3darkgrey}\textbf{In-Depth Changes}}\end{center}

\end{frame}

% ------------------------------------------------------------------------------
% LTXE-SLIDE-START
% LTXE-SLIDE-UID:		547708aa-92282fd1-6c27d618-4f13f86b
% LTXE-SLIDE-TITLE:		Add SVG support
% LTXE-SLIDE-REFERENCE:	Feature-50136-AddSVGSupport.rst
% ------------------------------------------------------------------------------
\begin{frame}[fragile]
	\frametitle{In-Depth Changes}
	\framesubtitle{SVG Support}

	\begin{itemize}
		\item TYPO3 CMS core supports SVG images now ("Scalable Vector Graphics")

		\item When an SVG image is scaled, a record with the calculated new dimensions
			is stored in \texttt{sys\_file\_processedfile} rather than creating a
			processed file\newline
			\small(except, is image is processed further, e.g. cropping)\normalsize.

		\item A fallback is added to determine SVG dimensions if ImageMagick/GraphicsMagick
			can not determine image dimensions. In this case, the contents of the XML file
			is read.

		\item SVG has also been added to the list of valid image files:\newline
			\texttt{\$GLOBALS['TYPO3\_CONF\_VARS']['GFX']['imagefile\_ext']}

	\end{itemize}

\end{frame}

% ------------------------------------------------------------------------------
% LTXE-SLIDE-START
% LTXE-SLIDE-UID:		bd03e141-3842b75e-8a44920f-a8c139e5
% LTXE-SLIDE-TITLE:		Add count methods and sort functionality to FAL drivers
% LTXE-SLIDE-REFERENCE:	Breaking-56746-AddCountMethodsAndSortFunctionalityToFalDrivers.rst
% ------------------------------------------------------------------------------
\begin{frame}[fragile]
	\frametitle{In-Depth Changes}
	\framesubtitle{Extending FAL Driver}

	% decrease font size for code listing
	\lstset{basicstyle=\tiny\ttfamily}

	\begin{itemize}
		\item In order to improve the performance of the file list when showing (remote) storages
			the FAL driver should take care of sorting, ordering and determining the number
			of files/folders. Two new parameters \texttt{sort} and \texttt{sortRev} have been
			added to allow that:

			\begin{lstlisting}
				public function getFilesInFolder($folderIdentifier, $start = 0, $numberOfItems = 0,
				  $recursive = FALSE, array $filenameFilterCallbacks = array(), $sort = '', $sortRev = FALSE);

				public function getFoldersInFolder($folderIdentifier, $start = 0, $numberOfItems = 0,
				  $recursive = FALSE, array $folderNameFilterCallbacks = array(), $sort = '', $sortRev = FALSE);
			\end{lstlisting}

		\item Additionally, two new methods have been implemented:

			\begin{lstlisting}
				public function getFilesInFolderCount($folderIdentifier, $recursive = FALSE,
				  array $filenameFilterCallbacks = array());

				public function getFoldersInFolderCount($folderIdentifier, $recursive = FALSE,
				  array $folderNameFilterCallbacks = array());
			\end{lstlisting}

	\end{itemize}

\end{frame}

% ------------------------------------------------------------------------------
% LTXE-SLIDE-START
% LTXE-SLIDE-UID:		d0df6060-4b8d33a9-69adad59-422e9ada
% LTXE-SLIDE-TITLE:		Introduce Backend Routing (1)
% LTXE-SLIDE-REFERENCE:	commit a08ce7238e583d1962edfe998e04c7d9c3d7c2ae
% ------------------------------------------------------------------------------
\begin{frame}[fragile]
	\frametitle{In-Depth Changes}
	\framesubtitle{Backend Routing API (1)}

	\begin{itemize}
		\item A Backend Routing API has been implemented, which manages the backend Entry Points
		\item Inspired by the Symfony Routing Framework, this API is compatible with it to a large extent\newline
			\small(however TYPO3 uses only approx. 20\% at this point in time)\normalsize

		\item Basically three classes implement the functionality:
			\begin{itemize}
				\item \texttt{class Route}:\tabto{3.6cm}contains details about path and options
				\item \texttt{class Router}:\tabto{3.6cm}API to match the route
				\item \texttt{class UrlGenerator}:\tabto{3.6cm}generates the URL
			\end{itemize}
	\end{itemize}

\end{frame}

% ------------------------------------------------------------------------------
% LTXE-SLIDE-START
% LTXE-SLIDE-UID:		fe1eec67-bbf31ed9-57b8e83a-3a922f0e
% LTXE-SLIDE-TITLE:		Introduce Backend Routing (2)
% LTXE-SLIDE-REFERENCE:	commit a08ce7238e583d1962edfe998e04c7d9c3d7c2ae
% ------------------------------------------------------------------------------
\begin{frame}[fragile]
	\frametitle{In-Depth Changes}
	\framesubtitle{Backend Routing API (2)}

	\begin{itemize}

		\item Routes are defined in the following file of an extension:
			\texttt{Configuration/Backend/Routes.php}\newline
			(see system extension \texttt{backend} as an example)

		\item Further details about the Backend Routing API:\newline
			\small\url{http://wiki.typo3.org/Blueprints/BackendRouting}\normalsize

	\end{itemize}

\end{frame}

% ------------------------------------------------------------------------------
% LTXE-SLIDE-START
% LTXE-SLIDE-UID:		9d56ca19-04a98896-ba0a973c-8a16e877
% LTXE-SLIDE-TITLE:		New system extension: mediace
% LTXE-SLIDE-REFERENCE:	Breaking-64719-MediaContentMovedToSystemExtension.rst
% LTXE-SLIDE-REFERENCE:	Breaking-65778-MediaWizardProviderMovedToSystemExtension.rst
% ------------------------------------------------------------------------------
\begin{frame}[fragile]
	\frametitle{In-Depth Changes}
	\framesubtitle{New System Extension for Media Content Elements}

	\begin{itemize}

		\item New system extension "\texttt{mediace}" contains the following cObjects:

			\begin{itemize}
				\item \texttt{MULTIMEDIA}
				\item \texttt{MEDIA}
				\item \texttt{SWFOBJECT}
				\item \texttt{FLOWPLAYER}
				\item \texttt{QTOBJECT}
			\end{itemize}

		\item Content elements \texttt{media} and \texttt{multimedia} have been moved to
			this system extension, too, as well as the "Media Wizard Provider"

		\item This extension is \underline{\textbf{not}} installed by default!

	\end{itemize}

\end{frame}

% ------------------------------------------------------------------------------
% LTXE-SLIDE-START
% LTXE-SLIDE-UID:		cc63cd98-524ee6df-38430963-f7c7067c
% LTXE-SLIDE-TITLE:		Composer vendor directory
% LTXE-SLIDE-REFERENCE:	Breaking-66001-ComposerVendorDirectoryChanged.rst
% ------------------------------------------------------------------------------
\begin{frame}[fragile]
	\frametitle{In-Depth Changes}
	\framesubtitle{Location of Third-party Libraries}

	\begin{itemize}

		\item Composer-installed third-party libraries are installed under \texttt{typo3/contrib/vendor} now\newline
			\small
				(TYPO3 CMS < 7.2: in folder \texttt{Packages/Libraries})
			\normalsize

		\item This way the packaging process for releasing TYPO3 CMS as tarball or zip can trigger
			a fully working installation without having to ship \texttt{Packages/} for third-party libraries

		\item Problems may occur with installations if they were set up via composer and use \texttt{phpunit}
			unless composer dependencies have been completely rebuilt. To fix this, execute:

			\begin{lstlisting}
				# cd htdocs/
				# rm -rf typo3/contrib/vendor/ bin/ Packages/Libraries/ composer.lock
				# composer install
			\end{lstlisting}
	\end{itemize}

\end{frame}

% ------------------------------------------------------------------------------
% LTXE-SLIDE-START
% LTXE-SLIDE-UID:		c9a5288c-f45aac1e-34606612-fdf6ec18
% LTXE-SLIDE-TITLE:		Introduce JavaScript notification API
% LTXE-SLIDE-REFERENCE:	Feature-66047-IntroduceJavascriptNotificationApi.rst
% ------------------------------------------------------------------------------
\begin{frame}[fragile]
	\frametitle{In-Depth Changes}
	\framesubtitle{JavaScript Notifications}

	\begin{itemize}

		\item A new JavaScript Notification API has been implemented:
			\begin{lstlisting}
				// old and deprecated:
				top.TYPO3.Flashmessages.display(TYPO3.Severity.notice)

				// new and the only correct way since TYPO3 CMS 7.2:
				top.TYPO3.Notification.notice(title, message)
    		\end{lstlisting}

		\item The following API functions exist:\newline
			\small(parameter \texttt{duration} is optional and features a default value of 5 seconds)\normalsize
			\begin{itemize}
				\item \normalsize\smaller\texttt{top.TYPO3.Notification.notice(title, message, duration)}\normalsize
				\item \smaller\texttt{top.TYPO3.Notification.info(title, message, duration)}\normalsize
				\item \smaller\texttt{top.TYPO3.Notification.success(title, message, duration)}\normalsize
				\item \smaller\texttt{top.TYPO3.Notification.warning(title, message, duration)}\normalsize
				\item \smaller\texttt{top.TYPO3.Notification.error(title, message, duration)}\normalsize
			\end{itemize}

	\end{itemize}

\end{frame}

% ------------------------------------------------------------------------------
% LTXE-SLIDE-START
% LTXE-SLIDE-UID:		b2acb5df-60b9a7dd-35cf697d-2a50d5db
% LTXE-SLIDE-TITLE:		System Information Dropdown (1)
% LTXE-SLIDE-REFERENCE:	Feature-65767-SystemInformationDropdown.rst
% ------------------------------------------------------------------------------
\begin{frame}[fragile]
	\frametitle{In-Depth Changes}
	\framesubtitle{System Information Dropdown (1)}

	% decrease font size for code listing
	\lstset{basicstyle=\tiny\ttfamily}

	\begin{itemize}
		\item Custom system information items can be added to the dropdown by creating a slot

		\item The slot must be registered in file \texttt{ext\_localconf.php}:

			\begin{lstlisting}
				$signalSlotDispatcher = \TYPO3\CMS\Core\Utility\GeneralUtility::makeInstance(
				  \TYPO3\CMS\Extbase\SignalSlot\Dispatcher::class);

				$signalSlotDispatcher->connect(
				  \TYPO3\CMS\Backend\Backend\ToolbarItems\SystemInformationToolbarItem::class,
				  'getSystemInformation',
				  \Vendor\Extension\SystemInformation\Item::class,
				  'getItem'
				);
			\end{lstlisting}

	\end{itemize}

\end{frame}

% ------------------------------------------------------------------------------
% LTXE-SLIDE-START
% LTXE-SLIDE-UID:		e0f13209-22e0ff8c-7ff23bf9-31b1bb52
% LTXE-SLIDE-TITLE:		System Information Dropdown (2)
% LTXE-SLIDE-REFERENCE:	Feature-65767-SystemInformationDropdown.rst
% ------------------------------------------------------------------------------
\begin{frame}[fragile]
	\frametitle{In-Depth Changes}
	\framesubtitle{System Information Dropdown (2)}

	% decrease font size for code listing
	\lstset{basicstyle=\tiny\ttfamily}

	\begin{itemize}

		\item Custom system information items can be added to the dropdown by creating a slot

		\item This requires class \texttt{Item} and its method \texttt{getItem()} in file
			\small
				\texttt{EXT:extension\textbackslash
					Classes\textbackslash
					SystemInformation\textbackslash
					Item.php}:
			\normalsize

		\begin{lstlisting}
			class Item {
			  public function getItem() {
			    return array(array(
			      'title' => 'The title shown on hover',
			      'value' => 'Description shown in the list',
			      'status' => SystemInformationHookInterface::STATUS_OK,
			      'count' => 4,
			      'icon' => \TYPO3\CMS\Backend\Utility\IconUtility::getSpriteIcon(
				    'extensions-example-information-icon')
			    ));
			  }
			}
		\end{lstlisting}

	\end{itemize}

\end{frame}

% ------------------------------------------------------------------------------
% LTXE-SLIDE-START
% LTXE-SLIDE-UID:		7d27df49-260500ef-0fc78a89-6e80069b
% LTXE-SLIDE-TITLE:		System Information Dropdown (3)
% LTXE-SLIDE-REFERENCE:	Feature-65767-SystemInformationDropdown.rst
% ------------------------------------------------------------------------------
\begin{frame}[fragile]
	\frametitle{In-Depth Changes}
	\framesubtitle{System Information Dropdown (3)}

	% decrease font size for code listing
	\lstset{basicstyle=\tiny\ttfamily}

	\begin{itemize}
		\item Icon \texttt{extensions-example-information-icon} must be registered in \texttt{ext\_localconf.php}:
		\begin{lstlisting}
			\TYPO3\CMS\Backend\Sprite\SpriteManager::addSingleIcons(
			  array(
			    'information-icon' => \TYPO3\CMS\Core\Utility\ExtensionManagementUtility::extRelPath(
			      $_EXTKEY) . 'Resources/Public/Images/Icons/information-icon.png'
			    ),
			   $_EXTKEY
			);
		\end{lstlisting}

	\end{itemize}

\end{frame}


% ------------------------------------------------------------------------------
% LTXE-SLIDE-START
% LTXE-SLIDE-UID:		d7449c27-fd9676cf-d12be80b-c24a4536
% LTXE-SLIDE-TITLE:		System Information Dropdown (4)
% LTXE-SLIDE-REFERENCE:	Feature-65767-SystemInformationDropdown.rst
% ------------------------------------------------------------------------------
\begin{frame}[fragile]
	\frametitle{In-Depth Changes}
	\framesubtitle{System Information Dropdown (4)}

	% decrease font size for code listing
	\lstset{basicstyle=\tiny\ttfamily}

	\begin{itemize}

		\item Messages are shown at the bottom of the dropdown

		\item Extensions can provide its own slot to fill messages:

		\begin{lstlisting}
			$signalSlotDispatcher = \TYPO3\CMS\Core\Utility\GeneralUtility::makeInstance(
			  \TYPO3\CMS\Extbase\SignalSlot\Dispatcher::class);

			$signalSlotDispatcher->connect(
			  \TYPO3\CMS\Backend\Backend\ToolbarItems\SystemInformationToolbarItem::class,
			  'loadMessages',
			  \Vendor\Extension\SystemInformation\Message::class,
			  'getMessage'
			);
		\end{lstlisting}

	\end{itemize}

\end{frame}

% ------------------------------------------------------------------------------
% LTXE-SLIDE-START
% LTXE-SLIDE-UID:		ab3b46ce-c7d7a228-7f7888af-4312fcdd
% LTXE-SLIDE-TITLE:		System Information Dropdown (5)
% LTXE-SLIDE-REFERENCE:	Feature-65767-SystemInformationDropdown.rst
% ------------------------------------------------------------------------------
\begin{frame}[fragile]
	\frametitle{In-Depth Changes}
	\framesubtitle{System Information Dropdown (5)}

	% decrease font size for code listing
	\lstset{basicstyle=\tiny\ttfamily}

	\begin{itemize}

		\item Messages are shown at the bottom of the dropdown

		\item This requires the class \texttt{Message} and its method \texttt{getMessage()} in file
			\small
				\texttt{EXT:extension\textbackslash
					Classes\textbackslash
					SystemInformation\textbackslash
					Message.php}:
			\normalsize

		\begin{lstlisting}
			class Message {
			  public function getMessage() {
			    return array(array(
			      'status' => SystemInformationHookInterface::STATUS_OK,
			      'text' => 'Something went wrong. Take a look at the reports module.'
			    ));
			  }
			}
		\end{lstlisting}

	\end{itemize}

\end{frame}


% ------------------------------------------------------------------------------
% LTXE-SLIDE-START
% LTXE-SLIDE-UID:		98e2056a-d165b638-b1a1ee08-206bc15a
% LTXE-SLIDE-TITLE:		Add image cropping
% LTXE-SLIDE-REFERENCE:	Feature-65584-AddImageCropping.rst
% ------------------------------------------------------------------------------
\begin{frame}[fragile]
	\frametitle{In-Depth Changes}
	\framesubtitle{Image Manipulation Configuration Options (1)}

	\begin{itemize}
		\item The following TypoScript configuration options are available:
			\begin{lstlisting}
				# disable cropping for all images
				tt_content.image.20.1.file.crop =

				# override or set cropping for all images
				# offsetX,offsetY,width,height
				tt_content.image.20.1.file.crop = 50,50,100,100
			\end{lstlisting}

		\item Fluid also supports cropping functions:
			\begin{lstlisting}
				# disable cropping for all images
				<f:image image="{imageObject}" crop="" ></f:image>

				# override or set cropping for all images
				# offsetX,offsetY,width,height
				<f:image image="{imageObject}" crop="50,50,100,100" ></f:image>
			\end{lstlisting}

	\end{itemize}

\end{frame}

% ------------------------------------------------------------------------------
% LTXE-SLIDE-START
% LTXE-SLIDE-UID:		e9cce085-6c300f1f-010de4ee-2c744704
% LTXE-SLIDE-TITLE:		Add image cropping (2)
% LTXE-SLIDE-REFERENCE:	Feature-65584-AddImageCropping.rst
% ------------------------------------------------------------------------------
\begin{frame}[fragile]
	\frametitle{In-Depth Changes}
	\framesubtitle{Image Manipulation Configuration Options (2)}

	% decrease font size for code listing
	\lstset{basicstyle=\smaller\ttfamily}

	\begin{itemize}
		\item TCA features the image cropping functionality, too:

			\begin{itemize}
				\item Column Type: \texttt{image\_manipulation}
				\item Config \texttt{file\_field}: string	\tabto{5.6cm}(default: \texttt{uid\_local})
				\item Config \texttt{enableZoom}: boolean	\tabto{5.6cm}(default: \texttt{FALSE})
				\item Config \texttt{allowedExtensions}: string\newline
					(default: \smaller\texttt{\$GLOBALS['TYPO3\_CONF\_VARS']['GFX']['imagefile\_ext']}\small)
				\item Config \texttt{ratios}: array, default:

					\begin{lstlisting}
						array(
						  '1.7777777777777777' => '16:9',
						  '1.3333333333333333' => '4:3',
						  '1' => '1:1',
						  'NaN' => 'Free'
						)
					\end{lstlisting}
			\end{itemize}

	\end{itemize}

\end{frame}

% ------------------------------------------------------------------------------
% LTXE-SLIDE-START
% LTXE-SLIDE-UID:		ed863269-574e4061-d69f855e-6494c062
% LTXE-SLIDE-TITLE:		Additional params for HTMLparser userFunc
% LTXE-SLIDE-REFERENCE:	Feature-59712-HtmlParserAdditionalUserFuncParams.rst
% ------------------------------------------------------------------------------
\begin{frame}[fragile]
	\frametitle{In-Depth Changes}
	\framesubtitle{Additional Parameters for HTMLparser userFunc}

	% decrease font size for code listing
	\lstset{basicstyle=\tiny\ttfamily}

	\begin{itemize}

		\item Additional parameters can be supplied to a userFunc of the HTMLparser:

			\begin{lstlisting}
				myobj = TEXT
				myobj.value = <a href="/" class="myclass">MyText</a>
				myobj.HTMLparser.tags.a.fixAttrib.class {
				  userFunc = Tx\MyExt\Myclass->htmlUserFunc
				  userFunc.myparam = test
				}
			\end{lstlisting}

		\item Access to these parameters in an extension as follows:

			\begin{lstlisting}
				function htmlUserFunc(array $params, HtmlParser $htmlParser) {
				  // $params['attributeValue'] contains the attribute value "myclass"
				  // $params['myparam'] is set to "test" in this example
				  ...
				}
			\end{lstlisting}

	\end{itemize}

\end{frame}

% ------------------------------------------------------------------------------
% LTXE-SLIDE-START
% LTXE-SLIDE-UID:		b04f8d44-fd695c73-b768f3cb-fd6d8643
% LTXE-SLIDE-TITLE:		New Locking API (1)
% LTXE-SLIDE-REFERENCE:	Feature-47712-NewLockingAPI.rst
% ------------------------------------------------------------------------------
\begin{frame}[fragile]
	\frametitle{In-Depth Changes}
	\framesubtitle{Locking API (1)}

	\begin{itemize}

		\item New Locking API has been introduced, which provides various locking methods (SimpleFile, Semaphore, ...)
		\item A locking method must implement the \small\texttt{LockingStrategyInterface}\normalsize:
		\begin{lstlisting}
			$lockFactory = GeneralUtility::makeInstance(LockFactory::class);
			$locker = $lockFactory->createLocker('someId');
			$locker->acquire() || die('Could not acquire lock.');
			...
			$locker->release();
		\end{lstlisting}

	\end{itemize}

\end{frame}

% ------------------------------------------------------------------------------
% LTXE-SLIDE-START
% LTXE-SLIDE-UID:		79a00412-c72401e6-4d9667c9-0e15d3f4
% LTXE-SLIDE-TITLE:		New Locking API (2)
% LTXE-SLIDE-REFERENCE:	Feature-47712-NewLockingAPI.rst
% ------------------------------------------------------------------------------
\begin{frame}[fragile]
	\frametitle{In-Depth Changes}
	\framesubtitle{Locking API (2)}

	% decrease font size for code listing
	\lstset{basicstyle=\tiny\ttfamily}

	\begin{itemize}

		\item Some methods also support non-blocking locks:

		\begin{lstlisting}
			$lockFactory = GeneralUtility::makeInstance(LockFactory::class);
			$locker = $lockFactory->createLocker(
			  'someId',
			  LockingStrategyInterface::LOCK_CAPABILITY_SHARED |
			    LockingStrategyInterface::LOCK_CAPABILITY_NOBLOCK
			);
			try {
			  $result = $locker->acquire(LockingStrategyInterface::LOCK_CAPABILITY_SHARED | LockingStrategyInterface::LOCK_CAPABILITY_NOBLOCK);
			  catch (\RuntimeException $e) {
			  if ($e->getCode() === 1428700748) {
			    // some process owns the lock
			    // let's do something else meanwhile
			    ...
			  }
			}
			if ($result) {
			  $locker->release();
			}
		\end{lstlisting}

	\end{itemize}

\end{frame}

% ------------------------------------------------------------------------------
% LTXE-SLIDE-START
% LTXE-SLIDE-UID:		108cb4a7-6e006507-c25a91e7-3a867f09
% LTXE-SLIDE-TITLE:		Add signal after extension installation
% LTXE-SLIDE-REFERENCE:	commit 39e0fd0a2c919da270cb49223433bcf95febbb10
% ------------------------------------------------------------------------------
\begin{frame}[fragile]
	\frametitle{In-Depth Changes}
	\framesubtitle{Signal after Extension Installation}

	% decrease font size for code listing
	\lstset{basicstyle=\tiny\ttfamily}

	\begin{itemize}

		\item New signal has been implemented in method
			\smaller
				\texttt{\textbackslash
					TYPO3\textbackslash
					CMS\textbackslash
					Extensionmanager\textbackslash
					Utility\textbackslash
					InstallUtility::install()}
			\normalsize
			which emits as soon as an extension has been installed and all imports/updates
			finished

		\begin{lstlisting}
			// execution
			$this->emitAfterExtensionInstallSignal($extensionKey);

			// methode
			protected function emitAfterExtensionInstallSignal($extensionKey) {
			  $this->signalSlotDispatcher->dispatch(
			    __CLASS__,
			    'afterExtensionInstall',
			    array($extensionKey, $this)
			  );
			}
		\end{lstlisting}

	\end{itemize}

\end{frame}

% ------------------------------------------------------------------------------
% LTXE-SLIDE-START
% LTXE-SLIDE-UID:		acb1ba95-1a401bef-549072a1-f32263a9
% LTXE-SLIDE-TITLE:		Registry for adding text extractor classes (1)
% LTXE-SLIDE-REFERENCE:	Feature-36743-FAL-TextExtractorRegistry.rst
% ------------------------------------------------------------------------------
\begin{frame}[fragile]
	\frametitle{In-Depth Changes}
	\framesubtitle{Registry for Text Extraction (1)}

	% decrease font size for code listing
	\lstset{basicstyle=\tiny\ttfamily}

	\begin{itemize}

		\item Multiple text extractors can be registered to allow dealing
			with different file types (e.g. Office, PDF files, etc.)

		\item TYPO3 core ships with an extractor for plain text files

		\item Every registered text extractor class needs to implement the \texttt{TextExtractorInterface}

		\item ...and the following methods:\newline
			\texttt{canExtractText()}\newline
			\small
				checks if text extraction from the given file is possible
			\normalsize
			\newline
			\texttt{extractText()}\newline
			\small
				returns the file's text content as a string
			\normalsize

	\end{itemize}

\end{frame}

% ------------------------------------------------------------------------------
% LTXE-SLIDE-START
% LTXE-SLIDE-UID:		acb1ba95-1a401bef-549072a1-f32263a9
% LTXE-SLIDE-TITLE:		Registry for adding text extractor classes (2)
% LTXE-SLIDE-REFERENCE:	Feature-36743-FAL-TextExtractorRegistry.rst
% ------------------------------------------------------------------------------
\begin{frame}[fragile]
	\frametitle{In-Depth Changes}
	\framesubtitle{Registry for Text Extraction (2)}

	% decrease font size for code listing
	\lstset{basicstyle=\tiny\ttfamily}

	\begin{itemize}

		\item Text extractor registration in file \texttt{ext\_localconf.php}:

			\begin{lstlisting}
				$textExtractorRegistry = \TYPO3\CMS\Core\Resource\TextExtraction\TextExtractorRegistry::getInstance();
				$textExtractorRegistry->registerTextExtractor(
				  \TYPO3\CMS\Core\Resource\TextExtraction\PlainTextExtractor::class
				);
			\end{lstlisting}

		\item Usage as follows:

			\begin{lstlisting}
				$textExtractorRegistry = \TYPO3\CMS\Core\Resource\TextExtraction\TextExtractorRegistry::getInstance();
				$extractor = $textExtractorRegistry->getTextExtractor($file);
				if($extractor !== NULL) {
				  $content = $extractor->extractText($file);
				}
			\end{lstlisting}
	\end{itemize}

\end{frame}

% ------------------------------------------------------------------------------
% LTXE-SLIDE-START
% LTXE-SLIDE-UID:		d9657b5d-5cd56c65-d7266a27-401dd233
% LTXE-SLIDE-TITLE:		Miscellaneous
% LTXE-SLIDE-REFERENCE:	Feature-66042-WebLibrariesLoadedViaBower.rst
% LTXE-SLIDE-REFERENCE:	Feature-63703-AddOptionToStopTask.rst
% LTXE-SLIDE-REFERENCE:	Feature-61463-AllowProcessedFoldersInDifferentStorage.rst
% LTXE-SLIDE-REFERENCE:	Feature-52693-TSFE-RequestedId.rst
% ------------------------------------------------------------------------------

\begin{frame}[fragile]
	\frametitle{In-Depth Changes}
	\framesubtitle{Miscellaneous}

	\begin{itemize}

		\item Web libraries (such as Twitter Bootstrap, jQuery, Font Awesome, etc.) use
			"Bower" (\url{http://bower.io}) and are not part of the TYPO3 core Git repository
			anymore\newline
			\small
				\texttt{\# bower install}	\tabto{3.4cm}executes an installation\newline
				\texttt{\# bower update}		\tabto{3.4cm}executes an update\newline
			\normalsize
			(file \texttt{bower.json} is located in directory \texttt{Build/})

		\item Scheduler CLI has received option "\texttt{-s}" to stop a running task

		\item The processing folder of a (remote) storage can be outside of the
			storage (useful in case of a read-only storage for instance)

		\item It is now possible to retrieve the page ID of the originally requested page:
			\texttt{\$TSFE->getRequestedId()}

	\end{itemize}

\end{frame}

% ------------------------------------------------------------------------------
