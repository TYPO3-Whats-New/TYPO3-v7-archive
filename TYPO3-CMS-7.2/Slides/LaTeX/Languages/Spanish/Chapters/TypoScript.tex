% ------------------------------------------------------------------------------
% TYPO3 CMS 7.2 - What's New - Chapter "TypoScript" (English Version)
%
% @author	Michael Schams <schams.net>
% @license	Creative Commons BY-NC-SA 3.0
% @link		http://typo3.org/download/release-notes/whats-new/
% @language	English
% ------------------------------------------------------------------------------
% LTXE-CHAPTER-UID:		12518a77-2b90a173-4e3f8420-485e0497
% LTXE-CHAPTER-NAME:	TypoScript
% ------------------------------------------------------------------------------

\section{TSconfig \& TypoScript}
\begin{frame}[fragile]
	\frametitle{TSconfig \& TypoScript}

	\begin{center}\huge{Capítulo 2:}\end{center}
	\begin{center}\huge{\color{typo3darkgrey}\textbf{TSconfig \& TypoScript}}\end{center}

\end{frame}

% ------------------------------------------------------------------------------
% LTXE-SLIDE-START
% LTXE-SLIDE-UID:		a22b1cc4-2f5edc86-be9d95c3-f88a042a
% LTXE-SLIDE-ORIGIN:	56b025e6-cf380b11-c9d1c009-0b548d6a English
% LTXE-SLIDE-TITLE:		Add flexible preview URL configuration (1)
% LTXE-SLIDE-REFERENCE:	Feature-66370-AddFlexiblePreviewUrlConfiguration.rst
% ------------------------------------------------------------------------------
\begin{frame}[fragile]
	\frametitle{TSconfig \& TypoScript}
	\framesubtitle{Configuración flexible de la URL de vista previa (1)}

	% decrease font size for code listing
	\lstset{basicstyle=\tiny\ttfamily}

	\begin{itemize}

		\item Ahora es posible configurar el enlace de vista previa generado por el botón
			"guardar \& ver" en el backend.

		\item Un caso de uso común es tener vistas previas para registros de noticias o blog,
			pero también puede definir páginas de vista previa diferentes para elementos
			de contenido normales.

			\begin{lstlisting}
				TCEMAIN.preview {
				  <table name> {
				    previewPageId = 123
				    useDefaultLanguageRecord = 0
				    fieldToParameterMap {
				      uid = tx_myext_pi1[showUid]
				    }
				    additionalGetParameters {
				      tx_myext_pi1[special] = HELLO
				    }
				  }
				}
			\end{lstlisting}

	\end{itemize}

\end{frame}

% ------------------------------------------------------------------------------
% LTXE-SLIDE-START
% LTXE-SLIDE-UID:		0a07266a-96b4f361-0a144e8d-13a6117a
% LTXE-SLIDE-ORIGIN:	6822c376-37649cd8-0acc836a-b856fc61 English
% LTXE-SLIDE-TITLE:		Add flexible preview URL configuration (2)
% LTXE-SLIDE-REFERENCE:	Feature-66370-AddFlexiblePreviewUrlConfiguration.rst
% ------------------------------------------------------------------------------
\begin{frame}[fragile]
	\frametitle{TSconfig \& TypoScript}
	\framesubtitle{Configuración flexible de la URL de vista previa (2)}

	\begin{itemize}
		\item \texttt{previewPageId}:\newline
			\smaller
				UID de la página a usar para vista previa\newline
				(si se omite este ajuste, la página actual será usada)
			\normalsize
		\item \texttt{useDefaultLanguageRecord}:\newline
			\smaller
				define si los registros traducidos usarán la UID del registro por defecto\newline
				(esto es activado por defecto, valor: 1)
			\normalsize
		\item \texttt{fieldToParameterMap}:\newline
			\smaller
				un mapeo que permite seleccionar campos del registro para ser incluidos como parámetros GET
			\normalsize
		\item \texttt{additionalGetParameters}:\newline
			\smaller
				permite añadir parámetros GET arbitrariamente e incluso sobrescribir otros
			\normalsize
	\end{itemize}

\end{frame}

% ------------------------------------------------------------------------------
% LTXE-SLIDE-START
% LTXE-SLIDE-UID:		74087d4b-616e3e02-50026df8-f902e827
% LTXE-SLIDE-ORIGIN:	c5dbf3c7-655fc6e4-227a3bf6-f5d89fb1 English
% LTXE-SLIDE-TITLE:
% LTXE-SLIDE-REFERENCE:	Feature-59646-AddRteConfigurationPropertyButtonsLinkTypePropertiesTargetDefault.rst
% ------------------------------------------------------------------------------
\begin{frame}[fragile]
	\frametitle{TSconfig \& TypoScript}
	\framesubtitle{Configuración RTE: Objetivo por Defecto}

	\begin{itemize}

		\item Una propiedad de configuración RTE puede ser usada en PageTSconfig
			para configurar un objetivo por defecto para los enlaces de un tipo dado\newline

			\small
				\texttt{buttons.link.[}
				\textit{type}
				\texttt{].properties.target.default = ...}
			\normalsize\newline

		\item Los tipos de enlace posibles son:\newline
			\small
				(tipos posteriores pueden ser proporcionados por extensiones)
			\normalsize

			\begin{itemize}
				\item \texttt{page}
				\item \texttt{file}
				\item \texttt{url}
				\item \texttt{mail}
				\item \texttt{spec}
			\end{itemize}
	\end{itemize}

\end{frame}

% ------------------------------------------------------------------------------
% LTXE-SLIDE-START
% LTXE-SLIDE-UID:		2d26c3a1-a0d84e61-ccd9dde3-f95e1e50
% LTXE-SLIDE-ORIGIN:	fd957580-301e4a0a-ad5325f5-ffa024c3 English
% LTXE-SLIDE-TITLE:		Strip empty HTML tags in HtmlParser
% LTXE-SLIDE-REFERENCE:	Feature-20555-StripEmptyHtmlTags.rst
% ------------------------------------------------------------------------------
\begin{frame}[fragile]
	\frametitle{TSconfig \& TypoScript}
	\framesubtitle{Retirar Etiquetas HTML Vacías en HTMLparser}

	% decrease font size for code listing
	\lstset{basicstyle=\tiny\ttfamily}

	\begin{itemize}
		\item Una funcionalidad nueva ha sido implementada en el HTMLparser
			que permite la retirada de etiquetas HTML vacías

			\begin{lstlisting}
				stdWrap {
				   // this removes all empty HTML tags
				   HTMLparser.stripEmptyTags = 1
				   // this removes empty h2 and h3 tags only
				   HTMLparser.stripEmptyTags.tags = h2, h3
				}

				RTE.default.proc.entryHTMLparser_db {
				   stripEmptyTags = 1
				   stripEmptyTags.tags = p
				   stripEmptyTags.treatNonBreakingSpaceAsEmpty = 1
				}
			\end{lstlisting}

			\underline{\textbf{Nota:}}
				HTMLparser retira todas las etiquetas desconocidas por defecto.\newline
				Por lo tanto podría ser útil para conservar éstas:\newline
				\texttt{HTMLparser.keepNonMatchedTags = 1}

	\end{itemize}

\end{frame}

% ------------------------------------------------------------------------------
% LTXE-SLIDE-START
% LTXE-SLIDE-UID:		096b504e-cb7c203e-d8022481-77c0355a
% LTXE-SLIDE-ORIGIN:	62960e76-c3dca953-fe5a1070-831633fc English
% LTXE-SLIDE-TITLE:		Miscellaneous
% LTXE-SLIDE-REFERENCE:	Feature-63040-AddRteConfigurationPropertyButtonsAbbreviationRemoveFieldsets.rst
% LTXE-SLIDE-REFERENCE:	commit 31c62f9311ee3d33bf792d548cc4a5fac83aa3d0
% ------------------------------------------------------------------------------
\begin{frame}[fragile]
	\frametitle{TSconfig \& TypoScript}
	\framesubtitle{Miscelánea}

	\begin{itemize}
		\item La nueva propiedad \texttt{buttons.abbreviation.removeFieldsets} pueden ser
			usada en PageTSconfig para configurar el cuadro de abreviaturas

			\begin{lstlisting}
				# Los valores posibles son:
				# acronym, definedAcronym, abbreviation, definedAbbreviation
				buttons.abbreviation.removeFieldsets = acronym,definedAcronym
			\end{lstlisting}

		\item Ahora la propiedad \texttt{inlineLanguageLabel} del objeto \texttt{PAGE}
			puede manejar referencias a \texttt{LLL:}

	\end{itemize}

\end{frame}

% ------------------------------------------------------------------------------
