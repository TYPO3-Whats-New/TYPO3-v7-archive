% ------------------------------------------------------------------------------
% TYPO3 CMS 7.2 - What's New - Chapter "In-Depth Changes" (English Version)
%
% @author	Michael Schams <schams.net>
% @license	Creative Commons BY-NC-SA 3.0
% @link		http://typo3.org/download/release-notes/whats-new/
% @language	English
% ------------------------------------------------------------------------------
% LTXE-CHAPTER-UID:		e28087f2-4bffcc7b-3c685cca-579f18a8
% LTXE-CHAPTER-NAME:	In-Depth Changes
% ------------------------------------------------------------------------------

\section{Cambios en Profundidad}
\begin{frame}[fragile]
	\frametitle{Cambios en Profundidad}

	\begin{center}\huge{Capítulo 3:}\end{center}
	\begin{center}\huge{\color{typo3darkgrey}\textbf{Cambios en Profundidad}}\end{center}

\end{frame}

% ------------------------------------------------------------------------------
% LTXE-SLIDE-START
% LTXE-SLIDE-UID:		a7457706-b5ece9e1-54bbf7ca-8ba7cab0
% LTXE-SLIDE-ORIGIN:	547708aa-92282fd1-6c27d618-4f13f86b English
% LTXE-SLIDE-TITLE:		Add SVG support
% LTXE-SLIDE-REFERENCE:	Feature-50136-AddSVGSupport.rst
% ------------------------------------------------------------------------------
\begin{frame}[fragile]
	\frametitle{Cambios en Profundidad}
	\framesubtitle{Soporta de SVG en el núcleo}

	\begin{itemize}
		\item Ahora el núcleo de TYPO3 CMS soporta imágenes SVG ("Gráficos Vectoriales Redimensionables")

		\item Cuando una imagen SVG es escalada, se almacena un registro con las nuevas
			dimensiones calculadas en \texttt{sys\_file\_processedfile} en vez de crear
			un archivo procesado\newline
			\small(excepto, si la imagen es procesada más adelante, por ejemplo, si es recortada)\normalsize.

		\item Una alternativa es añadida para determinar las dimensiones de SVG si
			ImageMagick/GraphicsMagick no puede determinar las dimensiones
			de la imagen. En este caso, se lee el contenido del archivo XML.

		\item SVG ha sido añadido también a la lista de archivos de imágenes válidos:\newline
			\texttt{\$GLOBALS['TYPO3\_CONF\_VARS']['GFX']['imagefile\_ext']}

	\end{itemize}

\end{frame}

% ------------------------------------------------------------------------------
% LTXE-SLIDE-START
% LTXE-SLIDE-UID:		24f10a01-a77a5af5-9560d27c-ae430f09
% LTXE-SLIDE-ORIGIN:	bd03e141-3842b75e-8a44920f-a8c139e5 English
% LTXE-SLIDE-TITLE:		Add count methods and sort functionality to FAL drivers
% LTXE-SLIDE-REFERENCE:	Breaking-56746-AddCountMethodsAndSortFunctionalityToFalDrivers.rst
% ------------------------------------------------------------------------------
\begin{frame}[fragile]
	\frametitle{Cambios en Profundidad}
	\framesubtitle{Extendiendo Driver FAL}

	% decrease font size for code listing
	\lstset{basicstyle=\tiny\ttfamily}

	\begin{itemize}
		\item Con el fin de mejorar el desempeño de la lista de archivos al mostrar
			almacenamientos (remotos), el driver FAL deberá ocuparse de organizar, ordenar y
			determinar el número de archivos/carpetas.
			Dos parámetros nuevos \texttt{sort} and \texttt{sortRev} han sido añadidos para permitir esto:

			\begin{lstlisting}
				public function getFilesInFolder($folderIdentifier, $start = 0, $numberOfItems = 0,
				  $recursive = FALSE, array $filenameFilterCallbacks = array(), $sort = '', $sortRev = FALSE);

				public function getFoldersInFolder($folderIdentifier, $start = 0, $numberOfItems = 0,
				  $recursive = FALSE, array $folderNameFilterCallbacks = array(), $sort = '', $sortRev = FALSE);
			\end{lstlisting}

		\item Adicionalmente, dos métodos nuevos han sido implementados:

			\begin{lstlisting}
				public function getFilesInFolderCount($folderIdentifier, $recursive = FALSE,
				  array $filenameFilterCallbacks = array());

				public function getFoldersInFolderCount($folderIdentifier, $recursive = FALSE,
				  array $folderNameFilterCallbacks = array());
			\end{lstlisting}

	\end{itemize}

\end{frame}

% ------------------------------------------------------------------------------
% LTXE-SLIDE-START
% LTXE-SLIDE-UID:		806f0e9a-6c34bf1f-de61e508-cb557a04
% LTXE-SLIDE-ORIGIN:	d0df6060-4b8d33a9-69adad59-422e9ada English
% LTXE-SLIDE-TITLE:		Introduce Backend Routing (1)
% LTXE-SLIDE-REFERENCE:	commit a08ce7238e583d1962edfe998e04c7d9c3d7c2ae
% ------------------------------------------------------------------------------
\begin{frame}[fragile]
	\frametitle{Cambios en Profundidad}
	\framesubtitle{Backend Routing API (1)}

	\begin{itemize}
		\item Un Backend Routing API ha sido implementado, el cual administra los Entry Points del backend
		\item Inspirándose en el Symfony Routing Framework, este API es compatible con él en gran medida\newline
			\small(sin embargo TYPO3 usa sólo aproximadamente 20\% en este momento)\normalsize

		\item Básicamente tres clases implementan la funcionalidad:
			\begin{itemize}
				\item \texttt{class Route}:\tabto{3.6cm}contiene detalles sobre la ruta y opciones
				\item \texttt{class Router}:\tabto{3.6cm}API que concuerde con la ruta
				\item \texttt{class UrlGenerator}:\tabto{3.6cm}genera la URL
			\end{itemize}
	\end{itemize}

\end{frame}

% ------------------------------------------------------------------------------
% LTXE-SLIDE-START
% LTXE-SLIDE-UID:		451c1d2e-881d0674-df1e3a4b-e53e783d
% LTXE-SLIDE-ORIGIN:	fe1eec67-bbf31ed9-57b8e83a-3a922f0e English
% LTXE-SLIDE-TITLE:		Introduce Backend Routing (2)
% LTXE-SLIDE-REFERENCE:	commit a08ce7238e583d1962edfe998e04c7d9c3d7c2ae
% ------------------------------------------------------------------------------
\begin{frame}[fragile]
	\frametitle{Cambios en Profundidad}
	\framesubtitle{Backend Routing API (2)}

	\begin{itemize}

		\item Las rutas son definidas en el siguiente archivo de una extensión:
			\texttt{Configuration/Backend/Routes.php}\newline
			(ver extensión de sistema \texttt{backend} como ejemplo)

		\item Detalles adicionales sobre el API Backend Routing:\newline
			\small\url{http://wiki.typo3.org/Blueprints/BackendRouting}\normalsize

	\end{itemize}

\end{frame}

% ------------------------------------------------------------------------------
% LTXE-SLIDE-START
% LTXE-SLIDE-UID:		dbd5e895-f3e195cb-3aa46278-10caf50e
% LTXE-SLIDE-ORIGIN:	9d56ca19-04a98896-ba0a973c-8a16e877 English
% LTXE-SLIDE-TITLE:		New system extension: mediace
% LTXE-SLIDE-REFERENCE:	Breaking-64719-MediaContentMovedToSystemExtension.rst
% LTXE-SLIDE-REFERENCE:	Breaking-65778-MediaWizardProviderMovedToSystemExtension.rst
% ------------------------------------------------------------------------------
\begin{frame}[fragile]
	\frametitle{Cambios en Profundidad}
	\framesubtitle{Nueva extensión de sistema para Elementos de Contenido Multimedia}

	\begin{itemize}

		\item La nueva extensión de sistema "\texttt{mediace}" contiene los siguientes cObjects:

			\begin{itemize}
				\item \texttt{MULTIMEDIA}
				\item \texttt{MEDIA}
				\item \texttt{SWFOBJECT}
				\item \texttt{FLOWPLAYER}
				\item \texttt{QTOBJECT}
			\end{itemize}

		\item Los elementos de contenido \texttt{media} y \texttt{multimedia} han sido movidos a esta
			extensión de sistema, así como también el "Media Wizard Provider"

		\item ¡Esta extensión \underline{\textbf{no}} está instalada por defecto!

	\end{itemize}

\end{frame}

% ------------------------------------------------------------------------------
% LTXE-SLIDE-START
% LTXE-SLIDE-UID:		b847d093-25cd565a-e1832f3e-244bd4b2
% LTXE-SLIDE-ORIGIN:	cc63cd98-524ee6df-38430963-f7c7067c English
% LTXE-SLIDE-TITLE:		Composer vendor directory
% LTXE-SLIDE-REFERENCE:	Breaking-66001-ComposerVendorDirectoryChanged.rst
% ------------------------------------------------------------------------------
\begin{frame}[fragile]
	\frametitle{Cambios en Profundidad}
	\framesubtitle{Ubicación de bibliotecas de terceros}

	\begin{itemize}

		\item Ahora las bibliotecas de terceros instaladas de Composer están instaladas bajo \texttt{typo3/contrib/vendor}\newline
			\small
				(TYPO3 CMS < 7.2: en carpeta \texttt{Packages/Libraries})
			\normalsize

		\item De esta manera el proceso de empaquetado para lanzar TYPO3 CMS como archivo tarball o zip
			puede desencadenar una instalación completa sin tener que enviar \texttt{Packages/} para bibliotecas de terceros

		\item Problemas pueden aparecer con instalaciones que fueron configuradas mediante composer y usan \texttt{phpunit},
			a menos que las dependencias de composer hayan sido completamente reconstruidas. Para arreglar esto, ejecute:

			\begin{lstlisting}
				# cd htdocs/
				# rm -rf typo3/contrib/vendor/ bin/ Packages/Libraries/ composer.lock
				# composer install
			\end{lstlisting}
	\end{itemize}

\end{frame}

% ------------------------------------------------------------------------------
% LTXE-SLIDE-START
% LTXE-SLIDE-UID:		37fbb443-1979f13f-c2a49853-7303b0c7
% LTXE-SLIDE-ORIGIN:	c9a5288c-f45aac1e-34606612-fdf6ec18 English
% LTXE-SLIDE-TITLE:		Introduce JavaScript notification API
% LTXE-SLIDE-REFERENCE:	Feature-66047-IntroduceJavascriptNotificationApi.rst
% ------------------------------------------------------------------------------
\begin{frame}[fragile]
	\frametitle{Cambios en Profundidad}
	\framesubtitle{Notificaciones JavaScript}

	\begin{itemize}

		\item Un nuevo JavaScript Notification API ha sido implementado:
			\begin{lstlisting}
				// old and deprecated:
				top.TYPO3.Flashmessages.display(TYPO3.Severity.notice)

				// new and the only correct way since TYPO3 CMS 7.2:
				top.TYPO3.Notification.notice(title, message)
    		\end{lstlisting}

		\item Las siguientes funciones API existen:\newline
			\small(parámetro \texttt{duration} es opcional y tiene un valor por defecto de 5 segundos)\normalsize
			\begin{itemize}
				\item \normalsize\smaller\texttt{top.TYPO3.Notification.notice(title, message, duration)}\normalsize
				\item \smaller\texttt{top.TYPO3.Notification.info(title, message, duration)}\normalsize
				\item \smaller\texttt{top.TYPO3.Notification.success(title, message, duration)}\normalsize
				\item \smaller\texttt{top.TYPO3.Notification.warning(title, message, duration)}\normalsize
				\item \smaller\texttt{top.TYPO3.Notification.error(title, message, duration)}\normalsize
			\end{itemize}

	\end{itemize}

\end{frame}

% ------------------------------------------------------------------------------
% LTXE-SLIDE-START
% LTXE-SLIDE-UID:		fcc37a7b-b8381ebe-0f9a955b-4625a8cf
% LTXE-SLIDE-ORIGIN:	b2acb5df-60b9a7dd-35cf697d-2a50d5db English
% LTXE-SLIDE-TITLE:		System Information Dropdown (1)
% LTXE-SLIDE-REFERENCE:	Feature-65767-SystemInformationDropdown.rst
% ------------------------------------------------------------------------------
\begin{frame}[fragile]
	\frametitle{Cambios en Profundidad}
	\framesubtitle{Menú desplegable de Información del Sistema (1)}

	% decrease font size for code listing
	\lstset{basicstyle=\tiny\ttfamily}

	\begin{itemize}
		\item Se pueden agregar elementos personalizados de información del sistema
			en el menú desplegable mediante la creación de una ranura

		\item La ranura debe ser registrada en el archivo  \texttt{ext\_localconf.php}:

			\begin{lstlisting}
				$signalSlotDispatcher = \TYPO3\CMS\Core\Utility\GeneralUtility::makeInstance(
				  \TYPO3\CMS\Extbase\SignalSlot\Dispatcher::class);

				$signalSlotDispatcher->connect(
				  \TYPO3\CMS\Backend\Backend\ToolbarItems\SystemInformationToolbarItem::class,
				  'getSystemInformation',
				  \Vendor\Extension\SystemInformation\Item::class,
				  'getItem'
				);
			\end{lstlisting}

	\end{itemize}

\end{frame}

% ------------------------------------------------------------------------------
% LTXE-SLIDE-START
% LTXE-SLIDE-UID:		fa594327-bbb11bf2-41167283-b4e20b77
% LTXE-SLIDE-ORIGIN:	e0f13209-22e0ff8c-7ff23bf9-31b1bb52 English
% LTXE-SLIDE-TITLE:		System Information Dropdown (2)
% LTXE-SLIDE-REFERENCE:	Feature-65767-SystemInformationDropdown.rst
% ------------------------------------------------------------------------------
\begin{frame}[fragile]
	\frametitle{Cambios en Profundidad}
	\framesubtitle{Menú desplegable de Información del Sistema (2)}

	% decrease font size for code listing
	\lstset{basicstyle=\tiny\ttfamily}

	\begin{itemize}

		\item Se pueden agregar elementos personalizados de información del sistema
			en el menú desplegable mediante la creación de una ranura

		\item Esto requiere la clase \texttt{Item} y su método \texttt{getItem()} en el archivo
			\small
				\texttt{EXT:extension\textbackslash
					Classes\textbackslash
					SystemInformation\textbackslash
					Item.php}:
			\normalsize

		\begin{lstlisting}
			class Item {
			  public function getItem() {
			    return array(array(
			      'title' => 'The title shown on hover',
			      'value' => 'Description shown in the list',
			      'status' => SystemInformationHookInterface::STATUS_OK,
			      'count' => 4,
			      'icon' => \TYPO3\CMS\Backend\Utility\IconUtility::getSpriteIcon(
				    'extensions-example-information-icon')
			    ));
			  }
			}
		\end{lstlisting}

	\end{itemize}

\end{frame}

% ------------------------------------------------------------------------------
% LTXE-SLIDE-START
% LTXE-SLIDE-UID:		c535788a-6e56e8f5-87d2b4da-812c5a1e
% LTXE-SLIDE-ORIGIN:	7d27df49-260500ef-0fc78a89-6e80069b English
% LTXE-SLIDE-TITLE:		System Information Dropdown (3)
% LTXE-SLIDE-REFERENCE:	Feature-65767-SystemInformationDropdown.rst
% ------------------------------------------------------------------------------
\begin{frame}[fragile]
	\frametitle{Cambios en Profundidad}
	\framesubtitle{Menú desplegable de Información del Sistema (3)}

	% decrease font size for code listing
	\lstset{basicstyle=\tiny\ttfamily}

	\begin{itemize}
		\item El icono \texttt{extensions-example-information-icon} debe registrarse en \texttt{ext\_localconf.php}:
		\begin{lstlisting}
			\TYPO3\CMS\Backend\Sprite\SpriteManager::addSingleIcons(
			  array(
			    'information-icon' => \TYPO3\CMS\Core\Utility\ExtensionManagementUtility::extRelPath(
			      $_EXTKEY) . 'Resources/Public/Images/Icons/information-icon.png'
			    ),
			   $_EXTKEY
			);
		\end{lstlisting}

	\end{itemize}

\end{frame}


% ------------------------------------------------------------------------------
% LTXE-SLIDE-START
% LTXE-SLIDE-UID:		a4f4bd12-da6d40fe-c46f199d-01d21f54
% LTXE-SLIDE-ORIGIN:	d7449c27-fd9676cf-d12be80b-c24a4536 English
% LTXE-SLIDE-TITLE:		System Information Dropdown (4)
% LTXE-SLIDE-REFERENCE:	Feature-65767-SystemInformationDropdown.rst
% ------------------------------------------------------------------------------
\begin{frame}[fragile]
	\frametitle{Cambios en Profundidad}
	\framesubtitle{Menú desplegable de Información del Sistema (4)}

	% decrease font size for code listing
	\lstset{basicstyle=\tiny\ttfamily}

	\begin{itemize}

		\item Se muestran los mensajes en la parte inferior del menú desplegable

		\item Las extensiones pueden proporcionar su propia ranura para rellenar mensajes:

		\begin{lstlisting}
			$signalSlotDispatcher = \TYPO3\CMS\Core\Utility\GeneralUtility::makeInstance(
			  \TYPO3\CMS\Extbase\SignalSlot\Dispatcher::class);

			$signalSlotDispatcher->connect(
			  \TYPO3\CMS\Backend\Backend\ToolbarItems\SystemInformationToolbarItem::class,
			  'loadMessages',
			  \Vendor\Extension\SystemInformation\Message::class,
			  'getMessage'
			);
		\end{lstlisting}

	\end{itemize}

\end{frame}

% ------------------------------------------------------------------------------
% LTXE-SLIDE-START
% LTXE-SLIDE-UID:		47c77871-b0de67de-fe2d3582-d364aa0c
% LTXE-SLIDE-ORIGIN:	ab3b46ce-c7d7a228-7f7888af-4312fcdd English
% LTXE-SLIDE-TITLE:		System Information Dropdown (5)
% LTXE-SLIDE-REFERENCE:	Feature-65767-SystemInformationDropdown.rst
% ------------------------------------------------------------------------------
\begin{frame}[fragile]
	\frametitle{Cambios en Profundidad}
	\framesubtitle{Menú desplegable de Información del Sistema (5)}

	% decrease font size for code listing
	\lstset{basicstyle=\tiny\ttfamily}

	\begin{itemize}

		\item Se muestran los mensajes en la parte inferior del menú desplegable

		\item Esto requiere la clase \texttt{Message} y su método \texttt{getMessage()} en el archivo
			\small
				\texttt{EXT:extension\textbackslash
					Classes\textbackslash
					SystemInformation\textbackslash
					Message.php}:
			\normalsize

		\begin{lstlisting}
			class Message {
			  public function getMessage() {
			    return array(array(
			      'status' => SystemInformationHookInterface::STATUS_OK,
			      'text' => 'Something went wrong. Take a look at the reports module.'
			    ));
			  }
			}
		\end{lstlisting}

	\end{itemize}

\end{frame}


% ------------------------------------------------------------------------------
% LTXE-SLIDE-START
% LTXE-SLIDE-UID:		3b5175fd-bad9968b-52e0ba99-e3cf98a7
% LTXE-SLIDE-ORIGIN:	98e2056a-d165b638-b1a1ee08-206bc15a English
% LTXE-SLIDE-TITLE:		Add image cropping
% LTXE-SLIDE-REFERENCE:	Feature-65584-AddImageCropping.rst
% ------------------------------------------------------------------------------
\begin{frame}[fragile]
	\frametitle{Cambios en Profundidad}
	\framesubtitle{Opciones de Configuración de Manipulación de Imágenes  (1)}

	\begin{itemize}
		\item Las siguientes opciones de configuración TypoScript están disponibles:
			\begin{lstlisting}
				# disable cropping for all images
				tt_content.image.20.1.file.crop =

				# override or set cropping for all images
				# offsetX,offsetY,width,height
				tt_content.image.20.1.file.crop = 50,50,100,100
			\end{lstlisting}

		\item Fluid también soporta funciones de recorte:
			\begin{lstlisting}
				# disable cropping for all images
				<f:image image="{imageObject}" crop="" ></f:image>

				# override or set cropping for all images
				# offsetX,offsetY,width,height
				<f:image image="{imageObject}" crop="50,50,100,100" ></f:image>
			\end{lstlisting}

	\end{itemize}

\end{frame}

% ------------------------------------------------------------------------------
% LTXE-SLIDE-START
% LTXE-SLIDE-UID:		6806a051-5aca5b06-c6e2c3a1-10892760
% LTXE-SLIDE-ORIGIN:	e9cce085-6c300f1f-010de4ee-2c744704 English
% LTXE-SLIDE-TITLE:		Add image cropping (2)
% LTXE-SLIDE-REFERENCE:	Feature-65584-AddImageCropping.rst
% ------------------------------------------------------------------------------
\begin{frame}[fragile]
	\frametitle{Cambios en Profundidad}
	\framesubtitle{Opciones de Configuración de Manipulación de Imágenes (2)}

	% decrease font size for code listing
	\lstset{basicstyle=\smaller\ttfamily}

	\begin{itemize}
		\item TCA tiene la función de recorte de imagen también:

			\begin{itemize}
				\item Tipo de Columna: \texttt{image\_manipulation}
				\item Config \texttt{file\_field}: string	\tabto{5.6cm}(por defecto: \texttt{uid\_local})
				\item Config \texttt{enableZoom}: boolean	\tabto{5.6cm}(por defecto: \texttt{FALSE})
				\item Config \texttt{allowedExtensions}: string\newline
					(default: \smaller\texttt{\$GLOBALS['TYPO3\_CONF\_VARS']['GFX']['imagefile\_ext']}\small)
				\item Config \texttt{ratios}: array, por defecto:

					\begin{lstlisting}
						array(
						  '1.7777777777777777' => '16:9',
						  '1.3333333333333333' => '4:3',
						  '1' => '1:1',
						  'NaN' => 'Free'
						)
					\end{lstlisting}
			\end{itemize}

	\end{itemize}

\end{frame}

% ------------------------------------------------------------------------------
% LTXE-SLIDE-START
% LTXE-SLIDE-UID:		3e84520f-e4b00ba4-db24c518-d1163bed
% LTXE-SLIDE-ORIGIN:	ed863269-574e4061-d69f855e-6494c062 English
% LTXE-SLIDE-TITLE:		Additional params for HTMLparser userFunc
% LTXE-SLIDE-REFERENCE:	Feature-59712-HtmlParserAdditionalUserFuncParams.rst
% ------------------------------------------------------------------------------
\begin{frame}[fragile]
	\frametitle{Cambios en Profundidad}
	\framesubtitle{Parámetros Adicionales para userFunc HTMLparser}

	% decrease font size for code listing
	\lstset{basicstyle=\tiny\ttfamily}

	\begin{itemize}

		\item Se pueden suministrar parámetros adicionales para una userFunc del HTMLparser:

			\begin{lstlisting}
				myobj = TEXT
				myobj.value = <a href="/" class="myclass">MyText</a>
				myobj.HTMLparser.tags.a.fixAttrib.class {
				  userFunc = Tx\MyExt\Myclass->htmlUserFunc
				  userFunc.myparam = test
				}
			\end{lstlisting}

		\item Acceso a estos parámetros en una extensión como se muestra a continuación:

			\begin{lstlisting}
				function htmlUserFunc(array $params, HtmlParser $htmlParser) {
				  // $params['attributeValue'] contains the attribute value "myclass"
				  // $params['myparam'] is set to "test" in this example
				  ...
				}
			\end{lstlisting}

	\end{itemize}

\end{frame}

% ------------------------------------------------------------------------------
% LTXE-SLIDE-START
% LTXE-SLIDE-UID:		1ded7916-663c09e2-5df7c33c-fe5311fe
% LTXE-SLIDE-ORIGIN:	b04f8d44-fd695c73-b768f3cb-fd6d8643 English
% LTXE-SLIDE-TITLE:		New Locking API (1)
% LTXE-SLIDE-REFERENCE:	Feature-47712-NewLockingAPI.rst
% ------------------------------------------------------------------------------
\begin{frame}[fragile]
	\frametitle{Cambios en Profundidad}
	\framesubtitle{API Locking(1)}

	\begin{itemize}

		\item Se ha introducido una nueva API Locking, la cual proporciona diferentes métodos de bloqueo (SimpleFile, Semaphore, ...)
		\item Un método de bloqueo debe implementar la \small\texttt{LockingStrategyInterface}\normalsize:
		\begin{lstlisting}
			$lockFactory = GeneralUtility::makeInstance(LockFactory::class);
			$locker = $lockFactory->createLocker('someId');
			$locker->acquire() || die('Could not acquire lock.');
			...
			$locker->release();
		\end{lstlisting}

	\end{itemize}

\end{frame}

% ------------------------------------------------------------------------------
% LTXE-SLIDE-START
% LTXE-SLIDE-UID:		9abea31b-32a68cdb-45fd7181-b1192fbe
% LTXE-SLIDE-ORIGIN:	79a00412-c72401e6-4d9667c9-0e15d3f4 English
% LTXE-SLIDE-TITLE:		New Locking API (2)
% LTXE-SLIDE-REFERENCE:	Feature-47712-NewLockingAPI.rst
% ------------------------------------------------------------------------------
\begin{frame}[fragile]
	\frametitle{Cambios en Profundidad}
	\framesubtitle{API Locking (2)}

	% decrease font size for code listing
	\lstset{basicstyle=\tiny\ttfamily}

	\begin{itemize}

		\item Algunos métodos también soportan cierres sin bloqueo:

		\begin{lstlisting}
			$lockFactory = GeneralUtility::makeInstance(LockFactory::class);
			$locker = $lockFactory->createLocker(
			  'someId',
			  LockingStrategyInterface::LOCK_CAPABILITY_SHARED |
			    LockingStrategyInterface::LOCK_CAPABILITY_NOBLOCK
			);
			try {
			  $result = $locker->acquire(LockingStrategyInterface::LOCK_CAPABILITY_SHARED | LockingStrategyInterface::LOCK_CAPABILITY_NOBLOCK);
			  catch (\RuntimeException $e) {
			  if ($e->getCode() === 1428700748) {
			    // some process owns the lock
			    // let's do something else meanwhile
			    ...
			  }
			}
			if ($result) {
			  $locker->release();
			}
		\end{lstlisting}

	\end{itemize}

\end{frame}

% ------------------------------------------------------------------------------
% LTXE-SLIDE-START
% LTXE-SLIDE-UID:		27c8c4b5-f13ee465-51524cfb-be7e7667
% LTXE-SLIDE-ORIGIN:	108cb4a7-6e006507-c25a91e7-3a867f09 English
% LTXE-SLIDE-TITLE:		Add signal after extension installation
% LTXE-SLIDE-REFERENCE:	commit 39e0fd0a2c919da270cb49223433bcf95febbb10
% ------------------------------------------------------------------------------
\begin{frame}[fragile]
	\frametitle{Cambios en Profundidad}
	\framesubtitle{Señal después de la Instalación de una Extensión}

	% decrease font size for code listing
	\lstset{basicstyle=\tiny\ttfamily}

	\begin{itemize}

		\item Se ha implementado una nueva señal en el método
			\smaller
				\texttt{\textbackslash
					TYPO3\textbackslash
					CMS\textbackslash
					Extensionmanager\textbackslash
					Utility\textbackslash
					InstallUtility::install()}
			\normalsize
			la cual emite tan pronto como se instala una extensión
			y se terminan todas las importaciones/actualizaciones

		\begin{lstlisting}
			// execution
			$this->emitAfterExtensionInstallSignal($extensionKey);

			// methode
			protected function emitAfterExtensionInstallSignal($extensionKey) {
			  $this->signalSlotDispatcher->dispatch(
			    __CLASS__,
			    'afterExtensionInstall',
			    array($extensionKey, $this)
			  );
			}
		\end{lstlisting}

	\end{itemize}

\end{frame}

% ------------------------------------------------------------------------------
% LTXE-SLIDE-START
% LTXE-SLIDE-UID:		0003f58f-60065ba2-06e468ec-9cdf00e3
% LTXE-SLIDE-ORIGIN:	acb1ba95-1a401bef-549072a1-f32263a9 English
% LTXE-SLIDE-TITLE:		Registry for adding text extractor classes (1)
% LTXE-SLIDE-REFERENCE:	Feature-36743-FAL-TextExtractorRegistry.rst
% ------------------------------------------------------------------------------
\begin{frame}[fragile]
	\frametitle{Cambios en Profundidad}
	\framesubtitle{Registro para Extracción de Texto (1)}

	% decrease font size for code listing
	\lstset{basicstyle=\tiny\ttfamily}

	\begin{itemize}

		\item Se pueden registrar múltiples extractores de texto para permitir que trabajen
			con diferentes tipos de archivos (p.e. Office, archivos PDF, etc.)

		\item El núcleo de TYPO3 viene con un extractor para archivos de texto sin formato

		\item Cada clase de extractor de texto registrado necesita implementar \texttt{TextExtractorInterface}

		\item ...y los siguientes métodos:\newline
			\texttt{canExtractText()}\newline
			\small
				revisa si es posible la extracción de texto del archivo
			\normalsize
			\newline
			\texttt{extractText()}\newline
			\small
				devuelve el contenido del texto del archivo como una cadena
			\normalsize

	\end{itemize}

\end{frame}

% ------------------------------------------------------------------------------
% LTXE-SLIDE-START
% LTXE-SLIDE-UID:		cb41a58e-a6624a00-d9120b1a-7eab0bca
% LTXE-SLIDE-ORIGIN:	acb1ba95-1a401bef-549072a1-f32263a9 English
% LTXE-SLIDE-TITLE:		Registry for adding text extractor classes (2)
% LTXE-SLIDE-REFERENCE:	Feature-36743-FAL-TextExtractorRegistry.rst
% ------------------------------------------------------------------------------
\begin{frame}[fragile]
	\frametitle{Cambios en Profundidad}
	\framesubtitle{Registro para Extracción de Texto (2)}

	% decrease font size for code listing
	\lstset{basicstyle=\tiny\ttfamily}

	\begin{itemize}

		\item Registro del extractor de texto en el archivo \texttt{ext\_localconf.php}:

			\begin{lstlisting}
				$textExtractorRegistry = \TYPO3\CMS\Core\Resource\TextExtraction\TextExtractorRegistry::getInstance();
				$textExtractorRegistry->registerTextExtractor(
				  \TYPO3\CMS\Core\Resource\TextExtraction\PlainTextExtractor::class
				);
			\end{lstlisting}

		\item Uso de la manera siguiente:

			\begin{lstlisting}
				$textExtractorRegistry = \TYPO3\CMS\Core\Resource\TextExtraction\TextExtractorRegistry::getInstance();
				$extractor = $textExtractorRegistry->getTextExtractor($file);
				if($extractor !== NULL) {
				  $content = $extractor->extractText($file);
				}
			\end{lstlisting}
	\end{itemize}

\end{frame}

% ------------------------------------------------------------------------------
% LTXE-SLIDE-START
% LTXE-SLIDE-UID:		af32d6f2-64f3462e-4275ccb8-815e1b73
% LTXE-SLIDE-ORIGIN:	d9657b5d-5cd56c65-d7266a27-401dd233 English
% LTXE-SLIDE-TITLE:		Miscellaneous
% LTXE-SLIDE-REFERENCE:	Feature-66042-WebLibrariesLoadedViaBower.rst
% LTXE-SLIDE-REFERENCE:	Feature-63703-AddOptionToStopTask.rst
% LTXE-SLIDE-REFERENCE:	Feature-61463-AllowProcessedFoldersInDifferentStorage.rst
% LTXE-SLIDE-REFERENCE:	Feature-52693-TSFE-RequestedId.rst
% ------------------------------------------------------------------------------

\begin{frame}[fragile]
	\frametitle{Cambios en Profundidad}
	\framesubtitle{Miscelánea}

	\begin{itemize}

		\item Las bibliotecas Web (como Twitter Bootstrap, jQuery, Font Awesome, etc.) usan
			"Bower" (\url{http://bower.io}) y ya no son parte del repositorio Git del núcleo de TYPO3 \newline
			\small
				\texttt{\# bower install}	\tabto{3.4cm}ejecuta una instalación \newline
				\texttt{\# bower update}		\tabto{3.4cm}ejecuta una actualización \newline
			\normalsize
			(el archivo \texttt{bower.json} está ubicado en el directorio \texttt{Build/})

		\item Programador CLI ha recibido la opción "\texttt{-s}" para detener una tarea en ejecución

		\item La carpeta de procesamiento de un almacenamiento (remoto) puede estar
			fuera del almacenamiento (útil en caso de almacenamiento de solo lectura, por ejemplo)

		\item Ahora es posible recuperar el ID de la página solicitada originalmente:\newline
			\texttt{\$TSFE->getRequestedId()}

	\end{itemize}

\end{frame}

% ------------------------------------------------------------------------------
