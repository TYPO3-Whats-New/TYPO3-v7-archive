% ------------------------------------------------------------------------------
% TYPO3 CMS 7.2 - What's New - Chapter "Extbase & Fluid" (German Version)
%
% @author	Patrick Lobacher <patrick@lobacher.de> and Michael Schams <schams.net>
% @license	Creative Commons BY-NC-SA 3.0
% @link		http://typo3.org/download/release-notes/whats-new/
% @language	German
% ------------------------------------------------------------------------------
% LTXE-CHAPTER-UID:		7895b9cc-5939ed80-62e6d9b9-bcb40d5e
% LTXE-CHAPTER-NAME:	Chapter: Extbase & Fluid
% ------------------------------------------------------------------------------

\section{Extbase \& Fluid}
\begin{frame}[fragile]
	\frametitle{Extbase \& Fluid}

	\begin{center}\huge{Kapitel 4:}\end{center}
	\begin{center}\huge{\color{typo3darkgrey}\textbf{Extbase \& Fluid}}\end{center}

\end{frame}

% ------------------------------------------------------------------------------
% LTXE-SLIDE-START
% LTXE-SLIDE-UID:		44467f9a-abe64946-3da2bf4c-5bd094eb
% LTXE-SLIDE-TITLE:		Introduce callouts to replace content alerts
% LTXE-SLIDE-REFERENCE:	Feature-66077-IntroduceCalloutsToReplaceContentAlerts.rst
% ------------------------------------------------------------------------------

\begin{frame}[fragile]
	\frametitle{Extbase \& Fluid}
	\framesubtitle{Callouts anstelle von FlashMessages}

	\begin{itemize}

		\item An einigen Stellen im Backend werden nun Callouts anstelle von FlashMessages verwendet

		\item Dafür wurde ein neuer Fluid ViewHelper \texttt{be.infobox} eingeführt:

			\begin{lstlisting}
				<f:be.infobox title="Message title">
				   Inhalt der Nachricht
				</f:be.infobox>

				<f:be.infobox
				   title="Message title"
				   message="your box content"
				   state="-2"
				   iconName="check"
				   disableIcon="TRUE" />
			\end{lstlisting}

	\end{itemize}

\end{frame}

% ------------------------------------------------------------------------------
% LTXE-SLIDE-START
% LTXE-SLIDE-UID:		b5ee213c-b04df592-d31ed084-b94e0f58
% LTXE-SLIDE-TITLE:		FormatCaseViewHelper
% LTXE-SLIDE-REFERENCE: Feature-58621-FormatCaseViewHelper.rst
% ------------------------------------------------------------------------------

\begin{frame}[fragile]
	\frametitle{Extbase \& Fluid}
	\framesubtitle{format.case ViewHelper}

	\begin{itemize}

		\item Es gibt nun einen \texttt{format.case} ViewHelper, der die Schreibweise von Strings verändert:
			\begin{itemize}
				\item \texttt{upper}: Führt zu "UPPERCASE" (Großbuchstaben)
				\item \texttt{lower}: Führt zu "lowercase" (Kleinbuchstaben)
				\item \texttt{capital}: Führt zu einem großen Anfangsbuchstaben
				\item \texttt{uncapital}: Führt zu einem kleinen Anfangsbuchstaben
			\end{itemize}

			\begin{lstlisting}
				// Fuehrt zu "SOME TEXT WITH MIXED CASE"
				<f:format.case>Some TeXt WiTh miXed cAse</f:format.case>

				// Fuehrt zu "SomeString"
				<f:format.case mode="capital">someString</f:format.case>
			\end{lstlisting}

	\end{itemize}

\end{frame}

% ------------------------------------------------------------------------------
% LTXE-SLIDE-START
% LTXE-SLIDE-UID:		586287f8-fa46da57-a201e1bf-87532889
% LTXE-SLIDE-TITLE:		Skip cache hash for URIs to non-cacheable actions
% LTXE-SLIDE-REFERENCE:	Breaking-60272-SkipCacheHashForUrisToNonCacheableActions.rst
% ------------------------------------------------------------------------------

\begin{frame}[fragile]
	\frametitle{Extbase \& Fluid}
	\framesubtitle{Diverses}

	\begin{itemize}

		\item Parameter \texttt{cHash} wird nicht mehr an URLs angehängt, die auf eine
			Action leiten, welche nicht gecached wird oder wenn der Request nicht gecached
			wird.

	\end{itemize}

\end{frame}
