% ------------------------------------------------------------------------------
% TYPO3 CMS 7.4 - What's New - Chapter "TypoScript" (English Version)
%
% @author	Michael Schams <schams.net>
% @license	Creative Commons BY-NC-SA 3.0
% @link		http://typo3.org/download/release-notes/whats-new/
% @language	English
% ------------------------------------------------------------------------------
% LTXE-CHAPTER-UID:		693ffeb9-5ce46f4e-8c0b7d7c-be9f3ee3
% LTXE-CHAPTER-NAME:	TypoScript
% ------------------------------------------------------------------------------

\section{TSconfig \& TypoScript}
\begin{frame}[fragile]
	\frametitle{TSconfig \& TypoScript}

	\begin{center}\huge{Capítulo 2:}\end{center}
	\begin{center}\huge{\color{typo3darkgrey}\textbf{TSconfig \& TypoScript}}\end{center}

\end{frame}

% ------------------------------------------------------------------------------
% LTXE-SLIDE-START
% LTXE-SLIDE-UID:		5a2ba363-cbbbdf98-dbce0969-0c36c1e2
% LTXE-SLIDE-ORIGIN:	69805fcd-5bf6738b-3d31552e-d0add3ee English
% LTXE-SLIDE-ORIGIN:	78394814-b88d7d32-54dd1c41-022129d0 German
% LTXE-SLIDE-TITLE:		Feature: #61903 - PageTS dataprovider for backend layouts (1)
% LTXE-SLIDE-REFERENCE:	Feature-61903-PageTSDataproviderForBackendLayouts.rst
% ------------------------------------------------------------------------------
\begin{frame}[fragile]
	\frametitle{TSconfig \& TypoScript}
	\framesubtitle{Proveedor de Datos para Backend Layouts (1)}

	% decrease font size for code listing
	\lstset{basicstyle=\tiny\ttfamily}

	\begin{itemize}
		\item Ahora es posible definir backend layouts a través de TSconfig de página y también almacenarlos en ficheros. Por ejemplo:

		\begin{lstlisting}
			mod {
			  web_layout {
			    BackendLayouts {
			      exampleKey {
			        title = Example
			        config {
			          backend_layout {
			            colCount = 1
			            rowCount = 2
			            rows {
			              1 {
			                columns {
			                  1 {
			                    name = LLL:EXT:frontend/ ... /locallang_ttc.xlf:colPos.I.3
			                    colPos = 3
			                    colspan = 1
			                  }
			                }
			              }
			[...]
		\end{lstlisting}

	\end{itemize}

\end{frame}

% ------------------------------------------------------------------------------
% LTXE-SLIDE-START
% LTXE-SLIDE-UID:		8563607c-d06343f6-5a890486-dedefd14
% LTXE-SLIDE-ORIGIN:	0691efc7-c5a339e2-1753851e-9a62d8ed English
% LTXE-SLIDE-ORIGIN:	b88d4814-7d327839-10221c41-54dd29d0 German
% LTXE-SLIDE-TITLE:		Feature: #61903 - PageTS dataprovider for backend layouts (2)
% LTXE-SLIDE-REFERENCE:	Feature-61903-PageTSDataproviderForBackendLayouts.rst
% ------------------------------------------------------------------------------
\begin{frame}[fragile]
	\frametitle{TSconfig \& TypoScript}
	\framesubtitle{Proveedor de Datos para Backend Layouts (2)}

	% decrease font size for code listing
	\lstset{basicstyle=\tiny\ttfamily}

	\begin{itemize}
		\item \smaller(continuación)\normalsize
		\begin{lstlisting}
			[...]
			              2 {
			                columns {
			                  1 {
			                    name = Main
			                    colPos = 0
			                    colspan = 1
			                  }
			                }
			              }
			            }
			          }
			        }
			        icon = EXT:example_extension/Resources/Public/Images/BackendLayouts/default.gif
			      }
			    }
			  }
			}
		\end{lstlisting}
	\end{itemize}

\end{frame}

% ------------------------------------------------------------------------------
% LTXE-SLIDE-START
% LTXE-SLIDE-UID:		24954b08-82ec43f7-a8077490-d04d1839
% LTXE-SLIDE-ORIGIN:	53593b48-60766071-136bf56b-f63caf39 English
% LTXE-SLIDE-ORIGIN:	9fc7749d-aa1d6042-b1b7699d-e9aee9ed German
% LTXE-SLIDE-TITLE:		Feature: #67360 - Custom attribute name and multiple values for meta tags
% LTXE-SLIDE-REFERENCE:	Feature-67360-CustomAttributeNameAndMultipleValuesForMetaTags.rst
% ------------------------------------------------------------------------------

\begin{frame}[fragile]
	\frametitle{TSconfig \& TypoScript}
	\framesubtitle{Meta Tags Extendidos}

	% decrease font size for code listing
	\lstset{basicstyle=\tiny\ttfamily}

	\begin{itemize}

		% Note for translators: make sure, the following bullet point is only ONE line long.
		% If it wraps and uses two lines, the code example below does not fit on the page.
		% If not avoidable, split the content and create a second slide for the code.

		\item Opción \texttt{page.meta} ahora soporta nombres de atributo \href{http://ogp.me}{Open Graph}

			\begin{lstlisting}
				page {
				  meta {
				    X-UA-Compatible = IE=edge,chrome=1
				    X-UA-Compatible.attribute = http-equiv
				    keywords = TYPO3
				    # <meta property="og:site_name" content="TYPO3" />
				    og:site_name = TYPO3
				    og:site_name.attribute = property
				    description = Inspiring people to share
				    og:description = Inspiring people to share
				    og:description.attribute = property
				    og:locale = en_GB
				    og:locale.attribute = property
				    og:locale:alternate {
				      attribute = property
				      value.1 = fr_FR
				      value.2 = de_DE
				    }
				    refresh = 5; url=http://example.com/
				    refresh.attribute = http-equiv
				  }
				}
			\end{lstlisting}

	\end{itemize}

\end{frame}

% ------------------------------------------------------------------------------
% LTXE-SLIDE-START
% LTXE-SLIDE-UID:		f0b2c862-d3507a44-e2eb4c4e-02821601
% LTXE-SLIDE-ORIGIN:	5df1ea93-74ddf5d4-c8acd647-7f47e896 English
% LTXE-SLIDE-ORIGIN:	c38a626f-40c7bda9-523650fb-6829234e German
% LTXE-SLIDE-TITLE:		Feature: #68191 - TypoScript .select option languageField is active by default
% LTXE-SLIDE-REFERENCE:	Feature-68191-TypoScriptSelectOptionLanguageFieldIsActiveByDefault.rst
% ------------------------------------------------------------------------------

\begin{frame}[fragile]
	\frametitle{TSconfig \& TypoScript}
	\framesubtitle{\texttt{languageField} Activado por Defecto}

	% decrease font size for code listing
	\lstset{basicstyle=\tiny\ttfamily}

	\begin{itemize}

		\item Opción TypoScript \texttt{select} (usada en cObject CONTENT por ejemplo) requerida para configurar
			\texttt{languageField} explícitamente

		\item Esto no es requerido más, ya que el ajuste es ahora recogido de la estructura de información de TCA automáticamente

			% reminder for translators: special characters such as German umlauts are not
			% allowed inside "lstlisting". They break the LaTeX code. Be careful when you
			% want to translate the sentence "the following line is not required anymore"
			% below (feel free to leave it in English).

			\begin{lstlisting}
				config.sys_language_uid = 2
				page.10 = CONTENT
				page.10 {
				  table = tt_content
				  select.where = colPos=0

				  # the following line is not required anymore:
				  #select.languageField = sys_language_uid

				  renderObj = TEXT
				  renderObj.field = header
				  renderObj.htmlSpecialChars = 1
				}
			\end{lstlisting}

	\end{itemize}

\end{frame}

% ------------------------------------------------------------------------------
% LTXE-SLIDE-START
% LTXE-SLIDE-UID:		75820a4b-3d053e17-1773ae25-a053d2bd
% LTXE-SLIDE-ORIGIN:	48fffdc3-1a0b74a9-f2415121-1e9f032a English
% LTXE-SLIDE-ORIGIN:	945ae684-9d826d2a-1c2e0d7b-c6c69465 German
% LTXE-SLIDE-TITLE:		Feature: #64200 - Allow individual content caching
% LTXE-SLIDE-REFERENCE:	Feature-64200-AllowIndividualContentCaching.rst
% ------------------------------------------------------------------------------
\begin{frame}[fragile]
	\frametitle{TSconfig \& TypoScript}
	\framesubtitle{Cacheo Individual de Contenido}

	% decrease font size for code listing
	\lstset{basicstyle=\tiny\ttfamily}

	\begin{itemize}

		\item Desde TYPO3 CMS 7.4 existe un cacheo individual de contenido que - comparado a \texttt{stdWrap.cache} -
			también trabaja con COA objects\newline(similar al "Cacheo de Bloque de Magento")

			\begin{columns}[T]
				\begin{column}{.07\textwidth}
                \end{column}
				\begin{column}{.465\textwidth}
					\begin{lstlisting}
						page = PAGE
						page.10 = COA
						page.10 {
						  cache.key = coaout
						  cache.lifetime = 60
						  #stdWrap.cache.key = coastdWrap
						  #stdWrap.cache.lifetime = 60
						  10 = TEXT
						  10 {
						    cache.key = mycurrenttimestamp
						    cache.lifetime = 60
						    data = date : U
						    strftime = %H:%M:%S
						    noTrimWrap = |10: | |
						  }
					[...]
					\end{lstlisting}
				\end{column}

				\begin{column}{.465\textwidth}
					\begin{lstlisting}
					[...]
						  20 = TEXT
						  20 {
						    data = date : U
						    strftime = %H:%M:%S
						    noTrimWrap = |20: | |
						  }
						}
					\end{lstlisting}

				\end{column}
			\end{columns}

	\end{itemize}

\end{frame}

% ------------------------------------------------------------------------------
% LTXE-SLIDE-START
% LTXE-SLIDE-UID:		60139033-bdbb4f6d-852a6474-45a82b7d
% LTXE-SLIDE-ORIGIN:	a9f58068-74a67b75-cd3b44ef-4c5422b9 English
% LTXE-SLIDE-ORIGIN:	d4fd55ab-f9f1cd6c-4eb2ca00-d50beeb2 German
% LTXE-SLIDE-TITLE:		Feature: #67880 - Added count to listNum
% LTXE-SLIDE-REFERENCE:	Feature-67880-AddedCountToListNum.rst
% ------------------------------------------------------------------------------

\begin{frame}[fragile]
	\frametitle{TSconfig \& TypoScript}
	\framesubtitle{Contar Elementos en una Lista}

	% decrease font size for code listing
	\lstset{basicstyle=\tiny\ttfamily}

	\begin{itemize}

		\item Se ha añadido una nueva propiedad \texttt{returnCount} a la propiedad stdWrap \texttt{split}

		\item Ésta permite contar el número de elementos en una lista separada por comas

		\item El siguiente código devuelve \texttt{9} por ejemplo:

			\begin{lstlisting}
				1 = TEXT
				1 {
				  value = x,y,z,1,2,3,a,b,c
				  split.token = ,
				  split.returnCount = 1
				}
			\end{lstlisting}

	\end{itemize}

\end{frame}

% ------------------------------------------------------------------------------
% LTXE-SLIDE-START
% LTXE-SLIDE-UID:		07e9c402-31a2030e-b6fae013-3bd0128a
% LTXE-SLIDE-ORIGIN:	525607c3-d2f26693-2b44076d-e90599ca English
% LTXE-SLIDE-ORIGIN:	a31ad0e4-857ff98a-239b23fd-c2e48958 German
% LTXE-SLIDE-TITLE:		Feature: #65550 - Make table display order configurable in List module
% LTXE-SLIDE-REFERENCE:	Feature-65550-MakeTableDisplayOrderConfigurableInListModule.rst
% ------------------------------------------------------------------------------

\begin{frame}[fragile]
	\frametitle{TSconfig \& TypoScript}
	\framesubtitle{Clasificar el Orden de Tablas en Vista de Lista}

	% decrease font size for code listing
	\lstset{basicstyle=\tiny\ttfamily}

	\begin{itemize}

		\item Se ha añadido una nueva opción TSconfig \texttt{mod.web\_list.tableDisplayOrder} al módulo "Lista"

		\item Con esta opción, es configurable el orden en el que se muestran las tablas

		\item Se pueden usar las palabras clave \texttt{before} y \texttt{after} para especificar un orden relativo a otros nombres de tabla

		\begin{columns}[T]
			\begin{column}{.07\textwidth}
			\end{column}
			\begin{column}{.465\textwidth}

				\small Sintaxis:\normalsize

				\begin{lstlisting}
					mod.web_list.tableDisplayOrder {
					  <tableName> {
					    before = <tableA>, <tableB>, ...
					    after = <tableA>, <tableB>, ...
					  }
					}
				\end{lstlisting}
			\end{column}
			\begin{column}{.465\textwidth}

				\small Por ejemplo:\normalsize

				\begin{lstlisting}
					mod.web_list.tableDisplayOrder {
					  be_users.after = be_groups
					  sys_filemounts.after = be_users
					  pages_language_overlay.before = pages
					  fe_users.after = fe_groups
					  fe_users.before = pages
					}
				\end{lstlisting}

			\end{column}
		\end{columns}

	\end{itemize}

\end{frame}

% ------------------------------------------------------------------------------
% LTXE-SLIDE-START
% LTXE-SLIDE-UID:		92fd38f5-af751bc2-c83f87c8-74ad96d0
% LTXE-SLIDE-ORIGIN:	9d764efe-2cd7e74e-f53f456b-f4a88331 English
% LTXE-SLIDE-ORIGIN:	35e3a45d-888c3fe2-26d320d1-ce6c65c3 German
% LTXE-SLIDE-TITLE:		Feature: #33071 - Add the http header "Content-Language" when rendering a page
% LTXE-SLIDE-REFERENCE:	Feature-33071-AddTheHttpHeaderContent-LanguageWhenRenderingAPage.rst
% ------------------------------------------------------------------------------

\begin{frame}[fragile]
	\frametitle{TSconfig \& TypoScript}
	\framesubtitle{Content-Language en Cabecera HTTP}

	\begin{itemize}

		\item Cabecera HTTP \texttt{Content-language: XX} se envía por defecto, donde "XX" es el código ISO de la
			configuración \texttt{sys\_language\_content}

		\item Usando \texttt{config.disableLanguageHeader = 1}, puede deshabilitarse esta característica
			(no enviar la cabecera \texttt{Content-language} en absoluto)

	\end{itemize}

\end{frame}

% ------------------------------------------------------------------------------
% LTXE-SLIDE-START
% LTXE-SLIDE-UID:		cd03167a-7e73cf49-a2269190-c389aeb7
% LTXE-SLIDE-ORIGIN:	17e7f151-0ebb8899-6209180f-bec0fc56 English
% LTXE-SLIDE-ORIGIN:	35a84207-7b86ee45-c39d815d-859811e1 German
% LTXE-SLIDE-TITLE:		Feature: #45725 - Added recursive option to folder based file collections
% LTXE-SLIDE-REFERENCE:	Feature-45725-AddedRecursiveOptionToFolderBasedFileCollections.rst
% ------------------------------------------------------------------------------

\begin{frame}[fragile]
	\frametitle{TSconfig \& TypoScript}
	\framesubtitle{Opción Recursiva para Colecciones de Ficheros}

	\begin{itemize}

		\item Colecciones de ficheros basadas en carpetas tienen una opción para recoger todos los ficheros recursivamente de la carpeta proporcionada ahora

		\item La opción está también disponible en el Objeto TypoScript \texttt{FILES}

			\begin{lstlisting}
				filecollection = FILES
				filecollection {
				  folders = 1:images/
				  folders.recursive = 1
				  renderObj = IMAGE
				  renderObj {
				    file.import.data = file:current:uid
				  }
				}
			\end{lstlisting}

	\end{itemize}

\end{frame}

% ------------------------------------------------------------------------------
% LTXE-SLIDE-START
% LTXE-SLIDE-UID:		f12a1aba-f5d51fc4-5a2bbb6d-bcbb95cc
% LTXE-SLIDE-ORIGIN:	b7337338-ab366edd-bd94e9c0-6d7e86de English
% LTXE-SLIDE-ORIGIN:	ece6c712-df09d880-9378940c-e658d6a5 German
% LTXE-SLIDE-TITLE:		Feature: #34922 - Allow .ts file extension for static TypoScript templates
% LTXE-SLIDE-REFERENCE:	Feature-34922-AllowTsFileExtensionForStaticTyposcriptTemplates.rst
% ------------------------------------------------------------------------------

\begin{frame}[fragile]
	\frametitle{TSconfig \& TypoScript}
	\framesubtitle{Extensión \texttt{.ts} para Plantillas Estáticas}

	\begin{itemize}

		\item En TYPO3 CMS < 7.4, sólo se permiten los siguientes ficheros de nombre como plantillas TypoScript estáticas:

			\begin{itemize}
				\item \texttt{constants.txt}
				\item \texttt{setup.txt}
				\item \texttt{include\_static.txt}
				\item \texttt{include\_static\_files.txt}
			\end{itemize}

		\item Para \texttt{constants} y \texttt{setup}, la extensión de fichero \texttt{.ts} también está permitida ahora

		\item En este contexto, \texttt{.ts} se prioriza sobre \texttt{.txt}

	\end{itemize}

\end{frame}

% ------------------------------------------------------------------------------
% LTXE-SLIDE-START
% LTXE-SLIDE-UID:		69a056d0-cfaf7a31-481665f5-7ea5a2d5
% LTXE-SLIDE-ORIGIN:	6e127c7e-38fe07f8-e6b9545b-9142c306 English
% LTXE-SLIDE-ORIGIN:	168d7ca5-b9a3c55b-d6300076-7abdfd09 German
% LTXE-SLIDE-TITLE:		Feature: #20194 - Configuration for displaying the "save & view" button
% LTXE-SLIDE-REFERENCE:	Feature-20194-ConfigurationForDisplayingTheSaveViewButton.rst
% ------------------------------------------------------------------------------

\begin{frame}[fragile]
	\frametitle{TSconfig \& TypoScript}
	\framesubtitle{Botón de Guardar \& ver}

	\begin{itemize}

		\item Ahora se puede configurar el botón de "guardar \& ver" a través de TSconfig

		\item TSconfig \texttt{TCEMAIN.preview.disableButtonForDokType} acepta una lista separada por comas de "doktypes"

		\item Valor por defecto es "254, 255, 199" (que es: Carpeta de Almacenamiento, Papelera de reciclaje y Separador de Menú)

		\item Como consecuencia, el botón de "guardar \& ver" no es mostrado en carpetas y páginas de reciclaje por defecto nunca más

	\end{itemize}

\end{frame}

% ------------------------------------------------------------------------------
% LTXE-SLIDE-START
% LTXE-SLIDE-UID:		b43f1ad3-608cfce9-21c892c4-c530080d
% LTXE-SLIDE-ORIGIN:	eb340ca0-00687bcb-51cda09f-82cb51ce English
% LTXE-SLIDE-ORIGIN:	80efec02-bec87365-a8bc2ab6-6741053f German
% LTXE-SLIDE-TITLE:		Feature: #43984 - Add stdWrap functionality to TreatIdAsReference TypoScript
% LTXE-SLIDE-REFERENCE:	Feature-43984-AddStdWrapFunctionalityToTreatIdAsReferenceTypoScript.rst
% ------------------------------------------------------------------------------
\begin{frame}[fragile]
	\frametitle{TSconfig \& TypoScript}
	\framesubtitle{stdWrap para \texttt{treatIdAsReference}}

	\begin{itemize}

		\item Para el objeto \texttt{getImgResource} existe la opción \texttt{treatIdAsReference},
			que puede ser usada para definir que los UIDs son tratados como UIDs de \texttt{sys\_file\_reference}
			en lugar de \texttt{sys\_file}.

		\item La opción \texttt{treatIdAsReference} recibe la funcionalidad stdWrap ahora

	\end{itemize}

\end{frame}

% ------------------------------------------------------------------------------
