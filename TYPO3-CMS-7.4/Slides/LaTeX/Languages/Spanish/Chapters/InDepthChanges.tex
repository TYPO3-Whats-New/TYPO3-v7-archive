% ------------------------------------------------------------------------------
% TYPO3 CMS 7.4 - What's New - Chapter "In-Depth Changes" (English Version)
%
% @author	Michael Schams <schams.net>
% @license	Creative Commons BY-NC-SA 3.0
% @link		http://typo3.org/download/release-notes/whats-new/
% @language	English
% ------------------------------------------------------------------------------
% LTXE-CHAPTER-UID:		9a69925b-ed88557b-d362988a-940ed520
% LTXE-CHAPTER-NAME:	In-Depth Changes
% ------------------------------------------------------------------------------

\section{Cambios en Profundidad}
\begin{frame}[fragile]
	\frametitle{Cambios en Profundidad}

	\begin{center}\huge{Capítulo 4:}\end{center}
	\begin{center}\huge{\color{typo3darkgrey}\textbf{Cambios en Profundidad}}\end{center}

\end{frame}

% ------------------------------------------------------------------------------
% LTXE-SLIDE-START
% LTXE-SLIDE-UID:		36aca8c8-0421e32d-654b28d7-1ba68d1f
% LTXE-SLIDE-ORIGIN:	5fa3f922-971c07f7-2ec96457-63394b8b English
% LTXE-SLIDE-ORIGIN:	5cfc1d44-3d6f3d87-789957ab-655f357d German
% LTXE-SLIDE-TITLE:		Breaking #65305: DriverInterface has been extended
% LTXE-SLIDE-REFERENCE:	Breaking-65305-AddFunctionsToDriverInterface.rst
% ------------------------------------------------------------------------------

\begin{frame}[fragile]
	\frametitle{Cambios en Profundidad}
	\framesubtitle{Interfaz Driver}

	% decrease font size for code listing
	\lstset{basicstyle=\tiny\ttfamily}

	\begin{itemize}

		\item Se han añadido los siguientes métodos a \texttt{DriverInterface}:

			\begin{itemize}
				\item \texttt{getFolderInFolder}
				\item \texttt{getFileInFolder}
			\end{itemize}

		\item Cada driver FAL debe implementar estos nuevos métodos:

			\begin{columns}[T]
				\begin{column}{.07\textwidth}
                \end{column}
				\begin{column}{.465\textwidth}

					\begin{lstlisting}
						public function getFoldersInFolder(
						  $folderIdentifier,
						  $start = 0,
						  $numberOfItems = 0,
						  $recursive = FALSE,
						  array $folderNameFilterCallbacks = array(),
						  $sort = '',
						  $sortRev = FALSE
						);
					\end{lstlisting}

                \end{column}
				\begin{column}{.465\textwidth}

					\begin{lstlisting}
						public function getFileInFolder(
						  $fileName,
						  $folderIdentifier
						);
					\end{lstlisting}

				\end{column}
			\end{columns}

	\end{itemize}

	\breakingchange

\end{frame}

% ------------------------------------------------------------------------------
% LTXE-SLIDE-START
% LTXE-SLIDE-UID:		8f3241ff-35d88fae-1be76dcf-4acbe422
% LTXE-SLIDE-ORIGIN:	603042c6-c0caed8e-a6565d08-09528be6 English
% LTXE-SLIDE-ORIGIN:	b719be34-650cf0d2-13ac26aa-dd3a4bbd German
% LTXE-SLIDE-TITLE:		Feature #22175: Support IEC/SI units in file size formatting
% LTXE-SLIDE-REFERENCE:	Feature-22175-SupportIecSiUnitsInFileSizeFormatting.rst
% ------------------------------------------------------------------------------

\begin{frame}[fragile]
	\frametitle{Cambios en Profundidad}
	\framesubtitle{Soporte IEC/SI en Formateo de Tamaño de Fichero}

	% decrease font size for code listing
	%\lstset{basicstyle=\tiny\ttfamily}

	\begin{itemize}

		\item Formateo de tamaño de fichero soporta dos palabras clave adicionalmente a la lista de etiquetas ahora:

			\begin{itemize}
				\item \small\texttt{iec} (por defecto)\newline
					\small(potencia de 2, etiquetas: \texttt{| Ki| Mi| Gi| Ti| Pi| Ei| Zi| Yi})\normalsize
				\item \small\texttt{si}\newline
					\small(potencia de 10, etiquetas: \texttt{| k| M| G| T| P| E| Z| Y})\normalsize
			\end{itemize}

		\item Fijar formateo en TypoScript por ejemplo:\newline
			\texttt{bytes.labels = iec}

		\begin{lstlisting}
			echo GeneralUtility::formatSize(85123);
			// => before "83.1 K"
			// => now "83.13 Ki"
		\end{lstlisting}

	\end{itemize}

\end{frame}

% ------------------------------------------------------------------------------
% LTXE-SLIDE-START
% LTXE-SLIDE-UID:		60340561-b7adff62-edab89ff-3213b43a
% LTXE-SLIDE-ORIGIN:	be6e3997-48f241bd-be7fb0f4-6fcded3f English
% LTXE-SLIDE-ORIGIN:	c5766b9f-4acc2f8d-dcadf0be-299ed259 German
% LTXE-SLIDE-TITLE:		Feature #67293: Dependency ordering service (1)
% LTXE-SLIDE-REFERENCE:	Feature-67293-DependencyOrderingService.rst
% ------------------------------------------------------------------------------

\begin{frame}[fragile]
	\frametitle{Cambios en Profundidad}
	\framesubtitle{Servicio de Orden de Dependencia (1)}

	\begin{itemize}

		\item En muchos casos es necesario crear una lista ordenada de items de un conjunto de "dependencias".
			La lista ordenada se usa entonces para ejecutar acciones en el orden proporcionado.

		\item Algunos ejemplos donde el núcleo de TYPO3 usa esto son:

			\begin{itemize}
				\item orden de ejecución de hook,
				\item orden de carga de extensiones,
				\item listado de ítems de menú,
				\item etc.
			\end{itemize}

		\item El \texttt{DependencyResolver} ha sido re-trabajado y provee un
			\texttt{DependencyOrderingService} ahora

	\end{itemize}

\end{frame}

% ------------------------------------------------------------------------------
% LTXE-SLIDE-START
% LTXE-SLIDE-UID:		0b01b4f7-042d41eb-841afc0f-e167a7c0
% LTXE-SLIDE-ORIGIN:	2fd8356d-45429468-7de53ebf-4c674627 English
% LTXE-SLIDE-ORIGIN:	4acc2f8d-6b9fc576-f0bedcad-d259299e German
% LTXE-SLIDE-TITLE:		Feature #67293: Dependency ordering service (2)
% LTXE-SLIDE-REFERENCE:	Feature-67293-DependencyOrderingService.rst
% ------------------------------------------------------------------------------

\begin{frame}[fragile]
	\frametitle{Cambios en Profundidad}
	\framesubtitle{Servicio de Pedido de Dependencia (2)}

	% decrease font size for code listing
	\lstset{basicstyle=\tiny\ttfamily}

	\begin{itemize}

		\item Uso:

			\begin{lstlisting}
				$GLOBALS['TYPO3_CONF_VARS']['EXTCONF']['someExt']['someHook'][<some id>] = [
				  'handler' => someClass::class,
				  'runBefore' => [ <some other ID> ],
				  'runAfter' => [ ... ],
				  ...
				];
			\end{lstlisting}

		\item Ejemplo:

			\begin{lstlisting}
				$hooks = $GLOBALS['TYPO3_CONF_VARS']['EXTCONF']['someExt']['someHook'];
				$sorted = GeneralUtility:makeInstance(DependencyOrderingService::class)->orderByDependencies(
				  $hooks, 'runBefore', 'runAfter'
				);
			\end{lstlisting}

	\end{itemize}

\end{frame}

% ------------------------------------------------------------------------------
% LTXE-SLIDE-START
% LTXE-SLIDE-UID:		07f5596f-00235a3c-b49a667e-00851d3d
% LTXE-SLIDE-ORIGIN:	18e5e8ba-e691a20f-3a6f211c-7c855633 English
% LTXE-SLIDE-ORIGIN:	3752346f-1d6aa26b-4b3dd1a7-f142d76b German
% LTXE-SLIDE-TITLE:		Feature #56644: Hook for InlineRecordContainer::checkAccess()
% LTXE-SLIDE-REFERENCE:	Feature-56644-AddHookToInlineRecordContainerCheckAccess.rst
% ------------------------------------------------------------------------------

\begin{frame}[fragile]
	\frametitle{Cambios en Profundidad}
	\framesubtitle{Hooks y Señales (1)}

	% decrease font size for code listing
	\lstset{basicstyle=\tiny\ttfamily}

	\begin{itemize}

		\item Se ha añadido hook para post-procesar resultados \texttt{InlineRecordContainer::checkAccess}

		\item \texttt{InlineRecordContainer::checkAccess} puede usarse para chequear el acceso a registros inline relacionados

		\item El siguiente código registra el hook:

			\begin{lstlisting}
				$GLOBALS['TYPO3_CONF_VARS']['SC_OPTIONS']['t3lib/class.t3lib_tceforms_inline.php']
				  ['checkAccess'][] = 'My\\Package\\HookClass->hookMethod';
			\end{lstlisting}

	\end{itemize}

\end{frame}

% ------------------------------------------------------------------------------
% LTXE-SLIDE-START
% LTXE-SLIDE-UID:		c166b64c-0ab25315-02f483d9-8a098d87
% LTXE-SLIDE-ORIGIN:	661cdcf3-cd43fd27-06ab1889-a7a22ea6 English
% LTXE-SLIDE-ORIGIN:	158946bb-f31ee48e-ee6d0762-6cac7a12 German
% LTXE-SLIDE-TITLE:		Feature #59231: Hook for AbstractUserAuthentication::checkAuthentication()
% LTXE-SLIDE-REFERENCE:	Feature-59231-AddHookToAbstractUserAuthenticationCheckAuthentication.rst
% ------------------------------------------------------------------------------

\begin{frame}[fragile]
	\frametitle{Cambios en Profundidad}
	\framesubtitle{Hooks y Señales (2)}

	% decrease font size for code listing
	\lstset{basicstyle=\tiny\ttfamily}

	\begin{itemize}

		\item Se ha añadido hook para post-procesar fallos de autenticación en \texttt{AbstractUserAuthentication::checkAuthentication}

		\item El proceso se detiene por defecto durante 5 seconds en el caso de una autenticación fallida

		\item Usando este nuevo hook, pueden implementarse soluciones alternativas
			(p.e. para prevenir ataques de fuerza bruta)

		\item El siguiente código registra el hook:

			\begin{lstlisting}
				$GLOBALS['TYPO3_CONF_VARS']['SC_OPTIONS']['t3lib/class.t3lib_userauth.php']
				  ['postLoginFailureProcessing'][] = 'My\\Package\\HookClass->hookMethod';
			\end{lstlisting}

	\end{itemize}

\end{frame}

% ------------------------------------------------------------------------------
% LTXE-SLIDE-START
% LTXE-SLIDE-UID:		779f7b7a-f6e1c3f5-5e2d2ff0-b6bb4db4
% LTXE-SLIDE-ORIGIN:	7cafd821-40468370-4a3daa2a-efdb473f English
% LTXE-SLIDE-ORIGIN:	cfa748d2-7c412900-462446d1-14dd41e4 German
% LTXE-SLIDE-TITLE:		Feature #67228 and #68047
% LTXE-SLIDE-REFERENCE:	Feature-67228-EmitSignalWhenAnIndexRecordIsMarkedAsMissing.rst
% LTXE-SLIDE-REFERENCE:	Feature-68047-EmitASignalForEachMappedObject.rst
% ------------------------------------------------------------------------------

\begin{frame}[fragile]
	\frametitle{Cambios en Profundidad}
	\framesubtitle{Hooks y Señales (3)}

	\begin{itemize}

		\item Nueva señal \texttt{recordMarkedAsMissing} es emitida cuando el indexador FAL encuentra un
			registro \texttt{sys\_file} que no tiene una entrada de sistema de ficheros correspondiente y lo marca como
			ausente. La señal pasa el UID del registro de \texttt{sys\_file}.

		\item Esto es útil en extensiones que proveen o extienden capacidades de manejo de ficheros tales como
			versionado, sincronizaciones, recuperación, etc.

		\item Señal \texttt{afterMappingSingleRow} es emitida siempre que el DataMapper crea un objeto

	\end{itemize}

\end{frame}

% ------------------------------------------------------------------------------
% LTXE-SLIDE-START
% LTXE-SLIDE-UID:		30806d40-e3cf2723-1846c7bc-ec304701
% LTXE-SLIDE-ORIGIN:	a1ba770c-9d79e259-d64ab6a4-453c16b1 English
% LTXE-SLIDE-ORIGIN:	24a6aa34-aa43c7ad-29712401-32d010f3 German
% LTXE-SLIDE-TITLE:		Breaking #55759: HTML (e.g. double quotes) in link titles
% LTXE-SLIDE-REFERENCE:	Breaking-55759-HTMLInLinkTitlesNotWorkingAnymore.rst
% ------------------------------------------------------------------------------
% Breaking #55759: HTML (e.g. double quotes) in link titles

\begin{frame}[fragile]
	\frametitle{Cambios en Profundidad}
	\framesubtitle{HTML en Títulos TypoLink}

	% decrease font size for code listing
	\lstset{basicstyle=\tiny\ttfamily}

	\begin{itemize}

		\item Comillas en títulos TypoLink se \textit{escapan} automáticamente ahora
		\item Esto significa que instancias donde se escapan manualmente códigos HTML,
			romperán la salida del frontend en TYPO CMS 7.4

			Antes:\tabto{1.8cm}'\texttt{Algún título \&quot;especial\&quot;}'\newline
			Se transforma en:\tabto{1.8cm}'\texttt{Algún título \&amp;quot;especial\&amp;quot;}'

		\item Es recomendable evitar el escapado, debido al hecho de que TYPO3 se hace cargo del escapado
			HTML en los títulos TypoLink ahora

	\end{itemize}

	\breakingchange

\end{frame}

% ------------------------------------------------------------------------------
% LTXE-SLIDE-START
% LTXE-SLIDE-UID:		608fdab3-340fa471-17b43c50-4e951b8c
% LTXE-SLIDE-ORIGIN:	30299840-02ec336c-0977320a-98f21078 English
% LTXE-SLIDE-ORIGIN:	8c2cf004-beb3b185-c41a8d75-35b4410a German
% LTXE-SLIDE-TITLE:		Breaking #56133 and #52705 - Miscellaneous (1)
% LTXE-SLIDE-REFERENCE:	Breaking-56133-NewBeUserPermissionFilesReplace.rst
% LTXE-SLIDE-REFERENCE:	Breaking-52705-DefaultLogConfigurationIsChanged.rst
% ------------------------------------------------------------------------------
% Feature #56133: New BE user permission "Files: replace"
% Breaking #52705: Default log configuration is changed

\begin{frame}[fragile]
	\frametitle{Cambios en Profundidad}
	\framesubtitle{Miscelánea (1)}

	% decrease font size for code listing
	\lstset{basicstyle=\tiny\ttfamily}

	\begin{itemize}

		\item Al configurar el permiso de usuario de backend \texttt{Files->replace}, los usuarios pueden ser
			permitidos o prevenidos para reemplazar Ficheros en el módulo "Lista de ficheros"

		% note: item above is classified as a breaking change, but I can not see why (Michael)
		% \breackingchange

		\item Se usa un hash en el nombre de fichero de archivos, generado por el FileWriter,
			si no se ha configurado otro fichero de log

			\begin{itemize}
				\item antes:\tabto{1.2cm}\texttt{typo3temp/logs/typo3.log}
				\item ahora:\tabto{1.2cm}\texttt{typo3temp/logs/typo3\_<hash>.log}
			\end{itemize}

			\small(valor \texttt{<hash>} es calculado en base a la clave de encriptación)\normalsize

		% note: item above is classified as a breaking change, but I can not see why (Michael)
		% \breackingchange

	\end{itemize}

\end{frame}

% ------------------------------------------------------------------------------
% LTXE-SLIDE-START
% LTXE-SLIDE-UID:		05dce154-da6c0e57-cb0665d6-f014f925
% LTXE-SLIDE-ORIGIN:	fe577b08-f30526e0-2d631fe7-642122be English
% LTXE-SLIDE-ORIGIN:	10f3aa43-c7ad32d0-29712401-aa3424a6 German
% LTXE-SLIDE-TITLE:		Miscellaneous (2)
% LTXE-SLIDE-REFERENCE:	unknown
% ------------------------------------------------------------------------------

\begin{frame}[fragile]
	\frametitle{Cambios en Profundidad}
	\framesubtitle{Miscelánea (2)}

	% decrease font size for code listing
	\lstset{basicstyle=\tiny\ttfamily}

	\begin{itemize}

		\item Clases usadas en hooks deben de seguir el mecanismo de autocargado
		\item Por ello la definición del hook puede acortarse ahora:

		\begin{lstlisting}
			$GLOBALS['TYPO3_CONF_VARS']['SC_OPTIONS']['tce']['formevals']
			  [\TYPO3\CMS\Saltedpasswords\Evaluation\FrontendEvaluator::class] = '';
		\end{lstlisting}

	\end{itemize}

	\breakingchange

\end{frame}

% ------------------------------------------------------------------------------
