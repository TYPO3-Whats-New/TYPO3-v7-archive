% ------------------------------------------------------------------------------
% TYPO3 CMS 7.4 - What's New - Chapter "In-Depth Changes" (French Version)
%
% @author	Michael Schams <schams.net>
% @license	Creative Commons BY-NC-SA 3.0
% @link		http://typo3.org/download/release-notes/whats-new/
% @language	French
% ------------------------------------------------------------------------------
% LTXE-CHAPTER-UID:		9a69925b-ed88557b-d362988a-940ed520
% LTXE-CHAPTER-NAME:	In-Depth Changes
% ------------------------------------------------------------------------------

\section{Changements en profondeur}
\begin{frame}[fragile]
	\frametitle{Changements en profondeur}

	\begin{center}\huge{Chapitre 4~:}\end{center}
	\begin{center}\huge{\color{typo3darkgrey}\textbf{Changements en profondeur}}\end{center}

\end{frame}

% ------------------------------------------------------------------------------
% LTXE-SLIDE-START
% LTXE-SLIDE-UID:		0709e576-c838fca8-4fef69d7-034e4507
% LTXE-SLIDE-ORIGIN:	5fa3f922-971c07f7-2ec96457-63394b8b English
% LTXE-SLIDE-ORIGIN:	5cfc1d44-3d6f3d87-789957ab-655f357d German
% LTXE-SLIDE-TITLE:		Breaking #65305: DriverInterface has been extended
% LTXE-SLIDE-REFERENCE:	Breaking-65305-AddFunctionsToDriverInterface.rst
% ------------------------------------------------------------------------------

\begin{frame}[fragile]
	\frametitle{Changements en profondeur}
	\framesubtitle{Driver Interface}

	% decrease font size for code listing
	\lstset{basicstyle=\tiny\ttfamily}

	\begin{itemize}

		\item Les méthodes suivantes sont ajoutées à  \texttt{DriverInterface}~:

			\begin{itemize}
				\item \texttt{getFolderInFolder}
				\item \texttt{getFileInFolder}
			\end{itemize}

		\item Tous les FAL driver doivent implémenter ces deux méthodes~:

			\begin{columns}[T]
				\begin{column}{.07\textwidth}
                \end{column}
				\begin{column}{.465\textwidth}

					\begin{lstlisting}
						public function getFoldersInFolder(
						  $folderIdentifier,
						  $start = 0,
						  $numberOfItems = 0,
						  $recursive = FALSE,
						  array $folderNameFilterCallbacks = array(),
						  $sort = '',
						  $sortRev = FALSE
						);
					\end{lstlisting}

                \end{column}
				\begin{column}{.465\textwidth}

					\begin{lstlisting}
						public function getFileInFolder(
						  $fileName,
						  $folderIdentifier
						);
					\end{lstlisting}

				\end{column}
			\end{columns}

	\end{itemize}

	\breakingchange

\end{frame}

% ------------------------------------------------------------------------------
% LTXE-SLIDE-START
% LTXE-SLIDE-UID:		7966214a-5b8a4973-01620a7a-62d66e39
% LTXE-SLIDE-ORIGIN:	603042c6-c0caed8e-a6565d08-09528be6 English
% LTXE-SLIDE-ORIGIN:	b719be34-650cf0d2-13ac26aa-dd3a4bbd German
% LTXE-SLIDE-TITLE:		Feature #22175: Support IEC/SI units in file size formatting
% LTXE-SLIDE-REFERENCE:	Feature-22175-SupportIecSiUnitsInFileSizeFormatting.rst
% ------------------------------------------------------------------------------

\begin{frame}[fragile]
	\frametitle{Changements en profondeur}
	\framesubtitle{Support de IEC/SI dans le formatage des tailles}

	% decrease font size for code listing
	%\lstset{basicstyle=\tiny\ttfamily}

	\begin{itemize}

		\item Le formatage des tailles de fichier supporte ces deux mots clés en plus des listes de libellés~:

			\begin{itemize}
				\item \small\texttt{iec} (par défaut)\newline
					\small(puissance de 2, libellés~: \texttt{| Ki| Mi| Gi| Ti| Pi| Ei| Zi| Yi})\normalsize
				\item \small\texttt{si}\newline
					\small(puissance de 10, libellés~: \texttt{| k| M| G| T| P| E| Z| Y})\normalsize
			\end{itemize}

		\item Formatage en TypoScript par exemple~:\newline
			\texttt{bytes.labels = iec}

		\begin{lstlisting}
			echo GeneralUtility::formatSize(85123);
			// => before "83.1 K"
			// => now "83.13 Ki"
		\end{lstlisting}

	\end{itemize}

\end{frame}

% ------------------------------------------------------------------------------
% LTXE-SLIDE-START
% LTXE-SLIDE-UID:		dca147d2-b2db2dd1-c7af3b0c-b26cd822
% LTXE-SLIDE-ORIGIN:	be6e3997-48f241bd-be7fb0f4-6fcded3f English
% LTXE-SLIDE-ORIGIN:	c5766b9f-4acc2f8d-dcadf0be-299ed259 German
% LTXE-SLIDE-TITLE:		Feature #67293: Dependency ordering service (1)
% LTXE-SLIDE-REFERENCE:	Feature-67293-DependencyOrderingService.rst
% ------------------------------------------------------------------------------

\begin{frame}[fragile]
	\frametitle{Changements en profondeur}
	\framesubtitle{Service d'ordonnancement des dépendances (1)}

	\begin{itemize}

		\item Dans de nombreux cas, il est nécessaire de générer une liste ordonnée depuis un ensemble de
			«~dépendances~».
			La liste est ensuite utilisée pour exécuter les actions dans l'ordre fourni.

		\item Exemples d'usages dans le cœur de TYPO3~:

			\begin{itemize}
				\item ordre d'exécution des hook,
				\item ordre de chargement des extensions,
				\item liste des éléments de menu,
				\item etc.
			\end{itemize}

		\item Le \texttt{DependencyResolver} est retravaillé et fourni
			maintenant un \texttt{DependencyOrderingService}

	\end{itemize}

\end{frame}

% ------------------------------------------------------------------------------
% LTXE-SLIDE-START
% LTXE-SLIDE-UID:		fcf10258-80316940-73815040-d1039aef
% LTXE-SLIDE-ORIGIN:	2fd8356d-45429468-7de53ebf-4c674627 English
% LTXE-SLIDE-ORIGIN:	4acc2f8d-6b9fc576-f0bedcad-d259299e German
% LTXE-SLIDE-TITLE:		Feature #67293: Dependency ordering service (2)
% LTXE-SLIDE-REFERENCE:	Feature-67293-DependencyOrderingService.rst
% ------------------------------------------------------------------------------

\begin{frame}[fragile]
	\frametitle{Changements en profondeur}
	\framesubtitle{Service d'ordonnancement des dépendances (2)}

	% decrease font size for code listing
	\lstset{basicstyle=\tiny\ttfamily}

	\begin{itemize}

		\item Utilisation~:

			\begin{lstlisting}
				$GLOBALS['TYPO3_CONF_VARS']['EXTCONF']['someExt']['someHook'][<some id>] = [
				  'handler' => someClass::class,
				  'runBefore' => [ <some other ID> ],
				  'runAfter' => [ ... ],
				  ...
				];
			\end{lstlisting}

		\item Exemple~:

			\begin{lstlisting}
				$hooks = $GLOBALS['TYPO3_CONF_VARS']['EXTCONF']['someExt']['someHook'];
				$sorted = GeneralUtility:makeInstance(DependencyOrderingService::class)->orderByDependencies(
				  $hooks, 'runBefore', 'runAfter'
				);
			\end{lstlisting}

	\end{itemize}

\end{frame}

% ------------------------------------------------------------------------------
% LTXE-SLIDE-START
% LTXE-SLIDE-UID:		b9f41e87-dcee083c-a61180a7-d93cb01e
% LTXE-SLIDE-ORIGIN:	18e5e8ba-e691a20f-3a6f211c-7c855633 English
% LTXE-SLIDE-ORIGIN:	3752346f-1d6aa26b-4b3dd1a7-f142d76b German
% LTXE-SLIDE-TITLE:		Feature #56644: Hook for InlineRecordContainer::checkAccess()
% LTXE-SLIDE-REFERENCE:	Feature-56644-AddHookToInlineRecordContainerCheckAccess.rst
% ------------------------------------------------------------------------------

\begin{frame}[fragile]
	\frametitle{Changements en profondeur}
	\framesubtitle{Hooks et Signals (1)}

	% decrease font size for code listing
	\lstset{basicstyle=\tiny\ttfamily}

	\begin{itemize}

		\item Un hook de traitement final des résultats de \texttt{InlineRecordContainer::checkAccess} est ajouté

		\item \texttt{InlineRecordContainer::checkAccess} s'utilise pour vérifier l'accès aux éléments inline liés

		\item Le code suivant inscrit au hook~:

			\begin{lstlisting}
				$GLOBALS['TYPO3_CONF_VARS']['SC_OPTIONS']['t3lib/class.t3lib_tceforms_inline.php']
				  ['checkAccess'][] = 'My\\Package\\HookClass->hookMethod';
			\end{lstlisting}

	\end{itemize}

\end{frame}

% ------------------------------------------------------------------------------
% LTXE-SLIDE-START
% LTXE-SLIDE-UID:		ba42c336-1668c9e5-07cc24a2-54ec4e2e
% LTXE-SLIDE-ORIGIN:	661cdcf3-cd43fd27-06ab1889-a7a22ea6 English
% LTXE-SLIDE-ORIGIN:	158946bb-f31ee48e-ee6d0762-6cac7a12 German
% LTXE-SLIDE-TITLE:		Feature #59231: Hook for AbstractUserAuthentication::checkAuthentication()
% LTXE-SLIDE-REFERENCE:	Feature-59231-AddHookToAbstractUserAuthenticationCheckAuthentication.rst
% ------------------------------------------------------------------------------

\begin{frame}[fragile]
	\frametitle{Changements en profondeur}
	\framesubtitle{Hooks et Signals (2)}

	% decrease font size for code listing
	\lstset{basicstyle=\tiny\ttfamily}

	\begin{itemize}

		\item Un hook de traitement des échecs de connexion de \texttt{AbstractUserAuthentication::checkAuthentication}
			est ajouté

		\item Le processus est arrêté pendant 5 secondes en cas d'échec par défaut

		\item Des méthodes alternatives sont utilisables par ce hook
			(i.e. afin de prévenir les attaques force brute)

		\item Le code suivant inscrit au hook~:

			\begin{lstlisting}
				$GLOBALS['TYPO3_CONF_VARS']['SC_OPTIONS']['t3lib/class.t3lib_userauth.php']
				  ['postLoginFailureProcessing'][] = 'My\\Package\\HookClass->hookMethod';
			\end{lstlisting}

	\end{itemize}

\end{frame}

% ------------------------------------------------------------------------------
% LTXE-SLIDE-START
% LTXE-SLIDE-UID:		f14145fd-d2b5a51a-3825696a-82aba28e
% LTXE-SLIDE-ORIGIN:	7cafd821-40468370-4a3daa2a-efdb473f English
% LTXE-SLIDE-ORIGIN:	cfa748d2-7c412900-462446d1-14dd41e4 German
% LTXE-SLIDE-TITLE:		Feature #67228 and #68047
% LTXE-SLIDE-REFERENCE:	Feature-67228-EmitSignalWhenAnIndexRecordIsMarkedAsMissing.rst
% LTXE-SLIDE-REFERENCE:	Feature-68047-EmitASignalForEachMappedObject.rst
% ------------------------------------------------------------------------------

\begin{frame}[fragile]
	\frametitle{Changements en profondeur}
	\framesubtitle{Hooks et Signals (3)}

	\begin{itemize}

		\item Le signal \texttt{recordMarkedAsMissing} est émis lorsque l'indexeur FAL rencontre un
			enregistrement \texttt{sys\_file} à une entrée du système de fichiers et le marque comme
			manquant. L'identifiant (UID) de l'enregistrement \texttt{sys\_file} est passé au signal.

		\item Le signal est utile aux extensions fournissant ou étendant les fonctions de gestion de fichiers
			comme le versionnement, la synchronisation, la récupération, etc.

		\item Le signal \texttt{afterMappingSingleRow} est émis à chaque création d'objet par le DataMapper

	\end{itemize}

\end{frame}

% ------------------------------------------------------------------------------
% LTXE-SLIDE-START
% LTXE-SLIDE-UID:		765704c4-e3ebd5a2-bc3cf488-a8b69ed5
% LTXE-SLIDE-ORIGIN:	a1ba770c-9d79e259-d64ab6a4-453c16b1 English
% LTXE-SLIDE-ORIGIN:	24a6aa34-aa43c7ad-29712401-32d010f3 German
% LTXE-SLIDE-TITLE:		Breaking #55759: HTML (e.g. double quotes) in link titles
% LTXE-SLIDE-REFERENCE:	Breaking-55759-HTMLInLinkTitlesNotWorkingAnymore.rst
% ------------------------------------------------------------------------------
% Breaking #55759: HTML (e.g. double quotes) in link titles

\begin{frame}[fragile]
	\frametitle{Changements en profondeur}
	\framesubtitle{Code HTML dans les titres de TypoLink}

	% decrease font size for code listing
	\lstset{basicstyle=\tiny\ttfamily}

	\begin{itemize}

		\item Les guillemets sont \textit{encodés} automatiquement dans les titres de TypoLink
		\item Dans le cas où le texte est déjà encodé manuellement,
			la sortie frontend sera cassée avec TYPO CMS 7.4

			Avant~:\tabto{2cm}'\texttt{Some \&quot;special\&quot; title}'\newline
			Maintenant~:\tabto{2cm}'\texttt{Some \&amp;quot;special\&amp;quot; title}'

		\item L'encodage préalable est déconseillé, parce que TYPO3 s'occupe maintenant
			d'encoder le texte dans les titres de TypoLink

	\end{itemize}

	\breakingchange

\end{frame}

% ------------------------------------------------------------------------------
% LTXE-SLIDE-START
% LTXE-SLIDE-UID:		f3841964-bab10119-7a851ef0-a6b09bd6
% LTXE-SLIDE-ORIGIN:	30299840-02ec336c-0977320a-98f21078 English
% LTXE-SLIDE-ORIGIN:	8c2cf004-beb3b185-c41a8d75-35b4410a German
% LTXE-SLIDE-TITLE:		Breaking #56133 and #52705 - Miscellaneous (1)
% LTXE-SLIDE-REFERENCE:	Breaking-56133-NewBeUserPermissionFilesReplace.rst
% LTXE-SLIDE-REFERENCE:	Breaking-52705-DefaultLogConfigurationIsChanged.rst
% ------------------------------------------------------------------------------
% Feature #56133: New BE user permission "Files: replace"
% Breaking #52705: Default log configuration is changed

\begin{frame}[fragile]
	\frametitle{Changements en profondeur}
	\framesubtitle{Divers (1)}

	% decrease font size for code listing
	\lstset{basicstyle=\tiny\ttfamily}

	\begin{itemize}

		\item La permission backend \texttt{Fichiers->remplacer} permet d'autoriser ou
			d'interdire le remplacement de fichiers dans le module Liste de fichiers.

		% note: item above is classified as a breaking change, but I can not see why (Michael)
		% \breackingchange

		\item Les fichiers générés par FileWriter contiennent un hash dans leurs noms.
			Si aucun autre fichier journal est configuré~:

			\begin{itemize}
				\item avant~:\tabto{2cm}\texttt{typo3temp/logs/typo3.log}
				\item maintenant~:\tabto{2cm}\texttt{typo3temp/logs/typo3\_<hash>.log}
			\end{itemize}

			\small(La valeur de \texttt{<hash>} est calculée en utilisant la clé de chiffrement (encryption key))\normalsize

		% note: item above is classified as a breaking change, but I can not see why (Michael)
		% \breackingchange

	\end{itemize}

\end{frame}

% ------------------------------------------------------------------------------
% LTXE-SLIDE-START
% LTXE-SLIDE-UID:		31f72482-af56b662-f5e98ed5-a1a84db4
% LTXE-SLIDE-ORIGIN:	fe577b08-f30526e0-2d631fe7-642122be English
% LTXE-SLIDE-ORIGIN:	10f3aa43-c7ad32d0-29712401-aa3424a6 German
% LTXE-SLIDE-TITLE:		Miscellaneous (2)
% LTXE-SLIDE-REFERENCE:	unknown
% ------------------------------------------------------------------------------

\begin{frame}[fragile]
	\frametitle{Changements en profondeur}
	\framesubtitle{Divers (2)}

	% decrease font size for code listing
	\lstset{basicstyle=\tiny\ttfamily}

	\begin{itemize}

		\item Les classes utilisées par les hook doivent suivre le mécanisme d'auto-chargement
		\item L'écriture des définitions des hook est donc réduite~:

		\begin{lstlisting}
			$GLOBALS['TYPO3_CONF_VARS']['SC_OPTIONS']['tce']['formevals']
			  [\TYPO3\CMS\Saltedpasswords\Evaluation\FrontendEvaluator::class] = '';
		\end{lstlisting}

	\end{itemize}

	\breakingchange

\end{frame}

% ------------------------------------------------------------------------------
