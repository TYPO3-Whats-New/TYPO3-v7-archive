% ------------------------------------------------------------------------------
% TYPO3 CMS 7.4 - What's New - Chapter "Deprecated Functions" (French Version)
%
% @author	Michael Schams <schams.net>
% @license	Creative Commons BY-NC-SA 3.0
% @link		http://typo3.org/download/release-notes/whats-new/
% @language	French
% ------------------------------------------------------------------------------
% LTXE-CHAPTER-UID:		052872e9-9f10b97b-dcf0f34b-0bb90d27
% LTXE-CHAPTER-NAME:	Deprecated Functions
% ------------------------------------------------------------------------------

\section{Fonctions dépréciées et retirées}
\begin{frame}[fragile]
	\frametitle{Fonctions dépréciées et retirées}

	\begin{center}\huge{Chapitre 6~:}\end{center}
	\begin{center}\huge{\color{typo3darkgrey}\textbf{Fonctions dépréciées et retirées}}\end{center}

\end{frame}

% ------------------------------------------------------------------------------
% LTXE-SLIDE-START
% LTXE-SLIDE-UID:		c1d9ce70-1efd276d-0ab964a3-35ed9971
% LTXE-SLIDE-ORIGIN:	685b0aeb-6654ab27-f8128f62-3454735b English
% LTXE-SLIDE-ORIGIN:	ebe40c2e-32da4e43-67f18e6c-866b0496 German
% LTXE-SLIDE-TITLE:		Deprecation: #67991 - Removed EXT:cms (1)
% LTXE-SLIDE-REFERENCE:	Deprecation-67991-RemovedExtCms.rst
% ------------------------------------------------------------------------------

\begin{frame}[fragile]
	\frametitle{Fonctions dépréciées et retirées}
	\framesubtitle{Extension système \texttt{cms} retirée (1)}

	% decrease font size for code listing
	\lstset{basicstyle=\tiny\ttfamily}

	\begin{itemize}

		\item L'extension système \texttt{cms} est retirée

		\item Les développeurs d'extension doivent revoir les dépendances indiquées dans le fichier \texttt{ext\_emconf.php}

			\begin{lstlisting}
				[...]
				'constraints' => array(
				  'depends' => array(
				    // 'cms' => ' ... ',           <= WRONG!
				    'typo3' => '7.0.0-7.99.99',
				  ),
				),
				[...]
			\end{lstlisting}

		\item La majorité des fonctionnalités sont migrées dans l'extension système \texttt{frontend}
			(une mise à jour des références de langue peut être nécessaire, voir diapositive suivante)

	\end{itemize}

\end{frame}

% ------------------------------------------------------------------------------
% LTXE-SLIDE-START
% LTXE-SLIDE-UID:		ecc2044b-31c48a8a-59112cab-8e398eeb
% LTXE-SLIDE-ORIGIN:	ff99fb62-81ad4d10-2367bc34-4985628d English
% LTXE-SLIDE-ORIGIN:	4e4332da-8e6c67f1-0496866b-ebe40c2e German
% LTXE-SLIDE-TITLE:		Deprecation: #67991 - Removed EXT:cms (2)
% LTXE-SLIDE-REFERENCE:	Deprecation-67991-RemovedExtCms.rst
% ------------------------------------------------------------------------------

\begin{frame}[fragile]
	\frametitle{Fonctions dépréciées et retirées}
	\framesubtitle{Extension système \texttt{cms} retirée (2)}

	% decrease font size for code listing
	\lstset{basicstyle=\tiny\ttfamily}

	\begin{itemize}

		\item Mise à jour des références aux fichiers de langue requise~:

			% hint for translators:
			% please translate words "old" and "new" in listing below,
			% but DO NOT USE special characters, like German umlauts, etc.
			% (special characters inside "lstlisting" break LaTeX).

			\begin{lstlisting}
				Ancien : typo3/sysext/cms/web_info/locallang.xlf
				Nouveau : typo3/sysext/frontend/Resources/Private/Language/locallang_webinfo.xlf
			\end{lstlisting}
			\vspace{-0.3cm}
			\begin{lstlisting}
				Ancien : typo3/sysext/cms/locallang_ttc.xlf
				Nouveau : typo3/sysext/frontend/Resources/Private/Language/locallang_ttc.xlf
			\end{lstlisting}
			\vspace{-0.3cm}
			\begin{lstlisting}
				Ancien : typo3/sysext/cms/locallang_tca.xlf
				Nouveau : typo3/sysext/frontend/Resources/Private/Language/locallang_tca.xlf
			\end{lstlisting}
			\vspace{-0.3cm}
			\begin{lstlisting}
				Ancien : typo3/sysext/cms/layout/locallang_db_new_content_el.xlf
				Nouveau : typo3/sysext/backend/Resources/Private/Language/locallang_db_new_content_el.xlf
			\end{lstlisting}
			\vspace{-0.3cm}
			\begin{lstlisting}
				Ancien : typo3/sysext/cms/layout/locallang.xlf
				Nouveau : typo3/sysext/backend/Resources/Private/Language/locallang_layout.xlf
			\end{lstlisting}
			\vspace{-0.3cm}
			\begin{lstlisting}
				Ancien : typo3/sysext/cms/layout/locallang_mod.xlf
				Nouveau : typo3/sysext/backend/Resources/Private/Language/locallang_mod.xlf
			\end{lstlisting}
			\vspace{-0.3cm}
			\begin{lstlisting}
				Ancien : typo3/sysext/cms/locallang_csh_webinfo.xlf
				Nouveau : typo3/sysext/frontend/Resources/Private/Language/locallang_csh_webinfo.xlf
			\end{lstlisting}
			\vspace{-0.3cm}
			\begin{lstlisting}
				Ancien : typo3/sysext/cms/locallang_csh_weblayout.xlf
				Nouveau : typo3/sysext/frontend/Resources/Private/Language/locallang_csh_weblayout.xlf
			\end{lstlisting}

	\end{itemize}

\end{frame}

% ------------------------------------------------------------------------------
% LTXE-SLIDE-START
% LTXE-SLIDE-UID:		5d9a79aa-009491e9-a445ed4a-c6a8506f
% LTXE-SLIDE-ORIGIN:	e1a896f5-538a7c07-e462c6f0-db80381c English
% LTXE-SLIDE-ORIGIN:	ae6fc887-9aeff493-158130de-3cdbf052 German
% LTXE-SLIDE-TITLE:		Deprecation: #68074 - Deprecate getPageRenderer() methods
% LTXE-SLIDE-REFERENCE:	Deprecation-68074-DeprecateGetPageRenderer.rst
% ------------------------------------------------------------------------------

\begin{frame}[fragile]
	\frametitle{Fonctions dépréciées et retirées}
	\framesubtitle{Méthodes de PageRenderer dépréciées}

	% decrease font size for code listing
	\lstset{basicstyle=\tiny\ttfamily}

	\begin{itemize}
		\item Les méthodes de \texttt{PageRenderer} suivantes sont classées \textbf{deprecated}~:

			\begin{lstlisting}
				TYPO3\CMS\Backend\Controller\BackendController::getPageRenderer()
				TYPO3\CMS\Backend\Template\DocumentTemplate::getPageRenderer()
				TYPO3\CMS\Backend\Template\FrontendDocumentTemplate::getPageRenderer()
				TYPO3\CMS\Frontend\Controller\TypoScriptFrontendController::getPageRenderer()
			\end{lstlisting}

		\item Le code suivant doit être utilisé pour récupérer une instance de \texttt{PageRenderer} à la place~:

			\begin{lstlisting}
				\TYPO3\CMS\Core\Utility\GeneralUtility::makeInstance(\TYPO3\CMS\Core\Page\PageRenderer::class)
			\end{lstlisting}

	\end{itemize}

\end{frame}

% ------------------------------------------------------------------------------
% LTXE-SLIDE-START
% LTXE-SLIDE-UID:		df8e97cc-fa2df859-766b2db8-a6b0c4a0
% LTXE-SLIDE-ORIGIN:	440429fa-30e8d4b1-4784ea6f-86bc343e English
% LTXE-SLIDE-ORIGIN:	a58c2f4f-3eb806e5-d256078a-70ed73c0 German
% LTXE-SLIDE-TITLE:		Deprecation #68098 and #68122
% LTXE-SLIDE-REFERENCE:	Deprecation-68098-GeneralUtilityMethods.rst
% LTXE-SLIDE-REFERENCE:	Deprecation-68122-GeneralUtilityReadLLfile.rst
% ------------------------------------------------------------------------------
% Deprecation: #68098 - Deprecate GeneralUtility methods
% Deprecation: #68122 - Deprecate GeneralUtility::readLLfile

\begin{frame}[fragile]
	\frametitle{Fonctions dépréciées et retirées}
	\framesubtitle{Méthodes de \texttt{GeneralUtility} dépréciées}

	% decrease font size for code listing
	\lstset{basicstyle=\tiny\ttfamily}

	\begin{itemize}
		\item Les méthodes de \texttt{GeneralUtility} suivantes sont classées \textbf{deprecated}
			et seront retirées à la version 8 de TYPO3 CMS~:

			\begin{lstlisting}
				GeneralUtility::modifyHTMLColor()
				GeneralUtility::modifyHTMLColorAll()
				GeneralUtility::isBrokenEmailEnvironment()
				GeneralUtility::normalizeMailAddress()
				GeneralUtility::formatForTextarea()
				GeneralUtility::getThisUrl()
				GeneralUtility::cleanOutputBuffers()
				GeneralUtility::readLLfile()
			\end{lstlisting}

		\item La méthode \texttt{readLLfile()} est remplaçable par le code suivant~:

			\begin{lstlisting}
				/** @var \TYPO3\CMS\Core\Localization\LocalizationFactory $languageFactory */
				$languageFactory = GeneralUtility::makeInstance(
				  \TYPO3\CMS\Core\Localization\LocalizationFactory::class
				);
				$languageFactory->getParsedData($fileToParse, $language, $renderCharset, $errorMode);
			\end{lstlisting}

	\end{itemize}

\end{frame}

% ------------------------------------------------------------------------------
% LTXE-SLIDE-START
% LTXE-SLIDE-UID:		0317fb66-a20eac9f-d86b0bf3-db3524bd
% LTXE-SLIDE-ORIGIN:	0d77c7a7-fd5da718-210555bf-80e688f0 English
% LTXE-SLIDE-ORIGIN:	b0c4d95f-d9fd698f-2d9dfdec-42aaf1af German
% LTXE-SLIDE-TITLE:		Breaking: #39721 - Prototype.js and Scriptaculous removed
% LTXE-SLIDE-REFERENCE:	Breaking-39721-PrototypejsAndScriptaculousRemoved.rst
% ------------------------------------------------------------------------------

\begin{frame}[fragile]
	\frametitle{Fonctions dépréciées et retirées}
	\framesubtitle{Bibliothèques JavaScript retirées}

	\begin{itemize}

		\item Les bibliothèques JavaScript \texttt{prototype.js} et \texttt{scriptaculous} sont retirées.
			Ainsi, les propriétés TypoScript suivantes n'ont plus de fonction~:

			\begin{itemize}
				\item \texttt{page.javascriptLibs.Prototype}
				\item \texttt{page.javascriptLibs.Scriptaculous.*}
			\end{itemize}

		\item L'usage des attributs suivants du ViewHelper \texttt{be.container} résulte en une erreur~:
			\begin{lstlisting}
				<f:be.container loadPrototype="false" loadScriptaculous="false" scriptaculousModule="someModule,someOtherModule">
			\end{lstlisting}

		\item En remplacement, \texttt{jQuery} et \texttt{RequireJS} doivent être utilisés\newline
			(elles sont déjà chargées par défaut en backend)

	\end{itemize}

\end{frame}

% ------------------------------------------------------------------------------
% LTXE-SLIDE-START
% LTXE-SLIDE-UID:		3b558f50-f0ece3a0-bed38b20-c612e70d
% LTXE-SLIDE-ORIGIN:	4bb27612-517a802f-3b7a7dae-900630fc English
% LTXE-SLIDE-ORIGIN:	1375cbd0-05287ba0-3235df98-19473bf0 German
% LTXE-SLIDE-TITLE:		Deprecation: #68183 - typo3/mod.php
% LTXE-SLIDE-REFERENCE:	Deprecation-68183-Typo3modphp.rst
% ------------------------------------------------------------------------------

\begin{frame}[fragile]
	\frametitle{Fonctions dépréciées et retirées}
	\framesubtitle{Dépréciés~: \texttt{init.php}, \texttt{mod.php} et \texttt{ajax.php}}

	% decrease font size for code listing
	%\lstset{basicstyle=\tiny\ttfamily}

	\begin{itemize}

		\item Afin de nettoyer le contenu du dossier \texttt{typo3}, les fichiers suivants sont
			marqués \textbf{deprecated}~: \texttt{init.php}, \texttt{mod.php} et \texttt{ajax.php}

		\item Le code suivant est utilisable pour les points d'entrée Init~:

			\begin{lstlisting}
				call_user_func(function() {
				  $classLoader = require __DIR__ . '/vendor/autoload.php';
				  (new \TYPO3\CMS\Backend\Http\Application($classLoader))->run();
				});
			\end{lstlisting}

		\item Et la méthode suivante pour accéder à \texttt{mod.php}~:

			\begin{lstlisting}
				BackendUtility::getModuleUrl()
			\end{lstlisting}

	\end{itemize}

\end{frame}

% ------------------------------------------------------------------------------
% LTXE-SLIDE-START
% LTXE-SLIDE-UID:		6d73a682-47a1b05f-b7b45d18-287123b0
% LTXE-SLIDE-ORIGIN:	a09aedae-f4adde91-f6b0d3a8-ada31684 English
% LTXE-SLIDE-ORIGIN:	749d4c59-1ec6087a-45020fbd-2f7b8646 German
% LTXE-SLIDE-TITLE:		Deprecation: #67737 - TCA: Drop additional palette
% LTXE-SLIDE-REFERENCE:	Deprecation-67737-TcaDropAdditionalPalette.rst
% ------------------------------------------------------------------------------

\begin{frame}[fragile]
	\frametitle{Fonctions dépréciées et retirées}
	\framesubtitle{TCA~: Options additionnelles retirées}

	% decrease font size for code listing
	\lstset{basicstyle=\tiny\ttfamily}

	\begin{itemize}

		\item La description \texttt{showitem} de la clé TCA \texttt{types} permettait aux développeurs
			de définir des options additionnelles (palette)

		\item Ce n'est plus possible et est migré en tant qu'élément indépendant

		\item Avant~:

			\begin{lstlisting}
				'types' => array(
				  'aType' => array(
				    'showitem' => 'aField;aLabel;anAdditionalPaletteName',
				  ),
				),
			\end{lstlisting}

		\item Maintenant~:

			\begin{lstlisting}
				'types' => array(
				  'aType' => array(
				    'showitem' => 'aField;aLabel, --palette--;;anAdditionalPaletteName',
				  ),
				),
			\end{lstlisting}

	\end{itemize}

\end{frame}

% ------------------------------------------------------------------------------
% LTXE-SLIDE-START
% LTXE-SLIDE-UID:		aaceae30-5dd40523-9c1fa8fb-2b15a6a1
% LTXE-SLIDE-ORIGIN:	e27efa5c-c3999b32-6715dd2f-da8acf1f English
% LTXE-SLIDE-ORIGIN:	6398e32e-7f467ac7-463d9b47-bc0fd47c German
% LTXE-SLIDE-TITLE:		Breaking: #67577, #67646 and #67824
% LTXE-SLIDE-REFERENCE: Breaking-67577-RteEnabledFlagHandling.rst
% LTXE-SLIDE-REFERENCE: Breaking-67646-LibraryInclusionInFrontend.rst
% LTXE-SLIDE-REFERENCE: Breaking-67824-Typo3ExtFolderRemoved.rst
% ------------------------------------------------------------------------------
% Breaking: #67577 - rte_enabled and flag handling
% Breaking: #67646 - PHP library inclusion in frontend removed
% Breaking: #67824 - typo3/ext folder removed

\begin{frame}[fragile]
	\frametitle{Fonctions dépréciées et retirées}
	\framesubtitle{Divers (1)}

	\begin{itemize}

		\item Les cObject «~Texte~» et «~Texte et images~» possédaient une case «~RTE enabled~».
			Elle est retirée, inclue l'option TCA \texttt{flag}.

		\item Les options TypoScript suivantes pour inclure des fichiers PHP sont retirées~:

			\begin{itemize}
				\item \texttt{config.includeLibrary}
				\item \texttt{config.includeLibs}
			\end{itemize}

		\item Le dossier \texttt{typo3/ext} est retiré\newline
			\small
				(mais pas l'option des extensions globales~: le dossier peut être créé manuellement)
			\normalsize

	\end{itemize}

\end{frame}

% ------------------------------------------------------------------------------
% LTXE-SLIDE-START
% LTXE-SLIDE-UID:		a2f20048-16cbb681-42bd330c-e5e3a0b0
% LTXE-SLIDE-ORIGIN:	7b47e35b-9ae0fd9a-13d2db50-ebe93a84 English
% LTXE-SLIDE-ORIGIN:	463d9b47-bc0fd47c-6398e32e-7f467ac7 German
% LTXE-SLIDE-TITLE:		Breaking: #68001
% LTXE-SLIDE-REFERENCE: Breaking-68001-RemovedExtJSCoreAndExtJSAdapters.rst
% ------------------------------------------------------------------------------
% Breaking: #68001 - Removed ExtJS Core and ExtJS Adapters
% Breaking: #68020 - Dropped DisableBigButtons

\begin{frame}[fragile]
	\frametitle{Fonctions dépréciées et retirées}
	\framesubtitle{Divers (2)}

	\begin{itemize}

		\item ExtCore (adapteur ExtJS léger et indépendant) est retiré, incluant les options TypoScript suivantes~:

			\begin{itemize}
				\item \texttt{page.javascriptLibs.ExtCore.*}
				\item \texttt{page.javascriptLibs.ExtJs.*}
			\end{itemize}

			Sont bien évidemment incluses les options du ViewHelper \texttt{<f:be.container>}

		\item Les nommés «~BigButtons~» («~Éditer les propriétés de la page~», «~Déplacer la page~», …)
			sont retirés, incluant leur configuration TSconfig \texttt{mod.we\_layout.disableBigButtons}

	\end{itemize}

\end{frame}

% ------------------------------------------------------------------------------
% LTXE-SLIDE-START
% LTXE-SLIDE-UID:		53f19383-57abf2cc-607b975c-28f4766c
% LTXE-SLIDE-ORIGIN:	8052d34e-ccc2def0-c43053a0-c3c8eba3 English
% LTXE-SLIDE-ORIGIN:	d3a774ba-c1fd32c6-099aa85f-96b6a2a2 German
% LTXE-SLIDE-TITLE:		Breaking: #68020, #68131, #65790 and #67506
% LTXE-SLIDE-REFERENCE: Breaking-68020-DroppedDisableBigButtons.rst
% LTXE-SLIDE-REFERENCE: Breaking-68131-StreamlineErrorAndExceptionHandling.rst
% LTXE-SLIDE-REFERENCE: Deprecation-65790-PagesStoragePidDeprecated.rst
% LTXE-SLIDE-REFERENCE: Deprecation-67506-DeprecateIconUtilitygetIcon.rst
% ------------------------------------------------------------------------------
% Breaking: #68131 - Streamline error and exception handling
% Deprecation: #65790 - Remove pages.storage_pid and logic
% Deprecation: #67506 - Deprecate IconUtility::getIcon

\begin{frame}[fragile]
	\frametitle{Fonctions dépréciées et retirées}
	\framesubtitle{Divers (3)}

	\begin{itemize}

		\item Les gestionnaires d'erreur et d'exception ne sont configurables plus que dans le fichier
			\texttt{LocalConfiguration.php} ou \texttt{AdditionalConfiguration.php}. Leur surcharge n'est plus
			possible dans \texttt{ext\_localconf.php}

		\item Le champ «~Dossier de stockage général~», contenant l'identifiant de stockage des enregistrements
			d'une page, est retiré.
			Le dossier de stockage doit être configuré en utilisant le TypoScript ou le FlexForms.

		\item La méthode \texttt{IconUtility::getIcon()} est classée \textbf{deprecated} (utilisez
			la méthode \texttt{IconUtility::getSpriteIconForRecord()} en remplacement)

	\end{itemize}

\end{frame}

% ------------------------------------------------------------------------------
