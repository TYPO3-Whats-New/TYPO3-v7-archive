% ------------------------------------------------------------------------------
% TYPO3 CMS 7.4 - What's New - Chapter "In-Depth Changes" (English Version)
%
% @author	Michael Schams <schams.net>
% @license	Creative Commons BY-NC-SA 3.0
% @link		http://typo3.org/download/release-notes/whats-new/
% @language	English
% ------------------------------------------------------------------------------
% LTXE-CHAPTER-UID:		9a69925b-ed88557b-d362988a-940ed520
% LTXE-CHAPTER-NAME:	In-Depth Changes
% ------------------------------------------------------------------------------

\section{Wijzigingen nader bekeken}
\begin{frame}[fragile]
	\frametitle{Wijzigingen nader bekeken}

	\begin{center}\huge{Hoofdstuk 4:}\end{center}
	\begin{center}\huge{\color{typo3darkgrey}\textbf{Wijzigingen nader bekeken}}\end{center}

\end{frame}

% ------------------------------------------------------------------------------
% LTXE-SLIDE-START
% LTXE-SLIDE-UID:		562129e0-3d5ba310-cce81c05-1fe4d258
% LTXE-SLIDE-ORIGIN:	5fa3f922-971c07f7-2ec96457-63394b8b English
% LTXE-SLIDE-ORIGIN:	5cfc1d44-3d6f3d87-789957ab-655f357d German
% LTXE-SLIDE-TITLE:		Breaking #65305: DriverInterface has been extended
% LTXE-SLIDE-REFERENCE:	Breaking-65305-AddFunctionsToDriverInterface.rst
% ------------------------------------------------------------------------------

\begin{frame}[fragile]
	\frametitle{Wijzigingen nader bekeken}
	\framesubtitle{Interface Stuurprogramma}

	% decrease font size for code listing
	\lstset{basicstyle=\tiny\ttfamily}

	\begin{itemize}

		\item De volgende functies zijn toegevoegd aan de \texttt{DriverInterface}:

			\begin{itemize}
				\item \texttt{getFolderInFolder}
				\item \texttt{getFileInFolder}
			\end{itemize}

		\item Elk FAL-stuurprogramma moet deze nieuwe functies bevatten:

			\begin{columns}[T]
				\begin{column}{.07\textwidth}
                \end{column}
				\begin{column}{.465\textwidth}

					\begin{lstlisting}
						public function getFoldersInFolder(
						  $folderIdentifier,
						  $start = 0,
						  $numberOfItems = 0,
						  $recursive = FALSE,
						  array $folderNameFilterCallbacks = array(),
						  $sort = '',
						  $sortRev = FALSE
						);
					\end{lstlisting}

                \end{column}
				\begin{column}{.465\textwidth}

					\begin{lstlisting}
						public function getFileInFolder(
						  $fileName,
						  $folderIdentifier
						);
					\end{lstlisting}

				\end{column}
			\end{columns}

	\end{itemize}

	\breakingchange

\end{frame}

% ------------------------------------------------------------------------------
% LTXE-SLIDE-START
% LTXE-SLIDE-UID:		0be2bc60-42490834-9b40c3ce-5f606112
% LTXE-SLIDE-ORIGIN:	603042c6-c0caed8e-a6565d08-09528be6 English
% LTXE-SLIDE-ORIGIN:	b719be34-650cf0d2-13ac26aa-dd3a4bbd German
% LTXE-SLIDE-TITLE:		Feature #22175: Support IEC/SI units in file size formatting
% LTXE-SLIDE-REFERENCE:	Feature-22175-SupportIecSiUnitsInFileSizeFormatting.rst
% ------------------------------------------------------------------------------

\begin{frame}[fragile]
	\frametitle{Wijzigingen nader bekeken}
	\framesubtitle{IEC/SI ondersteuning in formattering bestandsgrootte}

	% decrease font size for code listing
	%\lstset{basicstyle=\tiny\ttfamily}

	\begin{itemize}

		\item Formattering van bestandsgrootte ondersteunt twee extra opties met een lijst labels:

			\begin{itemize}
				\item \small\texttt{iec} (standaard)\newline
					\small(macht van 2, labels: \texttt{| Ki| Mi| Gi| Ti| Pi| Ei| Zi| Yi})\normalsize
				\item \small\texttt{si}\newline
					\small(macht van 10, labels: \texttt{| k| M| G| T| P| E| Z| Y})\normalsize
			\end{itemize}

		\item Voorbeeldinstelling voor formattering in TypoScript:\newline
			\texttt{bytes.labels = iec}

		\begin{lstlisting}
			echo GeneralUtility::formatSize(85123);
			// => voor "83.1 K"
			// => na "83.13 Ki"
		\end{lstlisting}

	\end{itemize}

\end{frame}

% ------------------------------------------------------------------------------
% LTXE-SLIDE-START
% LTXE-SLIDE-UID:		7b93106d-261ce981-f18b8225-b0b48bca
% LTXE-SLIDE-ORIGIN:	be6e3997-48f241bd-be7fb0f4-6fcded3f English
% LTXE-SLIDE-ORIGIN:	c5766b9f-4acc2f8d-dcadf0be-299ed259 German
% LTXE-SLIDE-TITLE:		Feature #67293: Dependency ordering service (1)
% LTXE-SLIDE-REFERENCE:	Feature-67293-DependencyOrderingService.rst
% ------------------------------------------------------------------------------

\begin{frame}[fragile]
	\frametitle{Wijzigingen nader bekeken}
	\framesubtitle{Service voor volgorde van afhankelijkheden (1)}

	\begin{itemize}

		\item Vaak is het nodig om een geordende lijst te maken van "dependencies".
			De lijst wordt gebruikt om acties in die volgorde uit te voeren.

		\item Voorbeelden voor gebruik in de TYPO3 core:

			\begin{itemize}
				\item volgorde uitvoeren van hooks,
				\item volgorde laden van extensies,
				\item lijst menu-items,
				\item etc.
			\end{itemize}

		\item De \texttt{DependencyResolver} is herbouwd en bevat nu een
			\texttt{DependencyOrderingService}

	\end{itemize}

\end{frame}

% ------------------------------------------------------------------------------
% LTXE-SLIDE-START
% LTXE-SLIDE-UID:		15f9c9af-530a7738-f3b670b0-8dda23f4
% LTXE-SLIDE-ORIGIN:	2fd8356d-45429468-7de53ebf-4c674627 English
% LTXE-SLIDE-ORIGIN:	4acc2f8d-6b9fc576-f0bedcad-d259299e German
% LTXE-SLIDE-TITLE:		Feature #67293: Dependency ordering service (2)
% LTXE-SLIDE-REFERENCE:	Feature-67293-DependencyOrderingService.rst
% ------------------------------------------------------------------------------

\begin{frame}[fragile]
	\frametitle{Wijzigingen nader bekeken}
	\framesubtitle{Service voor volgorde van afhankelijkheden (2)}

	% decrease font size for code listing
	\lstset{basicstyle=\tiny\ttfamily}

	\begin{itemize}

		\item Gebruik:

			\begin{lstlisting}
				$GLOBALS['TYPO3_CONF_VARS']['EXTCONF']['someExt']['someHook'][<some id>] = [
				  'handler' => someClass::class,
				  'runBefore' => [ <some other ID> ],
				  'runAfter' => [ ... ],
				  ...
				];
			\end{lstlisting}

		\item Voorbeeld:

			\begin{lstlisting}
				$hooks = $GLOBALS['TYPO3_CONF_VARS']['EXTCONF']['someExt']['someHook'];
				$sorted = GeneralUtility:makeInstance(DependencyOrderingService::class)->orderByDependencies(
				  $hooks, 'runBefore', 'runAfter'
				);
			\end{lstlisting}

	\end{itemize}

\end{frame}

% ------------------------------------------------------------------------------
% LTXE-SLIDE-START
% LTXE-SLIDE-UID:		13f5af6b-925e39f3-104c4613-dcf58734
% LTXE-SLIDE-ORIGIN:	18e5e8ba-e691a20f-3a6f211c-7c855633 English
% LTXE-SLIDE-ORIGIN:	3752346f-1d6aa26b-4b3dd1a7-f142d76b German
% LTXE-SLIDE-TITLE:		Feature #56644: Hook for InlineRecordContainer::checkAccess()
% LTXE-SLIDE-REFERENCE:	Feature-56644-AddHookToInlineRecordContainerCheckAccess.rst
% ------------------------------------------------------------------------------

\begin{frame}[fragile]
	\frametitle{Wijzigingen nader bekeken}
	\framesubtitle{Hooks en Signals (1)}

	% decrease font size for code listing
	\lstset{basicstyle=\tiny\ttfamily}

	\begin{itemize}

		\item Hook voor post-proces \texttt{InlineRecordContainer::checkAccess} resultaten is toegevoegd

		\item \texttt{InlineRecordContainer::checkAccess} kan gebruikt worden om toegang te regelen voor gerelateerde inline records

		\item De volgende code registreert de hook:

			\begin{lstlisting}
				$GLOBALS['TYPO3_CONF_VARS']['SC_OPTIONS']['t3lib/class.t3lib_tceforms_inline.php']
				  ['checkAccess'][] = 'My\\Package\\HookClass->hookMethod';
			\end{lstlisting}

	\end{itemize}

\end{frame}

% ------------------------------------------------------------------------------
% LTXE-SLIDE-START
% LTXE-SLIDE-UID:		f8ddf64a-9f857d85-5728bf6a-11ce9c42
% LTXE-SLIDE-ORIGIN:	661cdcf3-cd43fd27-06ab1889-a7a22ea6 English
% LTXE-SLIDE-ORIGIN:	158946bb-f31ee48e-ee6d0762-6cac7a12 German
% LTXE-SLIDE-TITLE:		Feature #59231: Hook for AbstractUserAuthentication::checkAuthentication()
% LTXE-SLIDE-REFERENCE:	Feature-59231-AddHookToAbstractUserAuthenticationCheckAuthentication.rst
% ------------------------------------------------------------------------------

\begin{frame}[fragile]
	\frametitle{Wijzigingen nader bekeken}
	\framesubtitle{Hooks en Signals (2)}

	% decrease font size for code listing
	\lstset{basicstyle=\tiny\ttfamily}

	\begin{itemize}

		\item Hook voor post-process van mislukte aanmeldingen in \texttt{AbstractUserAuthentication::checkAuthentication}
			is toegevoegd

		\item Proces stopt standaard gedurende 5 sec bij een mislukte aanmelding

		\item Met deze hook kunnen alternatieve oplossingen gemaakt worden
			(bijv. ter voorkoming van brute force aanvallen)

		\item De volgende code registreert de hook:

			\begin{lstlisting}
				$GLOBALS['TYPO3_CONF_VARS']['SC_OPTIONS']['t3lib/class.t3lib_userauth.php']
				  ['postLoginFailureProcessing'][] = 'My\\Package\\HookClass->hookMethod';
			\end{lstlisting}

	\end{itemize}

\end{frame}

% ------------------------------------------------------------------------------
% LTXE-SLIDE-START
% LTXE-SLIDE-UID:		fcf79f77-84f0fa41-8c77d81a-fb683a80
% LTXE-SLIDE-ORIGIN:	7cafd821-40468370-4a3daa2a-efdb473f English
% LTXE-SLIDE-ORIGIN:	cfa748d2-7c412900-462446d1-14dd41e4 German
% LTXE-SLIDE-TITLE:		Feature #67228 and #68047
% LTXE-SLIDE-REFERENCE:	Feature-67228-EmitSignalWhenAnIndexRecordIsMarkedAsMissing.rst
% LTXE-SLIDE-REFERENCE:	Feature-68047-EmitASignalForEachMappedObject.rst
% ------------------------------------------------------------------------------

\begin{frame}[fragile]
	\frametitle{Wijzigingen nader bekeken}
	\framesubtitle{Hooks en Signals (3)}

	\begin{itemize}

		\item Nieuw signal \texttt{recordMarkedAsMissing} wordt verzonden als de FAL indexer een
			\texttt{sys\_file} record vindt dat geen overeenkomstig bestand heeft en markeert het als
			vermist. Het signaal geeft het record UID van \texttt{sys\_file} door.

		\item Dit is handig in extensies die bestandsbeheer bieden of uitbreiden, zoals
			versiebeheer, synchronisatie, herstel, enz.

		\item Signaal \texttt{afterMappingSingleRow} wordt verzonden als de DataMapper een object aanmaakt

	\end{itemize}

\end{frame}

% ------------------------------------------------------------------------------
% LTXE-SLIDE-START
% LTXE-SLIDE-UID:		4f2f7c7d-e9d16695-3fbd2217-0f9e4dd4
% LTXE-SLIDE-ORIGIN:	a1ba770c-9d79e259-d64ab6a4-453c16b1 English
% LTXE-SLIDE-ORIGIN:	24a6aa34-aa43c7ad-29712401-32d010f3 German
% LTXE-SLIDE-TITLE:		Breaking #55759: HTML (e.g. double quotes) in link titles
% LTXE-SLIDE-REFERENCE:	Breaking-55759-HTMLInLinkTitlesNotWorkingAnymore.rst
% ------------------------------------------------------------------------------
% Breaking #55759: HTML (e.g. double quotes) in link titles

\begin{frame}[fragile]
	\frametitle{Wijzigingen nader bekeken}
	\framesubtitle{HTML in titels in TypoLink}

	% decrease font size for code listing
	\lstset{basicstyle=\tiny\ttfamily}

	\begin{itemize}

		\item Aanhalingstekens in TypoLink titels worden automatisch \textit{escaped}
		\item Dit betekent dat installaties waar HTML code handmatig escaped wordt incorrect output tonen
			bij TYPO CMS 7.4

			Voor:\tabto{1.8cm}'\texttt{Een \&quot;speciale\&quot; titel}'\newline
			Wordt:\tabto{1.8cm}'\texttt{Een \&amp;quot;speciale\&amp;quot; titel}'

		\item Aanbevolen wordt om escape-acties uit te schakelen omdat TYPO3 nu zorgt voor het escapen van
			HTML in TypoLink titels

	\end{itemize}

	\breakingchange

\end{frame}

% ------------------------------------------------------------------------------
% LTXE-SLIDE-START
% LTXE-SLIDE-UID:		cdaeefc8-c6466891-9053db7c-a33a9daf
% LTXE-SLIDE-ORIGIN:	30299840-02ec336c-0977320a-98f21078 English
% LTXE-SLIDE-ORIGIN:	8c2cf004-beb3b185-c41a8d75-35b4410a German
% LTXE-SLIDE-TITLE:		Breaking #56133 and #52705 - Miscellaneous (1)
% LTXE-SLIDE-REFERENCE:	Breaking-56133-NewBeUserPermissionFilesReplace.rst
% LTXE-SLIDE-REFERENCE:	Breaking-52705-DefaultLogConfigurationIsChanged.rst
% ------------------------------------------------------------------------------
% Feature #56133: New BE user permission "Files: replace"
% Breaking #52705: Default log configuration is changed

\begin{frame}[fragile]
	\frametitle{Wijzigingen nader bekeken}
	\framesubtitle{Diversen (1)}

	% decrease font size for code listing
	\lstset{basicstyle=\tiny\ttfamily}

	\begin{itemize}

		\item Via de rechten van een Backendgebruiker \texttt{Files->replace} kan het een gebruiker toegestaan of
			verboden worden om bestanden in de Bestandslijst module te vervangen

		% note: item above is classified as a breaking change, but I can not see why (Michael)
		% \breackingchange

		\item Een hash wordt gebruikt in de bestandsnaam van bestanden die FileWriter maakt als er geen ander
			logbestand is geconfigureerd

			\begin{itemize}
				\item voor:\tabto{1.2cm}\texttt{typo3temp/logs/typo3.log}
				\item na:\tabto{1.2cm}\texttt{typo3temp/logs/typo3\_<hash>.log}
			\end{itemize}

			\small(waarde van \texttt{<hash>} wordt gebaseerd op de encryptiesleutel)\normalsize

		% note: item above is classified as a breaking change, but I can not see why (Michael)
		% \breackingchange

	\end{itemize}

\end{frame}

% ------------------------------------------------------------------------------
% LTXE-SLIDE-START
% LTXE-SLIDE-UID:		2a617640-4cf1cb56-5f12b89f-099cdc75
% LTXE-SLIDE-ORIGIN:	fe577b08-f30526e0-2d631fe7-642122be English
% LTXE-SLIDE-ORIGIN:	10f3aa43-c7ad32d0-29712401-aa3424a6 German
% LTXE-SLIDE-TITLE:		Miscellaneous (2)
% LTXE-SLIDE-REFERENCE:	unknown
% ------------------------------------------------------------------------------

\begin{frame}[fragile]
	\frametitle{Wijzigingen nader bekeken}
	\framesubtitle{Diversen (2)}

	% decrease font size for code listing
	\lstset{basicstyle=\tiny\ttfamily}

	\begin{itemize}

		\item Klassen gebruikt in hooks moeten het autoload-mechanisme volgen
		\item Daarom kan de hook-definitie ingekort worden:

		\begin{lstlisting}
			$GLOBALS['TYPO3_CONF_VARS']['SC_OPTIONS']['tce']['formevals']
			  [\TYPO3\CMS\Saltedpasswords\Evaluation\FrontendEvaluator::class] = '';
		\end{lstlisting}

	\end{itemize}

	\breakingchange

\end{frame}

% ------------------------------------------------------------------------------
