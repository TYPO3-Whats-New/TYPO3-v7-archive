% ------------------------------------------------------------------------------
% TYPO3 CMS 7.4 - What's New - Chapter "Extbase & Fluid" (English Version)
%
% @author	Michael Schams <schams.net>
% @license	Creative Commons BY-NC-SA 3.0
% @link		http://typo3.org/download/release-notes/whats-new/
% @language	English
% ------------------------------------------------------------------------------
% LTXE-CHAPTER-UID:		d1a9ab65-989c4bc7-f84807fd-d96febe7
% LTXE-CHAPTER-NAME:	Extbase & Fluid
% ------------------------------------------------------------------------------

\section{Extbase i Fluid}
\begin{frame}[fragile]
	\frametitle{Extbase i Fluid}

	\begin{center}\huge{Poglavlje 5:}\end{center}
	\begin{center}\huge{\color{typo3darkgrey}\textbf{Extbase i Fluid}}\end{center}

\end{frame}

% ------------------------------------------------------------------------------
% LTXE-SLIDE-START
% LTXE-SLIDE-UID:		f4656f87-32fb9136-02af8356-9e0473f7
% LTXE-SLIDE-ORIGIN:	235ca4ed-fafe968a-814d4bd4-a8f936d0 English
% LTXE-SLIDE-ORIGIN:	7b7e03da-55cc30d6-8f375c39-c82ffc4f German
% LTXE-SLIDE-TITLE:		Feature #66070: Configure anchor for pagination widget
% LTXE-SLIDE-REFERENCE:	Feature-66070-ConfigureSectionForPaginationWidget.rst
% ------------------------------------------------------------------------------

\begin{frame}[fragile]
	\frametitle{Extbase i Fluid}
	\framesubtitle{Sidro za Pagination Widget}

	% decrease font size for code listing
	\lstset{basicstyle=\tiny\ttfamily}

	\begin{itemize}

		\item Ova nova karakteristika dozvoljava da se kljuc \texttt{section} doda konfiguraciji za Fluid pagination widget

		\item Ankor se dodaje na svaki link pagination widget-a

		\item Sledeci kod dodaje sidro \texttt{\#archive}:

			\begin{lstlisting}
				<f:widget.paginate objects="{plantpestWarnings}" as="paginatedWarnings"
				  configuration="{section: 'archive', itemsPerPage: 10, insertAbove: 0, insertBelow: 1,
				  maximumNumberOfLinks: 10}">

				   [...]

				</f:widget.paginate>
			\end{lstlisting}

	\end{itemize}

\end{frame}

% ------------------------------------------------------------------------------
% LTXE-SLIDE-START
% LTXE-SLIDE-UID:		ab1deac0-0c703fe0-2e6b97ad-3cc5c981
% LTXE-SLIDE-ORIGIN:	db9ef18b-234cf92c-ef6869a5-248692f4 English
% LTXE-SLIDE-ORIGIN:	3356cd43-6aa02ac1-54fd276d-0c509286 German
% LTXE-SLIDE-TITLE:		Feature #68022: Added base date attribute to DateViewHelper
% LTXE-SLIDE-REFERENCE:	Feature-68022-AddedBaseDateAttributeToDateViewHelper.rst
% ------------------------------------------------------------------------------

\begin{frame}[fragile]
	\frametitle{Extbase i Fluid}
	\framesubtitle{Atribut \texttt{base} za DateViewHelper}

	% decrease font size for code listing
	%\lstset{basicstyle=\tiny\ttfamily}

	\begin{itemize}

		\item DateViewHelper prosiren je za opcionalni atribut pod imenom \texttt{base}
		\item Atribut se moze koristiti kako bi se izracunale informacije datuma
		\item Ako je datum DateTime objekat, \texttt{base} se ignorise
		\item Sledeci primer vraca "2016", ako je \texttt{dateObject} datum iz 2017:

			\begin{lstlisting}
				<f:format.date format="Y" base="{dateObject}">-1 year</f:format.date>
			\end{lstlisting}

		\small
			(pogledati \href{http://www.php.net/manual/en/datetime.formats.relative.php}{PHP dokumentaciju} za listu validnih vrednosti)
		\normalsize

	\end{itemize}

\end{frame}

% ------------------------------------------------------------------------------
% LTXE-SLIDE-START
% LTXE-SLIDE-UID:		385a93c2-81afa894-a4e92a85-c91e3c99
% LTXE-SLIDE-ORIGIN:	aa476541-0e34625d-a79996ee-ecbf2e62 English
% LTXE-SLIDE-ORIGIN:	097a17c3-6c43ed7b-3f2db1ff-35b37660 German
% LTXE-SLIDE-TITLE:		Breaking #67890: Redesign FluidTemplateDataProcessorInterface to DataProcessorInterface
% LTXE-SLIDE-REFERENCE:	Breaking-67890-RedesignFluidTemplateDataProcessorInterfaceToDataProcessorInterface.rst
% ------------------------------------------------------------------------------

\begin{frame}[fragile]
	\frametitle{Extbase i Fluid}
	\framesubtitle{Opcija \texttt{dataProcessing} za FLUIDTEMPLATE}

	% decrease font size for code listing
	\lstset{basicstyle=\tiny\ttfamily}

	\begin{itemize}

		\item U TYPO3 CMS 7.3, predstavljena je opcija \texttt{dataProcessing} za cObject \texttt{FLUIDTEMPLATE}

		\item Njegov \texttt{FluidTemplateDataProcessorInterface} je refaktorisan u \texttt{DataProcessorInterface},
			koji isto utice na metod \texttt{process()}

			\begin{lstlisting}
				public function process(
				  ContentObjectRenderer $cObj,
				  array $contentObjectConfiguration,
				  array $processorConfiguration,
				  array $processedData
				);
			\end{lstlisting}

	\end{itemize}

	\breakingchange

\end{frame}

% ------------------------------------------------------------------------------
