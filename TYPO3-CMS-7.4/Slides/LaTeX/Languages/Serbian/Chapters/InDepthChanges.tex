% ------------------------------------------------------------------------------
% TYPO3 CMS 7.4 - What's New - Chapter "In-Depth Changes" (English Version)
%
% @author	Michael Schams <schams.net>
% @license	Creative Commons BY-NC-SA 3.0
% @link		http://typo3.org/download/release-notes/whats-new/
% @language	English
% ------------------------------------------------------------------------------
% LTXE-CHAPTER-UID:		9a69925b-ed88557b-d362988a-940ed520
% LTXE-CHAPTER-NAME:	In-Depth Changes
% ------------------------------------------------------------------------------

\section{Korenite promene}
\begin{frame}[fragile]
	\frametitle{Korenite promene}

	\begin{center}\huge{Poglavlje 4:}\end{center}
	\begin{center}\huge{\color{typo3darkgrey}\textbf{Korenite promene}}\end{center}

\end{frame}

% ------------------------------------------------------------------------------
% LTXE-SLIDE-START
% LTXE-SLIDE-UID:		5f06bf16-98a5e8f7-05abc446-cf12e93a
% LTXE-SLIDE-ORIGIN:	5fa3f922-971c07f7-2ec96457-63394b8b English
% LTXE-SLIDE-ORIGIN:	5cfc1d44-3d6f3d87-789957ab-655f357d German
% LTXE-SLIDE-TITLE:		Breaking #65305: DriverInterface has been extended
% LTXE-SLIDE-REFERENCE:	Breaking-65305-AddFunctionsToDriverInterface.rst
% ------------------------------------------------------------------------------

\begin{frame}[fragile]
	\frametitle{Korenite promene}
	\framesubtitle{Driver Interface}

	% decrease font size for code listing
	\lstset{basicstyle=\tiny\ttfamily}

	\begin{itemize}

		\item Sledece metode pridodate su \texttt{DriverInterface}:

			\begin{itemize}
				\item \texttt{getFolderInFolder}
				\item \texttt{getFileInFolder}
			\end{itemize}

		\item Svaki FAL drajver trebalo bi da implementira ove nove metode:

			\begin{columns}[T]
				\begin{column}{.07\textwidth}
                \end{column}
				\begin{column}{.465\textwidth}

					\begin{lstlisting}
						public function getFoldersInFolder(
						  $folderIdentifier,
						  $start = 0,
						  $numberOfItems = 0,
						  $recursive = FALSE,
						  array $folderNameFilterCallbacks = array(),
						  $sort = '',
						  $sortRev = FALSE
						);
					\end{lstlisting}

                \end{column}
				\begin{column}{.465\textwidth}

					\begin{lstlisting}
						public function getFileInFolder(
						  $fileName,
						  $folderIdentifier
						);
					\end{lstlisting}

				\end{column}
			\end{columns}

	\end{itemize}

	\breakingchange

\end{frame}

% ------------------------------------------------------------------------------
% LTXE-SLIDE-START
% LTXE-SLIDE-UID:		4aba2397-08577ce4-d7092e6b-fb1739bf
% LTXE-SLIDE-ORIGIN:	603042c6-c0caed8e-a6565d08-09528be6 English
% LTXE-SLIDE-ORIGIN:	b719be34-650cf0d2-13ac26aa-dd3a4bbd German
% LTXE-SLIDE-TITLE:		Feature #22175: Support IEC/SI units in file size formatting
% LTXE-SLIDE-REFERENCE:	Feature-22175-SupportIecSiUnitsInFileSizeFormatting.rst
% ------------------------------------------------------------------------------

\begin{frame}[fragile]
	\frametitle{Korenite promene}
	\framesubtitle{IEC/SI podrska za formatiranje velicine fajla}

	% decrease font size for code listing
	%\lstset{basicstyle=\tiny\ttfamily}

	\begin{itemize}

		\item Formatiranje velicine fajla sada podrzava dve kljucne reci dodatno u odnosu na listu labela:

			\begin{itemize}
				\item \small\texttt{iec} (podrazumevano)\newline
					\small(moc 2, labele: \texttt{| Ki| Mi| Gi| Ti| Pi| Ei| Zi| Yi})\normalsize
				\item \small\texttt{si}\newline
					\small(moc 10, labele: \texttt{| k| M| G| T| P| E| Z| Y})\normalsize
			\end{itemize}

		\item Podesiti formatiranje u TypoScript-u na primer:\newline
			\texttt{bytes.labels = iec}

		\begin{lstlisting}
			echo GeneralUtility::formatSize(85123);
			// => before "83.1 K"
			// => now "83.13 Ki"
		\end{lstlisting}

	\end{itemize}

\end{frame}

% ------------------------------------------------------------------------------
% LTXE-SLIDE-START
% LTXE-SLIDE-UID:		8c637855-bfa71a37-0c5cae8e-f7e10487
% LTXE-SLIDE-ORIGIN:	be6e3997-48f241bd-be7fb0f4-6fcded3f English
% LTXE-SLIDE-ORIGIN:	c5766b9f-4acc2f8d-dcadf0be-299ed259 German
% LTXE-SLIDE-TITLE:		Feature #67293: Dependency ordering service (1)
% LTXE-SLIDE-REFERENCE:	Feature-67293-DependencyOrderingService.rst
% ------------------------------------------------------------------------------

\begin{frame}[fragile]
	\frametitle{Korenite promene}
	\framesubtitle{Dependency Ordering Service (1)}

	\begin{itemize}

		\item U mnogo slucajeva je neophodno napraviti sortiranu listu od seta zavisnih funkcija - "dependencies". Ovako sortirana lista koristi se kako bi se akcije izvrsile u datom redosledu.

		\item Neki od primera u kojima ovo koristi TYPO3 srz su:

			\begin{itemize}
				\item redosled pozivanja hukova,
				\item redosled ucitavanja prosirenja,
				\item listanje stavki iz menija,
				\item itd.
			\end{itemize}

		\item The \texttt{DependencyResolver} je preradjen i sada dozvoljava
			\texttt{DependencyOrderingService}

	\end{itemize}

\end{frame}

% ------------------------------------------------------------------------------
% LTXE-SLIDE-START
% LTXE-SLIDE-UID:		43738f50-719bb32a-b4ea9626-744d7486
% LTXE-SLIDE-ORIGIN:	2fd8356d-45429468-7de53ebf-4c674627 English
% LTXE-SLIDE-ORIGIN:	4acc2f8d-6b9fc576-f0bedcad-d259299e German
% LTXE-SLIDE-TITLE:		Feature #67293: Dependency ordering service (2)
% LTXE-SLIDE-REFERENCE:	Feature-67293-DependencyOrderingService.rst
% ------------------------------------------------------------------------------

\begin{frame}[fragile]
	\frametitle{Korenite promene}
	\framesubtitle{Dependency Ordering Service (2)}

	% decrease font size for code listing
	\lstset{basicstyle=\tiny\ttfamily}

	\begin{itemize}

		\item Primena:

			\begin{lstlisting}
				$GLOBALS['TYPO3_CONF_VARS']['EXTCONF']['someExt']['someHook'][<some id>] = [
				  'handler' => someClass::class,
				  'runBefore' => [ <some other ID> ],
				  'runAfter' => [ ... ],
				  ...
				];
			\end{lstlisting}

		\item Primer:

			\begin{lstlisting}
				$hooks = $GLOBALS['TYPO3_CONF_VARS']['EXTCONF']['someExt']['someHook'];
				$sorted = GeneralUtility:makeInstance(DependencyOrderingService::class)->orderByDependencies(
				  $hooks, 'runBefore', 'runAfter'
				);
			\end{lstlisting}

	\end{itemize}

\end{frame}

% ------------------------------------------------------------------------------
% LTXE-SLIDE-START
% LTXE-SLIDE-UID:		9a7774f5-b27432d3-760031b9-8137bf81
% LTXE-SLIDE-ORIGIN:	18e5e8ba-e691a20f-3a6f211c-7c855633 English
% LTXE-SLIDE-ORIGIN:	3752346f-1d6aa26b-4b3dd1a7-f142d76b German
% LTXE-SLIDE-TITLE:		Feature #56644: Hook for InlineRecordContainer::checkAccess()
% LTXE-SLIDE-REFERENCE:	Feature-56644-AddHookToInlineRecordContainerCheckAccess.rst
% ------------------------------------------------------------------------------

\begin{frame}[fragile]
	\frametitle{Korenite promene}
	\framesubtitle{Hukovi i signali (1)}

	% decrease font size for code listing
	\lstset{basicstyle=\tiny\ttfamily}

	\begin{itemize}

		\item Dodat je huk za post-process \texttt{InlineRecordContainer::checkAccess} rezultate

		\item \texttt{InlineRecordContainer::checkAccess} se moze koristit da se proveri pristup povezanim rekordima istog nivoa

		\item Sledeci kod registruje huk:

			\begin{lstlisting}
				$GLOBALS['TYPO3_CONF_VARS']['SC_OPTIONS']['t3lib/class.t3lib_tceforms_inline.php']
				  ['checkAccess'][] = 'My\\Package\\HookClass->hookMethod';
			\end{lstlisting}

	\end{itemize}

\end{frame}

% ------------------------------------------------------------------------------
% LTXE-SLIDE-START
% LTXE-SLIDE-UID:		7d3bcb9f-8a029351-74a9853b-1328e544
% LTXE-SLIDE-ORIGIN:	661cdcf3-cd43fd27-06ab1889-a7a22ea6 English
% LTXE-SLIDE-ORIGIN:	158946bb-f31ee48e-ee6d0762-6cac7a12 German
% LTXE-SLIDE-TITLE:		Feature #59231: Hook for AbstractUserAuthentication::checkAuthentication()
% LTXE-SLIDE-REFERENCE:	Feature-59231-AddHookToAbstractUserAuthenticationCheckAuthentication.rst
% ------------------------------------------------------------------------------

\begin{frame}[fragile]
	\frametitle{Korenite promene}
	\framesubtitle{Hukovi i signali (2)}

	% decrease font size for code listing
	\lstset{basicstyle=\tiny\ttfamily}

	\begin{itemize}

		\item Dodat je huk za post-process neuspesnih prijava na sajt u \texttt{AbstractUserAuthentication::checkAuthentication}

		\item Proces po podrazumevanom podesavanju odugovlaci 5 sekundi ukoliko je login bio neuspesan

		\item Koristeci ovaj novi huk mogu se implementirati alternativna resenja (npr. kako bi se sprecili napadi)

		\item Sledeci kod registruje huk:

			\begin{lstlisting}
				$GLOBALS['TYPO3_CONF_VARS']['SC_OPTIONS']['t3lib/class.t3lib_userauth.php']
				  ['postLoginFailureProcessing'][] = 'My\\Package\\HookClass->hookMethod';
			\end{lstlisting}

	\end{itemize}

\end{frame}

% ------------------------------------------------------------------------------
% LTXE-SLIDE-START
% LTXE-SLIDE-UID:		e363e99b-7f92780a-9841993b-030aff2b
% LTXE-SLIDE-ORIGIN:	7cafd821-40468370-4a3daa2a-efdb473f English
% LTXE-SLIDE-ORIGIN:	cfa748d2-7c412900-462446d1-14dd41e4 German
% LTXE-SLIDE-TITLE:		Feature #67228 and #68047
% LTXE-SLIDE-REFERENCE:	Feature-67228-EmitSignalWhenAnIndexRecordIsMarkedAsMissing.rst
% LTXE-SLIDE-REFERENCE:	Feature-68047-EmitASignalForEachMappedObject.rst
% ------------------------------------------------------------------------------

\begin{frame}[fragile]
	\frametitle{Korenite promene}
	\framesubtitle{Hukovi i signali (3)}

	\begin{itemize}

		\item Novi signal \texttt{recordMarkedAsMissing} emituje se kada FAL indekser naidje na
			\texttt{sys\_file} rekord koji nema odgovarajuci fajl sistem unos i kada nedostaju njegove oznake. Signal preskoci UID \texttt{sys\_file} rekorda.

		\item Ovo je korisno kod prosirenja koje obezbedjuju ili prosiruju mogucnosti upravljanja fajlovima kao sto su verzionisanje, sinhronizacija, oporavak, itd.

		\item Signal \texttt{afterMappingSingleRow} emituje se kad god DataMapper napravi objekat

	\end{itemize}

\end{frame}

% ------------------------------------------------------------------------------
% LTXE-SLIDE-START
% LTXE-SLIDE-UID:		89a39869-a9321540-af35d4b1-c4b32944
% LTXE-SLIDE-ORIGIN:	a1ba770c-9d79e259-d64ab6a4-453c16b1 English
% LTXE-SLIDE-ORIGIN:	24a6aa34-aa43c7ad-29712401-32d010f3 German
% LTXE-SLIDE-TITLE:		Breaking #55759: HTML (e.g. double quotes) in link titles
% LTXE-SLIDE-REFERENCE:	Breaking-55759-HTMLInLinkTitlesNotWorkingAnymore.rst
% ------------------------------------------------------------------------------
% Breaking #55759: HTML (e.g. double quotes) in link titles

\begin{frame}[fragile]
	\frametitle{Korenite promene}
	\framesubtitle{HTML u TypoLink naslovima}

	% decrease font size for code listing
	\lstset{basicstyle=\tiny\ttfamily}

	\begin{itemize}

		\item Navodnici u TypoLink naslovima sada se automatski \textit{izbegavaju}
		\item Ovo znaci da instance gde je HTML kod izbegnut rucno u TYPO CMS 7.4 ce popucati na korisnickom interfejsu

			Pre:\tabto{1.8cm}'\texttt{Some \&quot;special\&quot; title}'\newline
			Postaje:\tabto{1.8cm}'\texttt{Some \&amp;quot;special\&amp;quot; title}'

		\item Preporucuje se da se izbegavanje zaobidje, zato sto TYPO3 sada sam brine o izbegavanju HTML-a u TypoLink naslovima

	\end{itemize}

	\breakingchange

\end{frame}

% ------------------------------------------------------------------------------
% LTXE-SLIDE-START
% LTXE-SLIDE-UID:		3849f684-7163a506-99ce10de-36d7ca39
% LTXE-SLIDE-ORIGIN:	30299840-02ec336c-0977320a-98f21078 English
% LTXE-SLIDE-ORIGIN:	8c2cf004-beb3b185-c41a8d75-35b4410a German
% LTXE-SLIDE-TITLE:		Breaking #56133 and #52705 - Miscellaneous (1)
% LTXE-SLIDE-REFERENCE:	Breaking-56133-NewBeUserPermissionFilesReplace.rst
% LTXE-SLIDE-REFERENCE:	Breaking-52705-DefaultLogConfigurationIsChanged.rst
% ------------------------------------------------------------------------------
% Feature #56133: New BE user permission "Files: replace"
% Breaking #52705: Default log configuration is changed

\begin{frame}[fragile]
	\frametitle{Korenite promene}
	\framesubtitle{Razno (1)}

	% decrease font size for code listing
	\lstset{basicstyle=\tiny\ttfamily}

	\begin{itemize}

		\item Konfigurisanjem premisija za korisnike administratorskog interfejsa \texttt{Files->replace}, korisnicima se moze dozvoliti ili zabraniti zamena fajlova u Filelist modulu

		% note: item above is classified as a breaking change, but I can not see why (Michael)
		% \breackingchange

		\item Ukoliko nijedan drugi log nije konfigurisan, koristi se hes u imenu fajla, generisan uz pomoc FileWriter-a

			\begin{itemize}
				\item pre:\tabto{1.2cm}\texttt{typo3temp/logs/typo3.log}
				\item sada:\tabto{1.2cm}\texttt{typo3temp/logs/typo3\_<hash>.log}
			\end{itemize}

			\small(vrednost \texttt{<hash>} je izracunata na osnovu enkripcionog kljuca)\normalsize

		% note: item above is classified as a breaking change, but I can not see why (Michael)
		% \breackingchange

	\end{itemize}

\end{frame}

% ------------------------------------------------------------------------------
% LTXE-SLIDE-START
% LTXE-SLIDE-UID:		735a1da6-6da4725e-bac37a41-aa504214
% LTXE-SLIDE-ORIGIN:	fe577b08-f30526e0-2d631fe7-642122be English
% LTXE-SLIDE-ORIGIN:	10f3aa43-c7ad32d0-29712401-aa3424a6 German
% LTXE-SLIDE-TITLE:		Miscellaneous (2)
% LTXE-SLIDE-REFERENCE:	unknown
% ------------------------------------------------------------------------------

\begin{frame}[fragile]
	\frametitle{Korenite promene}
	\framesubtitle{Razno (2)}

	% decrease font size for code listing
	\lstset{basicstyle=\tiny\ttfamily}

	\begin{itemize}

		\item Klase koje se koriste u huku moraju da prate mehanizam samoucitavanja
		\item Stoga se definisanje huka sada moze skratiti:

		\begin{lstlisting}
			$GLOBALS['TYPO3_CONF_VARS']['SC_OPTIONS']['tce']['formevals']
			  [\TYPO3\CMS\Saltedpasswords\Evaluation\FrontendEvaluator::class] = '';
		\end{lstlisting}

	\end{itemize}

	\breakingchange

\end{frame}

% ------------------------------------------------------------------------------
