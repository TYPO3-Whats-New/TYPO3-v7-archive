% ------------------------------------------------------------------------------
% TYPO3 CMS 7.4 - What's New - Chapter "In-Depth Changes" (English Version)
%
% @author	Michael Schams <schams.net>
% @license	Creative Commons BY-NC-SA 3.0
% @link		http://typo3.org/download/release-notes/whats-new/
% @language	English
% ------------------------------------------------------------------------------
% LTXE-CHAPTER-UID:		9a69925b-ed88557b-d362988a-940ed520
% LTXE-CHAPTER-NAME:	In-Depth Changes
% ------------------------------------------------------------------------------

\section{Modifiche rilevanti}
\begin{frame}[fragile]
	\frametitle{Modifiche rilevanti}

	\begin{center}\huge{Capitolo 4:}\end{center}
	\begin{center}\huge{\color{typo3darkgrey}\textbf{Modifiche rilevanti}}\end{center}

\end{frame}

% ------------------------------------------------------------------------------
% LTXE-SLIDE-START
% LTXE-SLIDE-UID:		b4dec2bb-96ab8669-4a32868d-ec4e7b14
% LTXE-SLIDE-ORIGIN:	5fa3f922-971c07f7-2ec96457-63394b8b English
% LTXE-SLIDE-ORIGIN:	5cfc1d44-3d6f3d87-789957ab-655f357d German
% LTXE-SLIDE-TITLE:		Breaking #65305: DriverInterface has been extended
% LTXE-SLIDE-REFERENCE:	Breaking-65305-AddFunctionsToDriverInterface.rst
% ------------------------------------------------------------------------------

\begin{frame}[fragile]
	\frametitle{Modifiche rilevanti}
	\framesubtitle{Driver Interface}

	% decrease font size for code listing
	\lstset{basicstyle=\tiny\ttfamily}

	\begin{itemize}

		\item I seguenti metodi sono stati aggiunti a \texttt{DriverInterface}:

			\begin{itemize}
				\item \texttt{getFolderInFolder}
				\item \texttt{getFileInFolder}
			\end{itemize}

		\item Ogni driver FAL deve implementare questi nuovi metodi:

			\begin{columns}[T]
				\begin{column}{.07\textwidth}
                \end{column}
				\begin{column}{.465\textwidth}

					\begin{lstlisting}
						public function getFoldersInFolder(
						  $folderIdentifier,
						  $start = 0,
						  $numberOfItems = 0,
						  $recursive = FALSE,
						  array $folderNameFilterCallbacks = array(),
						  $sort = '',
						  $sortRev = FALSE
						);
					\end{lstlisting}

                \end{column}
				\begin{column}{.465\textwidth}

					\begin{lstlisting}
						public function getFileInFolder(
						  $fileName,
						  $folderIdentifier
						);
					\end{lstlisting}

				\end{column}
			\end{columns}

	\end{itemize}

	\breakingchange

\end{frame}

% ------------------------------------------------------------------------------
% LTXE-SLIDE-START
% LTXE-SLIDE-UID:		2417d9b0-8f27c85d-792173ad-fd711e7d
% LTXE-SLIDE-ORIGIN:	603042c6-c0caed8e-a6565d08-09528be6 English
% LTXE-SLIDE-ORIGIN:	b719be34-650cf0d2-13ac26aa-dd3a4bbd German
% LTXE-SLIDE-TITLE:		Feature #22175: Support IEC/SI units in file size formatting
% LTXE-SLIDE-REFERENCE:	Feature-22175-SupportIecSiUnitsInFileSizeFormatting.rst
% ------------------------------------------------------------------------------

\begin{frame}[fragile]
	\frametitle{Modifiche rilevanti}
	\framesubtitle{Supporto IEC/SI nella formattazione della dimensione di file}

	% decrease font size for code listing
	%\lstset{basicstyle=\tiny\ttfamily}

	\begin{itemize}

		\item La formattazione della dimensione di file supporta due chiavi aggiuntive alla lista delle etichette:

			\begin{itemize}
				\item \small\texttt{iec} (default)\newline
					\small(power of 2, labels: \texttt{| Ki| Mi| Gi| Ti| Pi| Ei| Zi| Yi})\normalsize
				\item \small\texttt{si}\newline
					\small(power of 10, labels: \texttt{| k| M| G| T| P| E| Z| Y})\normalsize
			\end{itemize}

		\item Impostazione del formato in TypoScript ad esempio:\newline
			\texttt{bytes.labels = iec}

		\begin{lstlisting}
			echo GeneralUtility::formatSize(85123);
			// => before "83.1 K"
			// => now "83.13 Ki"
		\end{lstlisting}

	\end{itemize}

\end{frame}

% ------------------------------------------------------------------------------
% LTXE-SLIDE-START
% LTXE-SLIDE-UID:		ec3774d5-f2272e00-ea79fa45-864022ac
% LTXE-SLIDE-ORIGIN:	be6e3997-48f241bd-be7fb0f4-6fcded3f English
% LTXE-SLIDE-ORIGIN:	c5766b9f-4acc2f8d-dcadf0be-299ed259 German
% LTXE-SLIDE-TITLE:		Feature #67293: Dependency ordering service (1)
% LTXE-SLIDE-REFERENCE:	Feature-67293-DependencyOrderingService.rst
% ------------------------------------------------------------------------------

\begin{frame}[fragile]
	\frametitle{Modifiche rilevanti}
	\framesubtitle{Ordinamento servizio dipendenza (1)}

	\begin{itemize}

		\item In vari casi è necessario creare un lista ordinata di elementi da una serie di "dipendenze".
			L'elenco ordinato è quindi usato per eseguire operazioni nell'ordine dato.

		\item Alcuni esempi di dove il core di TYPO3 li utilizza sono:

			\begin{itemize}
				\item ordine di esecuzione degli hook,
				\item ordine di caricamento estensioni,
				\item elenco delle voci di menu,
				\item ecc.
			\end{itemize}

		\item Il \texttt{DependencyResolver} è stato riscritto ed ora fornisce un 
			\texttt{DependencyOrderingService}

	\end{itemize}

\end{frame}

% ------------------------------------------------------------------------------
% LTXE-SLIDE-START
% LTXE-SLIDE-UID:		16ed86aa-e6f3c79e-ed777aab-b22f8afc
% LTXE-SLIDE-ORIGIN:	2fd8356d-45429468-7de53ebf-4c674627 English
% LTXE-SLIDE-ORIGIN:	4acc2f8d-6b9fc576-f0bedcad-d259299e German
% LTXE-SLIDE-TITLE:		Feature #67293: Dependency ordering service (2)
% LTXE-SLIDE-REFERENCE:	Feature-67293-DependencyOrderingService.rst
% ------------------------------------------------------------------------------

\begin{frame}[fragile]
	\frametitle{Modifiche rilevanti}
	\framesubtitle{Ordinamento servizio dipendenza (2)}

	% decrease font size for code listing
	\lstset{basicstyle=\tiny\ttfamily}

	\begin{itemize}

		\item Uso:

			\begin{lstlisting}
				$GLOBALS['TYPO3_CONF_VARS']['EXTCONF']['someExt']['someHook'][<some id>] = [
				  'handler' => someClass::class,
				  'runBefore' => [ <some other ID> ],
				  'runAfter' => [ ... ],
				  ...
				];
			\end{lstlisting}

		\item Esempio:

			\begin{lstlisting}
				$hooks = $GLOBALS['TYPO3_CONF_VARS']['EXTCONF']['someExt']['someHook'];
				$sorted = GeneralUtility:makeInstance(DependencyOrderingService::class)->orderByDependencies(
				  $hooks, 'runBefore', 'runAfter'
				);
			\end{lstlisting}

	\end{itemize}

\end{frame}

% ------------------------------------------------------------------------------
% LTXE-SLIDE-START
% LTXE-SLIDE-UID:		e4934a23-0d151fc9-186d7cdd-9fbc2f10
% LTXE-SLIDE-ORIGIN:	18e5e8ba-e691a20f-3a6f211c-7c855633 English
% LTXE-SLIDE-ORIGIN:	3752346f-1d6aa26b-4b3dd1a7-f142d76b German
% LTXE-SLIDE-TITLE:		Feature #56644: Hook for InlineRecordContainer::checkAccess()
% LTXE-SLIDE-REFERENCE:	Feature-56644-AddHookToInlineRecordContainerCheckAccess.rst
% ------------------------------------------------------------------------------

\begin{frame}[fragile]
	\frametitle{Modifiche rilevanti}
	\framesubtitle{Hooks e Segnali (1)}

	% decrease font size for code listing
	\lstset{basicstyle=\tiny\ttfamily}

	\begin{itemize}

		\item  E' stato aggiunto un hook al risultato del post-process \texttt{InlineRecordContainer::checkAccess}

		\item \texttt{InlineRecordContainer::checkAccess} può essere usato per verificare l'accesso ai relativi record inline

		\item Il codice seguente registra l'hook:

			\begin{lstlisting}
				$GLOBALS['TYPO3_CONF_VARS']['SC_OPTIONS']['t3lib/class.t3lib_tceforms_inline.php']
				  ['checkAccess'][] = 'My\\Package\\HookClass->hookMethod';
			\end{lstlisting}

	\end{itemize}

\end{frame}

% ------------------------------------------------------------------------------
% LTXE-SLIDE-START
% LTXE-SLIDE-UID:		5b871b3b-8a3e73df-4fce69c7-629119db
% LTXE-SLIDE-ORIGIN:	661cdcf3-cd43fd27-06ab1889-a7a22ea6 English
% LTXE-SLIDE-ORIGIN:	158946bb-f31ee48e-ee6d0762-6cac7a12 German
% LTXE-SLIDE-TITLE:		Feature #59231: Hook for AbstractUserAuthentication::checkAuthentication()
% LTXE-SLIDE-REFERENCE:	Feature-59231-AddHookToAbstractUserAuthenticationCheckAuthentication.rst
% ------------------------------------------------------------------------------

\begin{frame}[fragile]
	\frametitle{Modifiche rilevanti}
	\framesubtitle{Hooks e Segnali (2)}

	% decrease font size for code listing
	\lstset{basicstyle=\tiny\ttfamily}

	\begin{itemize}

		\item E' stato aggiunto l'hook al post-process login failures in \texttt{AbstractUserAuthentication::checkAuthentication}

		\item Il processo si ferma per 5 secondi nel caso di login fallito

		\item Utilizzando questo hook, possono essere implementate soluzioni alternative
			(es. per prevenire brute force attacks)

		\item Il codice seguente registra l'hook:

			\begin{lstlisting}
				$GLOBALS['TYPO3_CONF_VARS']['SC_OPTIONS']['t3lib/class.t3lib_userauth.php']
				  ['postLoginFailureProcessing'][] = 'My\\Package\\HookClass->hookMethod';
			\end{lstlisting}

	\end{itemize}

\end{frame}

% ------------------------------------------------------------------------------
% LTXE-SLIDE-START
% LTXE-SLIDE-UID:		483b95bc-76e38b6b-bc03e04a-0cfc81d4
% LTXE-SLIDE-ORIGIN:	7cafd821-40468370-4a3daa2a-efdb473f English
% LTXE-SLIDE-ORIGIN:	cfa748d2-7c412900-462446d1-14dd41e4 German
% LTXE-SLIDE-TITLE:		Feature #67228 and #68047
% LTXE-SLIDE-REFERENCE:	Feature-67228-EmitSignalWhenAnIndexRecordIsMarkedAsMissing.rst
% LTXE-SLIDE-REFERENCE:	Feature-68047-EmitASignalForEachMappedObject.rst
% ------------------------------------------------------------------------------

\begin{frame}[fragile]
	\frametitle{Modifiche rilevanti}
	\framesubtitle{Hooks e Segnali (3)}

	\begin{itemize}

		\item Il nuovo segnale \texttt{recordMarkedAsMissing} è emesso quando l'idexer del FAL incontra un 
			record \texttt{sys\_file} che non ha una corrispondente voce di filesystem ed è marcato come 
			mancante. Il segnala passa l'UID del record \texttt{sys\_file}.

		\item Questo è utile nelle estensioni che forniscono o estendono le funzionalità di gestione dei file, come il controllo
			di versione, sincronizzazioni, recupero, ecc.

		\item Il segnale \texttt{afterMappingSingleRow} è emesso ognivolta che il DataMapper crea un oggetto

	\end{itemize}

\end{frame}

% ------------------------------------------------------------------------------
% LTXE-SLIDE-START
% LTXE-SLIDE-UID:		2de659c7-23fc61b8-fa222234-1489df72
% LTXE-SLIDE-ORIGIN:	a1ba770c-9d79e259-d64ab6a4-453c16b1 English
% LTXE-SLIDE-ORIGIN:	24a6aa34-aa43c7ad-29712401-32d010f3 German
% LTXE-SLIDE-TITLE:		Breaking #55759: HTML (e.g. double quotes) in link titles
% LTXE-SLIDE-REFERENCE:	Breaking-55759-HTMLInLinkTitlesNotWorkingAnymore.rst
% ------------------------------------------------------------------------------
% Breaking #55759: HTML (e.g. double quotes) in link titles

\begin{frame}[fragile]
	\frametitle{Modifiche rilevanti}
	\framesubtitle{HTML nel titolo di TypoLink}

	% decrease font size for code listing
	\lstset{basicstyle=\tiny\ttfamily}

	\begin{itemize}

		\item I riferimenti nei titoli di TypoLink sono \textit{gestiti} automaticamente
		\item Questo significa che le istanze dove il codice HTML è già gestito manualmente,
			varierà l'output di frontend in TYPO CMS 7.4

			Prima:\tabto{1.8cm}'\texttt{Some \&quot;special\&quot; title}'\newline
			Diventa:\tabto{1.8cm}'\texttt{Some \&amp;quot;special\&amp;quot; title}'

		\item Si raccomanda di evitare l'escaping, visto il fatto che TYPO3 si prende cura di eseguire l'escaping
			HTML nei titoli di TypoLink

	\end{itemize}

	\breakingchange

\end{frame}

% ------------------------------------------------------------------------------
% LTXE-SLIDE-START
% LTXE-SLIDE-UID:		f0be74dd-78659c1f-0af957a4-c3d86078
% LTXE-SLIDE-ORIGIN:	30299840-02ec336c-0977320a-98f21078 English
% LTXE-SLIDE-ORIGIN:	8c2cf004-beb3b185-c41a8d75-35b4410a German
% LTXE-SLIDE-TITLE:		Breaking #56133 and #52705 - Miscellaneous (1)
% LTXE-SLIDE-REFERENCE:	Breaking-56133-NewBeUserPermissionFilesReplace.rst
% LTXE-SLIDE-REFERENCE:	Breaking-52705-DefaultLogConfigurationIsChanged.rst
% ------------------------------------------------------------------------------
% Feature #56133: New BE user permission "Files: replace"
% Breaking #52705: Default log configuration is changed

\begin{frame}[fragile]
	\frametitle{Modifiche rilevanti}
	\framesubtitle{Varie (1)}

	% decrease font size for code listing
	\lstset{basicstyle=\tiny\ttfamily}

	\begin{itemize}

		\item Configurando il permesso dell'utente di backend \texttt{Files->replace}, l'utente è
			autorizzato o limitato a sostituire il nome dei file nel modulo Filelist

		% note: item above is classified as a breaking change, but I can not see why (Michael)
		% \breackingchange

		\item Una hash è usata nel nome dei file, generati da FileWriter,
			se nessun altro file di registro è stato configurato

			\begin{itemize}
				\item prima:\tabto{1.2cm}\texttt{typo3temp/logs/typo3.log}
				\item ora:\tabto{1.2cm}\texttt{typo3temp/logs/typo3\_<hash>.log}
			\end{itemize}

			\small(il valore \texttt{<hash>} è calcolato basando sulla chiave encryption)\normalsize

		% note: item above is classified as a breaking change, but I can not see why (Michael)
		% \breackingchange

	\end{itemize}

\end{frame}

% ------------------------------------------------------------------------------
% LTXE-SLIDE-START
% LTXE-SLIDE-UID:		d07c6cb1-b7ca3767-da46aab6-f14c6cc4
% LTXE-SLIDE-ORIGIN:	fe577b08-f30526e0-2d631fe7-642122be English
% LTXE-SLIDE-ORIGIN:	10f3aa43-c7ad32d0-29712401-aa3424a6 German
% LTXE-SLIDE-TITLE:		Miscellaneous (2)
% LTXE-SLIDE-REFERENCE:	unknown
% ------------------------------------------------------------------------------

\begin{frame}[fragile]
	\frametitle{Modifiche rilevanti}
	\framesubtitle{Varie (2)}

	% decrease font size for code listing
	\lstset{basicstyle=\tiny\ttfamily}

	\begin{itemize}

		\item Le classi usate negli hook devono seguire il meccanismo di autoloading
		\item Pertanto la definizione degli hook ora può essere abbreviata:

		\begin{lstlisting}
			$GLOBALS['TYPO3_CONF_VARS']['SC_OPTIONS']['tce']['formevals']
			  [\TYPO3\CMS\Saltedpasswords\Evaluation\FrontendEvaluator::class] = '';
		\end{lstlisting}

	\end{itemize}

	\breakingchange

\end{frame}

% ------------------------------------------------------------------------------
