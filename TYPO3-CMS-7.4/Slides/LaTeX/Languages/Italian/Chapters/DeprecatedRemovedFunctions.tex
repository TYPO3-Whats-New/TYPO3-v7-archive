% ------------------------------------------------------------------------------
% TYPO3 CMS 7.4 - What's New - Chapter "Deprecated Functions" (English Version)
%
% @author	Michael Schams <schams.net>
% @license	Creative Commons BY-NC-SA 3.0
% @link		http://typo3.org/download/release-notes/whats-new/
% @language	English
% ------------------------------------------------------------------------------
% LTXE-CHAPTER-UID:		052872e9-9f10b97b-dcf0f34b-0bb90d27
% LTXE-CHAPTER-NAME:	Deprecated Functions
% ------------------------------------------------------------------------------

\section{Funzionalità deprecate/rimosse}
\begin{frame}[fragile]
	\frametitle{Funzionalità deprecate/rimosse}

	\begin{center}\huge{Capitolo 6:}\end{center}
	\begin{center}\huge{\color{typo3darkgrey}\textbf{Funzionalità deprecate/rimosse}}\end{center}

\end{frame}

% ------------------------------------------------------------------------------
% LTXE-SLIDE-START
% LTXE-SLIDE-UID:		3274557d-ceafa04a-b6b8c0e2-e12045e9
% LTXE-SLIDE-ORIGIN:	685b0aeb-6654ab27-f8128f62-3454735b English
% LTXE-SLIDE-ORIGIN:	ebe40c2e-32da4e43-67f18e6c-866b0496 German
% LTXE-SLIDE-TITLE:		Deprecation: #67991 - Removed EXT:cms (1)
% LTXE-SLIDE-REFERENCE:	Deprecation-67991-RemovedExtCms.rst
% ------------------------------------------------------------------------------

\begin{frame}[fragile]
	\frametitle{Funzionalità deprecate/rimosse}
	\framesubtitle{Rimossa l'estensione di sistema \texttt{cms} (1)}

	% decrease font size for code listing
	\lstset{basicstyle=\tiny\ttfamily}

	\begin{itemize}

		\item L'estensione di sistema \texttt{cms} è stata rimossa

		\item Gli sviluppatori devono rivedere le impostazioni di dipendenza nel file \texttt{ext\_emconf.php}

			\begin{lstlisting}
				[...]
				'constraints' => array(
				  'depends' => array(
				    // 'cms' => ' ... ',           <= SBAGLIATO!
				    'typo3' => '7.0.0-7.99.99',
				  ),
				),
				[...]
			\end{lstlisting}

		\item La maggior parte delle funzionalità sono state spostate nell'estensione di sistema \texttt{frontend}
			(questo richiede un aggiornamento dei riferimenti di lingua, vedi la slide seguente)

	\end{itemize}

\end{frame}

% ------------------------------------------------------------------------------
% LTXE-SLIDE-START
% LTXE-SLIDE-UID:		b39db2dd-9d9e6f20-084c6483-f0220f09
% LTXE-SLIDE-ORIGIN:	ff99fb62-81ad4d10-2367bc34-4985628d English
% LTXE-SLIDE-ORIGIN:	4e4332da-8e6c67f1-0496866b-ebe40c2e German
% LTXE-SLIDE-TITLE:		Deprecation: #67991 - Removed EXT:cms (2)
% LTXE-SLIDE-REFERENCE:	Deprecation-67991-RemovedExtCms.rst
% ------------------------------------------------------------------------------

\begin{frame}[fragile]
	\frametitle{Funzionalità deprecate/rimosse}
	\framesubtitle{Rimossa l'estensione di sistema \texttt{cms} (2)}

	% decrease font size for code listing
	\lstset{basicstyle=\tiny\ttfamily}

	\begin{itemize}

		\item Richiede l'aggiornamento dei riferimenti ai file di lingua:

			% hint for translators:
			% please translate words "old" and "new" in listing below,
			% but DO NOT USE special characters, like German umlauts, etc.
			% (special characters inside "lstlisting" break LaTeX).

			\begin{lstlisting}
				VECCHIO: typo3/sysext/cms/web_info/locallang.xlf
				NUOVO: typo3/sysext/frontend/Resources/Private/Language/locallang_webinfo.xlf
			\end{lstlisting}
			\vspace{-0.3cm}
			\begin{lstlisting}
				VECCHIO: typo3/sysext/cms/locallang_ttc.xlf
				NUOVO: typo3/sysext/frontend/Resources/Private/Language/locallang_ttc.xlf
			\end{lstlisting}
			\vspace{-0.3cm}
			\begin{lstlisting}
				VECCHIO: typo3/sysext/cms/locallang_tca.xlf
				NUOVO: typo3/sysext/frontend/Resources/Private/Language/locallang_tca.xlf
			\end{lstlisting}
			\vspace{-0.3cm}
			\begin{lstlisting}
				VECCHIO: typo3/sysext/cms/layout/locallang_db_new_content_el.xlf
				NUOVO: typo3/sysext/backend/Resources/Private/Language/locallang_db_new_content_el.xlf
			\end{lstlisting}
			\vspace{-0.3cm}
			\begin{lstlisting}
				VECCHIO: typo3/sysext/cms/layout/locallang.xlf
				NUOVO: typo3/sysext/backend/Resources/Private/Language/locallang_layout.xlf
			\end{lstlisting}
			\vspace{-0.3cm}
			\begin{lstlisting}
				VECCHIO: typo3/sysext/cms/layout/locallang_mod.xlf
				NUOVO: typo3/sysext/backend/Resources/Private/Language/locallang_mod.xlf
			\end{lstlisting}
			\vspace{-0.3cm}
			\begin{lstlisting}
				VECCHIO: typo3/sysext/cms/locallang_csh_webinfo.xlf
				NUOVO: typo3/sysext/frontend/Resources/Private/Language/locallang_csh_webinfo.xlf
			\end{lstlisting}
			\vspace{-0.3cm}
			\begin{lstlisting}
				VECCHIO: typo3/sysext/cms/locallang_csh_weblayout.xlf
				NUOVO: typo3/sysext/frontend/Resources/Private/Language/locallang_csh_weblayout.xlf
			\end{lstlisting}

	\end{itemize}

\end{frame}

% ------------------------------------------------------------------------------
% LTXE-SLIDE-START
% LTXE-SLIDE-UID:		6d351d90-e45e4844-6f313e41-18166031
% LTXE-SLIDE-ORIGIN:	e1a896f5-538a7c07-e462c6f0-db80381c English
% LTXE-SLIDE-ORIGIN:	ae6fc887-9aeff493-158130de-3cdbf052 German
% LTXE-SLIDE-TITLE:		Deprecation: #68074 - Deprecate getPageRenderer() methods
% LTXE-SLIDE-REFERENCE:	Deprecation-68074-DeprecateGetPageRenderer.rst
% ------------------------------------------------------------------------------

\begin{frame}[fragile]
	\frametitle{Funzionalità deprecate/rimosse}
	\framesubtitle{Deprecato il metodo PageRenderer}

	% decrease font size for code listing
	\lstset{basicstyle=\tiny\ttfamily}

	\begin{itemize}
		\item I seguenti metodi \texttt{PageRenderer} sono stati classificati come \textbf{deprecati}:

			\begin{lstlisting}
				TYPO3\CMS\Backend\Controller\BackendController::getPageRenderer()
				TYPO3\CMS\Backend\Template\DocumentTemplate::getPageRenderer()
				TYPO3\CMS\Backend\Template\FrontendDocumentTemplate::getPageRenderer()
				TYPO3\CMS\Frontend\Controller\TypoScriptFrontendController::getPageRenderer()
			\end{lstlisting}

		\item Il seguente codice va utilizzato per ottenere un instanza di PageRenderer al loro posto:

			\begin{lstlisting}
				\TYPO3\CMS\Core\Utility\GeneralUtility::makeInstance(\TYPO3\CMS\Core\Page\PageRenderer::class)
			\end{lstlisting}

	\end{itemize}

\end{frame}

% ------------------------------------------------------------------------------
% LTXE-SLIDE-START
% LTXE-SLIDE-UID:		cfa177b7-4036cd52-eff5de3d-98b57952
% LTXE-SLIDE-ORIGIN:	440429fa-30e8d4b1-4784ea6f-86bc343e English
% LTXE-SLIDE-ORIGIN:	a58c2f4f-3eb806e5-d256078a-70ed73c0 German
% LTXE-SLIDE-TITLE:		Deprecation #68098 and #68122
% LTXE-SLIDE-REFERENCE:	Deprecation-68098-GeneralUtilityMethods.rst
% LTXE-SLIDE-REFERENCE:	Deprecation-68122-GeneralUtilityReadLLfile.rst
% ------------------------------------------------------------------------------
% Deprecation: #68098 - Deprecate GeneralUtility methods
% Deprecation: #68122 - Deprecate GeneralUtility::readLLfile

\begin{frame}[fragile]
	\frametitle{Funzionalità deprecate/rimosse}
	\framesubtitle{Deprecati i metodi \texttt{GeneralUtility}}

	% decrease font size for code listing
	\lstset{basicstyle=\tiny\ttfamily}

	\begin{itemize}
		\item I seguenti metodi \texttt{GeneralUtility} sono stati classificati come \textbf{deprecati}
			e saranno rimossi in TYPO3 CMS versione 8:

			\begin{lstlisting}
				GeneralUtility::modifyHTMLColor()
				GeneralUtility::modifyHTMLColorAll()
				GeneralUtility::isBrokenEmailEnvironment()
				GeneralUtility::normalizeMailAddress()
				GeneralUtility::formatForTextarea()
				GeneralUtility::getThisUrl()
				GeneralUtility::cleanOutputBuffers()
				GeneralUtility::readLLfile()
			\end{lstlisting}

		\item Il metodo \texttt{readLLfile()} può essere sostituito con il codice seguente:

			\begin{lstlisting}
				/** @var $languageFactory \TYPO3\CMS\Core\Localization\LocalizationFactory */
				$languageFactory = GeneralUtility::makeInstance(
				  \TYPO3\CMS\Core\Localization\LocalizationFactory::class
				);
				$languageFactory->getParsedData($fileToParse, $language, $renderCharset, $errorMode);
			\end{lstlisting}

	\end{itemize}

\end{frame}

% ------------------------------------------------------------------------------
% LTXE-SLIDE-START
% LTXE-SLIDE-UID:		471542a6-1b8b6958-ecec52ac-9b0d4b2f
% LTXE-SLIDE-ORIGIN:	0d77c7a7-fd5da718-210555bf-80e688f0 English
% LTXE-SLIDE-ORIGIN:	b0c4d95f-d9fd698f-2d9dfdec-42aaf1af German
% LTXE-SLIDE-TITLE:		Breaking: #39721 - Prototype.js and Scriptaculous removed
% LTXE-SLIDE-REFERENCE:	Breaking-39721-PrototypejsAndScriptaculousRemoved.rst
% ------------------------------------------------------------------------------

\begin{frame}[fragile]
	\frametitle{Funzionalità deprecate/rimosse}
	\framesubtitle{Rimosse librerie JavaScript}

	\begin{itemize}

		\item Le librerie JavaScript \texttt{prototype.js} e \texttt{scriptaculous} sono state rimosse.
			Di conseguenza, le seguenti proprietà TypoScript non hanno più nessuna funzione:

			\begin{itemize}
				\item \texttt{page.javascriptLibs.Prototype}
				\item \texttt{page.javascriptLibs.Scriptaculous.*}
			\end{itemize}

		\item L'utilizzo dei seguenti attributi nei ViewHelper \texttt{be.container} genera un errore:
			\begin{lstlisting}
				<f:be.container loadPrototype="false" loadScriptaculous="false" scriptaculousModule="someModule,someOtherModule">
			\end{lstlisting}

		\item In sostituzione, possono essere usati \texttt{jQuery} e \texttt{RequireJS}\newline
			(che sono caricati di default nel backend)

	\end{itemize}

\end{frame}

% ------------------------------------------------------------------------------
% LTXE-SLIDE-START
% LTXE-SLIDE-UID:		5c776355-58363a17-39bc9e96-4f961676
% LTXE-SLIDE-ORIGIN:	4bb27612-517a802f-3b7a7dae-900630fc English
% LTXE-SLIDE-ORIGIN:	1375cbd0-05287ba0-3235df98-19473bf0 German
% LTXE-SLIDE-TITLE:		Deprecation: #68183 - typo3/mod.php
% LTXE-SLIDE-REFERENCE:	Deprecation-68183-Typo3modphp.rst
% ------------------------------------------------------------------------------

\begin{frame}[fragile]
	\frametitle{Funzionalità deprecate/rimosse}
	\framesubtitle{Deprecati: \texttt{init.php}, \texttt{mod.php} e \texttt{ajax.php}}

	% decrease font size for code listing
	%\lstset{basicstyle=\tiny\ttfamily}

	\begin{itemize}

		\item Al fine di ripulire il contenuto della directory \texttt{typo3}, i seguenti file sono stati
			marcati come \textbf{deprecati}: \texttt{init.php}, \texttt{mod.php} e \texttt{ajax.php}

		\item Il codice seguente può essere utilizzato in Init Entry Points:

			\begin{lstlisting}
				call_user_func(function() {
				  $classLoader = require __DIR__ . '/vendor/autoload.php';
				  (new \TYPO3\CMS\Backend\Http\Application($classLoader))->run();
				});
			\end{lstlisting}

		\item La chiamata al metodo seguente può essere usata per accedere a \texttt{mod.php}:

			\begin{lstlisting}
				BackendUtility::getModuleUrl()
			\end{lstlisting}

	\end{itemize}

\end{frame}

% ------------------------------------------------------------------------------
% LTXE-SLIDE-START
% LTXE-SLIDE-UID:		52282ac7-be3866b9-465e7886-5978e86b
% LTXE-SLIDE-ORIGIN:	a09aedae-f4adde91-f6b0d3a8-ada31684 English
% LTXE-SLIDE-ORIGIN:	749d4c59-1ec6087a-45020fbd-2f7b8646 German
% LTXE-SLIDE-TITLE:		Deprecation: #67737 - TCA: Drop additional palette
% LTXE-SLIDE-REFERENCE:	Deprecation-67737-TcaDropAdditionalPalette.rst
% ------------------------------------------------------------------------------

\begin{frame}[fragile]
	\frametitle{Funzionalità deprecate/rimosse}
	\framesubtitle{TCA: Rimossa Palette aggiuntiva}

	% decrease font size for code listing
	\lstset{basicstyle=\tiny\ttfamily}

	\begin{itemize}

		\item La stringa \texttt{showitem} della chiave TCA \texttt{types} permetteva agli sviluppatori di definire
			una palette aggiuntiva

		\item Questa è stata rimossa e spostata alla palette normale

		\item Prima:

			\begin{lstlisting}
				'types' => array(
				  'aType' => array(
				    'showitem' => 'aField;aLabel;anAdditionalPaletteName',
				  ),
				),
			\end{lstlisting}

		\item Ora:

			\begin{lstlisting}
				'types' => array(
				  'aType' => array(
				    'showitem' => 'aField;aLabel, --palette--;;anAdditionalPaletteName',
				  ),
				),
			\end{lstlisting}

	\end{itemize}

\end{frame}

% ------------------------------------------------------------------------------
% LTXE-SLIDE-START
% LTXE-SLIDE-UID:		f8db47e0-00981127-707753c0-eea6f279
% LTXE-SLIDE-ORIGIN:	e27efa5c-c3999b32-6715dd2f-da8acf1f English
% LTXE-SLIDE-ORIGIN:	6398e32e-7f467ac7-463d9b47-bc0fd47c German
% LTXE-SLIDE-TITLE:		Breaking: #67577, #67646 and #67824
% LTXE-SLIDE-REFERENCE: Breaking-67577-RteEnabledFlagHandling.rst
% LTXE-SLIDE-REFERENCE: Breaking-67646-LibraryInclusionInFrontend.rst
% LTXE-SLIDE-REFERENCE: Breaking-67824-Typo3ExtFolderRemoved.rst
% ------------------------------------------------------------------------------
% Breaking: #67577 - rte_enabled and flag handling
% Breaking: #67646 - PHP library inclusion in frontend removed
% Breaking: #67824 - typo3/ext folder removed

\begin{frame}[fragile]
	\frametitle{Funzionalità deprecate/rimosse}
	\framesubtitle{Varie (1)}

	\begin{itemize}
 
		\item I cObject "Text" e "Text with Images" avevano nel passato un checkbox "abilita RTE".
			Questo è stato rimosso, ed anche l'opzione TCA \texttt{flag}.

		\item Le seguenti opzioni TypoScript per includere file PHP sono state rimosse:

			\begin{itemize}
				\item \texttt{config.includeLibrary}
				\item \texttt{config.includeLibs}
			\end{itemize}

		\item La directory \texttt{typo3/ext} è stata rimossa\newline
			\small
				(ma non la possibilità di utilizzare estensioni globali: la directory va creata manualmente)
			\normalsize

	\end{itemize}

\end{frame}

% ------------------------------------------------------------------------------
% LTXE-SLIDE-START
% LTXE-SLIDE-UID:		89b13fd1-08c4d4b4-0e7ff68d-487bb7ab
% LTXE-SLIDE-ORIGIN:	7b47e35b-9ae0fd9a-13d2db50-ebe93a84 English
% LTXE-SLIDE-ORIGIN:	463d9b47-bc0fd47c-6398e32e-7f467ac7 German
% LTXE-SLIDE-TITLE:		Breaking: #68001
% LTXE-SLIDE-REFERENCE: Breaking-68001-RemovedExtJSCoreAndExtJSAdapters.rst
% ------------------------------------------------------------------------------
% Breaking: #68001 - Removed ExtJS Core and ExtJS Adapters
% Breaking: #68020 - Dropped DisableBigButtons

\begin{frame}[fragile]
	\frametitle{Funzionalità deprecate/rimosse}
	\framesubtitle{Varie (2)}

	\begin{itemize}

		\item ExtCore (un adattatore ExtJs) è stato rimosso, incluse le seguenti opzioni TypoScript:

			\begin{itemize}
				\item \texttt{page.javascriptLibs.ExtCore.*}
				\item \texttt{page.javascriptLibs.ExtJs.*}
			\end{itemize}

			Questo include anche l'opzione nel ViewHelper \texttt{<f:be.container>} 

		\item I cosidetti "BigButtons" ("Modifica proprietà pagina", "Sposta pagina",...) sono stati rimossi,
			inclusa l'opzioni TSconfig \texttt{mod.we\_layout.disableBigButtons}

	\end{itemize}

\end{frame}

% ------------------------------------------------------------------------------
% LTXE-SLIDE-START
% LTXE-SLIDE-UID:		41d5f2f4-15f1a478-8e01fcf9-81d3dd16
% LTXE-SLIDE-ORIGIN:	8052d34e-ccc2def0-c43053a0-c3c8eba3 English
% LTXE-SLIDE-ORIGIN:	d3a774ba-c1fd32c6-099aa85f-96b6a2a2 German
% LTXE-SLIDE-TITLE:		Breaking: #68020, #68131, #65790 and #67506
% LTXE-SLIDE-REFERENCE: Breaking-68020-DroppedDisableBigButtons.rst
% LTXE-SLIDE-REFERENCE: Breaking-68131-StreamlineErrorAndExceptionHandling.rst
% LTXE-SLIDE-REFERENCE: Deprecation-65790-PagesStoragePidDeprecated.rst
% LTXE-SLIDE-REFERENCE: Deprecation-67506-DeprecateIconUtilitygetIcon.rst
% ------------------------------------------------------------------------------
% Breaking: #68131 - Streamline error and exception handling
% Deprecation: #65790 - Remove pages.storage_pid and logic
% Deprecation: #67506 - Deprecate IconUtility::getIcon

\begin{frame}[fragile]
	\frametitle{Funzionalità deprecate/rimosse}
	\framesubtitle{Varie (3)}

	\begin{itemize}

		\item Gli errori e la gestione delle eccezioni non possono più essere configurate nelle estensioni (es. sovrascritto in
			\texttt{ext\_localconf.php}), ma solo nei file \texttt{LocalConfiguration.php} o
			\texttt{AdditionalConfiguration.php} 

		\item Il campo "General Record Storage Page", che conteneva il PID di pagina di archivio, è stato rimosso.
			Ora il PID di archivio deve essere configurato usando TypoScript o FlexForms.

		\item La funzione \texttt{IconUtility::getIcon()} è stata classificata come \textbf{deprecata} (va
			usato il metodo \texttt{IconUtility::getSpriteIconForRecord()} al suo posto)

	\end{itemize}

\end{frame}

% ------------------------------------------------------------------------------
