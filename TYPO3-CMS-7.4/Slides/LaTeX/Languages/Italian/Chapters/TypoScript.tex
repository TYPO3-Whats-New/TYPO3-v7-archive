% ------------------------------------------------------------------------------
% TYPO3 CMS 7.4 - What's New - Chapter "TypoScript" (English Version)
%
% @author	Michael Schams <schams.net>
% @license	Creative Commons BY-NC-SA 3.0
% @link		http://typo3.org/download/release-notes/whats-new/
% @language	English
% ------------------------------------------------------------------------------
% LTXE-CHAPTER-UID:		693ffeb9-5ce46f4e-8c0b7d7c-be9f3ee3
% LTXE-CHAPTER-NAME:	TypoScript
% ------------------------------------------------------------------------------

\section{TSconfig \& TypoScript}
\begin{frame}[fragile]
	\frametitle{TSconfig \& TypoScript}

	\begin{center}\huge{Capitolo 2:}\end{center}
	\begin{center}\huge{\color{typo3darkgrey}\textbf{TSconfig \& TypoScript}}\end{center}

\end{frame}

% ------------------------------------------------------------------------------
% LTXE-SLIDE-START
% LTXE-SLIDE-UID:		120895e1-679a198c-95387c1d-83d4fa59
% LTXE-SLIDE-ORIGIN:	69805fcd-5bf6738b-3d31552e-d0add3ee English
% LTXE-SLIDE-ORIGIN:	78394814-b88d7d32-54dd1c41-022129d0 German
% LTXE-SLIDE-TITLE:		Feature: #61903 - PageTS dataprovider for backend layouts (1)
% LTXE-SLIDE-REFERENCE:	Feature-61903-PageTSDataproviderForBackendLayouts.rst
% ------------------------------------------------------------------------------
\begin{frame}[fragile]
	\frametitle{TSconfig \& TypoScript}
	\framesubtitle{Data-Provider per layout di backend (1)}

	% decrease font size for code listing
	\lstset{basicstyle=\tiny\ttfamily}

	\begin{itemize}
		\item Ora è possibile definire i layout di backend tramite TSconfig di pagina e anche attraverso la registrazione di file. Ad esempio:

		\begin{lstlisting}
			mod {
			  web_layout {
			    BackendLayouts {
			      exampleKey {
			        title = Esempio
			        config {
			          backend_layout {
			            colCount = 1
			            rowCount = 2
			            rows {
			              1 {
			                columns {
			                  1 {
			                    name = LLL:EXT:frontend/ ... /locallang_ttc.xlf:colPos.I.3
			                    colPos = 3
			                    colspan = 1
			                  }
			                }
			              }
			[...]
		\end{lstlisting}

	\end{itemize}

\end{frame}

% ------------------------------------------------------------------------------
% LTXE-SLIDE-START
% LTXE-SLIDE-UID:		65dbdd22-facf42ce-1721ac1c-28be31c7
% LTXE-SLIDE-ORIGIN:	0691efc7-c5a339e2-1753851e-9a62d8ed English
% LTXE-SLIDE-ORIGIN:	b88d4814-7d327839-10221c41-54dd29d0 German
% LTXE-SLIDE-TITLE:		Feature: #61903 - PageTS dataprovider for backend layouts (2)
% LTXE-SLIDE-REFERENCE:	Feature-61903-PageTSDataproviderForBackendLayouts.rst
% ------------------------------------------------------------------------------
\begin{frame}[fragile]
	\frametitle{TSconfig \& TypoScript}
	\framesubtitle{Data-Provider per layout di backend (2)}

	% decrease font size for code listing
	\lstset{basicstyle=\tiny\ttfamily}

	\begin{itemize}
		\item \smaller(continua)\normalsize
		\begin{lstlisting}
			[...]
			              2 {
			                columns {
			                  1 {
			                    name = Principale
			                    colPos = 0
			                    colspan = 1
			                  }
			                }
			              }
			            }
			          }
			        }
			        icon = EXT:example_extension/Resources/Public/Images/BackendLayouts/default.gif
			      }
			    }
			  }
			}
		\end{lstlisting}
	\end{itemize}

\end{frame}

% ------------------------------------------------------------------------------
% LTXE-SLIDE-START
% LTXE-SLIDE-UID:		ed2bf8e7-c1b5413a-6ec76bb5-29253744
% LTXE-SLIDE-ORIGIN:	53593b48-60766071-136bf56b-f63caf39 English
% LTXE-SLIDE-ORIGIN:	9fc7749d-aa1d6042-b1b7699d-e9aee9ed German
% LTXE-SLIDE-TITLE:		Feature: #67360 - Custom attribute name and multiple values for meta tags
% LTXE-SLIDE-REFERENCE:	Feature-67360-CustomAttributeNameAndMultipleValuesForMetaTags.rst
% ------------------------------------------------------------------------------

\begin{frame}[fragile]
	\frametitle{TSconfig \& TypoScript}
	\framesubtitle{Meta Tag ampliati}

	% decrease font size for code listing
	\lstset{basicstyle=\tiny\ttfamily}

	\begin{itemize}

		% Note for translators: make sure, the following bullet point is only ONE line long.
		% If it wraps and uses two lines, the code example below does not fit on the page.
		% If not avoidable, split the content and create a second slide for the code.

		\item L'opzione \texttt{page.meta} supporta gli attributi \href{http://ogp.me}{Open Graph}

			\begin{lstlisting}
				page {
				  meta {
				    X-UA-Compatible = IE=edge,chrome=1
				    X-UA-Compatible.attribute = http-equiv
				    keywords = TYPO3
				    # <meta property="og:site_name" content="TYPO3" />
				    og:site_name = TYPO3
				    og:site_name.attribute = property
				    description = Inspiring people to share
				    og:description = Inspiring people to share
				    og:description.attribute = property
				    og:locale = en_GB
				    og:locale.attribute = property
				    og:locale:alternate {
				      attribute = property
				      value.1 = fr_FR
				      value.2 = de_DE
				    }
				    refresh = 5; url=http://example.com/
				    refresh.attribute = http-equiv
				  }
				}
			\end{lstlisting}

	\end{itemize}

\end{frame}

% ------------------------------------------------------------------------------
% LTXE-SLIDE-START
% LTXE-SLIDE-UID:		16f9d3f3-dcc4bf0e-1fcf705c-8b374c8a
% LTXE-SLIDE-ORIGIN:	5df1ea93-74ddf5d4-c8acd647-7f47e896 English
% LTXE-SLIDE-ORIGIN:	c38a626f-40c7bda9-523650fb-6829234e German
% LTXE-SLIDE-TITLE:		Feature: #68191 - TypoScript .select option languageField is active by default
% LTXE-SLIDE-REFERENCE:	Feature-68191-TypoScriptSelectOptionLanguageFieldIsActiveByDefault.rst
% ------------------------------------------------------------------------------

\begin{frame}[fragile]
	\frametitle{TSconfig \& TypoScript}
	\framesubtitle{\texttt{languageField} Impostato di default}

	% decrease font size for code listing
	\lstset{basicstyle=\tiny\ttfamily}

	\begin{itemize}

		\item L'opzione TypoScript \texttt{select} (usata nel cObject CONTENT per esempio) richiedeva l'impostazione
			\texttt{languageField} esplicitamente

		\item Questo non è più necessario, l'impostazione è recuperata automaticamente dallle informazione della struttura TCA

			% reminder for translators: special characters such as German umlauts are not
			% allowed inside "lstlisting". They break the LaTeX code. Be careful when you
			% want to translate the sentence "the following line is not required anymore"
			% below (feel free to leave it in English).

			\begin{lstlisting}
				config.sys_language_uid = 2
				page.10 = CONTENT
				page.10 {
				  table = tt_content
				  select.where = colPos=0

				  # the following line is not required anymore:
				  #select.languageField = sys_language_uid

				  renderObj = TEXT
				  renderObj.field = header
				  renderObj.htmlSpecialChars = 1
				}
			\end{lstlisting}

	\end{itemize}

\end{frame}

% ------------------------------------------------------------------------------
% LTXE-SLIDE-START
% LTXE-SLIDE-UID:		852edc7b-d4661598-71862e8d-a8a39d4f
% LTXE-SLIDE-ORIGIN:	48fffdc3-1a0b74a9-f2415121-1e9f032a English
% LTXE-SLIDE-ORIGIN:	945ae684-9d826d2a-1c2e0d7b-c6c69465 German
% LTXE-SLIDE-TITLE:		Feature: #64200 - Allow individual content caching
% LTXE-SLIDE-REFERENCE:	Feature-64200-AllowIndividualContentCaching.rst
% ------------------------------------------------------------------------------
\begin{frame}[fragile]
	\frametitle{TSconfig \& TypoScript}
	\framesubtitle{Caching individuale dei contenuti}

	% decrease font size for code listing
	\lstset{basicstyle=\tiny\ttfamily}

	\begin{itemize}

		\item Da TYPO3 CMS 7.4 esiste una cache individuale dei contenuti che - paragonabile a \texttt{stdWrap.cache} -
			funziona anche con gli oggetti COA\newline(simile a "Magento Block Caching")

			\begin{columns}[T]
				\begin{column}{.07\textwidth}
                \end{column}
				\begin{column}{.465\textwidth}
					\begin{lstlisting}
						page = PAGE
						page.10 = COA
						page.10 {
						  cache.key = coaout
						  cache.lifetime = 60
						  #stdWrap.cache.key = coastdWrap
						  #stdWrap.cache.lifetime = 60
						  10 = TEXT
						  10 {
						    cache.key = mycurrenttimestamp
						    cache.lifetime = 60
						    data = date : U
						    strftime = %H:%M:%S
						    noTrimWrap = |10: | |
						  }
					[...]
					\end{lstlisting}
				\end{column}

				\begin{column}{.465\textwidth}
					\begin{lstlisting}
					[...]
						  20 = TEXT
						  20 {
						    data = date : U
						    strftime = %H:%M:%S
						    noTrimWrap = |20: | |
						  }
						}
					\end{lstlisting}

				\end{column}
			\end{columns}

	\end{itemize}

\end{frame}

% ------------------------------------------------------------------------------
% LTXE-SLIDE-START
% LTXE-SLIDE-UID:		42010dde-c8e7d1c2-982554a0-1dd87dd9
% LTXE-SLIDE-ORIGIN:	a9f58068-74a67b75-cd3b44ef-4c5422b9 English
% LTXE-SLIDE-ORIGIN:	d4fd55ab-f9f1cd6c-4eb2ca00-d50beeb2 German
% LTXE-SLIDE-TITLE:		Feature: #67880 - Added count to listNum
% LTXE-SLIDE-REFERENCE:	Feature-67880-AddedCountToListNum.rst
% ------------------------------------------------------------------------------

\begin{frame}[fragile]
	\frametitle{TSconfig \& TypoScript}
	\framesubtitle{Contatore di elementi in una lista}

	% decrease font size for code listing
	\lstset{basicstyle=\tiny\ttfamily}

	\begin{itemize}

		\item Una nuova proprietà \texttt{returnCount} è stata aggiunta alla proprietà stdWrap \texttt{split}

		\item Questa permette di contare il numero di elementi in una lista

		\item Il codice seguente restituisce \texttt{9} per esempio:

			\begin{lstlisting}
				1 = TEXT
				1 {
				  value = x,y,z,1,2,3,a,b,c
				  split.token = ,
				  split.returnCount = 1
				}
			\end{lstlisting}

	\end{itemize}

\end{frame}

% ------------------------------------------------------------------------------
% LTXE-SLIDE-START
% LTXE-SLIDE-UID:		a3582479-90eabdc8-f935e4ba-16550e35
% LTXE-SLIDE-ORIGIN:	525607c3-d2f26693-2b44076d-e90599ca English
% LTXE-SLIDE-ORIGIN:	a31ad0e4-857ff98a-239b23fd-c2e48958 German
% LTXE-SLIDE-TITLE:		Feature: #65550 - Make table display order configurable in List module
% LTXE-SLIDE-REFERENCE:	Feature-65550-MakeTableDisplayOrderConfigurableInListModule.rst
% ------------------------------------------------------------------------------

\begin{frame}[fragile]
	\frametitle{TSconfig \& TypoScript}
	\framesubtitle{Ordinemaneto delle tabelle nel modulo "Lista"}

	% decrease font size for code listing
	\lstset{basicstyle=\tiny\ttfamily}

	\begin{itemize}

		\item La nuova opzione TSconfig \texttt{mod.web\_list.tableDisplayOrder} è stata aggiunta al modulo "Lista"

		\item Con questa opzione si configura l'ordine delle tabelle con cui devono essere visualizzate

		\item Le chiavi \texttt{before} e \texttt{after} possono essere usate per specificare un ordine riferendo al nome di altre tabelle

		\begin{columns}[T]
			\begin{column}{.07\textwidth}
			\end{column}
			\begin{column}{.465\textwidth}

				\small Sintassi:\normalsize

				\begin{lstlisting}
					mod.web_list.tableDisplayOrder {
					  <tableName> {
					    before = <tableA>, <tableB>, ...
					    after = <tableA>, <tableB>, ...
					  }
					}
				\end{lstlisting}
			\end{column}
			\begin{column}{.465\textwidth}

				\small Per esempio:\normalsize

				\begin{lstlisting}
					mod.web_list.tableDisplayOrder {
					  be_users.after = be_groups
					  sys_filemounts.after = be_users
					  pages_language_overlay.before = pages
					  fe_users.after = fe_groups
					  fe_users.before = pages
					}
				\end{lstlisting}

			\end{column}
		\end{columns}

	\end{itemize}

\end{frame}

% ------------------------------------------------------------------------------
% LTXE-SLIDE-START
% LTXE-SLIDE-UID:		5c520ec7-c7d1b518-e3040e4f-bc70764b
% LTXE-SLIDE-ORIGIN:	9d764efe-2cd7e74e-f53f456b-f4a88331 English
% LTXE-SLIDE-ORIGIN:	35e3a45d-888c3fe2-26d320d1-ce6c65c3 German
% LTXE-SLIDE-TITLE:		Feature: #33071 - Add the http header "Content-Language" when rendering a page
% LTXE-SLIDE-REFERENCE:	Feature-33071-AddTheHttpHeaderContent-LanguageWhenRenderingAPage.rst
% ------------------------------------------------------------------------------

\begin{frame}[fragile]
	\frametitle{TSconfig \& TypoScript}
	\framesubtitle{Content-Language nell'intestazione HTTP}

	\begin{itemize}

		\item L'intestazione HTTP \texttt{Content-language: XX} è inviata per default, dove "XX" è il codice ISO della
			configurazione \texttt{sys\_language\_content} 

		\item Con l'utilizzo di \texttt{config.disableLanguageHeader = 1}, questa funzione può essere disabilitata
			(non invia l'intestazione \texttt{Content-language})

	\end{itemize}

\end{frame}

% ------------------------------------------------------------------------------
% LTXE-SLIDE-START
% LTXE-SLIDE-UID:		8c0db55c-3e315ad3-64331d64-32b724e8
% LTXE-SLIDE-ORIGIN:	17e7f151-0ebb8899-6209180f-bec0fc56 English
% LTXE-SLIDE-ORIGIN:	35a84207-7b86ee45-c39d815d-859811e1 German
% LTXE-SLIDE-TITLE:		Feature: #45725 - Added recursive option to folder based file collections
% LTXE-SLIDE-REFERENCE:	Feature-45725-AddedRecursiveOptionToFolderBasedFileCollections.rst
% ------------------------------------------------------------------------------

\begin{frame}[fragile]
	\frametitle{TSconfig \& TypoScript}
	\framesubtitle{Opzione ricorsiva per File Collections}

	\begin{itemize}

		\item Le collezioni di file basate su una directory hanno un opzione per recuperare tutti i file recursivamente

		\item L'opzione è disponibile anche nell'oggetto TypoScript \texttt{FILES}

			\begin{lstlisting}
				filecollection = FILES
				filecollection {
				  folders = 1:images/
				  folders.recursive = 1
				  renderObj = IMAGE
				  renderObj {
				    file.import.data = file:current:uid
				  }
				}
			\end{lstlisting}

	\end{itemize}

\end{frame}

% ------------------------------------------------------------------------------
% LTXE-SLIDE-START
% LTXE-SLIDE-UID:		a43c16c4-08d469a2-54068fdb-bbd8bb74
% LTXE-SLIDE-ORIGIN:	b7337338-ab366edd-bd94e9c0-6d7e86de English
% LTXE-SLIDE-ORIGIN:	ece6c712-df09d880-9378940c-e658d6a5 German
% LTXE-SLIDE-TITLE:		Feature: #34922 - Allow .ts file extension for static TypoScript templates
% LTXE-SLIDE-REFERENCE:	Feature-34922-AllowTsFileExtensionForStaticTyposcriptTemplates.rst
% ------------------------------------------------------------------------------

\begin{frame}[fragile]
	\frametitle{TSconfig \& TypoScript}
	\framesubtitle{L'estensione \texttt{.ts} per i template statici}

	\begin{itemize}

		\item In TYPO3 CMS < 7.4, solo i seguenti file erano validi come template TypoScript statici:

			\begin{itemize}
				\item \texttt{constants.txt}
				\item \texttt{setup.txt}
				\item \texttt{include\_static.txt}
				\item \texttt{include\_static\_files.txt}
			\end{itemize}

		\item Per \texttt{constants} e \texttt{setup} ora è possibile usare l'estensione di file \texttt{.ts}

		\item In questo contesto \texttt{.ts} ha la priorità su \texttt{.txt}

	\end{itemize}

\end{frame}

% ------------------------------------------------------------------------------
% LTXE-SLIDE-START
% LTXE-SLIDE-UID:		495f80f5-f8fa864d-1357ad70-e85c0de6
% LTXE-SLIDE-ORIGIN:	6e127c7e-38fe07f8-e6b9545b-9142c306 English
% LTXE-SLIDE-ORIGIN:	168d7ca5-b9a3c55b-d6300076-7abdfd09 German
% LTXE-SLIDE-TITLE:		Feature: #20194 - Configuration for displaying the "save & view" button
% LTXE-SLIDE-REFERENCE:	Feature-20194-ConfigurationForDisplayingTheSaveViewButton.rst
% ------------------------------------------------------------------------------

\begin{frame}[fragile]
	\frametitle{TSconfig \& TypoScript}
	\framesubtitle{Bottone Salva \& vedi}

	\begin{itemize}

		\item Il bottone "salva \& vedi" è ora configurabile via TSconfig

		\item TSconfig \texttt{TCEMAIN.preview.disableButtonForDokType} accetta una lista separata da virgole di "doktypes"

		\item Il valore di default è "254, 255, 199" (che è: Storage Folder, Recycler e Menu Separator)

		\item Di conseguenza, il bottone "salva \& vedi" non è visualizzato di default in folders e pagine recycler

	\end{itemize}

\end{frame}

% ------------------------------------------------------------------------------
% LTXE-SLIDE-START
% LTXE-SLIDE-UID:		cf774569-37bcb7ed-af7e4d49-00f91ed8
% LTXE-SLIDE-ORIGIN:	eb340ca0-00687bcb-51cda09f-82cb51ce English
% LTXE-SLIDE-ORIGIN:	80efec02-bec87365-a8bc2ab6-6741053f German
% LTXE-SLIDE-TITLE:		Feature: #43984 - Add stdWrap functionality to TreatIdAsReference TypoScript
% LTXE-SLIDE-REFERENCE:	Feature-43984-AddStdWrapFunctionalityToTreatIdAsReferenceTypoScript.rst
% ------------------------------------------------------------------------------
\begin{frame}[fragile]
	\frametitle{TSconfig \& TypoScript}
	\framesubtitle{stdWrap per \texttt{treatIdAsReference}}

	\begin{itemize}

		\item Per l'oggetto \texttt{getImgResource} esiste l'opzione \texttt{treatIdAsReference},
			che può essere utilizzata per definire quali UID sono trattati come UID del \texttt{sys\_file\_reference}
			anzichè \texttt{sys\_file}.

		\item L'opzione \texttt{treatIdAsReference} dispone ora della funzionalità stdWrap

	\end{itemize}

\end{frame}

% ------------------------------------------------------------------------------
