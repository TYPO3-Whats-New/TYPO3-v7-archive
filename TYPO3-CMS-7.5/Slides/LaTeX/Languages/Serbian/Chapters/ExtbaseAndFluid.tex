% ------------------------------------------------------------------------------
% TYPO3 CMS 7.5 - What's New - Chapter "Extbase & Fluid" (Serbian Version)
%
% @author	Michael Schams <schams.net>
% @license	Creative Commons BY-NC-SA 3.0
% @link		http://typo3.org/download/release-notes/whats-new/
% @language	Serbian
% ------------------------------------------------------------------------------
% LTXE-CHAPTER-UID:		c9b2e403-e06bb691-ce202054-149c5089
% LTXE-CHAPTER-NAME:	Extbase & Fluid
% ------------------------------------------------------------------------------

\section{Extbase i Fluid}
\begin{frame}[fragile]
	\frametitle{Extbase i Fluid}

	\begin{center}\huge{Poglavlje 4:}\end{center}
	\begin{center}\huge{\color{typo3darkgrey}\textbf{Extbase i Fluid}}\end{center}

\end{frame}

% ------------------------------------------------------------------------------
% LTXE-SLIDE-START
% LTXE-SLIDE-UID:		48cc41a6-9a809142-ad8c93ea-9bb6fb4f
% LTXE-SLIDE-ORIGIN:	8bf4a04e-9453b098-84fa7616-1df8ea8f English
% LTXE-SLIDE-ORIGIN:	7528691a-d33af3f4-edab28ae-ba7c8016 German
% LTXE-SLIDE-TITLE:		Severity-filtering for FlashMessageQueue
% LTXE-SLIDE-REFERENCE:	Feature-7098-SeverityFilteringForFlashMessageQueue.rst
% ------------------------------------------------------------------------------

\begin{frame}[fragile]
	\frametitle{Extbase i Fluid}
	\framesubtitle{Stroze filtriranje za FlashMessageQueue}

	\begin{itemize}

		\item U TYPO3 CMS < 7.5, \underline{sve} poruke iz FlashMessageQueue mogle biti samo dohvacene i/ili uklonjene

		\item U TYPO3 CMS >= 7.5, ovo meze da se uradi za odredjen nivo storgoce:

			\begin{lstlisting}
				FlashMessageQueue::getAllMessages($severity);
				FlashMessageQueue::getAllMessagesAndFlush($severity);
				FlashMessageQueue::removeAllFlashMessagesFromSession($severity);
				FlashMessageQueue::clear($severity);
			\end{lstlisting}

	\end{itemize}

\end{frame}

% ------------------------------------------------------------------------------
% LTXE-SLIDE-START
% LTXE-SLIDE-UID:		ad2fe9f0-aba51ea3-c5040897-3c6aacb1
% LTXE-SLIDE-ORIGIN:	1728dca6-0c88a636-5dc7cc00-dfa3266f English
% LTXE-SLIDE-ORIGIN:	484e03e4-cb7828c2-8bf31f48-cabf004a German
% LTXE-SLIDE-TITLE:		Query support for BETWEEN added
% LTXE-SLIDE-REFERENCE:	Feature-47812-QuerySupportForBETWEENAdded.rst
% ------------------------------------------------------------------------------

\begin{frame}[fragile]
	\frametitle{Extbase i Fluid}
	\framesubtitle{Dodata podrska za upit "\texttt{between}"}

	\begin{itemize}

		\item Podrska za \texttt{between} je dodata za Extbase query objekat

		\item Ne postoji poboljsanje performansi jer je DBMS vec interno konvertovao
			"\texttt{between}" u: \texttt{min <= expr AND expr <= max}

		\item Ova nova Extbase opcija kopira ponasanje DBMS' gradjenjem logickih 
			\texttt{AND} uslova, tako da ovo radi na svim DBMS

			\begin{lstlisting}
				$query->matching(
				  $query->between('uid', 3, 5)
				);
			\end{lstlisting}

	\end{itemize}

\end{frame}

% ------------------------------------------------------------------------------
% LTXE-SLIDE-START
% LTXE-SLIDE-UID:		d80d6078-39020aa3-64148248-32f076ee
% LTXE-SLIDE-ORIGIN:	0f2d87b1-c907b8a6-44612e51-e84e6490 English
% LTXE-SLIDE-ORIGIN:	12b78f4c-7291d788-ea865da7-e6501a2e German
% LTXE-SLIDE-TITLE:		Added support for multiple FlashMessage queues
% LTXE-SLIDE-REFERENCE:	Feature-64726-UsingArbitraryFlashmessageQueues.rst
% ------------------------------------------------------------------------------

\begin{frame}[fragile]
	\frametitle{Extbase i Fluid}
	\framesubtitle{Visestruki FlashMessage Queues}

	\begin{itemize}

		\item Sada je moguce implementirati vise FlashMessageQueues:

			\begin{lstlisting}
				$queueIdentifier = 'myQueue';
				$this->controllerContext->getFlashMessageQueue($queueIdentifier);
			\end{lstlisting}

		\item Pristup koristeci Fluid radi na sledeci nacin::

			\begin{lstlisting}
				<f:flashMessages queueIdentifier="myQueue" ></f:flashMessages>
			\end{lstlisting}

	\end{itemize}

\end{frame}

% ------------------------------------------------------------------------------
% LTXE-SLIDE-START
% LTXE-SLIDE-UID:		b7e5778d-9467c582-05f6f8d8-475fcd1f
% LTXE-SLIDE-ORIGIN:	6915482b-a7f5bd30-f04b3dc6-d94a5f2a English
% LTXE-SLIDE-ORIGIN:	2f1002c1-6bde766c-3b1ad500-ab9ccf0c German
% LTXE-SLIDE-TITLE:		Introduced MediaViewHelper (1)
% LTXE-SLIDE-REFERENCE:	Feature-66366-IntroducedMediaViewHelper.rst
% ------------------------------------------------------------------------------

\begin{frame}[fragile]
	\frametitle{Extbase i Fluid}
	\framesubtitle{Media ViewHelper (1)}

	% decrease font size for code listing
	\lstset{basicstyle=\tiny\ttfamily}

	\begin{itemize}

		\item Da bi se lako prikazali video, audio ili neki drugi tipovi fajlova sa registrovanom
			Renderer klasom za korisnicki interfejs, implementiran je \texttt{MediaViewHelper}

		\item \texttt{MediaViewHelper} prvo proverava da li je prisutan Renderer za dati fajl - 
			ako nije, rendera se podrazumevani image tag

		\item Primeri:

			\begin{lstlisting}
				<code title="Image Object">
				  <f:media file="{file}" width="400" height="375" ></f:media>
				</code>

				<output>
				  <img alt="alt set in image record" src="fileadmin/_processed_/323223424.png"
				    width="396" height="375" />
				</output>
			\end{lstlisting}

	\end{itemize}

\end{frame}

% ------------------------------------------------------------------------------
% LTXE-SLIDE-START
% LTXE-SLIDE-UID:		97f71647-405e8f87-5bf3c02b-c7e5d15a
% LTXE-SLIDE-ORIGIN:	8bd8314d-631b7ae9-3102867f-804046fa English
% LTXE-SLIDE-ORIGIN:	73fe507d-772b6011-7457429b-4039dd5c German
% LTXE-SLIDE-TITLE:		Introduced MediaViewHelper (2)
% LTXE-SLIDE-REFERENCE:	Feature-66366-IntroducedMediaViewHelper.rst
% ------------------------------------------------------------------------------

\begin{frame}[fragile]
	\frametitle{Extbase i Fluid}
	\framesubtitle{Media ViewHelper (2)}

	% decrease font size for code listing
	\lstset{basicstyle=\tiny\ttfamily}

	\begin{itemize}

		\item Primeri (nastavak):

			\begin{lstlisting}
				<code title="MP4 Video Object">
				  <f:media file="{file}" width="400" height="375" ></f:media>
				</code>

				<output>
				  <video width="400" height="375" controls>
				    <source src="fileadmin/user_upload/my-video.mp4" type="video/mp4">
				  </video>
				</output>

				<code title="MP4 Video Object with loop and autoplay option set">
				  <f:media file="{file}" width="400" height="375"
				    additionalConfig="{loop: '1', autoplay: '1'}" ></f:media>
				</code>

				<output>
				  <video width="400" height="375" controls loop>
				    <source src="fileadmin/user_upload/my-video.mp4" type="video/mp4">
				  </video>
				</output>
			\end{lstlisting}

	\end{itemize}

\end{frame}

% ------------------------------------------------------------------------------
% LTXE-SLIDE-START
% LTXE-SLIDE-UID:		f07d2fde-70eac16f-bcf7cd91-5c83f4e8
% LTXE-SLIDE-ORIGIN:	5d858cf5-81f4cf18-1b13e093-6a00c98a English
% LTXE-SLIDE-ORIGIN:	2abe5d2e-c1c7e93b-c86559da-81196fdb German
% LTXE-SLIDE-TITLE:		Adopt form to support the Extbase/ Fluid MVC stack (1)
% LTXE-SLIDE-REFERENCE:	Feature-69401-AdoptFormToSupportTheExtbaseFluidMVCStack.rst
% ------------------------------------------------------------------------------

\begin{frame}[fragile]
	\frametitle{Extbase i Fluid}
	\framesubtitle{Sistemsko prosirenje \texttt{form} (1)}

	\begin{itemize}

		\item Sistemsko prosirenje \texttt{form} (ukljucuje jedinstveni data model, kontroler, logiku, validaciju osobina, prikaze i sablone) je unapredjeno da podrzava Extbase/Fluid MVC 

		\item Ovo omogucava bolje upravljanje i kontrolu ponasanja i izgleda jednostavnim izmenama u Fluid sablonima ili koriscenjem spostvenih view helper-a

		\item Svaki element forme koristi svoje parsale, koji mogu da se konfigurisu putem 
			TypoScript opcije \texttt{partialPath = ...}

	\end{itemize}

\end{frame}

% ------------------------------------------------------------------------------
% LTXE-SLIDE-START
% LTXE-SLIDE-UID:		b5407f66-6d65abac-444d303e-0b806d95
% LTXE-SLIDE-ORIGIN:	20c7837c-5b841f21-fb3ead22-265aebe4 English
% LTXE-SLIDE-ORIGIN:	d45facb0-73989d92-ffbc7c2b-46f583ba German
% LTXE-SLIDE-TITLE:		Adopt form to support the Extbase/ Fluid MVC stack (2)
% LTXE-SLIDE-REFERENCE:	Feature-69401-AdoptFormToSupportTheExtbaseFluidMVCStack.rst
% ------------------------------------------------------------------------------

\begin{frame}[fragile]
	\frametitle{Extbase i Fluid}
	\framesubtitle{Sistemsko prosirenje \texttt{form} (2)}

	\begin{itemize}

		\item Postoje sledeca nova tri ViewHelper-a:

			\begin{itemize}
				\item \texttt{AggregateSelectOptionsViewHelper} (za optgroup tagove)
				\item \texttt{SelectViewHelper} (za optgroup tagove)
				\item \texttt{PlainMailViewHelper} (za ispisivanje mejlova kao obican tekst)
			\end{itemize}

		\item Kao dodatak, tu su i tri View-a::

			\begin{itemize}
				\item \texttt{show} (sama forma)
				\item \texttt{confirmation} (potvrdna strana)
				\item \texttt{postProcessor}/\texttt{mail} (email)
			\end{itemize}

		\item Putanje sablona i vidljivost polja mogu se podesavati za svaki View posebno

	\end{itemize}

\end{frame}

% ------------------------------------------------------------------------------
% LTXE-SLIDE-START
% LTXE-SLIDE-UID:		9f2742c5-59462295-f844af3a-a197e030
% LTXE-SLIDE-ORIGIN:	20b68c91-1b501686-f97f7b55-0822d9d1 English
% LTXE-SLIDE-ORIGIN:	cd99ea8c-e6234f0f-caf559ed-c3f825c0 German
% LTXE-SLIDE-TITLE:		Add annotation for CLI only commands
% LTXE-SLIDE-REFERENCE:	Feature-68746-AddAnnotationForCLIOnlyCommands.rst
% ------------------------------------------------------------------------------

\begin{frame}[fragile]
	\frametitle{Extbase i Fluid}
	\framesubtitle{Anotacija \texttt{@cli}}

	\begin{itemize}

		\item Koristeci novu anotaciju \texttt{@cli}, komande u Extbase CommandController 
			mogu se oznaciti kao samo CLI komande

		\item Ove komande se ne uzimaju u obzir u selekciji komandi scheduler-a

		\item Na primer, tipicno koriscenje je koriscenje komande kao sto je \texttt{extbase:help:help} 

	\end{itemize}

\end{frame}

% ------------------------------------------------------------------------------
