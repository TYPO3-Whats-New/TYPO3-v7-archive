% ------------------------------------------------------------------------------
% TYPO3 CMS 7.5 - What's New - Chapter "Changements en profondeur" (French Version)
%
% @author	Michael Schams <schams.net>
% @license	Creative Commons BY-NC-SA 3.0
% @link		http://typo3.org/download/release-notes/whats-new/
% @language	French
% ------------------------------------------------------------------------------
% LTXE-CHAPTER-UID:		b8f1c138-2fb2f5ce-15d1ffc0-e80171e2
% LTXE-CHAPTER-NAME:	Changements en profondeur
% ------------------------------------------------------------------------------

\section{Changements en profondeur}
\begin{frame}[fragile]
	\frametitle{Changements en profondeur}

	\begin{center}\huge{Chapitre 3~:}\end{center}
	\begin{center}\huge{\color{typo3darkgrey}\textbf{Changements en profondeur}}\end{center}

\end{frame}

% ------------------------------------------------------------------------------
% LTXE-SLIDE-START
% LTXE-SLIDE-UID:		c9298d41-eed47832-ccf14bdc-291c517d
% LTXE-SLIDE-ORIGIN:	16e557e2-fa6a5035-5b28d66e-27c0c626 English
% LTXE-SLIDE-ORIGIN:	03be70bc-886235ca-10f0cef5-6b33fe60 German
% LTXE-SLIDE-TITLE:		Fluid-based Content Elements Introduced (1)
% LTXE-SLIDE-REFERENCE:	Feature-38732-Fluid-basedContentElementsIntroduced.rst
% ------------------------------------------------------------------------------

\begin{frame}[fragile]
	\frametitle{Changements en profondeur}
	\framesubtitle{Fluid-based Content Elements (1)}

	\begin{itemize}

		\item La nouvelle extension système \textbf{"Fluid-based Content Elements"} est implémentée

		\item Des modèles Fluid sont utilisés pour le rendu des éléments de contenu à la place du TypoScript

		\item Elle pourra être une alternative à \textit{CSS Styled Content} dans le futur

		\item Les gabarits statiques suivants sont à inclure pour utiliser la fonctionnalité~:

			\begin{itemize}
				\item Content Elements (\texttt{fluid\_styled\_content})
				\item Content Elements CSS (optional) (\texttt{fluid\_styled\_content})
			\end{itemize}

	\end{itemize}

\end{frame}



% ------------------------------------------------------------------------------
% LTXE-SLIDE-START
% LTXE-SLIDE-UID:		23860963-3ad78c66-b6ecd689-a3e67f06
% LTXE-SLIDE-ORIGIN:	d8383008-e3bbb7e2-91720bea-3b93db62 English
% LTXE-SLIDE-ORIGIN:	bb7124ec-4658414b-757ca8ca-f0addee3 German
% LTXE-SLIDE-TITLE:		Fluid-based Content Elements Introduced (2)
% LTXE-SLIDE-REFERENCE:	Feature-38732-Fluid-basedContentElementsIntroduced.rst
% ------------------------------------------------------------------------------

\begin{frame}[fragile]
	\frametitle{Changements en profondeur}
	\framesubtitle{Fluid-based Content Elements (2)}

	% decrease font size for code listing
	\lstset{basicstyle=\tiny\ttfamily}

	\begin{itemize}

		\item De plus le gabarit TSconfig de page suivant doit être inclus~:\newline
			\small
				\texttt{Fluid-based Content Elements (fluid\_styled\_content)}
			\normalsize

		\item Surchargez les modèles par défaut en ajoutant vos propres chemins en configuration TypoScript~:

			\begin{lstlisting}
				lib.fluidContent.templateRootPaths.50 = EXT:site_example/Resources/Private/Templates/
				lib.fluidContent.partialRootPaths.50 = EXT:site_example/Resources/Private/Partials/
				lib.fluidContent.layoutRootPaths.50 = EXT:site_example/Resources/Private/Layouts/
			\end{lstlisting}

	\end{itemize}

\end{frame}

% ------------------------------------------------------------------------------
% LTXE-SLIDE-START
% LTXE-SLIDE-UID:		139325fb-417243ca-e25b0603-2cb02a6d
% LTXE-SLIDE-ORIGIN:	d25f0c05-31c928e4-e7439714-d87f883b English
% LTXE-SLIDE-ORIGIN:	04cd65d3-e013813d-d2eac450-7d3a79a8 German
% LTXE-SLIDE-TITLE:		Fluid-based Content Elements Introduced (3)
% LTXE-SLIDE-REFERENCE:	Feature-38732-Fluid-basedContentElementsIntroduced.rst
% LTXE-SLIDE-REFERENCE:	Important-67954-MigrateCTypesTextImageAndTextpicToTextmedia.rst
% ------------------------------------------------------------------------------

\begin{frame}[fragile]
	\frametitle{Changements en profondeur}
	\framesubtitle{Fluid-based Content Elements (3)}

	\begin{itemize}

		\item Migration de \textit{CSS Styled Content} vers \textit{Fluid-based Content Elements}~:

			\begin{itemize}

				\item Désinstaller l'extension \texttt{css\_styled\_content}

				\item Installer l'extension \texttt{fluid\_styled\_content}

				\item Utiliser l'assistant de migration de l'Install Tool pour migrer les éléments
					\texttt{text}, \texttt{image} et \texttt{textpic} vers \texttt{textmedia}

			\end{itemize}
	\end{itemize}

	\vspace{1.4cm}

	\begingroup
		\color{red}
			\small
				\underline{Note~:} \textit{"Fluid-based Content Elements"} est dans un stage préliminaire
				et des changements importants peuvent toujours avoir lieu avant TYPO3 CMS 7 LTS.
				Des conflits avec \textit{CSS Styled Content} peuvent toujours exister.
			\normalsize
	\endgroup

\end{frame}


% ------------------------------------------------------------------------------
% LTXE-SLIDE-START
% LTXE-SLIDE-UID:		55d1e214-f400878e-951082f2-6561423e
% LTXE-SLIDE-ORIGIN:	89b8c8c2-3a458bcf-5d61b025-6b4fb470 English
% LTXE-SLIDE-ORIGIN:	8807e1a0-2b3fadac-ac842970-b85e5130 German
% LTXE-SLIDE-TITLE:		Add SELECTmmQuery method to DatabaseConnection
% LTXE-SLIDE-REFERENCE:	Feature-19494-AddSELECTmmQueryMethodToDatabaseConnection.rst
% ------------------------------------------------------------------------------

\begin{frame}[fragile]
	\frametitle{Changements en profondeur}
	\framesubtitle{Méthode SELECTmmQuery}

	% decrease font size for code listing
	\lstset{basicstyle=\tiny\ttfamily}

	\begin{itemize}

		\item La méthode \texttt{SELECT\_mm\_query} est ajoutée à la classe \texttt{DatabaseConnection}

		\item Extraite de \texttt{exec\_SELECT\_mm\_query} pour séparer la construction et l'exécution des
			requêtes M:M.

		\item Permettant l'usage de la construction de requête dans la couche d'abstraction des bases de données.

			\begin{lstlisting}
				$query = SELECT_mm_query('*', 'table1', 'table1_table2_mm', 'table2', 'AND table1.uid = 1',
				'', 'table1.title DESC');
			\end{lstlisting}

	\end{itemize}

\end{frame}

% ------------------------------------------------------------------------------
% LTXE-SLIDE-START
% LTXE-SLIDE-UID:		3565c5f1-9566f3ce-9023f9ab-d7bdd85d
% LTXE-SLIDE-ORIGIN:	f2c4f1a9-6cd020c0-518650d6-2a31d0de English
% LTXE-SLIDE-ORIGIN:	398fdcb2-1cc9a862-42e59a09-e4ed65ba German
% LTXE-SLIDE-TITLE:		Scheduler task to optimize database tables
% LTXE-SLIDE-REFERENCE:	Feature-25341-SchedulerTaskToOptimizeDatabaseTables.rst
% ------------------------------------------------------------------------------

\begin{frame}[fragile]
	\frametitle{Changements en profondeur}
	\framesubtitle{Optimiser les tables MySQL}

	\begin{itemize}

		\item Nouvelle tâche du planificateur pour exécuter la commande MySQL \texttt{OPTIMIZE TABLE}
			sur les tables sélectionnées

		\item La commande réorganise physiquement le stockage des données des tables et de leurs indexes
			afin de réduire l'occupation et améliorer les performances d'E/S

		\item Ces types de tables sont supportés~:\newline
			\texttt{MyISAM}, \texttt{InnoDB} et \texttt{ARCHIVE}

		\item L'utilisation de cette tâche avec DBAL et d'autre SGBD \underline{n'}est
			\underline{pas} supportée car la commande est spécifique à MySQL

	\end{itemize}

	% Note to translators: if this does not fit on one slide, you could try to change
	% \small to \smaller or shorten the sentences below or the bullet points above.

	\begingroup
		\color{red}
			\small
				\underline{Note~:} L'optimisation des tables est coûteuse en performances d'E/S.
				Aussi, avant MySQL 5.6.17 le processus verrouille les tables pendant son exécution,
				pouvant impacter le site.
			\normalsize
	\endgroup

\end{frame}

% ------------------------------------------------------------------------------
% LTXE-SLIDE-START
% LTXE-SLIDE-UID:		dd2a1e1b-8d756536-e12d64df-e97b3b5e
% LTXE-SLIDE-ORIGIN:	f0da3ed6-ab7b2518-402338db-a1842865 English
% LTXE-SLIDE-ORIGIN:	f9cbeb69-d08efdaf-a8c00d48-353f667e German
% LTXE-SLIDE-TITLE:		Improved handling of online media (1)
% LTXE-SLIDE-REFERENCE:	Feature-61799-ImprovedHandlingOfOnlineMedia.rst
% ------------------------------------------------------------------------------

\begin{frame}[fragile]
	\frametitle{Changements en profondeur}
	\framesubtitle{Gestion des contenus en ligne (1)}

	\begin{itemize}

		\item Les contenus externes (online media) sont supportés par défaut

		\item Comme exemples, le support des vidéos YouTube et Vimeo est ajouté au noyau

		\item Les ressources s'ajoutent en tant qu'URL dans le contenu \textbf{"Text \& Media"},
			par exemple

		\item La classe d'assistance correspondante récupère les métadonnées et fournie
			l'image de vignette si disponible

	\end{itemize}

\end{frame}

% ------------------------------------------------------------------------------
% LTXE-SLIDE-START
% LTXE-SLIDE-UID:		09a7e62b-7d475318-b7f9d4f6-526918fe
% LTXE-SLIDE-ORIGIN:	485f95a8-a9ec5940-e9aa7322-688630cb English
% LTXE-SLIDE-ORIGIN:	76e87d6a-cd1ebbf9-9a4d49ec-49cc569f German
% LTXE-SLIDE-TITLE:		Improved handling of online media (2)
% LTXE-SLIDE-REFERENCE:	Feature-61799-ImprovedHandlingOfOnlineMedia.rst
% ------------------------------------------------------------------------------

\begin{frame}[fragile]
	\frametitle{Changements en profondeur}
	\framesubtitle{Gestion des contenus en ligne (2)}

	Les motifs d'URL suivants sont disponibles~:
	\vspace{0.4cm}

	\begin{columns}[T]
		\begin{column}{.52\textwidth}
			\smaller
				\tabto{0.2cm}\textbf{\underline{YouTube:}}\newline
				\tabto{0.2cm}\texttt{youtu.be/<code>}\newline
				\tabto{0.2cm}\texttt{www.youtube.com/watch?v=<code>}\newline
				\tabto{0.2cm}\texttt{www.youtube.com/v/<code>}\newline
				\tabto{0.2cm}\texttt{www.youtube-nocookie.com/v/<code>}\newline
				\tabto{0.2cm}\texttt{www.youtube.com/embed/<code>}\newline
		\end{column}
		\begin{column}{.48\textwidth}
			\vspace{-0.25cm}\smaller
				\textbf{\underline{Vimeo:}}\newline
				\texttt{vimeo.com/<code>}\newline
				\texttt{player.vimeo.com/video/<code>}\newline
		\end{column}
	\end{columns}

\end{frame}

% ------------------------------------------------------------------------------
% LTXE-SLIDE-START
% LTXE-SLIDE-UID:		7983483c-e7138672-fce7f5c6-eec3e92c
% LTXE-SLIDE-ORIGIN:	dc79335c-d9e51543-ff5759e0-5005bff8 English
% LTXE-SLIDE-ORIGIN:	aef8746a-e1babbc9-4dd74eec-cf379dae German
% LTXE-SLIDE-TITLE:		Improved handling of online media (3)
% LTXE-SLIDE-REFERENCE:	Feature-61799-ImprovedHandlingOfOnlineMedia.rst
% ------------------------------------------------------------------------------

\begin{frame}[fragile]
	\frametitle{Changements en profondeur}
	\framesubtitle{Gestion des contenus en ligne (3)}

	% decrease font size for code listing
	\lstset{basicstyle=\tiny\ttfamily}

	\begin{itemize}

		\item L'accès aux ressources par Fluid s'effectue comme suit~:

		\begin{lstlisting}
			<!-- enable js api and set no-cookie support for YouTube videos -->
			<f:media file="{file}" additionalConfig="{enablejsapi:1, 'no-cookie': true}" ></f:media>

			<!-- show title and uploader for YouTube and Vimeo before video starts playing -->
			<f:media file="{file}" additionalConfig="{showinfo:1}" ></f:media>
		\end{lstlisting}

		\item Options de configuration pour les vidéos YouTube~:\newline
			\small
				\texttt{autoplay}, \texttt{controls}, \texttt{loop}, \texttt{enablejsapi}, \texttt{showinfo}, \texttt{no-cookie}
			\normalsize

		\item Options de configuration pour les vidéos Vimeo~:\newline
			\small
				\texttt{autoplay}, \texttt{loop}, \texttt{showinfo}
			\normalsize
	\end{itemize}

\end{frame}

% ------------------------------------------------------------------------------
% LTXE-SLIDE-START
% LTXE-SLIDE-UID:		3e30915a-32962c25-dca16271-c26a3f98
% LTXE-SLIDE-ORIGIN:	2b41f269-eaa80c0c-89bcd0f0-ee66248b English
% LTXE-SLIDE-ORIGIN:	9ed71aaa-a7615392-78f0ac0c-f9bd7400 German
% LTXE-SLIDE-TITLE:		Improved handling of online media (4)
% LTXE-SLIDE-REFERENCE:	Feature-61799-ImprovedHandlingOfOnlineMedia.rst
% ------------------------------------------------------------------------------

\begin{frame}[fragile]
	\frametitle{Changements en profondeur}
	\framesubtitle{Gestion des contenus en ligne (4)}

	% decrease font size for code listing
	\lstset{basicstyle=\tiny\ttfamily}

	\begin{itemize}

		\item Pour inscrire votre propre gestionnaire de contenu en ligne, vous avez besoin
			d'une classe \small\texttt{OnlineMediaHelper}\normalsize\space implémentant
			\small\texttt{OnlineMediaHelperInterface}\normalsize\space et une classe
			\small\texttt{FileRenderer}\normalsize\space implémentant
			\small\texttt{FileRendererInterface}\normalsize\space

			\begin{lstlisting}
				// register your own online video service (the used key is also the bind file extension name)
				$GLOBALS['TYPO3_CONF_VARS']['SYS']['OnlineMediaHelpers']['myvideo'] =
				  \MyCompany\Myextension\Helpers\MyVideoHelper::class;

				$rendererRegistry = \TYPO3\CMS\Core\Resource\Rendering\RendererRegistry::getInstance();
				$rendererRegistry->registerRendererClass(
				  \MyCompany\Myextension\Rendering\MyVideoRenderer::class
				);

				// register an custom mime-type for your videos
				$GLOBALS['TYPO3_CONF_VARS']['SYS']['FileInfo']['fileExtensionToMimeType']['myvideo'] =
				  'video/myvideo';

				// register your custom file extension as allowed media file
				$GLOBALS['TYPO3_CONF_VARS']['SYS']['mediafile_ext'] .= ',myvideo';
			\end{lstlisting}

	\end{itemize}

\end{frame}

% ------------------------------------------------------------------------------
% LTXE-SLIDE-START
% LTXE-SLIDE-UID:		809d2abb-dd9ad4e8-889d750b-e67efe49
% LTXE-SLIDE-ORIGIN:	0d9d73fc-2a68f137-f5ff5876-74aeef2b English
% LTXE-SLIDE-ORIGIN:	c675e3a8-9a429b0b-61a4a603-d5b0740f German
% LTXE-SLIDE-TITLE:		Unified Backend Routing
% LTXE-SLIDE-REFERENCE:	Feature-65493-BackendRouting.rst
% ------------------------------------------------------------------------------

\begin{frame}[fragile]
	\frametitle{Changements en profondeur}
	\framesubtitle{Routage Backend}

	% decrease font size for code listing
	\lstset{basicstyle=\tiny\ttfamily}

	\begin{itemize}

		\item Un nouveau composant de routage est ajouté au backend de TYPO3 prenant en
			charge l'adressage des différents appels et modules de TYPO3 CMS

		\item Les routes sont à définir dans la classe suivante~:\newline
			\small
				\texttt{Configuration/Backend/Routes.php}
			\normalsize

			\begin{lstlisting}
				return [
				  'myRouteIdentifier' => [
				    'path' => '/document/edit',
				    'controller' => Acme\MyExtension\Controller\MyExampleController::class . '::methodToCall'
				  ]
				];
			\end{lstlisting}

		\item Les méthodes appelées contiennent des objets de requête et réponse compatible avec PSR-7~:

			\begin{lstlisting}
				public function methodToCall(ServerRequestInterface $request, ResponseInterface $response) {
				  ...
				}
			\end{lstlisting}

	\end{itemize}

\end{frame}

% ------------------------------------------------------------------------------
% LTXE-SLIDE-START
% LTXE-SLIDE-UID:		708a0a7e-2efa9a57-9ab42b84-3f38b15d
% LTXE-SLIDE-ORIGIN:	a38b3e30-a34a3bb1-ee94992a-1c91db32 English
% LTXE-SLIDE-ORIGIN:	df56086a-498ae937-9b2bf393-a95828f3 German
% LTXE-SLIDE-TITLE:		Autoload definition can be provided in ext_emconf.php
% LTXE-SLIDE-REFERENCE:	Feature-68700-AutoloadDefinitionCanBeProvidedInExt_emconfphp.rst
% ------------------------------------------------------------------------------

\begin{frame}[fragile]
	\frametitle{Changements en profondeur}
	\framesubtitle{Définition de l'auto-chargement dans \texttt{ext\_emconf.php}}

	% decrease font size for code listing
	\lstset{basicstyle=\tiny\ttfamily}

	\begin{itemize}

		\item Les extensions peuvent contenir une ou plusieurs définitions PSR-4 dans le fichier \texttt{ext\_emconf.php}

		\item C'était déjà possible dans \texttt{composer.json}, mais avec cette nouvelle fonctionnalité,
			les développeurs d'extension n'ont plus besoin de fournir un fichier composer juste pour ca

			\begin{lstlisting}
				$EM_CONF[$_EXTKEY] = array (
				  'title' => 'Extension Skeleton for TYPO3 CMS 7',
				  ...
				'autoload' =>
				  array(
				    'psr-4' => array(
				      'Helhum\\ExtScaffold\\' => 'Classes'
				    )
				  )
				);
			\end{lstlisting}

			\small
				(c'est la méthode d'enregistrement des classes recommandée de TYPO3)
			\normalsize

	\end{itemize}

\end{frame}

% ------------------------------------------------------------------------------
% LTXE-SLIDE-START
% LTXE-SLIDE-UID:		ab77d4e2-a2eee710-ec81f17f-f4ba6926
% LTXE-SLIDE-ORIGIN:	6d370930-d24dd6c2-54e2330d-b873a914 English
% LTXE-SLIDE-ORIGIN:	137a9fca-b6558571-346141d2-9304a223 German
% LTXE-SLIDE-TITLE:		Icon-Factory (1)
% LTXE-SLIDE-REFERENCE:	Feature-68741-IntroduceNewIconFactoryAsBaseForReplaceTheIconSkinningAPI.rst
% LTXE-SLIDE-REFERENCE:	Feature-69095-IntroduceIconStateForIconFactory.rst
% ------------------------------------------------------------------------------

\begin{frame}[fragile]
	\frametitle{Changements en profondeur}
	\framesubtitle{Nouvelle fabrique d'icônes (1)}

	% decrease font size for code listing
	\lstset{basicstyle=\smaller\ttfamily}

	\begin{itemize}

		\item La logique pour travailler avec les icônes, leurs tailles et
			les icônes de recouvrement est maintenant dans la classe \texttt{IconFactory}

		\item La nouvelle fabrique remplacera l'ancienne API de thème petit à petit

		\item Tous les icônes du noyau seront enregistrés avec la classe \texttt{IconRegistry}

		\item Les extensions doivent utiliser \texttt{IconRegistry::registerIcon()} pour surcharger un icône
			ou en ajouter un nouveau à la fabrique~:

			\begin{lstlisting}
				IconRegistry::registerIcon(
				  $identifier,
				  $iconProviderClassName,
				  array $options = array()
				);
			\end{lstlisting}

	\end{itemize}

\end{frame}

% ------------------------------------------------------------------------------
% LTXE-SLIDE-START
% LTXE-SLIDE-UID:		5a5b18b6-f44134d6-102034f9-682093cd
% LTXE-SLIDE-ORIGIN:	4e9f1524-c3c04972-bf6bd14b-19193ab2 English
% LTXE-SLIDE-ORIGIN:	6b59b4c7-21a7ff5c-4e67b53d-e4509355 German
% LTXE-SLIDE-TITLE:		Icon-Factory (2)
% LTXE-SLIDE-REFERENCE:	Feature-68741-IntroduceNewIconFactoryAsBaseForReplaceTheIconSkinningAPI.rst
% LTXE-SLIDE-REFERENCE:	Feature-69095-IntroduceIconStateForIconFactory.rst
% ------------------------------------------------------------------------------

\begin{frame}[fragile]
	\frametitle{Changements en profondeur}
	\framesubtitle{Nouvelle fabrique d'icônes (2)}

	% decrease font size for code listing
	\lstset{basicstyle=\tiny\ttfamily}

	\begin{itemize}

		\item Le noyau de TYPO3 CMS implémente trois fournisseurs d'icônes~:\newline
			\smaller
				\texttt{BitmapIconProvider}, \texttt{FontawesomeIconProvider} et \texttt{SvgIconProvider}
			\normalsize

		\item Usages d'exemple~:

			\begin{lstlisting}
				$iconFactory = GeneralUtility::makeInstance(IconFactory::class);
				$iconFactory->getIcon(
				  $identifier,
				  Icon::SIZE_SMALL,
				  $overlay,
				  IconState::cast(IconState::STATE_DEFAULT)
				)->render();
			\end{lstlisting}

		\item Les valeurs valides pour \texttt{Icon::SIZE\_…} sont~:\newline
			\small\texttt{SIZE\_SMALL}, \texttt{SIZE\_DEFAULT} et \texttt{SIZE\_LARGE}\normalsize
			\vspace{0.4cm}

		\item Les valeurs valides pour \texttt{Icon::STATE\_…} sont~:\newline
			\small\texttt{STATE\_DEFAULT} and \texttt{STATE\_DISABLED}\normalsize

	\end{itemize}

\end{frame}

% ------------------------------------------------------------------------------
% LTXE-SLIDE-START
% LTXE-SLIDE-UID:		fc8844cd-9a6fc4fd-b625e36b-97051e47
% LTXE-SLIDE-ORIGIN:	ffa90088-571dcf20-3d315854-a05814da English
% LTXE-SLIDE-ORIGIN:	21a7eeb5-b4c7ff5c-b53de450-4e679355 German
% LTXE-SLIDE-TITLE:		Icon-Factory (3)
% LTXE-SLIDE-REFERENCE:	Feature-68741-IntroduceNewIconFactoryAsBaseForReplaceTheIconSkinningAPI.rst
% LTXE-SLIDE-REFERENCE:	Feature-69095-IntroduceIconStateForIconFactory.rst
% ------------------------------------------------------------------------------

\begin{frame}[fragile]
	\frametitle{Changements en profondeur}
	\framesubtitle{Nouvelle fabrique d'icônes (3)}

	% decrease font size for code listing
	\lstset{basicstyle=\tiny\ttfamily}

	\begin{itemize}

		\item Le noyau de TYPO3 CMS fourni un ViewHelper Fluid permettant l'utilisation facile d'un icône dans une vue Fluid~:

			\begin{lstlisting}
				{namespace core = TYPO3\CMS\Core\ViewHelpers}

				<core:icon identifier="my-icon-identifier"></core:icon>

				<!-- use the "small" size if none given ->
				<core:icon identifier="my-icon-identifier"></core:icon>
				<core:icon identifier="my-icon-identifier" size="large"></core:icon>
				<core:icon identifier="my-icon-identifier" overlay="overlay-identifier"></core:icon>

				<core:icon identifier="my-icon-identifier" size="default" overlay="overlay-identifier">
				</core:icon>

				<core:icon identifier="my-icon-identifier" size="large" overlay="overlay-identifier">
				</core:icon>
			\end{lstlisting}

	\end{itemize}

\end{frame}

% ------------------------------------------------------------------------------
% LTXE-SLIDE-START
% LTXE-SLIDE-UID:		65c67c53-f4f575ac-8b1bc99c-a1da8ae1
% LTXE-SLIDE-ORIGIN:	10739ce3-88cfb3f6-d8364643-424cbf3c English
% LTXE-SLIDE-ORIGIN:	d9de708f-474f2f36-c104b01c-0d23a352 German
% LTXE-SLIDE-TITLE:		Signal for pre processing linkvalidator records
% LTXE-SLIDE-REFERENCE:	Feature-52217-SignalForPreProcessingLinkvalidatorRecords.rst
% ------------------------------------------------------------------------------

\begin{frame}[fragile]
	\frametitle{Changements en profondeur}
	\framesubtitle{Hooks et Signals}

	% decrease font size for code listing
	\lstset{basicstyle=\tiny\ttfamily}

	\begin{itemize}

		\item Un nouveau signal est ajouté à LinkValidator, permettant des opérations
			supplémentaires lors de l'initialisation d'un enregistrement\newline
			\small
				(ex. récupérer du contenu depuis la configuration du plugin dans l'enregistrement)
			\normalsize

		\item Enregistrer le signal dans \texttt{ext\_localconf.php}~:

			\begin{lstlisting}
				$signalSlotDispatcher = \TYPO3\CMS\Core\Utility\GeneralUtility::makeInstance(
				  \TYPO3\CMS\Extbase\SignalSlot\Dispatcher::class
				);

				$signalSlotDispatcher->connect(
				  \TYPO3\CMS\Linkvalidator\LinkAnalyzer::class,
				  'beforeAnalyzeRecord',
				  \Vendor\Package\Slots\RecordAnalyzerSlot::class,
				  'beforeAnalyzeRecord'
				);
			\end{lstlisting}

	\end{itemize}

\end{frame}

% ------------------------------------------------------------------------------
% LTXE-SLIDE-START
% LTXE-SLIDE-UID:		6ad9b0bf-578b2981-4744d6d8-4ecd6020
% LTXE-SLIDE-ORIGIN:	8b50edb3-2363f2a9-3120a711-db8b6ede English
% LTXE-SLIDE-ORIGIN:	bcb8ef73-b8669a17-102da209-5f5c9adc German
% LTXE-SLIDE-TITLE:		Replaced JumpURL features with hooks (1)
% LTXE-SLIDE-REFERENCE:	Breaking-52156-ReplaceJumpUrlWithHooks.rst
% ------------------------------------------------------------------------------

\begin{frame}[fragile]
	\frametitle{Changements en profondeur}
	\framesubtitle{JumpUrl en extension système (1)}

	% decrease font size for code listing
	\lstset{basicstyle=\tiny\ttfamily}

	\begin{itemize}

		\item La génération et la prise en charge des JumpURLs sont déplacées dans l'extension système \texttt{jumpurl}

		\item De nouveaux hooks sont introduits pour permettre la génération et la prise en charge personnalisée (voir page suivante)

	\end{itemize}

	\breakingchange

\end{frame}

% ------------------------------------------------------------------------------
% LTXE-SLIDE-START
% LTXE-SLIDE-UID:		f0ae0a58-2886d122-8c55d99a-133efbf0
% LTXE-SLIDE-ORIGIN:	68003b46-065b81f8-5206141c-77a35b88 English
% LTXE-SLIDE-ORIGIN:	ecb8fef3-69b86a17-02da2109-9adc5f5c German
% LTXE-SLIDE-TITLE:		Replaced JumpURL features with hooks (2)
% LTXE-SLIDE-REFERENCE:	Breaking-52156-ReplaceJumpUrlWithHooks.rst
% ------------------------------------------------------------------------------

\begin{frame}[fragile]
	\frametitle{Changements en profondeur}
	\framesubtitle{JumpUrl en extension système (2)}

	% decrease font size for code listing
	\lstset{basicstyle=\tiny\ttfamily}

	\begin{itemize}

		\item Hook 1~: manipuler les \textbf{URLs} pendant la génération de liens

			\begin{lstlisting}
				$GLOBALS['TYPO3_CONF_VARS']['SC_OPTIONS']['urlProcessing']['urlHandlers']
				  ['myext_myidentifier']['handler'] = \Company\MyExt\MyUrlHandler::class;

				// class needs to implement the UrlHandlerInterface:
				class MyUrlHandler implements \TYPO3\CMS\Frontend\Http\UrlHandlerInterface {
				  ...
				}
			\end{lstlisting}

		\item Hook 2~: prise en charge des liens \textbf{links}

			\begin{lstlisting}
				$GLOBALS['TYPO3_CONF_VARS']['SC_OPTIONS']['urlProcessing']['urlProcessors']
				  ['myext_myidentifier']['processor'] = \Company\MyExt\MyUrlProcessor::class;

				// class needs to implement the UrlProcessorInterface:
				class MyUrlProcessor implements \TYPO3\CMS\Frontend\Http\UrlProcessorInterface {
				  ...
				}
			\end{lstlisting}

	\end{itemize}

\end{frame}

% ------------------------------------------------------------------------------
% LTXE-SLIDE-START
% LTXE-SLIDE-UID:		3993f89f-f546d35a-16d6286e-50b9c18d
% LTXE-SLIDE-ORIGIN:	ff50feed-0f2dfe5a-1f960c51-e493d3b0 English
% LTXE-SLIDE-ORIGIN:	08a70f0f-528daf11-ee485c29-d9707f25 German
% LTXE-SLIDE-TITLE:		Colored Output for CLI Calls on Errors
% LTXE-SLIDE-REFERENCE:	Feature-67056-AddOptionToDisableMoveButtonsTCAGroupType.rst
% LTXE-SLIDE-REFERENCE:	Feature-67875-OverrideCategoryRegistryEntry.rst
% LTXE-SLIDE-REFERENCE:	Feature-68804-ColoredOutputForCLI-relevantErrorMessages.rst
% ------------------------------------------------------------------------------

\begin{frame}[fragile]
	\frametitle{Changements en profondeur}
	\framesubtitle{Interface ligne de commande (CLI)}

	% decrease font size for code listing
	\lstset{basicstyle=\tiny\ttfamily}

	\begin{itemize}

		\item Appeler \texttt{typo3/cli\_dispatch.phpsh} en ligne de commande affiche un message d'erreur colorisé
			si aucune ou une clé invalide a été spécifiée en premier paramètre

		\item Les contrôleurs de commande d'Extbase peuvent être placés dans des sous-dossiers dans le dossier
			\texttt{Command}

		\item Exemple~:\newline

			Controller in file:\newline
			\smaller\texttt{my\_ext/Classes/Command/Hello/WorldCommandController.php}\normalsize\newline
			...can be called via CLI:\newline
			\smaller\texttt{typo3/cli\_dispatch.sh extbase my\_ext:hello:world <arguments>}\normalsize

	\end{itemize}

\end{frame}

% ------------------------------------------------------------------------------
% LTXE-SLIDE-START
% LTXE-SLIDE-UID:		3d3d69cc-4efab8ba-8ae1fc6a-8bc3931c
% LTXE-SLIDE-ORIGIN:	df540c68-edd41657-8854e46a-a716dccb English
% LTXE-SLIDE-ORIGIN:	a708daf1-e48f525c-7010fe7f-2529d9e4 German
% LTXE-SLIDE-TITLE:		Miscellaneous (1)
% LTXE-SLIDE-REFERENCE:	Feature-68804-ColoredOutputForCLI-relevantErrorMessages.rst
% LTXE-SLIDE-REFERENCE:	Feature-69512-SupportTyposcriptFilesAsTextFileType.rst
% LTXE-SLIDE-REFERENCE:	Feature-69543-IntroducedGLOBALSTYPO3_CONF_VARSSYSmediafile_ext.rst
% ------------------------------------------------------------------------------

\begin{frame}[fragile]
	\frametitle{Changements en profondeur}
	\framesubtitle{Divers (1)}

	\begin{itemize}

		\item Les boutons de déplacement du type TCA \texttt{group} peuvent être désactivés
			explicitement en utilisant l'option \texttt{hideMoveIcons = TRUE}

		\item La méthode \texttt{makeCategorizable} est étendue avec un nouveau paramètre
			\texttt{override} pour définir une nouvelle configuration de catégorie pour une
			combinaison de table et champ existante

		\item Exemple~:

			\begin{lstlisting}
				\TYPO3\CMS\Core\Utility\ExtensionManagementUtility::makeCategorizable(
				  'css_styled_content', 'tt_content', 'categories', array(), TRUE
				);
			\end{lstlisting}

			\small
				Le dernier paramètre (ici~: \texttt{TRUE}) force la surcharge (par défaut \texttt{FALSE}).
			\normalsize

	\end{itemize}

\end{frame}

% ------------------------------------------------------------------------------
% LTXE-SLIDE-START
% LTXE-SLIDE-UID:		c23ba1fa-ff03cbe8-691f3138-df89de44
% LTXE-SLIDE-ORIGIN:	5a6db8b0-ff039479-8ba12ed0-d50d8a0f English
% LTXE-SLIDE-ORIGIN:	d52d6418-554d801f-352b85cb-54046d28 German
% LTXE-SLIDE-TITLE:		Miscellaneous (2)
% LTXE-SLIDE-REFERENCE:	Feature-69730-IntroduceUniqueIdGenerator.rst
% LTXE-SLIDE-REFERENCE:	Important-68758-CommandControllersAllowedInSubfolders.rst
% ------------------------------------------------------------------------------

\begin{frame}[fragile]
	\frametitle{Changements en profondeur}
	\framesubtitle{Divers (2)}

	% decrease font size for code listing
	%\lstset{basicstyle=\tiny\ttfamily}

	\begin{itemize}

		\item Nouvelle fonction pour générer un identifiant unique

			\begin{lstlisting}
				$uniqueId = \TYPO3\CMS\Core\Utility\StringUtility::getUniqueId('Prefix');
			\end{lstlisting}

		\item Le type de fichier \texttt{.typoscript} est ajouté à la liste des types de fichier texte valides

		\item Une nouvelle option de configuration définie les extensions des fichiers multimédia~:

			\begin{lstlisting}
				$GLOBALS['TYPO3_CONF_VARS']['SYS']['mediafile_ext'] =
				  'gif,jpg,jpeg,bmp,png,pdf,svg,ai,mov,avi';
			\end{lstlisting}

	\end{itemize}

	\breakingchange

\end{frame}

% ------------------------------------------------------------------------------
