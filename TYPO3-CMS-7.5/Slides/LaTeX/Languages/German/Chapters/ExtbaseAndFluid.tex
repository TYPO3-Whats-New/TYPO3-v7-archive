% ------------------------------------------------------------------------------
% TYPO3 CMS 7.5 - What's New - Chapter "Extbase & Fluid" (German Version)
%
% @author	Patrick Lobacher <patrick@lobacher.de> and Michael Schams <schams.net>
% @license	Creative Commons BY-NC-SA 3.0
% @link		http://typo3.org/download/release-notes/whats-new/
% @language	German
% ------------------------------------------------------------------------------
% LTXE-CHAPTER-UID:		c9b2e403-e06bb691-ce202054-149c5089
% LTXE-CHAPTER-NAME:	Extbase & Fluid
% ------------------------------------------------------------------------------

\section{Extbase \& Fluid}
\begin{frame}[fragile]
	\frametitle{Extbase \& Fluid}

	\begin{center}\huge{Kapitel 4:}\end{center}
	\begin{center}\huge{\color{typo3darkgrey}\textbf{Extbase \& Fluid}}\end{center}

\end{frame}

% ------------------------------------------------------------------------------
% LTXE-SLIDE-START
% LTXE-SLIDE-UID:		7528691a-d33af3f4-edab28ae-ba7c8016
% LTXE-SLIDE-TITLE:		Severity-filtering for FlashMessageQueue
% LTXE-SLIDE-REFERENCE:	Feature-7098-SeverityFilteringForFlashMessageQueue.rst
% ------------------------------------------------------------------------------

\begin{frame}[fragile]
	\frametitle{Extbase \& Fluid}
	\framesubtitle{Severity-Filter für FlashMessages}

	\begin{itemize}

		\item Bislang konnten nur alle FlashMessages auf einmal ermittelt und/oder gelöscht werden
		\item Nun kann man diese entsprechend der Severity (Gewichtung) filtern

			\begin{lstlisting}
				FlashMessageQueue::getAllMessages($severity);
				FlashMessageQueue::getAllMessagesAndFlush($severity);
				FlashMessageQueue::removeAllFlashMessagesFromSession($severity);
				FlashMessageQueue::clear($severity);
			\end{lstlisting}

	\end{itemize}

\end{frame}

% ------------------------------------------------------------------------------
% LTXE-SLIDE-START
% LTXE-SLIDE-UID:		484e03e4-cb7828c2-8bf31f48-cabf004a
% LTXE-SLIDE-TITLE:		Query support for BETWEEN added
% LTXE-SLIDE-REFERENCE:	Feature-47812-QuerySupportForBETWEENAdded.rst
% ------------------------------------------------------------------------------

\begin{frame}[fragile]
	\frametitle{Extbase \& Fluid}
	\framesubtitle{Query-Support für \texttt{between} hinzugefügt}

	\begin{itemize}

		\item Es wurde \texttt{between} zum Extbase Query Objekt hinzugefügt,
			welches prüft, ob sich ein Wert innerhalb einer oberen und unteren
			Grenze (einschließlich) befindet

		\item Dies wird zu \texttt{(min <= expr AND expr <= max)} übersetzt.\newline
			Dadurch hat dies keine Performance-Auswirkungen und funktioniert auf jedem DBMS

			\begin{lstlisting}
				$query->matching(
				  $query->between('uid', 3, 5)
				);
			\end{lstlisting}

	\end{itemize}

\end{frame}

% ------------------------------------------------------------------------------
% LTXE-SLIDE-START
% LTXE-SLIDE-UID:		12b78f4c-7291d788-ea865da7-e6501a2e
% LTXE-SLIDE-TITLE:		Added support for multiple FlashMessage queues
% LTXE-SLIDE-REFERENCE:	Feature-64726-UsingArbitraryFlashmessageQueues.rst
% ------------------------------------------------------------------------------

\begin{frame}[fragile]
	\frametitle{Extbase \& Fluid}
	\framesubtitle{Mehrere Message-Queues}

	\begin{itemize}

		\item Es können nun mehrere Message-Queues in Extbase realisiert werden:

			\begin{lstlisting}
				$queueIdentifier = 'myQueue';
				$this->controllerContext->getFlashMessageQueue($queueIdentifier);
			\end{lstlisting}

		\item In Fluid kann wie folgt darauf zugegriffen werden:

			\begin{lstlisting}
				<f:flashMessages queueIdentifier="myQueue" />
			\end{lstlisting}

	\end{itemize}

\end{frame}

% ------------------------------------------------------------------------------
% LTXE-SLIDE-START
% LTXE-SLIDE-UID:		2f1002c1-6bde766c-3b1ad500-ab9ccf0c
% LTXE-SLIDE-TITLE:		Introduced MediaViewHelper (1)
% LTXE-SLIDE-REFERENCE:	Feature-66366-IntroducedMediaViewHelper.rst
% ------------------------------------------------------------------------------

\begin{frame}[fragile]
	\frametitle{Extbase \& Fluid}
	\framesubtitle{Media-ViewHelper (1)}

	% decrease font size for code listing
	\lstset{basicstyle=\tiny\ttfamily}

	\begin{itemize}

		\item Um Medien komfortabel im Frontend rendern zu können (z.B. Video, Audio,
			registrierte Renderer), wurde ein MediaViewHelper zugefügt

		\item Zuerst versucht der ViewHelper den Renderer aufzurufen;
			schlägt dies fehl, wird ein Image-Tag gerendert

		\item Beispiel:

			\begin{lstlisting}
				<code title="Image Object">
				  <f:media file="{file}" width="400" height="375" />
				</code>

				<output>
				  <img alt="alt set in image record" src="fileadmin/_processed_/323223424.png"
				    width="396" height="375" />
				</output>
			\end{lstlisting}

	\end{itemize}

\end{frame}

% ------------------------------------------------------------------------------
% LTXE-SLIDE-START
% LTXE-SLIDE-UID:		73fe507d-772b6011-7457429b-4039dd5c
% LTXE-SLIDE-TITLE:		Introduced MediaViewHelper (2)
% LTXE-SLIDE-REFERENCE:	Feature-66366-IntroducedMediaViewHelper.rst
% ------------------------------------------------------------------------------

\begin{frame}[fragile]
	\frametitle{Extbase \& Fluid}
	\framesubtitle{Media-ViewHelper (2)}

	% decrease font size for code listing
	\lstset{basicstyle=\tiny\ttfamily}

	\begin{itemize}

		\item Beispiel (Fortsetzung):

			\begin{lstlisting}
				<code title="MP4 Video Object">
				  <f:media file="{file}" width="400" height="375" />
				</code>

				<output>
				  <video width="400" height="375" controls>
				    <source src="fileadmin/user_upload/my-video.mp4" type="video/mp4">
				  </video>
				</output>

				<code title="MP4 Video Object with loop and autoplay option set">
				  <f:media file="{file}" width="400" height="375"
				    additionalConfig="{loop: '1', autoplay: '1'}" />
				</code>

				<output>
				  <video width="400" height="375" controls loop>
				    <source src="fileadmin/user_upload/my-video.mp4" type="video/mp4">
				  </video>
				</output>
			\end{lstlisting}

	\end{itemize}

\end{frame}

% ------------------------------------------------------------------------------
% LTXE-SLIDE-START
% LTXE-SLIDE-UID:		2abe5d2e-c1c7e93b-c86559da-81196fdb
% LTXE-SLIDE-TITLE:		Adopt form to support the Extbase/ Fluid MVC stack (1)
% LTXE-SLIDE-REFERENCE:	Feature-69401-AdoptFormToSupportTheExtbaseFluidMVCStack.rst
% ------------------------------------------------------------------------------

\begin{frame}[fragile]
	\frametitle{Extbase \& Fluid}
	\framesubtitle{System-Extension \texttt{form} (1)}

	\begin{itemize}

		\item Die System-Extension \texttt{form} (inkl. Daten-Model, Controller-Logig,
			Property Validation, Views und Templating) wurde so adaptiert, dass der
			Extbase/Fluid MVC Stack unterstützt wird

		\item Die Ausgabe basiert nun komplett auf Fluid und kann somit entsprechend
			angepasst werden. Pro Form-Element gibt es ein eigenes Partial, welches
			nun auch individuell über die TypoScript-Option \texttt{partialPath = ...}
			angepasst werden kann

		\item Es wurden drei neue ViewHelper implementiert:

			\begin{itemize}
				\item \texttt{AggregateSelectOptionsViewHelper} (für optgroup Tags)
				\item \texttt{SelectViewHelper} (für von optgroup Tags)
				\item \texttt{PlainMailViewHelper} (zum Rendern von Plaintext Mails)
			\end{itemize}

	\end{itemize}

\end{frame}

% ------------------------------------------------------------------------------
% LTXE-SLIDE-START
% LTXE-SLIDE-UID:		d45facb0-73989d92-ffbc7c2b-46f583ba
% LTXE-SLIDE-TITLE:		Adopt form to support the Extbase/ Fluid MVC stack (2)
% LTXE-SLIDE-REFERENCE:	Feature-69401-AdoptFormToSupportTheExtbaseFluidMVCStack.rst
% ------------------------------------------------------------------------------

\begin{frame}[fragile]
	\frametitle{Extbase \& Fluid}
	\framesubtitle{System-Extension \texttt{form} (2)}

	\begin{itemize}

		\item Außerdem gibt drei Views:

			\begin{itemize}
				\item \texttt{show} (das Formular selbst)
				\item \texttt{confirmation} (die Bestätigungsseite)
				\item \texttt{postProcessor}/\texttt{mail} (die Email)
			\end{itemize}

		\item Die Template-Pfade und Sichtbarkeiten der Felder können für jeden
			View individuell angepasst werden

	\end{itemize}

\end{frame}

% ------------------------------------------------------------------------------
% LTXE-SLIDE-START
% LTXE-SLIDE-UID:		cd99ea8c-e6234f0f-caf559ed-c3f825c0
% LTXE-SLIDE-TITLE:		Add annotation for CLI only commands
% LTXE-SLIDE-REFERENCE:	Feature-68746-AddAnnotationForCLIOnlyCommands.rst
% ------------------------------------------------------------------------------

\begin{frame}[fragile]
	\frametitle{Extbase \& Fluid}
	\framesubtitle{Annotation \texttt{@cli}}

	\begin{itemize}

		\item Eine neue Annotation \texttt{@cli} wurde eingeführt:\newline
			wird diese beim CommandController verwendet, so kann dieser nur auf der
			Kommandozeile, aber nicht im Scheduler verwendet werden

	\end{itemize}

\end{frame}

% ------------------------------------------------------------------------------