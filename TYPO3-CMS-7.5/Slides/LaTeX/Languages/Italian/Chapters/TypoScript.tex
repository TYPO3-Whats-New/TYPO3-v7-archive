% ------------------------------------------------------------------------------
% TYPO3 CMS 7.5 - What's New - Chapter "TypoScript" (Italian Version)
%
% @author	Michael Schams <schams.net>
% @license	Creative Commons BY-NC-SA 3.0
% @link		http://typo3.org/download/release-notes/whats-new/
% @language	Italian
% ------------------------------------------------------------------------------
% LTXE-CHAPTER-UID:		f607fb32-7f7b68d8-68d0e69a-16738006
% LTXE-CHAPTER-NAME:	TypoScript
% ------------------------------------------------------------------------------

\section{TSconfig \& TypoScript}
\begin{frame}[fragile]
	\frametitle{TSconfig \& TypoScript}

	\begin{center}\huge{Capitolo 2:}\end{center}
	\begin{center}\huge{\color{typo3darkgrey}\textbf{TSconfig \& TypoScript}}\end{center}

\end{frame}

% ------------------------------------------------------------------------------
% LTXE-SLIDE-START
% LTXE-SLIDE-UID:		0fb1df45-1d720827-9d0c1b72-54558365
% LTXE-SLIDE-ORIGIN:	195a3fa3-207484a7-94e7ce76-c821394c English
% LTXE-SLIDE-ORIGIN:	c76757e2-fc02b94b-166a8168-d68774c9 German
% LTXE-SLIDE-TITLE:		Feature #16525: Add conditions to INCLUDE_TYPOSCRIPT
% LTXE-SLIDE-REFERENCE:	Feature-16525-AddConditionsToINCLUDE_TYPOSCRIPT.rst
% ------------------------------------------------------------------------------
\begin{frame}[fragile]
	\frametitle{TSconfig \& TypoScript}
	\framesubtitle{Condizioni per l'inclusione di TypoScript}

	% decrease font size for code listing
	\lstset{basicstyle=\tiny\ttfamily}

	\begin{itemize}

		\item \texttt{INCLUDE\_TYPOSCRIPT} ha una "condizione" extra (opzionale), con la quale
			includere i file/directory solamente se la condizione è soddisfatta.

		\begin{lstlisting}
			// include il file TypoScript solamente con utenti loggati:
			<INCLUDE_TYPOSCRIPT: source="FILE:EXT:my_extension/Configuration/TypoScript/feuser.ts"
			  condition="[loginUser = *]">

			// include il file TypoScript in base al contesto dell'applicazione:
			<INCLUDE_TYPOSCRIPT: source="FILE:EXT:my_extension/Configuration/TypoScript/staging.ts"
			  condition="applicationContext = /^Production\\/Staging\\/Server\\d+$/">
		\end{lstlisting}

	\end{itemize}

\end{frame}

% ------------------------------------------------------------------------------
% LTXE-SLIDE-START
% LTXE-SLIDE-UID:		664b10f6-ae21a1ed-403d520b-af94c926
% LTXE-SLIDE-ORIGIN:	4003d374-92104168-0eb97927-726037fa English
% LTXE-SLIDE-ORIGIN:	1bde9e8a-82352f02-c5c9f3db-b194cfb1 German
% LTXE-SLIDE-TITLE:		Feature #28243: Introduce TCA option to disable age display of dates per field
% LTXE-SLIDE-REFERENCE:	Feature-28243-IntroduceTcaOptionToDisableAgeDisplay.rst
% ------------------------------------------------------------------------------
\begin{frame}[fragile]
	\frametitle{TSconfig \& TypoScript}
	\framesubtitle{TCA-Option: Visualizzazione della scadenza di data}

	% decrease font size for code listing
	\lstset{basicstyle=\tiny\ttfamily}

	\begin{itemize}

		\item L'opzione TCA \texttt{disableAgeDisplay} disabilita la visualizzazione della scadenza\newline
			\small
				(per esempio: \texttt{"2015-08-30 (-27 days)"})
			\normalsize

			\begin{lstlisting}
				$GLOBALS['TCA']['tt_content']['columns']['date']['config']['disableAgeDisplay'] = true;
			\end{lstlisting}

		\item Come requisito, il \texttt{type} del campo deve essere \texttt{input}
			e \texttt{eval} deve essere impostato a \texttt{date}

	\end{itemize}

\end{frame}

% ------------------------------------------------------------------------------
% LTXE-SLIDE-START
% LTXE-SLIDE-UID:		b41f0486-aabd8aad-221998a0-f75fd954
% LTXE-SLIDE-ORIGIN:	68c8a68b-f7e50bf3-dc67e0e5-d3a33fd8 English
% LTXE-SLIDE-ORIGIN:	c3aa0981-411fcabb-bd9dca10-16f42898 German
% LTXE-SLIDE-TITLE:		Feature #57632: Include inline language label files with TypoScript (1)
% LTXE-SLIDE-REFERENCE:	Feature-57632-AddInlineLanguageLabelFilesWithTypoScript.rst
% ------------------------------------------------------------------------------
\begin{frame}[fragile]
	\frametitle{TSconfig \& TypoScript}
	\framesubtitle{Inline Language Label Files con TypoScript (1)}

	% decrease font size for code listing
	\lstset{basicstyle=\tiny\ttfamily}

	\begin{itemize}

		\item I file di lingua XLF possono essere letti e convertiti in un array inline

		\item Questo permette ad esempio l'accesso alle label in lingua da JavaScript

		\item I seguenti tre parametri opzionali sono supportati:

			\begin{itemize}
				\item \texttt{selectionPrefix}:\newline
					solo le chiavi delle label che iniziano con questo prefisso sono incluse
				\item \texttt{stripFromSelectionName}:\newline
					stringa che sarà rimossa dalle chiavi delle label incluse
				\item \texttt{errorMode}:\newline
					modalità di errore se il file non verrà trovato:\newline
					0: scrive in syslog (default), 1: ignora, 3: crea eccezione
			\end{itemize}

	\end{itemize}

\end{frame}

% ------------------------------------------------------------------------------
% LTXE-SLIDE-START
% LTXE-SLIDE-UID:		0ff33c8e-be78baae-f9456a26-ec356842
% LTXE-SLIDE-ORIGIN:	bb800987-b0e74193-c6ec6370-3b6a062d English
% LTXE-SLIDE-ORIGIN:	289816f4-411fcabb-0981cabb-bd9dca10 German
% LTXE-SLIDE-TITLE:		Feature #57632: Include inline language label files with TypoScript (2)
% LTXE-SLIDE-REFERENCE:	Feature-57632-AddInlineLanguageLabelFilesWithTypoScript.rst
% ------------------------------------------------------------------------------
\begin{frame}[fragile]
	\frametitle{TSconfig \& TypoScript}
	\framesubtitle{Inline Language Label Files con TypoScript (2)}

	% decrease font size for code listing
	\lstset{basicstyle=\tiny\ttfamily}

	\begin{itemize}

		\item Esempio:

			\begin{lstlisting}
				page = PAGE
				page.inlineLanguageLabelFiles {
				  someLabels = EXT:myExt/Resources/Private/Language/locallang.xlf
				  someLabels.selectionPrefix = idPrefix
				  someLabels.stripFromSelectionName = strip_me
				  someLabels.errorMode = 2
				}
			\end{lstlisting}

		\item Output:

			\begin{lstlisting}
				<script type="text/javascript">
				/*<![CDATA[*/
				  var TYPO3 = TYPO3 || {};
				  TYPO3.lang = {"firstLabel":[{"source":"first Label","target":"erstes Label"}],
				  "secondLabel":[{"source":"second Label","target":"zweites Label"}]};
				/*]]>*/
				</script>
			\end{lstlisting}

	\end{itemize}

\end{frame}

% ------------------------------------------------------------------------------
% LTXE-SLIDE-START
% LTXE-SLIDE-UID:		c54b49ac-4ce356c9-e0cf813d-6fce1f44
% LTXE-SLIDE-ORIGIN:	bac61b19-18a4db7b-af13de6c-5dbed67b English
% LTXE-SLIDE-ORIGIN:	f14c90e2-bf35bf40-e38d74a7-f4a2e973 German
% LTXE-SLIDE-TITLE:		Feature #59144: Previewing workspace records using Page TSconfig
% LTXE-SLIDE-REFERENCE:	Feature-59144-PageTSconfigWorkspacePreview.rst
% ------------------------------------------------------------------------------
\begin{frame}[fragile]
	\frametitle{TSconfig \& TypoScript}
	\framesubtitle{Anteprima del Workspace da TSconfig}

	% decrease font size for code listing
	\lstset{basicstyle=\tiny\ttfamily}

	\begin{itemize}

		\item TYPO3 CMS, di default, genera link di anteprima solo per le tabelle \texttt{tt\_content}, \texttt{pages} e
			\texttt{pages\_language\_overlay}

		\item Questo può ora essere configurato usando PageTSconfig:

			\begin{lstlisting}
				# use page 123 for previewing workspaces records (in general)
				options.workspaces.previewPageId = 123

				# use the pid field of each record for previewing (in general)
				options.workspaces.previewPageId = field:pid

				# use page 123 for previewing workspaces records (for table tx_myext_table)
				options.workspaces.previewPageId.tx_myext_table = 123

				# use the pid field of each record for previewing (or table tx_myext_table)
				options.workspaces.previewPageId.tx_myext_table = field:pid
			\end{lstlisting}

	\end{itemize}

\end{frame}

% ------------------------------------------------------------------------------
% LTXE-SLIDE-START
% LTXE-SLIDE-UID:		368bf9d0-815c64de-21a256ef-9b338658
% LTXE-SLIDE-ORIGIN:	c4b7358c-07059082-861a3b2d-929e1cc6 English
% LTXE-SLIDE-ORIGIN:	faeca558-5a675884-a1b77fae-36e316eb German
% LTXE-SLIDE-TITLE:		Feature #59591: Image quality definable per sourceCollection
% LTXE-SLIDE-REFERENCE:	Feature-59591-ImageQualityDefinablePerSourceCollection.rst
% ------------------------------------------------------------------------------
\begin{frame}[fragile]
	\frametitle{TSconfig \& TypoScript}
	\framesubtitle{Qualità dell'immagine in \texttt{sourceCollection}}

	% decrease font size for code listing
	\lstset{basicstyle=\tiny\ttfamily}

	\begin{itemize}

		\item La qualità dell'immagine di ogni voce in \texttt{sourceCollection} può essere ora configurata

		\item Questa impostazione ha la precedenza sulla configurazione in Install Tool\newline
			(registrata nel file \texttt{LocalConfiguration.php})

		\item Esempio:

			\begin{lstlisting}
				# per immagini retina piccole
				tt_content.image.20.1.sourceCollection.smallRetina.quality = 80

				# per immagini retina grandi
				tt_content.image.20.1.sourceCollection.largeRetina.quality = 65
			\end{lstlisting}

	\end{itemize}

\end{frame}

% ------------------------------------------------------------------------------
% LTXE-SLIDE-START
% LTXE-SLIDE-UID:		6e4eff97-738eaf59-2f41dcf7-b87753d2
% LTXE-SLIDE-ORIGIN:	ad691a22-7ab387b6-6e00c0aa-e24efd91 English
% LTXE-SLIDE-ORIGIN:	0073160e-6c95f70d-fcaf008f-3fd84819 German
% LTXE-SLIDE-TITLE:		Feature #67880: Added count to split
% LTXE-SLIDE-REFERENCE:	Feature-67880-AddedCountToSplit.rst
% ------------------------------------------------------------------------------
% Note: feature #67880 has been introduced in TYPO3 CMS 7.4 already
% and we mentioned it in the last version of our What's New Slides.
% It is not clear, why this feature is included in the ChangeLog of
% 7.5 again (with a slightly different title), but we should have it
% in our Slides, too.

\begin{frame}[fragile]
	\frametitle{TSconfig \& TypoScript}
	\framesubtitle{Contatore di elementi in una lista}

	% decrease font size for code listing
	\lstset{basicstyle=\tiny\ttfamily}

	\begin{itemize}

		\item Una nuova proprietà \texttt{returnCount} è stata aggiunta alla proprietà stdWrap \texttt{split}

		\item Questa permette di contare il numero di elementi di una lista

		\item Il codice seguente restituisce \texttt{9} per esempio:

			\begin{lstlisting}
				1 = TEXT
				1 {
				  value = x,y,z,1,2,3,a,b,c
				  split.token = ,
				  split.returnCount = 1
				}
			\end{lstlisting}

	\end{itemize}

\end{frame}
% ------------------------------------------------------------------------------
% LTXE-SLIDE-START
% LTXE-SLIDE-UID:		d872948d-3173c88c-4ea51f3b-beae63f7
% LTXE-SLIDE-ORIGIN:	1a1b3a3f-a12510ef-003db5f4-2fc4f613 English
% LTXE-SLIDE-ORIGIN:	da2aa521-48abf012-e8b4b95f-0e3f3556 German
% LTXE-SLIDE-TITLE:		Feature #69602: Simplify handling of backend layouts in frontend (1)
% LTXE-SLIDE-REFERENCE:	Feature-69602-SimplifyHandlingOfBackendLayoutsInFrontend.rst
% ------------------------------------------------------------------------------

\begin{frame}[fragile]
	\frametitle{TSconfig \& TypoScript}
	\framesubtitle{Interventi su Backend Layout (1)}

	% decrease font size for code listing
	\lstset{basicstyle=\tiny\ttfamily}

	\begin{itemize}

		\item Gli interventi su backend layout sono stati semplificati per il frontend

		\item La nuova opzione \texttt{pagelayout} può essere usata in TypoScript

		\item Esempio:

			\begin{lstlisting}
				page.10 = FLUIDTEMPLATE
				page.10 {
				  file.stdWrap.cObject = CASE
				  file.stdWrap.cObject {
				    key.data = pagelayout
				    default = TEXT
				    default.value = EXT:sitepackage/Resources/Private/Templates/Home.html
				    3 = TEXT
				    3.value = EXT:sitepackage/Resources/Private/Templates/1-col.html
				    4 = TEXT
				    4.value = EXT:sitepackage/Resources/Private/Templates/2-col.html
				  }
				}
			\end{lstlisting}

			\smaller
				(continua sulla prossima pagina)
			\normalsize

	\end{itemize}

\end{frame}

% ------------------------------------------------------------------------------
% LTXE-SLIDE-START
% LTXE-SLIDE-UID:		c4d4bff6-373a9a77-63a55796-8c1f185d
% LTXE-SLIDE-ORIGIN:	4054dec5-eb1750c8-20a10e74-0935b709 English
% LTXE-SLIDE-ORIGIN:	b95fe8b4-a521da2a-f01248ab-35560e3f German
% LTXE-SLIDE-TITLE:		Feature #69602: Simplify handling of backend layouts in frontend (2)
% LTXE-SLIDE-REFERENCE:	Feature-69602-SimplifyHandlingOfBackendLayoutsInFrontend.rst
% ------------------------------------------------------------------------------

\begin{frame}[fragile]
	\frametitle{TSconfig \& TypoScript}
	\framesubtitle{Interventi su Backend Layout (2)}

	% decrease font size for code listing
	\lstset{basicstyle=\tiny\ttfamily}

	\begin{itemize}

		\item ...dove \texttt{key.data = pagelayout} sostituisce il codice seguente:

			\begin{lstlisting}
				field = backend_layout
				ifEmpty.data = levelfield:-2,backend_layout_next_level,slide
				ifEmpty.ifEmpty = default
			\end{lstlisting}

	\end{itemize}

\end{frame}

% ------------------------------------------------------------------------------
% LTXE-SLIDE-START
% LTXE-SLIDE-UID:		9510e232-206be1b8-cd485d57-5f468e1f
% LTXE-SLIDE-ORIGIN:	07353cea-dac9032f-52a37a42-a261e38a English
% LTXE-SLIDE-ORIGIN:	6f1f7bb6-b50c019e-29c9be36-7d30c87d German
% LTXE-SLIDE-TITLE:		Feature #68756: Add config "base" to stdWrap
% LTXE-SLIDE-REFERENCE:	Feature-68756-AddConfigBaseToStdWrap.rst
% ------------------------------------------------------------------------------
\begin{frame}[fragile]
	\frametitle{TSconfig \& TypoScript}
	\framesubtitle{Varie}

	\begin{itemize}

		\item La proprietà stdWrap \texttt{bytes} è stata inserita in TYPO3 CMS 7.4

		\item La possibilità di impostare la \texttt{base} è stata aggiunta in TYPO3 CMS 7.5,
			e permette di definire se usare una base 1000 o 1024 per il calcolo

			\begin{lstlisting}
				bytes.labels = " | K| M| G"
				bytes.base = 1000
			\end{lstlisting}

	\end{itemize}

\end{frame}

% ------------------------------------------------------------------------------
