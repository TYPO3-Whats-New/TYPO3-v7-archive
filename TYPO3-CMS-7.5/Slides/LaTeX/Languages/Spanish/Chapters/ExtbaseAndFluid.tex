% ------------------------------------------------------------------------------
% TYPO3 CMS 7.5 - What's New - Chapter "Extbase & Fluid" (Spanish Version)
%
% @author	Michael Schams <schams.net>
% @license	Creative Commons BY-NC-SA 3.0
% @link		http://typo3.org/download/release-notes/whats-new/
% @language	Spanish
% ------------------------------------------------------------------------------
% LTXE-CHAPTER-UID:		c9b2e403-e06bb691-ce202054-149c5089
% LTXE-CHAPTER-NAME:	Extbase & Fluid
% ------------------------------------------------------------------------------

\section{Extbase \& Fluid}
\begin{frame}[fragile]
	\frametitle{Extbase \& Fluid}

	\begin{center}\huge{Capítulo 4:}\end{center}
	\begin{center}\huge{\color{typo3darkgrey}\textbf{Extbase \& Fluid}}\end{center}

\end{frame}

% ------------------------------------------------------------------------------
% LTXE-SLIDE-START
% LTXE-SLIDE-UID:		89b21094-8952cee5-1a1b86d3-e281f800
% LTXE-SLIDE-ORIGIN:	8bf4a04e-9453b098-84fa7616-1df8ea8f English
% LTXE-SLIDE-ORIGIN:	7528691a-d33af3f4-edab28ae-ba7c8016 German
% LTXE-SLIDE-TITLE:		Severity-filtering for FlashMessageQueue
% LTXE-SLIDE-REFERENCE:	Feature-7098-SeverityFilteringForFlashMessageQueue.rst
% ------------------------------------------------------------------------------

\begin{frame}[fragile]
	\frametitle{Extbase \& Fluid}
	\framesubtitle{Filtrado de Severidad para FlashMessageQueue}

	\begin{itemize}

		\item En TYPO3 CMS < 7.5, \underline{todos} los mensajes de la FlashMessageQueue pueden ser
			obtenidos y/o eliminados sólo

		\item En TYPO3 CMS >= 7.5, esto puede hacerse para una severidad específica:

			\begin{lstlisting}
				FlashMessageQueue::getAllMessages($severity);
				FlashMessageQueue::getAllMessagesAndFlush($severity);
				FlashMessageQueue::removeAllFlashMessagesFromSession($severity);
				FlashMessageQueue::clear($severity);
			\end{lstlisting}

	\end{itemize}

\end{frame}

% ------------------------------------------------------------------------------
% LTXE-SLIDE-START
% LTXE-SLIDE-UID:		906ed5fc-1b3aafda-b6141780-8cc37d54
% LTXE-SLIDE-ORIGIN:	1728dca6-0c88a636-5dc7cc00-dfa3266f English
% LTXE-SLIDE-ORIGIN:	484e03e4-cb7828c2-8bf31f48-cabf004a German
% LTXE-SLIDE-TITLE:		Query support for BETWEEN added
% LTXE-SLIDE-REFERENCE:	Feature-47812-QuerySupportForBETWEENAdded.rst
% ------------------------------------------------------------------------------

\begin{frame}[fragile]
	\frametitle{Extbase \& Fluid}
	\framesubtitle{Soporte Query añadido para "\texttt{between}"}

	\begin{itemize}

		\item Se ha añadido soporte para \texttt{between} al objeto Extbase Query

		\item No hay beneficio de funcionamiento debido al hecho de que la DBMS convierte
			"\texttt{between}" internamente de todos modos: \texttt{min <= expr AND expr <= max}

		\item La nueva característica Extbase replica el comportamiento de la DBMS construyendo una
			condición lógica \texttt{AND}, de manera que ésta funcione a lo largo de todas las DBMS

			\begin{lstlisting}
				$query->matching(
				  $query->between('uid', 3, 5)
				);
			\end{lstlisting}

	\end{itemize}

\end{frame}

% ------------------------------------------------------------------------------
% LTXE-SLIDE-START
% LTXE-SLIDE-UID:		286d85d4-73672a98-002b3287-671f8312
% LTXE-SLIDE-ORIGIN:	0f2d87b1-c907b8a6-44612e51-e84e6490 English
% LTXE-SLIDE-ORIGIN:	12b78f4c-7291d788-ea865da7-e6501a2e German
% LTXE-SLIDE-TITLE:		Added support for multiple FlashMessage queues
% LTXE-SLIDE-REFERENCE:	Feature-64726-UsingArbitraryFlashmessageQueues.rst
% ------------------------------------------------------------------------------

\begin{frame}[fragile]
	\frametitle{Extbase \& Fluid}
	\framesubtitle{Colas de FlashMessage Múltiples}

	\begin{itemize}

		\item Ahora es posible implementar múltiples FlashMessageQueues:

			\begin{lstlisting}
				$queueIdentifier = 'myQueue';
				$this->controllerContext->getFlashMessageQueue($queueIdentifier);
			\end{lstlisting}

		\item El acceso usando Fluid funciona como sigue:

			\begin{lstlisting}
				<f:flashMessages queueIdentifier="myQueue" ></f:flashMessages>
			\end{lstlisting}

	\end{itemize}

\end{frame}

% ------------------------------------------------------------------------------
% LTXE-SLIDE-START
% LTXE-SLIDE-UID:		05c0c442-230b693b-ec0bd4fb-5d225461
% LTXE-SLIDE-ORIGIN:	6915482b-a7f5bd30-f04b3dc6-d94a5f2a English
% LTXE-SLIDE-ORIGIN:	2f1002c1-6bde766c-3b1ad500-ab9ccf0c German
% LTXE-SLIDE-TITLE:		Introduced MediaViewHelper (1)
% LTXE-SLIDE-REFERENCE:	Feature-66366-IntroducedMediaViewHelper.rst
% ------------------------------------------------------------------------------

\begin{frame}[fragile]
	\frametitle{Extbase \& Fluid}
	\framesubtitle{ViewHelper Media (1)}

	% decrease font size for code listing
	\lstset{basicstyle=\tiny\ttfamily}

	\begin{itemize}

		\item Para renderizar fácilmente vídeo, audio y todos los otros tipos de fichero con
			una clase registrada Renderer en el frontend, el \texttt{MediaViewHelper}
			ha sido implementado

		\item \texttt{MediaViewHelper} primero chequea si hay un Renderer presente para
			el fichero dado - si no, recurre y renderiza un tag image

		\item Ejemplos:

			\begin{lstlisting}
				<code title="Image Object">
				  <f:media file="{file}" width="400" height="375" ></f:media>
				</code>

				<output>
				  <img alt="alt set in image record" src="fileadmin/_processed_/323223424.png"
				    width="396" height="375" />
				</output>
			\end{lstlisting}

	\end{itemize}

\end{frame}

% ------------------------------------------------------------------------------
% LTXE-SLIDE-START
% LTXE-SLIDE-UID:		e4152e39-4a905697-b5bb0422-5a231886
% LTXE-SLIDE-ORIGIN:	8bd8314d-631b7ae9-3102867f-804046fa English
% LTXE-SLIDE-ORIGIN:	73fe507d-772b6011-7457429b-4039dd5c German
% LTXE-SLIDE-TITLE:		Introduced MediaViewHelper (2)
% LTXE-SLIDE-REFERENCE:	Feature-66366-IntroducedMediaViewHelper.rst
% ------------------------------------------------------------------------------

\begin{frame}[fragile]
	\frametitle{Extbase \& Fluid}
	\framesubtitle{ViewHelper Media (2)}

	% decrease font size for code listing
	\lstset{basicstyle=\tiny\ttfamily}

	\begin{itemize}

		\item Ejemplos (continuación):

			\begin{lstlisting}
				<code title="MP4 Video Object">
				  <f:media file="{file}" width="400" height="375" ></f:media>
				</code>

				<output>
				  <video width="400" height="375" controls>
				    <source src="fileadmin/user_upload/my-video.mp4" type="video/mp4">
				  </video>
				</output>

				<code title="MP4 Video Object with loop and autoplay option set">
				  <f:media file="{file}" width="400" height="375"
				    additionalConfig="{loop: '1', autoplay: '1'}" ></f:media>
				</code>

				<output>
				  <video width="400" height="375" controls loop>
				    <source src="fileadmin/user_upload/my-video.mp4" type="video/mp4">
				  </video>
				</output>
			\end{lstlisting}

	\end{itemize}

\end{frame}

% ------------------------------------------------------------------------------
% LTXE-SLIDE-START
% LTXE-SLIDE-UID:		72f66378-04c9404d-df0d385f-fa784077
% LTXE-SLIDE-ORIGIN:	5d858cf5-81f4cf18-1b13e093-6a00c98a English
% LTXE-SLIDE-ORIGIN:	2abe5d2e-c1c7e93b-c86559da-81196fdb German
% LTXE-SLIDE-TITLE:		Adopt form to support the Extbase/ Fluid MVC stack (1)
% LTXE-SLIDE-REFERENCE:	Feature-69401-AdoptFormToSupportTheExtbaseFluidMVCStack.rst
% ------------------------------------------------------------------------------

\begin{frame}[fragile]
	\frametitle{Extbase \& Fluid}
	\framesubtitle{Extensión del Sistema \texttt{form} (1)}

	\begin{itemize}

		\item Extensión del sistema \texttt{form} (incluyendo el modelo de datos personalizado, lógica de
			controlador, validación de propiedades, vistas y templating) ha sido adaptado para soportar
			el stack MVC Extbase/Fluid

		\item Esto permite mejor personalización y control del comportamiento generado y
			marcado simplemente modificando los templates Fluid o utilizando una lógica propia de view helper personalizada

		\item Cada elemento del form usa su propio Partial, que puede ser también configurado a través de
			la opción TypoScript \texttt{partialPath = ...}

	\end{itemize}

\end{frame}

% ------------------------------------------------------------------------------
% LTXE-SLIDE-START
% LTXE-SLIDE-UID:		eed02da6-11dd1d77-5c7c331f-a2b83ce2
% LTXE-SLIDE-ORIGIN:	20c7837c-5b841f21-fb3ead22-265aebe4 English
% LTXE-SLIDE-ORIGIN:	d45facb0-73989d92-ffbc7c2b-46f583ba German
% LTXE-SLIDE-TITLE:		Adopt form to support the Extbase/ Fluid MVC stack (2)
% LTXE-SLIDE-REFERENCE:	Feature-69401-AdoptFormToSupportTheExtbaseFluidMVCStack.rst
% ------------------------------------------------------------------------------

\begin{frame}[fragile]
	\frametitle{Extbase \& Fluid}
	\framesubtitle{Extensión del Sistema \texttt{form} (2)}

	\begin{itemize}

		\item Los siguientes tres nuevos ViewHelpers existen:

			\begin{itemize}
				\item \texttt{AggregateSelectOptionsViewHelper} (para tags optgroup)
				\item \texttt{SelectViewHelper} (para tags optgroup)
				\item \texttt{PlainMailViewHelper} (para visualizar mails de texto plano)
			\end{itemize}

		\item Además, hay tres Vistas:

			\begin{itemize}
				\item \texttt{show} (el form en sí mismo)
				\item \texttt{confirmation} (la página de confirmación)
				\item \texttt{postProcessor}/\texttt{mail} (el email)
			\end{itemize}

		\item Rutas de template y visibilidad de los campos puede ser personalizada
			para cada Vista individualmente

	\end{itemize}

\end{frame}

% ------------------------------------------------------------------------------
% LTXE-SLIDE-START
% LTXE-SLIDE-UID:		5eb31bfe-4630760e-30c17437-475ffcb2
% LTXE-SLIDE-ORIGIN:	20b68c91-1b501686-f97f7b55-0822d9d1 English
% LTXE-SLIDE-ORIGIN:	cd99ea8c-e6234f0f-caf559ed-c3f825c0 German
% LTXE-SLIDE-TITLE:		Add annotation for CLI only commands
% LTXE-SLIDE-REFERENCE:	Feature-68746-AddAnnotationForCLIOnlyCommands.rst
% ------------------------------------------------------------------------------

\begin{frame}[fragile]
	\frametitle{Extbase \& Fluid}
	\framesubtitle{Anotación \texttt{@cli}}

	\begin{itemize}

		\item Al usar la nueva anotación \texttt{@cli}, comandos en un Extbase
			CommandController pueden ser marcados como comandos CLI solamente

		\item Estos comandos son excluidos de la selección de comandos del programador

		\item Típicos casos de uso son comandos tales como \texttt{extbase:help:help} por ejemplo

	\end{itemize}

\end{frame}

% ------------------------------------------------------------------------------
