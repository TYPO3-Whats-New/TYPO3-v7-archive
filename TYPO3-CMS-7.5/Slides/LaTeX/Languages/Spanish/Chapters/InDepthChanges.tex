% ------------------------------------------------------------------------------
% TYPO3 CMS 7.5 - What's New - Chapter "Cambios en Profundidad" (Spanish Version)
%
% @author	Michael Schams <schams.net>
% @license	Creative Commons BY-NC-SA 3.0
% @link		http://typo3.org/download/release-notes/whats-new/
% @language	Spanish
% ------------------------------------------------------------------------------
% LTXE-CHAPTER-UID:		b8f1c138-2fb2f5ce-15d1ffc0-e80171e2
% LTXE-CHAPTER-NAME:	Cambios en Profundidad
% ------------------------------------------------------------------------------

\section{Cambios en Profundidad}
\begin{frame}[fragile]
	\frametitle{Cambios en Profundidad}

	\begin{center}\huge{Capítulo 3:}\end{center}
	\begin{center}\huge{\color{typo3darkgrey}\textbf{Cambios en Profundidad}}\end{center}

\end{frame}

% ------------------------------------------------------------------------------
% LTXE-SLIDE-START
% LTXE-SLIDE-UID:		49ae0cf1-98aefd14-3780b2e0-86095cd3
% LTXE-SLIDE-ORIGIN:	16e557e2-fa6a5035-5b28d66e-27c0c626 English
% LTXE-SLIDE-ORIGIN:	03be70bc-886235ca-10f0cef5-6b33fe60 German
% LTXE-SLIDE-TITLE:		Fluid-based Content Elements Introduced (1)
% LTXE-SLIDE-REFERENCE:	Feature-38732-Fluid-basedContentElementsIntroduced.rst
% ------------------------------------------------------------------------------

\begin{frame}[fragile]
	\frametitle{Cambios en Profundidad}
	\framesubtitle{Elementos de Contenido basados en Fluid (1)}

	\begin{itemize}

		\item Ha sido implementada una nueva extensión del sistema \textbf{"Elementos de Contenido basados en Fluid"}

		\item Se usan templates Fluid para el renderizado de elementos de contenido en lugar de TypoScript

		\item Podría ser una alternativa a \textit{CSS Styled Content} en un punto en el futuro

		\item Incluye los siguientes templates estáticos para usar esta característica:

			\begin{itemize}
				\item Elementos de Contenido (\texttt{fluid\_styled\_content})
				\item Elementos de Contenido CSS (opcional) (\texttt{fluid\_styled\_content})
			\end{itemize}

	\end{itemize}

\end{frame}



% ------------------------------------------------------------------------------
% LTXE-SLIDE-START
% LTXE-SLIDE-UID:		0ccf3ab2-4c318241-b9713c4a-39d3e685
% LTXE-SLIDE-ORIGIN:	d8383008-e3bbb7e2-91720bea-3b93db62 English
% LTXE-SLIDE-ORIGIN:	bb7124ec-4658414b-757ca8ca-f0addee3 German
% LTXE-SLIDE-TITLE:		Fluid-based Content Elements Introduced (2)
% LTXE-SLIDE-REFERENCE:	Feature-38732-Fluid-basedContentElementsIntroduced.rst
% ------------------------------------------------------------------------------

\begin{frame}[fragile]
	\frametitle{Cambios en Profundidad}
	\framesubtitle{Elementos de Contenido basados en Fluid (2)}

	% decrease font size for code listing
	\lstset{basicstyle=\tiny\ttfamily}

	\begin{itemize}

		\item Además, el siguiente template PageTSconfig tiene que añadirse a las propiedades de la página:\newline
			\small
				\texttt{Elementos de Contenido basados en Fluid (fluid\_styled\_content)}
			\normalsize

		\item Sobreescribe templates por defecto añadiendo rutas propias en la configuración TypoScript:

			\begin{lstlisting}
				lib.fluidContent.templateRootPaths.50 = EXT:site_example/Resources/Private/Templates/
				lib.fluidContent.partialRootPaths.50 = EXT:site_example/Resources/Private/Partials/
				lib.fluidContent.layoutRootPaths.50 = EXT:site_example/Resources/Private/Layouts/
			\end{lstlisting}

	\end{itemize}

\end{frame}

% ------------------------------------------------------------------------------
% LTXE-SLIDE-START
% LTXE-SLIDE-UID:		acfed67f-25afc3a4-7f2a164b-89368cb0
% LTXE-SLIDE-ORIGIN:	d25f0c05-31c928e4-e7439714-d87f883b English
% LTXE-SLIDE-ORIGIN:	04cd65d3-e013813d-d2eac450-7d3a79a8 German
% LTXE-SLIDE-TITLE:		Fluid-based Content Elements Introduced (3)
% LTXE-SLIDE-REFERENCE:	Feature-38732-Fluid-basedContentElementsIntroduced.rst
% LTXE-SLIDE-REFERENCE:	Important-67954-MigrateCTypesTextImageAndTextpicToTextmedia.rst
% ------------------------------------------------------------------------------

\begin{frame}[fragile]
	\frametitle{Cambios en Profundidad}
	\framesubtitle{Elementos de Contenido basados en Fluid (3)}

	\begin{itemize}

		\item Migre de \textit{CSS Styled Content} a \textit{Elementos de Contenido basados en Fluid}:

			\begin{itemize}

				\item Desinstale extensión \texttt{css\_styled\_content}

				\item Instale extensión \texttt{fluid\_styled\_content}

				\item Use el Asistente de Actualización en la Herramienta de Instalación para migrar Elementos de Contenido
					\texttt{text}, \texttt{image} y \texttt{textpic} a \texttt{textmedia}

			\end{itemize}
	\end{itemize}

	\vspace{1.4cm}

	\begingroup
		\color{red}
			\small
				\underline{Nota:} \textit{"Elementos de Contenido basados en Fluid"} está todavía en un estado temprano
				y cambios de última hora son posibles hasta TYPO3 CMS 7 LTS.
				También algunos conflictos referentes a \textit{CSS Styled Content} posiblemente todavía existan.
			\normalsize
	\endgroup

\end{frame}


% ------------------------------------------------------------------------------
% LTXE-SLIDE-START
% LTXE-SLIDE-UID:		390244a4-8854f11e-020cb44d-637226c9
% LTXE-SLIDE-ORIGIN:	89b8c8c2-3a458bcf-5d61b025-6b4fb470 English
% LTXE-SLIDE-ORIGIN:	8807e1a0-2b3fadac-ac842970-b85e5130 German
% LTXE-SLIDE-TITLE:		Add SELECTmmQuery method to DatabaseConnection
% LTXE-SLIDE-REFERENCE:	Feature-19494-AddSELECTmmQueryMethodToDatabaseConnection.rst
% ------------------------------------------------------------------------------

\begin{frame}[fragile]
	\frametitle{Cambios en Profundidad}
	\framesubtitle{Método SELECTmmQuery}

	% decrease font size for code listing
	\lstset{basicstyle=\tiny\ttfamily}

	\begin{itemize}

		\item Ha sido añadido a la clase \texttt{DatabaseConnection} un nuevo método \texttt{SELECT\_mm\_query}

		\item Extraído de \texttt{exec\_SELECT\_mm\_query} para separar la construcción y ejecución
			de consultas M:M.

		\item Esto habilita el uso de la construcción de consultas en la capa de abstracción de la base de datos

			\begin{lstlisting}
				$query = SELECT_mm_query('*', 'table1', 'table1_table2_mm', 'table2', 'AND table1.uid = 1',
				'', 'table1.title DESC');
			\end{lstlisting}

	\end{itemize}

\end{frame}

% ------------------------------------------------------------------------------
% LTXE-SLIDE-START
% LTXE-SLIDE-UID:		b6aac085-a315950c-c33b9162-b135da6c
% LTXE-SLIDE-ORIGIN:	f2c4f1a9-6cd020c0-518650d6-2a31d0de English
% LTXE-SLIDE-ORIGIN:	398fdcb2-1cc9a862-42e59a09-e4ed65ba German
% LTXE-SLIDE-TITLE:		Scheduler task to optimize database tables
% LTXE-SLIDE-REFERENCE:	Feature-25341-SchedulerTaskToOptimizeDatabaseTables.rst
% ------------------------------------------------------------------------------

\begin{frame}[fragile]
	\frametitle{Cambios en Profundidad}
	\framesubtitle{Optimizar Tablas de Base de Datos en MySQL}

	\begin{itemize}

		\item Nueva tarea del programador para ejecutar el comando MySQL \texttt{OPTIMIZE TABLE}
			en las tablas seleccionadas de la base de datos

		\item Este comando reorganiza el almacenaje físico de los datos de la tabla y los datos de
			índices asociados para reducir espacio de almacenaje y mejorar la eficiencia de E/S

		\item Se soportan los siguientes tipos de tablas:\newline
			\texttt{MyISAM}, \texttt{InnoDB} y \texttt{ARCHIVE}

		\item Usar esta tarea con DBAL y otros DBMS \underline{no} está soportado
			debido al hecho de que los comandos usados son específicos de MySQL

	\end{itemize}

	% Note to translators: if this does not fit on one slide, you could try to change
	% \small to \smaller or shorten the sentences below or the bullet points above.

	\begingroup
		\color{red}
			\small
				\underline{Nota:} optimizar tablas es un proceso intensivo de E/S.
				También en MySQL < 5.6.17 el proceso bloquea las tablas mientras se ejecuta,
				lo que puede impactar en la página web.
			\normalsize
	\endgroup

\end{frame}

% ------------------------------------------------------------------------------
% LTXE-SLIDE-START
% LTXE-SLIDE-UID:		b0c98d4e-fe04927e-dc65377b-ea327f3a
% LTXE-SLIDE-ORIGIN:	f0da3ed6-ab7b2518-402338db-a1842865 English
% LTXE-SLIDE-ORIGIN:	f9cbeb69-d08efdaf-a8c00d48-353f667e German
% LTXE-SLIDE-TITLE:		Improved handling of online media (1)
% LTXE-SLIDE-REFERENCE:	Feature-61799-ImprovedHandlingOfOnlineMedia.rst
% ------------------------------------------------------------------------------

\begin{frame}[fragile]
	\frametitle{Cambios en Profundidad}
	\framesubtitle{Manejo de Medios Online (1)}

	\begin{itemize}

		\item Ahora se soportan medios externos (medios online)

		\item Como ejemplos, se ha implementado el soporte de vídeos Youtube y Vimeo en el núcleo

		\item Pueden añadirse recursos como URLs usando el elemento de contenido \textbf{"Text \& Media"}
			por ejemplo

		\item La clase auxiliar apropiada recoge los meta datos y suministra una imagen que será
			usada como la previsualización si está disponible

	\end{itemize}

\end{frame}

% ------------------------------------------------------------------------------
% LTXE-SLIDE-START
% LTXE-SLIDE-UID:		3540fa0f-f101523b-9e93f6a4-a921f39e
% LTXE-SLIDE-ORIGIN:	485f95a8-a9ec5940-e9aa7322-688630cb English
% LTXE-SLIDE-ORIGIN:	76e87d6a-cd1ebbf9-9a4d49ec-49cc569f German
% LTXE-SLIDE-TITLE:		Improved handling of online media (2)
% LTXE-SLIDE-REFERENCE:	Feature-61799-ImprovedHandlingOfOnlineMedia.rst
% ------------------------------------------------------------------------------

\begin{frame}[fragile]
	\frametitle{Cambios en Profundidad}
	\framesubtitle{Manejo de Medios Online (2)}

	Las siguientes sintaxis de URL son posibles:
	\vspace{0.4cm}

	\begin{columns}[T]
		\begin{column}{.52\textwidth}
			\smaller
				\tabto{0.2cm}\textbf{\underline{YouTube:}}\newline
				\tabto{0.2cm}\texttt{youtu.be/<code>}\newline
				\tabto{0.2cm}\texttt{www.youtube.com/watch?v=<code>}\newline
				\tabto{0.2cm}\texttt{www.youtube.com/v/<code>}\newline
				\tabto{0.2cm}\texttt{www.youtube-nocookie.com/v/<code>}\newline
				\tabto{0.2cm}\texttt{www.youtube.com/embed/<code>}\newline
		\end{column}
		\begin{column}{.48\textwidth}
			\vspace{-0.25cm}\smaller
				\textbf{\underline{Vimeo:}}\newline
				\texttt{vimeo.com/<code>}\newline
				\texttt{player.vimeo.com/video/<code>}\newline
		\end{column}
	\end{columns}

\end{frame}

% ------------------------------------------------------------------------------
% LTXE-SLIDE-START
% LTXE-SLIDE-UID:		e4ff5e73-5d82c177-cd5a7716-1e3ddeb9
% LTXE-SLIDE-ORIGIN:	dc79335c-d9e51543-ff5759e0-5005bff8 English
% LTXE-SLIDE-ORIGIN:	aef8746a-e1babbc9-4dd74eec-cf379dae German
% LTXE-SLIDE-TITLE:		Improved handling of online media (3)
% LTXE-SLIDE-REFERENCE:	Feature-61799-ImprovedHandlingOfOnlineMedia.rst
% ------------------------------------------------------------------------------

\begin{frame}[fragile]
	\frametitle{Cambios en Profundidad}
	\framesubtitle{Manejo de Medios Online (3)}

	% decrease font size for code listing
	\lstset{basicstyle=\tiny\ttfamily}

	\begin{itemize}

		\item Puede lograrse el acceso a los recursos usando Fluid como sigue:

		\begin{lstlisting}
			<!-- enable js api and set no-cookie support for YouTube videos -->
			<f:media file="{file}" additionalConfig="{enablejsapi:1, 'no-cookie': true}" ></f:media>

			<!-- show title and uploader for YouTube and Vimeo before video starts playing -->
			<f:media file="{file}" additionalConfig="{showinfo:1}" ></f:media>
		\end{lstlisting}

		\item Opciones de configuración personalizadas para vídeos YouTube:\newline
			\small
				\texttt{autoplay}, \texttt{controls}, \texttt{loop}, \texttt{enablejsapi}, \texttt{showinfo}, \texttt{no-cookie}
			\normalsize

		\item Opciones de configuración personalizadas para vídeos Vimeo:\newline
			\small
				\texttt{autoplay}, \texttt{loop}, \texttt{showinfo}
			\normalsize
	\end{itemize}

\end{frame}

% ------------------------------------------------------------------------------
% LTXE-SLIDE-START
% LTXE-SLIDE-UID:		44e6723f-13cc842f-37f6d105-e22942db
% LTXE-SLIDE-ORIGIN:	2b41f269-eaa80c0c-89bcd0f0-ee66248b English
% LTXE-SLIDE-ORIGIN:	9ed71aaa-a7615392-78f0ac0c-f9bd7400 German
% LTXE-SLIDE-TITLE:		Improved handling of online media (4)
% LTXE-SLIDE-REFERENCE:	Feature-61799-ImprovedHandlingOfOnlineMedia.rst
% ------------------------------------------------------------------------------

\begin{frame}[fragile]
	\frametitle{Cambios en Profundidad}
	\framesubtitle{Manejo de Medios Online (4)}

	% decrease font size for code listing
	\lstset{basicstyle=\tiny\ttfamily}

	\begin{itemize}

		\item Para registrar tu propio servicio de medios online, necesitas una
			clase \small\texttt{OnlineMediaHelper}\normalsize\space que implemente
			\small\texttt{OnlineMediaHelperInterface}\normalsize\space y una
			clase \small\texttt{FileRenderer}\normalsize\space que implemente
			\small\texttt{FileRendererInterface}\normalsize\space

			\begin{lstlisting}
				// register your own online video service (the used key is also the bind file extension name)
				$GLOBALS['TYPO3_CONF_VARS']['SYS']['OnlineMediaHelpers']['myvideo'] =
				  \MyCompany\Myextension\Helpers\MyVideoHelper::class;

				$rendererRegistry = \TYPO3\CMS\Core\Resource\Rendering\RendererRegistry::getInstance();
				$rendererRegistry->registerRendererClass(
				  \MyCompany\Myextension\Rendering\MyVideoRenderer::class
				);

				// register an custom mime-type for your videos
				$GLOBALS['TYPO3_CONF_VARS']['SYS']['FileInfo']['fileExtensionToMimeType']['myvideo'] =
				  'video/myvideo';

				// register your custom file extension as allowed media file
				$GLOBALS['TYPO3_CONF_VARS']['SYS']['mediafile_ext'] .= ',myvideo';
			\end{lstlisting}

	\end{itemize}

\end{frame}

% ------------------------------------------------------------------------------
% LTXE-SLIDE-START
% LTXE-SLIDE-UID:		3d30b343-1c894329-54ea5136-8a2dba43
% LTXE-SLIDE-ORIGIN:	0d9d73fc-2a68f137-f5ff5876-74aeef2b English
% LTXE-SLIDE-ORIGIN:	c675e3a8-9a429b0b-61a4a603-d5b0740f German
% LTXE-SLIDE-TITLE:		Unified Backend Routing
% LTXE-SLIDE-REFERENCE:	Feature-65493-BackendRouting.rst
% ------------------------------------------------------------------------------

\begin{frame}[fragile]
	\frametitle{Cambios en Profundidad}
	\framesubtitle{Enrutamiento Backend}

	% decrease font size for code listing
	\lstset{basicstyle=\tiny\ttfamily}

	\begin{itemize}

		\item Se ha añadido un nuevo componente de enrutamiento al backend de TYPO3 que maneja
			dirigirse a diferentes módulos/llamadas dentro de TYPO3 CMS

		\item Pueden definirse rutas en la siguiente clase:\newline
			\small
				\texttt{Configuration/Backend/Routes.php}
			\normalsize

			\begin{lstlisting}
				return [
				  'myRouteIdentifier' => [
				    'path' => '/document/edit',
				    'controller' => Acme\MyExtension\Controller\MyExampleController::class . '::methodToCall'
				  ]
				];
			\end{lstlisting}

		\item El método llamado contiene objetos de petición y respuesta que cumplen con PSR-7:

			\begin{lstlisting}
				public function methodToCall(ServerRequestInterface $request, ResponseInterface $response) {
				  ...
				}
			\end{lstlisting}

	\end{itemize}

\end{frame}

% ------------------------------------------------------------------------------
% LTXE-SLIDE-START
% LTXE-SLIDE-UID:		4516506a-52178b4b-ca55cd42-d931c07b
% LTXE-SLIDE-ORIGIN:	a38b3e30-a34a3bb1-ee94992a-1c91db32 English
% LTXE-SLIDE-ORIGIN:	df56086a-498ae937-9b2bf393-a95828f3 German
% LTXE-SLIDE-TITLE:		Autoload definition can be provided in ext_emconf.php
% LTXE-SLIDE-REFERENCE:	Feature-68700-AutoloadDefinitionCanBeProvidedInExt_emconfphp.rst
% ------------------------------------------------------------------------------

\begin{frame}[fragile]
	\frametitle{Cambios en Profundidad}
	\framesubtitle{Definición de Autocargado en \texttt{ext\_emconf.php}}

	% decrease font size for code listing
	\lstset{basicstyle=\tiny\ttfamily}

	\begin{itemize}

		\item Extensiones pueden proporcionar ahora una o más definiciones PSR-4 en el fichero \texttt{ext\_emconf.php}

		\item Esto era ya posible en \texttt{composer.json}, pero con esta nueva característica,
			los desarrolladores de extensiones no necesitan proporcionar más un fichero composer sólo para esto

			\begin{lstlisting}
				$EM_CONF[$_EXTKEY] = array (
				  'title' => 'Extension Skeleton for TYPO3 CMS 7',
				  ...
				'autoload' =>
				  array(
				    'psr-4' => array(
				      'Helhum\\ExtScaffold\\' => 'Classes'
				    )
				  )
				);
			\end{lstlisting}

			\small
				(ésta es la nueva manera recomendada de registrar clases en TYPO3)
			\normalsize

	\end{itemize}

\end{frame}

% ------------------------------------------------------------------------------
% LTXE-SLIDE-START
% LTXE-SLIDE-UID:		ae502d2c-545c4528-b1b28143-559bbe50
% LTXE-SLIDE-ORIGIN:	6d370930-d24dd6c2-54e2330d-b873a914 English
% LTXE-SLIDE-ORIGIN:	137a9fca-b6558571-346141d2-9304a223 German
% LTXE-SLIDE-TITLE:		Icon-Factory (1)
% LTXE-SLIDE-REFERENCE:	Feature-68741-IntroduceNewIconFactoryAsBaseForReplaceTheIconSkinningAPI.rst
% LTXE-SLIDE-REFERENCE:	Feature-69095-IntroduceIconStateForIconFactory.rst
% ------------------------------------------------------------------------------

\begin{frame}[fragile]
	\frametitle{Cambios en Profundidad}
	\framesubtitle{Nueva Fábrica de Iconos (1)}

	% decrease font size for code listing
	\lstset{basicstyle=\smaller\ttfamily}

	\begin{itemize}

		\item Lógica para trabajar con iconos, tamaños de iconos y superposiciones de iconos está ahora agrupada en
			la nueva clase \texttt{IconFactory}

		\item La nueva fábrica de iconos reemplazará a la antigua API de skins de iconos paso por paso

		\item Se registrarán todos los iconos del núcleo directamente en la clase \texttt{IconRegistry}

		\item Extensiones deben usar \texttt{IconRegistry::registerIcon()} para sobreescribir iconos
			existentes o añadir iconos adicionales a la fábrica de iconos:

			\begin{lstlisting}
				IconRegistry::registerIcon(
				  $identifier,
				  $iconProviderClassName,
				  array $options = array()
				);
			\end{lstlisting}

	\end{itemize}

\end{frame}

% ------------------------------------------------------------------------------
% LTXE-SLIDE-START
% LTXE-SLIDE-UID:		86133d1c-174be1d8-4fd5e6fa-7ed7960f
% LTXE-SLIDE-ORIGIN:	4e9f1524-c3c04972-bf6bd14b-19193ab2 English
% LTXE-SLIDE-ORIGIN:	6b59b4c7-21a7ff5c-4e67b53d-e4509355 German
% LTXE-SLIDE-TITLE:		Icon-Factory (2)
% LTXE-SLIDE-REFERENCE:	Feature-68741-IntroduceNewIconFactoryAsBaseForReplaceTheIconSkinningAPI.rst
% LTXE-SLIDE-REFERENCE:	Feature-69095-IntroduceIconStateForIconFactory.rst
% ------------------------------------------------------------------------------

\begin{frame}[fragile]
	\frametitle{Cambios en Profundidad}
	\framesubtitle{Nueva Fábrica de Iconos (2)}

	% decrease font size for code listing
	\lstset{basicstyle=\tiny\ttfamily}

	\begin{itemize}

		\item El núcleo de TYPO3 CMS implementa tres clases proveedoras de iconos:\newline
			\smaller
				\texttt{BitmapIconProvider}, \texttt{FontawesomeIconProvider} y \texttt{SvgIconProvider}
			\normalsize

		\item Uso de ejemplo:

			\begin{lstlisting}
				$iconFactory = GeneralUtility::makeInstance(IconFactory::class);
				$iconFactory->getIcon(
				  $identifier,
				  Icon::SIZE_SMALL,
				  $overlay,
				  IconState::cast(IconState::STATE_DEFAULT)
				)->render();
			\end{lstlisting}

		\item Valores válidos para \texttt{Icon::SIZE\_...} son:\newline
			\small\texttt{SIZE\_SMALL}, \texttt{SIZE\_DEFAULT} y \texttt{SIZE\_LARGE}\normalsize
			\vspace{0.4cm}

		\item Valores válidos para \texttt{Icon::STATE\_...} son:\newline
			\small\texttt{STATE\_DEFAULT} y \texttt{STATE\_DISABLED}\normalsize

	\end{itemize}

\end{frame}

% ------------------------------------------------------------------------------
% LTXE-SLIDE-START
% LTXE-SLIDE-UID:		dec74337-54988621-ffd361b6-74db3531
% LTXE-SLIDE-ORIGIN:	ffa90088-571dcf20-3d315854-a05814da English
% LTXE-SLIDE-ORIGIN:	21a7eeb5-b4c7ff5c-b53de450-4e679355 German
% LTXE-SLIDE-TITLE:		Icon-Factory (3)
% LTXE-SLIDE-REFERENCE:	Feature-68741-IntroduceNewIconFactoryAsBaseForReplaceTheIconSkinningAPI.rst
% LTXE-SLIDE-REFERENCE:	Feature-69095-IntroduceIconStateForIconFactory.rst
% ------------------------------------------------------------------------------

\begin{frame}[fragile]
	\frametitle{Cambios en Profundidad}
	\framesubtitle{Nueva Fábrica de Iconos (3)}

	% decrease font size for code listing
	\lstset{basicstyle=\tiny\ttfamily}

	\begin{itemize}

		\item El núcleo de TYPO3 CMS proporciona un ViewHelper Fluid que facilita el usar iconos dentro de una vista Fluid:

			\begin{lstlisting}
				{namespace core = TYPO3\CMS\Core\ViewHelpers}

				<core:icon identifier="my-icon-identifier"></core:icon>

				<!-- use the "small" size if none given ->
				<core:icon identifier="my-icon-identifier"></core:icon>
				<core:icon identifier="my-icon-identifier" size="large"></core:icon>
				<core:icon identifier="my-icon-identifier" overlay="overlay-identifier"></core:icon>

				<core:icon identifier="my-icon-identifier" size="default" overlay="overlay-identifier">
				</core:icon>

				<core:icon identifier="my-icon-identifier" size="large" overlay="overlay-identifier">
				</core:icon>
			\end{lstlisting}

	\end{itemize}

\end{frame}

% ------------------------------------------------------------------------------
% LTXE-SLIDE-START
% LTXE-SLIDE-UID:		f3a6db94-09ec23b2-8e855963-dcf422e6
% LTXE-SLIDE-ORIGIN:	10739ce3-88cfb3f6-d8364643-424cbf3c English
% LTXE-SLIDE-ORIGIN:	d9de708f-474f2f36-c104b01c-0d23a352 German
% LTXE-SLIDE-TITLE:		Signal for pre processing linkvalidator records
% LTXE-SLIDE-REFERENCE:	Feature-52217-SignalForPreProcessingLinkvalidatorRecords.rst
% ------------------------------------------------------------------------------

\begin{frame}[fragile]
	\frametitle{Cambios en Profundidad}
	\framesubtitle{Hooks y Señales}

	% decrease font size for code listing
	\lstset{basicstyle=\tiny\ttfamily}

	\begin{itemize}

		\item Ha sido añadida una nueva señal al LinkValidator, que permite procesamiento
			adicional sobre la inicialización de un registro específico\newline
			\small
				(p.e. obteniendo datos de contenido desde la configuración del plugin en registro)
			\normalsize

		\item Registrando la señal en el fichero \texttt{ext\_localconf.php}:

			\begin{lstlisting}
				$signalSlotDispatcher = \TYPO3\CMS\Core\Utility\GeneralUtility::makeInstance(
				  \TYPO3\CMS\Extbase\SignalSlot\Dispatcher::class
				);

				$signalSlotDispatcher->connect(
				  \TYPO3\CMS\Linkvalidator\LinkAnalyzer::class,
				  'beforeAnalyzeRecord',
				  \Vendor\Package\Slots\RecordAnalyzerSlot::class,
				  'beforeAnalyzeRecord'
				);
			\end{lstlisting}

	\end{itemize}

\end{frame}

% ------------------------------------------------------------------------------
% LTXE-SLIDE-START
% LTXE-SLIDE-UID:		db4ba4f0-035ad421-15c04901-a5a42429
% LTXE-SLIDE-ORIGIN:	8b50edb3-2363f2a9-3120a711-db8b6ede English
% LTXE-SLIDE-ORIGIN:	bcb8ef73-b8669a17-102da209-5f5c9adc German
% LTXE-SLIDE-TITLE:		Replaced JumpURL features with hooks (1)
% LTXE-SLIDE-REFERENCE:	Breaking-52156-ReplaceJumpUrlWithHooks.rst
% ------------------------------------------------------------------------------

\begin{frame}[fragile]
	\frametitle{Cambios en Profundidad}
	\framesubtitle{JumpUrl como Extensión del Sistema (1)}

	% decrease font size for code listing
	\lstset{basicstyle=\tiny\ttfamily}

	\begin{itemize}

		\item Se ha movido la generación y manejo de JumpURLs a una nueva extensión del sistema \texttt{jumpurl}

		\item Fueron introducidos nuevos hooks que permiten la generación y manejo personalizado de URL (ver siguiente página)

	\end{itemize}

	\breakingchange

\end{frame}

% ------------------------------------------------------------------------------
% LTXE-SLIDE-START
% LTXE-SLIDE-UID:		8d03200b-bfe15084-7489ab89-c5bc69e3
% LTXE-SLIDE-ORIGIN:	68003b46-065b81f8-5206141c-77a35b88 English
% LTXE-SLIDE-ORIGIN:	ecb8fef3-69b86a17-02da2109-9adc5f5c German
% LTXE-SLIDE-TITLE:		Replaced JumpURL features with hooks (2)
% LTXE-SLIDE-REFERENCE:	Breaking-52156-ReplaceJumpUrlWithHooks.rst
% ------------------------------------------------------------------------------

\begin{frame}[fragile]
	\frametitle{Cambios en Profundidad}
	\framesubtitle{JumpUrl como Extensión del Sistema (2)}

	% decrease font size for code listing
	\lstset{basicstyle=\tiny\ttfamily}

	\begin{itemize}

		\item Hook 1: manipulando \textbf{URLs} durante la generación del enlace

			\begin{lstlisting}
				$GLOBALS['TYPO3_CONF_VARS']['SC_OPTIONS']['urlProcessing']['urlHandlers']
				  ['myext_myidentifier']['handler'] = \Company\MyExt\MyUrlHandler::class;

				// class needs to implement the UrlHandlerInterface:
				class MyUrlHandler implements \TYPO3\CMS\Frontend\Http\UrlHandlerInterface {
				  ...
				}
			\end{lstlisting}

		\item Hook 2: manejo de \textbf{enlaces}

			\begin{lstlisting}
				$GLOBALS['TYPO3_CONF_VARS']['SC_OPTIONS']['urlProcessing']['urlProcessors']
				  ['myext_myidentifier']['processor'] = \Company\MyExt\MyUrlProcessor::class;

				// class needs to implement the UrlProcessorInterface:
				class MyUrlProcessor implements \TYPO3\CMS\Frontend\Http\UrlProcessorInterface {
				  ...
				}
			\end{lstlisting}

	\end{itemize}

\end{frame}

% ------------------------------------------------------------------------------
% LTXE-SLIDE-START
% LTXE-SLIDE-UID:		a42e0493-d9a903df-1f88e5ef-2e04f884
% LTXE-SLIDE-ORIGIN:	ff50feed-0f2dfe5a-1f960c51-e493d3b0 English
% LTXE-SLIDE-ORIGIN:	08a70f0f-528daf11-ee485c29-d9707f25 German
% LTXE-SLIDE-TITLE:		Colored Output for CLI Calls on Errors
% LTXE-SLIDE-REFERENCE:	Feature-67056-AddOptionToDisableMoveButtonsTCAGroupType.rst
% LTXE-SLIDE-REFERENCE:	Feature-67875-OverrideCategoryRegistryEntry.rst
% LTXE-SLIDE-REFERENCE:	Feature-68804-ColoredOutputForCLI-relevantErrorMessages.rst
% ------------------------------------------------------------------------------

\begin{frame}[fragile]
	\frametitle{Cambios en Profundidad}
	\framesubtitle{Interfaz de Línea de Comandos (CLI)}

	% decrease font size for code listing
	\lstset{basicstyle=\tiny\ttfamily}

	\begin{itemize}

		\item Llamando a \texttt{typo3/cli\_dispatch.phpsh} vía la línea de comandos ahora muestra un
			mensaje coloreado de error si una clave CLI inválida o ninguna clave es suministrada como primer parámetro

		\item Controladores de comando Extbase pueden ahora residir en subcarpetas arbistrarias dentro de la
			carpeta \texttt{Command}

		\item Ejemplo:\newline

			Controlador en fichero:\newline
			\smaller\texttt{my\_ext/Classes/Command/Hello/WorldCommandController.php}\normalsize\newline
			...puede ser llamado vía CLI:\newline
			\smaller\texttt{typo3/cli\_dispatch.sh extbase my\_ext:hello:world <arguments>}\normalsize

	\end{itemize}

\end{frame}

% ------------------------------------------------------------------------------
% LTXE-SLIDE-START
% LTXE-SLIDE-UID:		5c0454e8-257bc787-38c563ae-15e3393d
% LTXE-SLIDE-ORIGIN:	df540c68-edd41657-8854e46a-a716dccb English
% LTXE-SLIDE-ORIGIN:	a708daf1-e48f525c-7010fe7f-2529d9e4 German
% LTXE-SLIDE-TITLE:		Miscellaneous (1)
% LTXE-SLIDE-REFERENCE:	Feature-68804-ColoredOutputForCLI-relevantErrorMessages.rst
% LTXE-SLIDE-REFERENCE:	Feature-69512-SupportTyposcriptFilesAsTextFileType.rst
% LTXE-SLIDE-REFERENCE:	Feature-69543-IntroducedGLOBALSTYPO3_CONF_VARSSYSmediafile_ext.rst
% ------------------------------------------------------------------------------

\begin{frame}[fragile]
	\frametitle{Cambios en Profundidad}
	\framesubtitle{Miscelánea (1)}

	\begin{itemize}

		\item Los botones de mover del tipo \texttt{group} TCA pueden ser ahora explícitamente
			deshabilitados usando la opción \texttt{hideMoveIcons = TRUE}

		\item Método \texttt{makeCategorizable} ha sido extendido con un nuevo parámetro
			\texttt{override} para ajustar una nueva configuración de categoría para una combinación ya
			registrada de table/field

		\item Ejemplo:

			\begin{lstlisting}
				\TYPO3\CMS\Core\Utility\ExtensionManagementUtility::makeCategorizable(
				  'css_styled_content', 'tt_content', 'categories', array(), TRUE
				);
			\end{lstlisting}

			\small
				Último parámetro (aquí: \texttt{TRUE}) fuerza la sobreescritura (valor por defecto es \texttt{FALSE}).
			\normalsize

	\end{itemize}

\end{frame}

% ------------------------------------------------------------------------------
% LTXE-SLIDE-START
% LTXE-SLIDE-UID:		2bab1724-c322e8cb-1ecd0695-31107cc1
% LTXE-SLIDE-ORIGIN:	5a6db8b0-ff039479-8ba12ed0-d50d8a0f English
% LTXE-SLIDE-ORIGIN:	d52d6418-554d801f-352b85cb-54046d28 German
% LTXE-SLIDE-TITLE:		Miscellaneous (2)
% LTXE-SLIDE-REFERENCE:	Feature-69730-IntroduceUniqueIdGenerator.rst
% LTXE-SLIDE-REFERENCE:	Important-68758-CommandControllersAllowedInSubfolders.rst
% ------------------------------------------------------------------------------

\begin{frame}[fragile]
	\frametitle{Cambios en Profundidad}
	\framesubtitle{Miscelánea (2)}

	% decrease font size for code listing
	%\lstset{basicstyle=\tiny\ttfamily}

	\begin{itemize}

		\item Nueva función genera un ID único

			\begin{lstlisting}
				$uniqueId = \TYPO3\CMS\Core\Utility\StringUtility::getUniqueId('Prefix');
			\end{lstlisting}

		\item El tipo de fichero \texttt{.typoscript} ha sido añadido a la lista de tipos de ficheros de texto plano válidos

		\item Nueva opción de configuración define extensiones de ficheros media

			\begin{lstlisting}
				$GLOBALS['TYPO3_CONF_VARS']['SYS']['mediafile_ext'] =
				  'gif,jpg,jpeg,bmp,png,pdf,svg,ai,mov,avi';
			\end{lstlisting}

	\end{itemize}

	\breakingchange

\end{frame}

% ------------------------------------------------------------------------------
