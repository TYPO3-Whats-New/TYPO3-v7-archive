% ------------------------------------------------------------------------------
% TYPO3 CMS 7.3 - What's New - Chapter "TypoScript" (English Version)
%
% @author	Michael Schams <schams.net>
% @license	Creative Commons BY-NC-SA 3.0
% @link		http://typo3.org/download/release-notes/whats-new/
% @language	English
% ------------------------------------------------------------------------------
% LTXE-CHAPTER-UID:		e8cc33e1-d7bdb74f-0421b75f-7be06b19
% LTXE-CHAPTER-NAME:	TypoScript
% ------------------------------------------------------------------------------

\section{TSconfig \& TypoScript}
\begin{frame}[fragile]
	\frametitle{TSconfig \& TypoScript}

	\begin{center}\huge{Capitolo 2:}\end{center}
	\begin{center}\huge{\color{typo3darkgrey}\textbf{TSconfig \& TypoScript}}\end{center}

\end{frame}

% ------------------------------------------------------------------------------
% LTXE-SLIDE-START
% LTXE-SLIDE-UID:		5b0119c7-44a3a291-112d48d9-7163b9f4
% LTXE-SLIDE-ORIGIN:	f8419191-d59094a8-01e156a1-908dac0f English
% LTXE-SLIDE-ORIGIN:	56b025e6-cf380b11-c9d1c009-0b548d6a German
% LTXE-SLIDE-TITLE:		Feature #63561: Add TypoScript stdWrap strtotime
% LTXE-SLIDE-REFERENCE:	Feature-63561-AddTypoScriptStdWrapStrtotime.rst
% ------------------------------------------------------------------------------
\begin{frame}[fragile]
	\frametitle{TSconfig \& TypoScript}
	\framesubtitle{Nuova funzione stdWrap \texttt{strtotime}}

	\begin{itemize}

		\item La funzione stdWrap di TypoScript \texttt{strtotime} permette di convertire da
			data formattata a Unix timestamps, es. calcoli tra date

		\item Valori validi sono \texttt{1} o qualsiasi stringa time che è usata come primo
			argomento nella funzione PHP \texttt{strtotime()}

			\begin{lstlisting}
				date_as_timestamp = TEXT
				date_as_timestamp {
				  value = 2015-04-15
				  strtotime = 1
				} 

				next_weekday = TEXT
				next_weekday {
				  data = GP:selected_date
				  strtotime = + 2 weekdays
				  strftime = %Y-%m-%d
				}
			\end{lstlisting}

	\end{itemize}

\end{frame}

% ------------------------------------------------------------------------------
% LTXE-SLIDE-START
% LTXE-SLIDE-UID:		7a8702f8-37807363-9d82c43d-2125440b
% LTXE-SLIDE-ORIGIN:	687af5fc-b201bac6-e4ad678b-392fb9f8 English
% LTXE-SLIDE-ORIGIN:	c5dbf3c7-655fc6e4-227a3bf6-f5d89fb1 German
% LTXE-SLIDE-TITLE:		Feature #65250: TypoScript condition add GPmerged
% LTXE-SLIDE-REFERENCE:	Feature-65250-TypoScriptConditionAddGPmerged.rst
% ------------------------------------------------------------------------------

\begin{frame}[fragile]
	\frametitle{TSconfig \& TypoScript}
	\framesubtitle{\texttt{GPmerged} nelle condizioni}

	\begin{itemize}

		\item Usando \text{GP} nelle condizioni TypoScript sono restituite le variabili \text{POST},
			e se la request contiene entrambe, le variabili \text{POST} \underline{e}
			\text{GET}

		\item La nuova opzione \texttt{GPmerged} unisce i due metodi e ne restituisce il risultato

			\begin{lstlisting}
				[globalVar = GPmerged:tx_demo|foo = 1]
				  page.90 = TEXT
				  page.90.value = DEMO
				[global]
			\end{lstlisting}

	\end{itemize}

\end{frame}

% ------------------------------------------------------------------------------
% LTXE-SLIDE-START
% LTXE-SLIDE-UID:		3d78351c-c5001f7f-fa5321f4-bb3527be
% LTXE-SLIDE-ORIGIN:	2901d1f6-3d8e9a15-ad0268f1-4e30c04b English
% LTXE-SLIDE-ORIGIN:	05b2f338-7bcb651f-a4cebabd-2c677a20 German
% LTXE-SLIDE-TITLE:		Feature #66697: Add uppercamelcase and lowercamelcase to stdWrap.case
% LTXE-SLIDE-REFERENCE:	Feature-66697-AddUppercamelcaseAndLowercamelcaseToStdWrap.case.rst
% ------------------------------------------------------------------------------

\begin{frame}[fragile]
	\frametitle{TSconfig \& TypoScript}
	\framesubtitle{Nuova opzione per \texttt{stdWrap.case}}

	% decrease font size for code listing
	\lstset{basicstyle=\tiny\ttfamily}

	\begin{itemize}

		\item Le opzioni \texttt{uppercamelcase} e \texttt{lowercamelcase} sono state aggiunte
			a \texttt{stdWrap.case}

		\item Esempio:

			\begin{lstlisting}
				tt_content = CASE
				tt_content {
				  key.field = CType
				  my_custom_ctype =< lib.userContent
				  my_custom_ctype {
				    file = EXT:site_base/Resources/Private/Templates/SomeOtherTemplate.html
				    settings.extraParam = 1
				  }
				  default =< lib.userContent
				  default {
				    file = TEXT
				    file.field = CType
				    file.stdWrap.case = uppercamelcase
				    file.wrap = EXT:site_base/Resources/Private/Templates/|.html
				  }
				}
			\end{lstlisting}

	\end{itemize}

\end{frame}

% ------------------------------------------------------------------------------
% LTXE-SLIDE-START
% LTXE-SLIDE-UID:		f69c9b08-1ea1f311-08ad5d10-dd1abe22
% LTXE-SLIDE-ORIGIN:	72bacec0-2fa0303b-33c4c4d6-a3bc0857 English
% LTXE-SLIDE-ORIGIN:	cd745fc5-b8bd6483-ba1cf2a5-0ebf4310 German
% LTXE-SLIDE-TITLE:		Feature #66698: Add integrity property to JavaScript files (1)
% LTXE-SLIDE-REFERENCE:	Feature-66698-AddIntegrityPropertyToJavaScriptFiles.rst
% ------------------------------------------------------------------------------

\begin{frame}[fragile]
	\frametitle{TSconfig \& TypoScript}
	\framesubtitle{La proprietà \texttt{integrity} è stata aggiunta per i file JavaScript (1)}

	% decrease font size for code listing
	\lstset{basicstyle=\tiny\ttfamily}

	\begin{itemize}

		\item La proprietà \texttt{integrity} è stata aggiunta ai file JavaScript inclusi con
			lo scopo di specificare una hash SRI per attivare la verifica della risorsa\newline
			(SRI: Sub-Resource Integrity, vedi slide seguente)

		\item Questo riguarda le proprietà TypoScript di PAGE \texttt{page.includeJSLibs},
			\texttt{page.includeJSFooterlibs}, \texttt{ includeJS} e
			\texttt{includeJSFooter}

		\item Esempio:

			\begin{lstlisting}
				page {
				  includeJS {
				    jQuery = https://code.jquery.com/jquery-1.11.3.min.js
				    jquery.external = 1
				    jQuery.disableCompression = 1
				    jQuery.excludeFromConcatenation = 1
				    jQuery.integrity = sha256-7LkWEzqTdpEfELxcZZlS6wAx5Ff13zZ83lYO2/ujj7g=
				  }
				}
			\end{lstlisting}

	\end{itemize}

\end{frame}

% ------------------------------------------------------------------------------
% LTXE-SLIDE-START
% LTXE-SLIDE-UID:		96243bc6-ae027d8e-0ad7aa1b-b6b68561
% LTXE-SLIDE-ORIGIN:	bf190d17-80b961a2-a0c001d4-6ee48dfe English
% LTXE-SLIDE-ORIGIN:	b8bd6483-cd745fc5-ba1cf2a5-0ebf4310 German
% LTXE-SLIDE-TITLE:		Feature #66698: Add integrity property to JavaScript files (2)
% LTXE-SLIDE-REFERENCE:	Feature-66698-AddIntegrityPropertyToJavaScriptFiles.rst
% ------------------------------------------------------------------------------

\begin{frame}[fragile]
	\frametitle{TSconfig \& TypoScript}
	\framesubtitle{La proprietà \texttt{integrity} è stata aggiunta per i file JavaScript (2)}

	% decrease font size for code listing
	%\lstset{basicstyle=\tiny\ttfamily}

	\begin{itemize}

		\item SRI è una specifica W3C che permette agli sviluppatori web di garantire che
			le risorse ospitate sui server di terze parti non siano state manomesse 
			

		\item Generando una hash di integrità:

			\begin{itemize}
				\item Opzione 1: \url{https://srihash.org}
				\item Opzione 2: usando il seguente comando di shell
			\end{itemize}

			\begin{lstlisting}
				cat FILENAME.js | openssl dgst -sha256 -binary | openssl enc -base64 -A
			\end{lstlisting}

		\item Maggiori informazioni:

			\begin{itemize}
				\item \url{http://www.w3.org/TR/SRI/}
			\end{itemize}

	\end{itemize}

\end{frame}

% ------------------------------------------------------------------------------
