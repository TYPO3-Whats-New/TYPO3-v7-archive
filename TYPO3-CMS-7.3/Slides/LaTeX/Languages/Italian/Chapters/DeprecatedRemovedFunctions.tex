% ------------------------------------------------------------------------------
% TYPO3 CMS 7.3 - What's New - Chapter "Deprecated Functions" (English Version)
%
% @author	Michael Schams <schams.net>
% @license	Creative Commons BY-NC-SA 3.0
% @link		http://typo3.org/download/release-notes/whats-new/
% @language	English
% ------------------------------------------------------------------------------
% LTXE-CHAPTER-UID:		46065d05-1bec23cb-82793761-dfb7b0b1
% LTXE-CHAPTER-NAME:	Deprecated Functions
% ------------------------------------------------------------------------------

\section{Funzionalità deprecate/rimosse}
\begin{frame}[fragile]
	\frametitle{Funzionalità deprecate/rimosse}

	\begin{center}\huge{Capitolo 5:}\end{center}
	\begin{center}\huge{\color{typo3darkgrey}\textbf{Funzionalità deprecate/rimosse}}\end{center}

\end{frame}

% ------------------------------------------------------------------------------
% LTXE-SLIDE-START
% LTXE-SLIDE-UID:		2397b755-bab2fa35-19328ee3-31d65a0f
% LTXE-SLIDE-ORIGIN:	909542ad-6c4e7767-0f73f29a-0249cb54 English
% LTXE-SLIDE-ORIGIN:	693c1737-af89d78f-84db7a5e-f0a05dea German
% LTXE-SLIDE-TITLE:		Breaking: #63846 - FormEngine refactoring
% LTXE-SLIDE-REFERENCE:	Breaking-63846-FormEngineRefactoring.rst
% ------------------------------------------------------------------------------

\begin{frame}[fragile]
	\frametitle{Funzionalità deprecate/rimosse}
	\framesubtitle{Rifacimento FormEngine}

		\textbf{TCA:}

			\small
			\begin{itemize}

				\item Le opzioni \texttt{\_PADDING}, \texttt{\_VALIGN} e \texttt{DISTANCE}
					sono state rimosse da
					\texttt{TCA['aTable']['columns']['aField']['config']['wizards']}

				\item La chiave \texttt{TCA['aTable']['ctrl']['mainPalette']} è stata rimossa

			\end{itemize}

		\textbf{TSconfig:}

			\small
			\begin{itemize}
				\item Le chiavi \texttt{mod.web\_layout.tt\_content.fieldOrder} e
					\texttt{TCEFORM.aTable.aField.linkTitleToSelf} sono state rimosse
			\end{itemize}

		\textbf{Hooks:}

			\small
			\begin{itemize}
				\item Ora gli hook usano la chiave \texttt{type} invece di \texttt{form\_type} 
				\item L'hook \texttt{getSingleFieldClass} è stato rimosso
			\end{itemize}

\end{frame}

% ------------------------------------------------------------------------------
% LTXE-SLIDE-START
% LTXE-SLIDE-UID:		f2f12732-68f0b6ee-67e1ac8a-3b102fed
% LTXE-SLIDE-ORIGIN:	063f16c1-c3d2f376-642e83f2-9d3eaed9 English
% LTXE-SLIDE-ORIGIN:	4991d739-1ccd5bde-f424f2fc-47d59881 German
% LTXE-SLIDE-TITLE:		Breaking - #66429: Remove IdentityMap from persistence
% LTXE-SLIDE-REFERENCE: Breaking-66429-RemoveIdentityMapFromPersistence.rst
% ------------------------------------------------------------------------------

\begin{frame}[fragile]
	\frametitle{Funzionalità deprecate/rimosse}
	\framesubtitle{Rimozione di \texttt{IdentityMap} dalla persistenza di Extbase}

	% decrease font size for code listing
	\lstset{basicstyle=\tiny\ttfamily}

	\begin{itemize}

		\item La classe \texttt{IdentityMap} è stata rimossa dalla persistenza di Extbase\newline
			\small(una \texttt{ReflectionException} è restituita se si usa ancora)\normalsize

		\item L'accesso alla precedente proprietà \texttt{IdentityMap} con
			\texttt{DataMapper} e \texttt{Repository} fallirà ora e la creazione
			di un istanza \texttt{IdentityMap} non è più possibile

		\item Usa la persistenza "Sessions" al suo posto:

			\begin{lstlisting}
				$session = GeneralUtility::makeInstance(ObjectManager::class)->get(
				  \TYPO3\CMS\Extbase\Persistence\Generic\Session::class
				);

				$session->registerObject($object, $identifier);

				if($session->hasIdentifier($identifier)) {
				  $object = $session->getObjectByIdentifier($identifier, $className);
				}
			\end{lstlisting}

	\end{itemize}

\end{frame}

% ------------------------------------------------------------------------------
% LTXE-SLIDE-START
% LTXE-SLIDE-UID:		adfd4f94-5042f3e3-ba2471aa-0bf87500
% LTXE-SLIDE-ORIGIN:	068af613-ab320130-00499229-6cb65ee6 English
% LTXE-SLIDE-ORIGIN:	dcf8fd4e-879461aa-5bb22ccc-7d50bcff German
% LTXE-SLIDE-TITLE:		Miscellaneous (1)
% LTXE-SLIDE-REFERENCE:	Deprecation-65344-ExtTables.rst
% LTXE-SLIDE-REFERENCE:	Breaking-66906-AutomaticPNGToGIFConversionRemoved.rst
% LTXE-SLIDE-REFERENCE: Breaking-66997-RemoveSuper-challengedPasswordSecurity.rst
% LTXE-SLIDE-REFERENCE: Breaking-67204-DatabaseConnectionexec_SELECTgetRowsMayThrowException.rst
% LTXE-SLIDE-REFERENCE: Deprecation-61829-DbalConfigClassFile.rst
% LTXE-SLIDE-REFERENCE: Deprecation-66789-DeprecateOptionsInCshViewHelper.rst
% ------------------------------------------------------------------------------

\begin{frame}[fragile]
	\frametitle{Funzionalità deprecate/rimosse}
	\framesubtitle{Varie (1)}

	\begin{itemize}

		\item Il file \texttt{typo3conf/extTables.php} è deprecato.\newline
			Utilizza il seguente file al suo posto:\newline
			\smaller\texttt{<your\_extension>/Configuration/TCA/Overrides/pages.php}\normalsize

		\item La configurazione \texttt{\$TYPO3\_CONF\_VARS[GFX][png\_to\_gif]} è stata rimossa

		\item Nell'installazione di TYPO3 CMS, dove \underline{non} è installata l'estensione
			\texttt{rsaauth}, la password del login di BE viene gestita in chiaro\newline
			\small(soluzione: installa l'estensione \texttt{rsaauth} o usa HTTPS per il BE)\normalsize

		\item Il metodo \texttt{exec\_SELECTgetRows()} valida il parametro \texttt{\$uidIndexField}.
			Se questo campo specifico non è presente nei risultati del database, è restituito una
			\texttt{InvalidArgumentException}.

	\end{itemize}

\end{frame}

% ------------------------------------------------------------------------------
% LTXE-SLIDE-START
% LTXE-SLIDE-UID:		45da2137-dee6c612-3f8fb60e-13e954cf
% LTXE-SLIDE-ORIGIN:	697a7dbb-6f3f402c-13be799d-0115ab4d English
% LTXE-SLIDE-ORIGIN:	38119304-60574ef0-2f5a116c-1c87717f German
% LTXE-SLIDE-TITLE:		Miscellaneous (2)
% LTXE-SLIDE-REFERENCE:	Deprecation-67029-DeprecatePageBgImgOption.rst
% LTXE-SLIDE-REFERENCE: Deprecation-67171-T3editorIsEnabled.rst
% LTXE-SLIDE-REFERENCE: Breaking-67212-DiscardLegacyClassLoader.rst
% ------------------------------------------------------------------------------

\begin{frame}[fragile]
	\frametitle{Funzionalità deprecate/rimosse}
	\framesubtitle{Varie (2)}

	\begin{itemize}

		\item L'ozione di DBAL \texttt{config.classFile} è stata rimossa

		\item Le opzioni \texttt{iconOnly} e \texttt{styleAttributes} di
			\texttt{CshViewHelper} sono segnate come deprecate

		\item L'opzione TypoScript \texttt{page.bgImg} ora è deprecata

		\item Il metodo \texttt{isEnabled()} della classe \texttt{T3editor} è deprecato

		\item La vecchia TYPO3 ClassLoader è stata rimossa in favore di Composer ClassLoader

	\end{itemize}

\end{frame}

% ------------------------------------------------------------------------------
