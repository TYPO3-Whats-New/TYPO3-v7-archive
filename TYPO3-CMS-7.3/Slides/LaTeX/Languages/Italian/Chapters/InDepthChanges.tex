% ------------------------------------------------------------------------------
% TYPO3 CMS 7.3 - What's New - Chapter "In-Depth Changes" (English Version)
%
% @author	Michael Schams <schams.net>
% @license	Creative Commons BY-NC-SA 3.0
% @link		http://typo3.org/download/release-notes/whats-new/
% @language	English
% ------------------------------------------------------------------------------
% LTXE-CHAPTER-UID:		e8cc33e1-d7bdb74f-0421b75f-7be06b19
% LTXE-CHAPTER-NAME:	In-Depth Changes
% ------------------------------------------------------------------------------

\section{Modifiche rilevanti}
\begin{frame}[fragile]
	\frametitle{Modifiche rilevanti}

	\begin{center}\huge{Capitolo 3:}\end{center}
	\begin{center}\huge{\color{typo3darkgrey}\textbf{Modifiche rilevanti}}\end{center}

\end{frame}

% ------------------------------------------------------------------------------
% LTXE-SLIDE-START
% LTXE-SLIDE-UID:		61b37467-13477a17-98d8123f-ffbba1d5
% LTXE-SLIDE-ORIGIN:	17eaaa35-2a21b728-4c3488a0-30a7ac7b English
% LTXE-SLIDE-ORIGIN:	bd03e141-3842b75e-8a44920f-a8c139e5 German
% LTXE-SLIDE-TITLE:		Feature #59606: Integrate Symfony/Console into CommandController (1)
% LTXE-SLIDE-REFERENCE:	Feature-59606-IntegrateSymfonyConsoleIntoCommandController.rst
% ------------------------------------------------------------------------------

\begin{frame}[fragile]
	\frametitle{Modifiche rilevanti}\frametitle{Modifiche rilevanti}
	\framesubtitle{Integrazione Symfony/Console in CommandController (1)}

	% decrease font size for code listing
	\lstset{basicstyle=\tiny\ttfamily}

	Il CommandController ora fa uso internamente di Symfony/Console e fornisce vari metodi:

	\begin{itemize}

		\item \smaller TableHelper
			\begin{itemize}
				\item\smaller\texttt{outputTable(\$rows, \$headers = NULL)}
			\end{itemize}

		\item DialogHelper
			\begin{itemize}
				\item\smaller\texttt{select(\$question, \$choices, \$default = NULL, \$multiSelect = false, \$attempts = FALSE)}
				\item\texttt{ask(\$question, \$default = NULL, array \$autocomplete = array())}
				\item\texttt{askConfirmation(\$question, \$default = TRUE)}
				\item\texttt{askHiddenResponse(\$question, \$fallback = TRUE)}
				\item\texttt{askAndValidate(\$question, \$validator, \$attempts = FALSE, \$default = NULL, array \$autocomplete = NULL)}
				\item\texttt{askHiddenResponseAndValidate(\$question, \$validator, \$attempts = FALSE, \$fallback = TRUE)}
			\end{itemize}

	\end{itemize}

\end{frame}

% ------------------------------------------------------------------------------
% LTXE-SLIDE-START
% LTXE-SLIDE-UID:		ab6457d6-d409ead0-39a9926f-06911cad
% LTXE-SLIDE-ORIGIN:	a24edba5-e325709b-54cabffa-5ba42de7 English
% LTXE-SLIDE-ORIGIN:	3842b75e-8a44920f-a8c139e5-bd03e141 German
% LTXE-SLIDE-TITLE:		Feature #59606: Integrate Symfony/Console into CommandController (2)
% LTXE-SLIDE-REFERENCE:	Feature-59606-IntegrateSymfonyConsoleIntoCommandController.rst
% ------------------------------------------------------------------------------

\begin{frame}[fragile]
	\frametitle{Modifiche rilevanti}
	\framesubtitle{Integrazione Symfony/Console in CommandController (2)}

	% decrease font size for code listing
	\lstset{basicstyle=\tiny\ttfamily}

	\begin{itemize}
		\item \smaller ProgressHelper
			\begin{itemize}
				\item \smaller\texttt{progressStart(\$max = NULL)}
				\item \texttt{progressSet(\$current)}
				\item \texttt{progressAdvance(\$step = 1)}
				\item \texttt{progressFinish()}
			\end{itemize}
	\end{itemize}

	\smaller
		(vedi le slide seguenti per esempi di codice)
	\normalsize

\end{frame}

% ------------------------------------------------------------------------------
% LTXE-SLIDE-START
% LTXE-SLIDE-UID:		3a356fd7-190739a2-4f400aca-98c9e835
% LTXE-SLIDE-ORIGIN:	26ee64a3-a8f3ef0d-247faf3b-7df91473 English
% LTXE-SLIDE-ORIGIN:	a8c139e5-8a44920f-3842b75e-bd03e141 German
% LTXE-SLIDE-TITLE:		Feature #59606: Integrate Symfony/Console into CommandController (3)
% LTXE-SLIDE-REFERENCE:	Feature-59606-IntegrateSymfonyConsoleIntoCommandController.rst
% ------------------------------------------------------------------------------

\begin{frame}[fragile]
	\frametitle{In-Depth Changes}
	\framesubtitle{Integrazione Symfony/Console in CommandController (3)}

	% decrease font size for code listing
	\lstset{basicstyle=\tiny\ttfamily}

		\begin{lstlisting}
			<?php
			namespace Acme\Demo\Command;
			use TYPO3\CMS\Extbase\Mvc\Controller\CommandController;

			class MyCommandController extends CommandController {
			  public function myCommand() {

			    // render a table
			    $this->output->outputTable(array(
			      array('Bob', 34, 'm'),
			      array('Sally', 21, 'f'),
			      array('Blake', 56, 'm')
			    ),
			    array('Name', 'Age', 'Gender'));

			    // select
			    $colors = array('red', 'blue', 'yellow');
			    $selectedColorIndex = $this->output->select('Please select one color', $colors, 'red');
			    $this->outputLine('You choose the color %s.', array($colors[$selectedColorIndex]));

			    [...]
		\end{lstlisting}

\end{frame}

% ------------------------------------------------------------------------------
% LTXE-SLIDE-START
% LTXE-SLIDE-UID:		ba03d117-b05870ee-9fe68ad5-6d537327
% LTXE-SLIDE-ORIGIN:	e7f1a1e9-cabd20ae-78d335fc-3868ca27 English
% LTXE-SLIDE-ORIGIN:	a8c139e5-8a44920f-3842b75e-bd03e141 German
% LTXE-SLIDE-TITLE:		Feature #59606: Integrate Symfony/Console into CommandController (4)
% LTXE-SLIDE-REFERENCE:	Feature-59606-IntegrateSymfonyConsoleIntoCommandController.rst
% ------------------------------------------------------------------------------

\begin{frame}[fragile]
	\frametitle{Modifiche rilevanti}
	\framesubtitle{Integrazione Symfony/Console in CommandController (4)}

	% decrease font size for code listing
	\lstset{basicstyle=\tiny\ttfamily}

		\begin{lstlisting}
			    [...]
			    // ask
			    $name = $this->output->ask('What is your name?' . PHP_EOL, 'Bob', array('Bob', 'Sally', 'Blake'));
			    $this->outputLine('Hello %s.', array($name));

			    // prompt
			    $likesDogs = $this->output->askConfirmation('Do you like dogs?');
			    if ($likesDogs) {
			      $this->outputLine('You do like dogs!');
			    }

			    // progress
			    $this->output->progressStart(600);
			    for ($i = 0; $i < 300; $i ++) {
			      $this->output->progressAdvance();
			      usleep(5000);
			    }
			    $this->output->progressFinish();
			  }
			}
			?>
		\end{lstlisting}

\end{frame}

% ------------------------------------------------------------------------------
% LTXE-SLIDE-START
% LTXE-SLIDE-UID:		71f4e059-3dd59cb3-b24782c0-beea6e9d
% LTXE-SLIDE-ORIGIN:	b9b53d17-7e2c0c83-dabec46a-1a756e39 English
% LTXE-SLIDE-ORIGIN:	9d56ca19-04a98896-ba0a973c-8a16e877 German
% LTXE-SLIDE-TITLE:		Feature #66669: BE login form API (1)
% LTXE-SLIDE-REFERENCE:	Feature-66669-BeLoginFormAPI.rst
% ------------------------------------------------------------------------------

\begin{frame}[fragile]
	\frametitle{Modifiche rilevanti}
	\framesubtitle{API per Login di Backend (1)}

	% decrease font size for code listing
	\lstset{basicstyle=\tiny\ttfamily}

	\begin{itemize}

		\item Il login di Backend è stato completamente rifatto e sono state introdotte delle nuove API

		\item Il form OpenID è stato separato ed ora utilizza le nuove API
			(rendendolo indipendente dalle classi del Core)

		\item Il concetto del nuovo login di backend si basa su "login providers", che possono essere
			registrati nel file \texttt{ext\_localconf.php} come segue:

			\begin{lstlisting}
				$GLOBALS['TYPO3_CONF_VARS']['EXTCONF']['backend']['loginProviders'][1433416020] = [
				  'provider' => \TYPO3\CMS\Backend\LoginProvider\UsernamePasswordLoginProvider::class,
				  'sorting' => 50,
				  'icon-class' => 'fa-key',
				  'label' => 'LLL:EXT:backend/Resources/Private/Language/locallang.xlf:login.link'
				];
			\end{lstlisting}

	\end{itemize}

\end{frame}

% ------------------------------------------------------------------------------
% LTXE-SLIDE-START
% LTXE-SLIDE-UID:		bbc3169f-41bb5dd8-821291ba-ff010ccb
% LTXE-SLIDE-ORIGIN:	c443db74-ed3267f9-a113f486-d586081d English
% LTXE-SLIDE-ORIGIN:	9889604a-ca199d56-73cba0a9-e8778a16 German
% LTXE-SLIDE-TITLE:		Feature #66669: BE login form API (2)
% LTXE-SLIDE-REFERENCE:	Feature-66669-BeLoginFormAPI.rst
% ------------------------------------------------------------------------------

\begin{frame}[fragile]
	\frametitle{Modifiche rilevanti}
	\framesubtitle{API per Login di Backend (2)}

	\begin{itemize}

		\item Le opzioni sono definite come di seguito:

			\begin{itemize}

				\item \texttt{provider}:\newline
					login provider class name, deve essere implementata
						\texttt{TYPO3\textbackslash
							CMS\textbackslash
							Backend\textbackslash
							LoginProvider\textbackslash
							LoginProviderInterface}

				\item \texttt{sorting}:\newline
					ordinamento dei link ai possibili fornitori di login nella schermata di login

				\item \texttt{icon-class}:\newline
					nome dell'icona font-awesome per il collegamento nella schermata di login
					
				\item \texttt{label}:\newline
					testo del link per il fornitore di login nella schermata di accesso
			\end{itemize}

	\end{itemize}

\end{frame}

% ------------------------------------------------------------------------------
% LTXE-SLIDE-START
% LTXE-SLIDE-UID:		fa758f46-0c41bf5b-dff0cfd0-dbfadc92
% LTXE-SLIDE-ORIGIN:	66005bf1-6706e6ba-98feeb3f-ac972c3d English
% LTXE-SLIDE-ORIGIN:	04a98896-ba0a973c-8a16e877-9d56ca19 German
% LTXE-SLIDE-TITLE:		Feature #66669: BE login form API (3)
% LTXE-SLIDE-REFERENCE:	Feature-66669-BeLoginFormAPI.rst
% ------------------------------------------------------------------------------

\begin{frame}[fragile]
	\frametitle{Modifiche rilevanti}
	\framesubtitle{API per Login di Backend (3)}

	\begin{itemize}

		\item La \texttt{LoginProviderInterface} contiene solamente il metodo\newline
			\smaller\texttt{public function render(StandaloneView \$view, PageRenderer \$pageRenderer, LoginController \$loginController);}\normalsize

		\item I parametri sono definiti come di seguito:

			\begin{itemize}

				\item \texttt{\$view}:\newline
					Fluid StandaloneView che definisce la schermata di login. E' necessario impostare
					il file di template e si possono aggiungere le variabili alla vista secondo le 
					proprie esigenze.

				\item \texttt{\$pageRenderer}:\newline
					L'istanza PageRenderer permette di aggiungere risorse JavaScript
					necessarie.

				\item \texttt{\$loginController}:\newline
					L'istanza LoginController.

			\end{itemize}

	\end{itemize}

\end{frame}

% ------------------------------------------------------------------------------
% LTXE-SLIDE-START
% LTXE-SLIDE-UID:		0b9b0252-97a16f50-7e171f87-33bf11be
% LTXE-SLIDE-ORIGIN:	dd0c782c-161abebc-ba03f841-6d108444 English
% LTXE-SLIDE-ORIGIN:	04a98896-ba0a973c-8a16e877-9d56ca19 German
% LTXE-SLIDE-TITLE:		Feature #66669: BE login form API (4)
% LTXE-SLIDE-REFERENCE:	Feature-66669-BeLoginFormAPI.rst
% ------------------------------------------------------------------------------

\begin{frame}[fragile]
	\frametitle{Modifiche rilevanti}
	\framesubtitle{API per Login di Backend (4)}

	% decrease font size for code listing
	\lstset{basicstyle=\tiny\ttfamily}

	\begin{itemize}

		\item Il template deve contenere \texttt{<f:layout name="Login">} e
			\texttt{<f:section name="loginFormFields">} (per i campi del modulo):

			\begin{lstlisting}
				<f:layout name="Login" />
				<f:section name="loginFormFields">
				  <div class="form-group t3js-login-openid-section" id="t3-login-openid_url-section">
				    <div class="input-group">
				      <input type="text" id="openid_url"
				        name="openid_url"
				        value="{presetOpenId}"
				        autofocus="autofocus"
				        placeholder="{f:translate(key: 'openId', extensionName: 'openid')}"
				        class="form-control input-login t3js-clearable t3js-login-openid-field" />
				      <div class="input-group-addon">
				        <span class="fa fa-openid"></span>
				      </div>
				    </div>
				  </div>
				</f:section>
			\end{lstlisting}

	\end{itemize}

\end{frame}

% ------------------------------------------------------------------------------
% LTXE-SLIDE-START
% LTXE-SLIDE-UID:		1242f8d5-7f46b66a-60271149-1a7bfabe
% LTXE-SLIDE-ORIGIN:	2051c5f1-bd00622f-dfcbba05-c329dd60 English
% LTXE-SLIDE-ORIGIN:	cc63cd98-524ee6df-38430963-f7c7067c German
% LTXE-SLIDE-TITLE:		Feature #66681: CategoryRegistry: add options to set l10n_mode and l10n_display
% LTXE-SLIDE-REFERENCE:	Feature-66681-CategoryRegistryAddOptionsToSetL10n_modeAndL10n_display.rst
% ------------------------------------------------------------------------------

\begin{frame}[fragile]
	\frametitle{Modifiche rilevanti}
	\framesubtitle{Nuove opzioni per \texttt{CategoryRegistry}}

	% decrease font size for code listing
	\lstset{basicstyle=\tiny\ttfamily}

	\begin{itemize}

		\item Il metodo \texttt{CategoryRegistry->addTcaColumn} gestisce le opzioni per impostare
			\texttt{l10n\_mode} e \texttt{l10n\_display}:

			\begin{lstlisting}
				\TYPO3\CMS\Core\Utility\ExtensionManagementUtility::makeCategorizable(
				  $extensionKey,
				  $tableName,
				  'categories',
				  array(
				    'l10n_mode' => 'string (keyword)',
				    'l10n_display' => 'list of keywords'
				  )
				);
			\end{lstlisting}

	\end{itemize}

\end{frame}

% ------------------------------------------------------------------------------
% LTXE-SLIDE-START
% LTXE-SLIDE-UID:		24b499e5-a8d9bd1d-cbcd6905-2fd9702a
% LTXE-SLIDE-ORIGIN:	52d9c9a1-37301644-b52c1658-372774a8 English
% LTXE-SLIDE-ORIGIN:	c9a5288c-f45aac1e-34606612-fdf6ec18 German
% LTXE-SLIDE-TITLE:		Feature #66822: Allow Sprites For Backend Modules
% LTXE-SLIDE-REFERENCE:	Feature-66822-SpriteIconsInBackendModules.rst
% ------------------------------------------------------------------------------

\begin{frame}[fragile]
	\frametitle{Modifiche rilevanti}
	\framesubtitle{Sprites nei moduli di Backend}

	% decrease font size for code listing
	\lstset{basicstyle=\tiny\ttfamily}

	\begin{itemize}

		\item I moduli di Backend (i moduli principali come ad esempio "Web" e i sottomoduli come
			"Filelist") possono ora utilizzare le icone sprites\newline
			\small
				(solo le icone sprite conosciute da TYPO3 sono disponibili!)
			\normalsize

		\item Esempio:

			\begin{lstlisting}
				\TYPO3\CMS\Core\Utility\ExtensionManagementUtility::addModule(
				  'web',
				  'layout',
				  'top',
				  \TYPO3\CMS\Core\Utility\ExtensionManagementUtility::extPath($_EXTKEY) . 'Modules/Layout/',
				  array(
				    'script' => '_DISPATCH',
				    'access' => 'user,group',
				    'name' => 'web_layout',
				    'configuration' => array('icon' => 'module-web'),
				    'labels' => array(
				      'll_ref' => 'LLL:EXT:cms/layout/locallang_mod.xlf',
				    ),
				  )
				);
			\end{lstlisting}

	\end{itemize}

\end{frame}

% ------------------------------------------------------------------------------
% LTXE-SLIDE-START
% LTXE-SLIDE-UID:		3956e5fc-af1f873a-fdbed5d2-70f50582
% LTXE-SLIDE-ORIGIN:	8d50a473-76b35cf0-4414be81-8efd94f3 English
% LTXE-SLIDE-ORIGIN:	d9657b5d-5cd56c65-d7266a27-401dd233 German
% LTXE-SLIDE-TITLE:		Feature #67229: FormEngine NodeFactory API
% LTXE-SLIDE-REFERENCE:	Feature-67229-FormEngineNodeFactoryApi.rst
% ------------------------------------------------------------------------------

\begin{frame}[fragile]
	\frametitle{Modifiche rilevanti}
	\framesubtitle{FormEngine NodeFactory API (1)}

	% decrease font size for code listing
	\lstset{basicstyle=\tiny\ttfamily}

	\begin{itemize}

		\item Ora è possibile registrare nuovi nodi e sovrascrivere quelli esistenti

		\begin{lstlisting}
			$GLOBALS['TYPO3_CONF_VARS']['SYS']['formEngine']['nodeRegistry'][1433196792] = array(
			  'nodeName' => 'input',
			  'priority' => 40,
			  'class' => \MyVendor\MyExtension\Form\Element\T3editorElement::class
			);
		\end{lstlisting}

		\item L'esempio sopra registra una nuova classe
			\texttt{MyVendor\textbackslash
				MyExtension\textbackslash
				Form\textbackslash
				Element\textbackslash
				T3editorElement}
			che renderizza la classe per il tipo \texttt{input} di TCA, il quale deve implementare l'interfaccia
			\texttt{TYPO3\textbackslash
				CMS\textbackslash
				Backend\textbackslash
				Form\textbackslash
				NodeInterface}

		\item La chiave dell'array è lo Unix timestamp della data di quando è stato registrato l'elemento aggiunto

	\end{itemize}

\end{frame}

% ------------------------------------------------------------------------------
% LTXE-SLIDE-START
% LTXE-SLIDE-UID:		260b96ae-95858a00-6108b569-24740af5
% LTXE-SLIDE-ORIGIN:	94bf7d97-b60d2c04-a1693909-e7397a81 English
% LTXE-SLIDE-ORIGIN:	5cd56c65-d7266a27-401dd233-d9657b5d German
% LTXE-SLIDE-TITLE:		Feature #67229: FormEngine NodeFactory API
% LTXE-SLIDE-REFERENCE:	Feature-67229-FormEngineNodeFactoryApi.rst
% ------------------------------------------------------------------------------

\begin{frame}[fragile]
	\frametitle{Modifiche rilevanti}
	\framesubtitle{FormEngine NodeFactory API (2)}

	% decrease font size for code listing
	\lstset{basicstyle=\tiny\ttfamily}

	\begin{itemize}

		\item Nel caso siano registrati elementi multipli dello stesso tipo, viene utilizzato quello con lapriorità più alta (da 0 a 100)

		\item Un nuovo tipo TCA può essere registrato come segue

		\smaller\textbf{\texttt{TCA}}
		\begin{lstlisting}
			'columns' => array(
			  'bodytext' => array(
			    'config' => array(
			      'type' => 'text',
			      'renderType' => '3dCloud',
			    ),
			  ),
			)
		\end{lstlisting}

		\smaller\textbf{\texttt{ext\_localconf.php}}
		\begin{lstlisting}
			$GLOBALS['TYPO3_CONF_VARS']['SYS']['formEngine']['nodeRegistry'][1433197759] = array(
			  'nodeName' => '3dCloud',
			  'priority' => 40,
			  'class' => \MyVendor\MyExtension\Form\Element\ShowTextAs3dCloudElement::class
			);
		\end{lstlisting}

	\end{itemize}

\end{frame}

% ------------------------------------------------------------------------------
% LTXE-SLIDE-START
% LTXE-SLIDE-UID:		0db93595-7b7fae01-c45a3f81-5ab9ba82
% LTXE-SLIDE-ORIGIN:	818411e6-dd2a7aba-3a7afb94-4e970169 English
% LTXE-SLIDE-ORIGIN:	d7266a27-401dd233-d9657b5d-5cd56c65 German
% LTXE-SLIDE-TITLE:		Breaking #62983: postProcessMirrorUrl signal has moved
% LTXE-SLIDE-REFERENCE:	Breaking-62983-PostProcessMirrorUrlSignalHasMoved.rst
% ------------------------------------------------------------------------------

\begin{frame}[fragile]
	\frametitle{Modifiche rilevanti}
	\framesubtitle{Signal \texttt{postProcessMirrorUrl}}

	% decrease font size for code listing
	\lstset{basicstyle=\tiny\ttfamily}

	\begin{itemize}

		\item il segnale \texttt{postProcessMirrorUrl} è stato spostato in una nuova classe

		\breakingchange

		\item Il seguente codice di esempio verifica la versione di TYPO3:

		\begin{lstlisting}
			$signalSlotDispatcher->connect(
			  version_compare(TYPO3_version, '7.0', '<')
			    ? 'TYPO3\\CMS\\Lang\\Service\\UpdateTranslationService'
			    : 'TYPO3\\CMS\\Lang\\Service\\TranslationService',
			  'postProcessMirrorUrl',
			  'Vendor\\Extension\\Slots\\CustomMirror',
			  'postProcessMirrorUrl'
			);
		\end{lstlisting}

	\end{itemize}

\end{frame}

% ------------------------------------------------------------------------------
