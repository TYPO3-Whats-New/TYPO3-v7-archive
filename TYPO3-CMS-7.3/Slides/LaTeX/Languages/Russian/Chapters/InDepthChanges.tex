% ------------------------------------------------------------------------------
% TYPO3 CMS 7.3 - What's New - Chapter "In-Depth Changes" (English Version)
%
% @author	Michael Schams <schams.net>
% @license	Creative Commons BY-NC-SA 3.0
% @link		http://typo3.org/download/release-notes/whats-new/
% @language	English
% ------------------------------------------------------------------------------
% LTXE-CHAPTER-UID:		e8cc33e1-d7bdb74f-0421b75f-7be06b19
% LTXE-CHAPTER-NAME:	In-Depth Changes
% ------------------------------------------------------------------------------

\section{Глубинные изменения}
\begin{frame}[fragile]
	\frametitle{Глубинные изменения}

	\begin{center}\huge{Глава 3:}\end{center}
	\begin{center}\huge{\color{typo3darkgrey}\textbf{Глубинные изменения}}\end{center}

\end{frame}

% ------------------------------------------------------------------------------
% LTXE-SLIDE-START
% LTXE-SLIDE-UID:		a9947afe-03f82edb-0191c538-4f5589c8
% LTXE-SLIDE-ORIGIN:	17eaaa35-2a21b728-4c3488a0-30a7ac7b English
% LTXE-SLIDE-ORIGIN:	bd03e141-3842b75e-8a44920f-a8c139e5 German
% LTXE-SLIDE-TITLE:		Feature #59606: Integrate Symfony/Console into CommandController (1)
% LTXE-SLIDE-REFERENCE:	Feature-59606-IntegrateSymfonyConsoleIntoCommandController.rst
% ------------------------------------------------------------------------------

\begin{frame}[fragile]
	\frametitle{Глубинные изменения}\frametitle{Глубинные изменения}
	\framesubtitle{Интеграция Symfony/Console в CommandController (1)}

	% decrease font size for code listing
	\lstset{basicstyle=\tiny\ttfamily}

	Теперь CommandController позволяет использовать Symfony/Console изнутри и предоставляет разные
	методы:

	\begin{itemize}

		\item \smaller TableHelper
			\begin{itemize}
				\item\smaller\texttt{outputTable(\$rows, \$headers = NULL)}
			\end{itemize}

		\item DialogHelper
			\begin{itemize}
				\item\smaller\texttt{select(\$question, \$choices, \$default = NULL, \$multiSelect = false, \$attempts = FALSE)}
				\item\texttt{ask(\$question, \$default = NULL, array \$autocomplete = array())}
				\item\texttt{askConfirmation(\$question, \$default = TRUE)}
				\item\texttt{askHiddenResponse(\$question, \$fallback = TRUE)}
				\item\texttt{askAndValidate(\$question, \$validator, \$attempts = FALSE, \$default = NULL, array \$autocomplete = NULL)}
				\item\texttt{askHiddenResponseAndValidate(\$question, \$validator, \$attempts = FALSE, \$fallback = TRUE)}
			\end{itemize}

	\end{itemize}

\end{frame}

% ------------------------------------------------------------------------------
% LTXE-SLIDE-START
% LTXE-SLIDE-UID:		ba777f79-5e9e9a71-472c150e-6f83735d
% LTXE-SLIDE-ORIGIN:	a24edba5-e325709b-54cabffa-5ba42de7 English
% LTXE-SLIDE-ORIGIN:	3842b75e-8a44920f-a8c139e5-bd03e141 German
% LTXE-SLIDE-TITLE:		Feature #59606: Integrate Symfony/Console into CommandController (2)
% LTXE-SLIDE-REFERENCE:	Feature-59606-IntegrateSymfonyConsoleIntoCommandController.rst
% ------------------------------------------------------------------------------

\begin{frame}[fragile]
	\frametitle{Глубинные изменения}
	\framesubtitle{Интеграция Symfony/Console в CommandController (2)}

	% decrease font size for code listing
	\lstset{basicstyle=\tiny\ttfamily}

	\begin{itemize}
		\item \smaller ProgressHelper
			\begin{itemize}
				\item \smaller\texttt{progressStart(\$max = NULL)}
				\item \texttt{progressSet(\$current)}
				\item \texttt{progressAdvance(\$step = 1)}
				\item \texttt{progressFinish()}
			\end{itemize}
	\end{itemize}

	\smaller
		(примеры кода на следующих слайдах)
	\normalsize

\end{frame}

% ------------------------------------------------------------------------------
% LTXE-SLIDE-START
% LTXE-SLIDE-UID:		c1242f5f-b0c108ba-863f5e22-a96e34b8
% LTXE-SLIDE-ORIGIN:	26ee64a3-a8f3ef0d-247faf3b-7df91473 English
% LTXE-SLIDE-ORIGIN:	a8c139e5-8a44920f-3842b75e-bd03e141 German
% LTXE-SLIDE-TITLE:		Feature #59606: Integrate Symfony/Console into CommandController (3)
% LTXE-SLIDE-REFERENCE:	Feature-59606-IntegrateSymfonyConsoleIntoCommandController.rst
% ------------------------------------------------------------------------------

\begin{frame}[fragile]
	\frametitle{Глубинные изменения}
	\framesubtitle{Интеграция Symfony/Console в CommandController (3)}

	% decrease font size for code listing
	\lstset{basicstyle=\tiny\ttfamily}

		\begin{lstlisting}
			<?php
			namespace Acme\Demo\Command;
			use TYPO3\CMS\Extbase\Mvc\Controller\CommandController;

			class MyCommandController extends CommandController {
			  public function myCommand() {

			    // render a table
			    $this->output->outputTable(array(
			      array('Bob', 34, 'm'),
			      array('Sally', 21, 'f'),
			      array('Blake', 56, 'm')
			    ),
			    array('Name', 'Age', 'Gender'));

			    // select
			    $colors = array('red', 'blue', 'yellow');
			    $selectedColorIndex = $this->output->select('Please select one color', $colors, 'red');
			    $this->outputLine('You choose the color %s.', array($colors[$selectedColorIndex]));

			    [...]
		\end{lstlisting}

\end{frame}

% ------------------------------------------------------------------------------
% LTXE-SLIDE-START
% LTXE-SLIDE-UID:		b4d65c27-2e1703ad-a9061dff-f41716e4
% LTXE-SLIDE-ORIGIN:	e7f1a1e9-cabd20ae-78d335fc-3868ca27 English
% LTXE-SLIDE-ORIGIN:	a8c139e5-8a44920f-3842b75e-bd03e141 German
% LTXE-SLIDE-TITLE:		Feature #59606: Integrate Symfony/Console into CommandController (4)
% LTXE-SLIDE-REFERENCE:	Feature-59606-IntegrateSymfonyConsoleIntoCommandController.rst
% ------------------------------------------------------------------------------

\begin{frame}[fragile]
	\frametitle{Глубинные изменения}
	\framesubtitle{Интеграция Symfony/Console в CommandController (4)}

	% decrease font size for code listing
	\lstset{basicstyle=\tiny\ttfamily}

		\begin{lstlisting}
			    [...]
			    // ask
			    $name = $this->output->ask('What is your name?' . PHP_EOL, 'Bob', array('Bob', 'Sally', 'Blake'));
			    $this->outputLine('Hello %s.', array($name));

			    // prompt
			    $likesDogs = $this->output->askConfirmation('Do you like dogs?');
			    if ($likesDogs) {
			      $this->outputLine('You do like dogs!');
			    }

			    // progress
			    $this->output->progressStart(600);
			    for ($i = 0; $i < 300; $i ++) {
			      $this->output->progressAdvance();
			      usleep(5000);
			    }
			    $this->output->progressFinish();
			  }
			}
			?>
		\end{lstlisting}

\end{frame}

% ------------------------------------------------------------------------------
% LTXE-SLIDE-START
% LTXE-SLIDE-UID:		3191324d-ca414b68-bc81e785-f00a462e
% LTXE-SLIDE-ORIGIN:	b9b53d17-7e2c0c83-dabec46a-1a756e39 English
% LTXE-SLIDE-ORIGIN:	9d56ca19-04a98896-ba0a973c-8a16e877 German
% LTXE-SLIDE-TITLE:		Feature #66669: BE login form API (1)
% LTXE-SLIDE-REFERENCE:	Feature-66669-BeLoginFormAPI.rst
% ------------------------------------------------------------------------------

\begin{frame}[fragile]
	\frametitle{Глубинные изменения}
	\framesubtitle{Backend Login API (1)}

	% decrease font size for code listing
	\lstset{basicstyle=\tiny\ttfamily}

	\begin{itemize}

		\item Полностью переработана авторизация во внутреннем интерфейсе и представлен новый API

		\item Извлечена форма OpenID, которая теперь использует новый API
			(давая независимость от главных классов ядра)

		\item Новая концепция авторизации базируется на "провайдерах авторизации", которые можно зарегистрировать
			в файле \texttt{ext\_localconf.php} следующим образом:

			\begin{lstlisting}
				$GLOBALS['TYPO3_CONF_VARS']['EXTCONF']['backend']['loginProviders'][1433416020] = [
				  'provider' => \TYPO3\CMS\Backend\LoginProvider\UsernamePasswordLoginProvider::class,
				  'sorting' => 50,
				  'icon-class' => 'fa-key',
				  'label' => 'LLL:EXT:backend/Resources/Private/Language/locallang.xlf:login.link'
				];
			\end{lstlisting}

	\end{itemize}

\end{frame}

% ------------------------------------------------------------------------------
% LTXE-SLIDE-START
% LTXE-SLIDE-UID:		01c3c579-4ee24f97-4e083a26-86045700
% LTXE-SLIDE-ORIGIN:	c443db74-ed3267f9-a113f486-d586081d English
% LTXE-SLIDE-ORIGIN:	9889604a-ca199d56-73cba0a9-e8778a16 German
% LTXE-SLIDE-TITLE:		Feature #66669: BE login form API (2)
% LTXE-SLIDE-REFERENCE:	Feature-66669-BeLoginFormAPI.rst
% ------------------------------------------------------------------------------

\begin{frame}[fragile]
	\frametitle{Глубинные изменения}
	\framesubtitle{Backend Login API (2)}

	\begin{itemize}

		\item Параметры задаются следующим образом:

			\begin{itemize}

				\item \texttt{provider}:\newline
					название класса провайдера, который должен реализовывать
						\texttt{TYPO3\textbackslash
							CMS\textbackslash
							Backend\textbackslash
							LoginProvider\textbackslash
							LoginProviderInterface}

				\item \texttt{sorting}:\newline
					порядок ссылок на возможные провайдеры авторизации на форме авторизации

				\item \texttt{icon-class}:\newline
					название значка-надписи для сылки на форме авторизации

				\item \texttt{label}:\newline
					метка для ссылки на провадера авторизации для формы авторизации
			\end{itemize}

	\end{itemize}

\end{frame}

% ------------------------------------------------------------------------------
% LTXE-SLIDE-START
% LTXE-SLIDE-UID:		07ceda5b-00787940-4499d835-b3125873
% LTXE-SLIDE-ORIGIN:	66005bf1-6706e6ba-98feeb3f-ac972c3d English
% LTXE-SLIDE-ORIGIN:	04a98896-ba0a973c-8a16e877-9d56ca19 German
% LTXE-SLIDE-TITLE:		Feature #66669: BE login form API (3)
% LTXE-SLIDE-REFERENCE:	Feature-66669-BeLoginFormAPI.rst
% ------------------------------------------------------------------------------

\begin{frame}[fragile]
	\frametitle{Глубинные изменения}
	\framesubtitle{Backend Login API (3)}

	\begin{itemize}

		\item The \texttt{LoginProviderInterface} содержит лишь метод\newline
			\smaller\texttt{public function render(StandaloneView \$view, PageRenderer \$pageRenderer, LoginController \$loginController);}\normalsize

		\item Параметры определяются таким образом:

			\begin{itemize}

				\item \texttt{\$view}:\newline
					Fluid StandaloneView выводящий форму авторизации. Необходимо указать
					файл шаблона, а также, по необходимости, добавить переменные для собственных нужд.

				\item \texttt{\$pageRenderer}:\newline
					Экземпляр PageRenderer имеет возможность добавления необходимых ресурсов JavaScript.

				\item \texttt{\$loginController}:\newline
					Экземпляр LoginController.

			\end{itemize}

	\end{itemize}

\end{frame}

% ------------------------------------------------------------------------------
% LTXE-SLIDE-START
% LTXE-SLIDE-UID:		a9df140b-e911e753-d0d67c1e-02ce8af6
% LTXE-SLIDE-ORIGIN:	dd0c782c-161abebc-ba03f841-6d108444 English
% LTXE-SLIDE-ORIGIN:	04a98896-ba0a973c-8a16e877-9d56ca19 German
% LTXE-SLIDE-TITLE:		Feature #66669: BE login form API (4)
% LTXE-SLIDE-REFERENCE:	Feature-66669-BeLoginFormAPI.rst
% ------------------------------------------------------------------------------

\begin{frame}[fragile]
	\frametitle{Глубинные изменения}
	\framesubtitle{Backend Login API (4)}

	% decrease font size for code listing
	\lstset{basicstyle=\tiny\ttfamily}

	\begin{itemize}

		\item В шаблоне должен быть раздел \texttt{<f:layout name="Login">} и
			\texttt{<f:section name="loginFormFields">} (для полей формы):

			\begin{lstlisting}
				<f:layout name="Login" />
				<f:section name="loginFormFields">
				  <div class="form-group t3js-login-openid-section" id="t3-login-openid_url-section">
				    <div class="input-group">
				      <input type="text" id="openid_url"
				        name="openid_url"
				        value="{presetOpenId}"
				        autofocus="autofocus"
				        placeholder="{f:translate(key: 'openId', extensionName: 'openid')}"
				        class="form-control input-login t3js-clearable t3js-login-openid-field" />
				      <div class="input-group-addon">
				        <span class="fa fa-openid"></span>
				      </div>
				    </div>
				  </div>
				</f:section>
			\end{lstlisting}

	\end{itemize}

\end{frame}

% ------------------------------------------------------------------------------
% LTXE-SLIDE-START
% LTXE-SLIDE-UID:		726e8161-40be933c-06f34654-351739ab
% LTXE-SLIDE-ORIGIN:	2051c5f1-bd00622f-dfcbba05-c329dd60 English
% LTXE-SLIDE-ORIGIN:	cc63cd98-524ee6df-38430963-f7c7067c German
% LTXE-SLIDE-TITLE:		Feature #66681: CategoryRegistry: add options to set l10n_mode and l10n_display
% LTXE-SLIDE-REFERENCE:	Feature-66681-CategoryRegistryAddOptionsToSetL10n_modeAndL10n_display.rst
% ------------------------------------------------------------------------------

\begin{frame}[fragile]
	\frametitle{Глубинные изменения}
	\framesubtitle{\texttt{CategoryRegistry} с новыми параметрами}

	% decrease font size for code listing
	\lstset{basicstyle=\tiny\ttfamily}

	\begin{itemize}

		\item Метод \texttt{CategoryRegistry->addTcaColumn} получает параметры для задания
			\texttt{l10n\_mode} и \texttt{l10n\_display}:

			\begin{lstlisting}
				\TYPO3\CMS\Core\Utility\ExtensionManagementUtility::makeCategorizable(
				  $extensionKey,
				  $tableName,
				  'categories',
				  array(
				    'l10n_mode' => 'string (keyword)',
				    'l10n_display' => 'list of keywords'
				  )
				);
			\end{lstlisting}

	\end{itemize}

\end{frame}

% ------------------------------------------------------------------------------
% LTXE-SLIDE-START
% LTXE-SLIDE-UID:		60646952-1585bf50-d3aa6d5e-270ed077
% LTXE-SLIDE-ORIGIN:	52d9c9a1-37301644-b52c1658-372774a8 English
% LTXE-SLIDE-ORIGIN:	c9a5288c-f45aac1e-34606612-fdf6ec18 German
% LTXE-SLIDE-TITLE:		Feature #66822: Allow Sprites For Backend Modules
% LTXE-SLIDE-REFERENCE:	Feature-66822-SpriteIconsInBackendModules.rst
% ------------------------------------------------------------------------------

\begin{frame}[fragile]
	\frametitle{Глубинные изменения}
	\framesubtitle{Спрайты в модулях внутреннего интерфейса}

	% decrease font size for code listing
	\lstset{basicstyle=\tiny\ttfamily}

	\begin{itemize}

		\item Модули внутреннего интерфейса (основные, вроде "Веб" и подмодули, вроде "Список файлов") 
		теперь могут использовать спрайты для значков\newline
			\small
				(можно использовать лишь спрайты, известные для TYPO3!)
			\normalsize

		\item Пример:

			\begin{lstlisting}
				\TYPO3\CMS\Core\Utility\ExtensionManagementUtility::addModule(
				  'web',
				  'layout',
				  'top',
				  \TYPO3\CMS\Core\Utility\ExtensionManagementUtility::extPath($_EXTKEY) . 'Modules/Layout/',
				  array(
				    'script' => '_DISPATCH',
				    'access' => 'user,group',
				    'name' => 'web_layout',
				    'configuration' => array('icon' => 'module-web'),
				    'labels' => array(
				      'll_ref' => 'LLL:EXT:cms/layout/locallang_mod.xlf',
				    ),
				  )
				);
			\end{lstlisting}

	\end{itemize}

\end{frame}

% ------------------------------------------------------------------------------
% LTXE-SLIDE-START
% LTXE-SLIDE-UID:		c8f00536-14b3c611-b7cd5cc9-26b88557
% LTXE-SLIDE-ORIGIN:	8d50a473-76b35cf0-4414be81-8efd94f3 English
% LTXE-SLIDE-ORIGIN:	d9657b5d-5cd56c65-d7266a27-401dd233 German
% LTXE-SLIDE-TITLE:		Feature #67229: FormEngine NodeFactory API
% LTXE-SLIDE-REFERENCE:	Feature-67229-FormEngineNodeFactoryApi.rst
% ------------------------------------------------------------------------------

\begin{frame}[fragile]
	\frametitle{Глубинные изменения}
	\framesubtitle{FormEngine NodeFactory API (1)}

	% decrease font size for code listing
	\lstset{basicstyle=\tiny\ttfamily}

	\begin{itemize}

		\item Теперь возможно зарегистрировать новые узлы и переназначать существующие

		\begin{lstlisting}
			$GLOBALS['TYPO3_CONF_VARS']['SYS']['formEngine']['nodeRegistry'][1433196792] = array(
			  'nodeName' => 'input',
			  'priority' => 40,
			  'class' => \MyVendor\MyExtension\Form\Element\T3editorElement::class
			);
		\end{lstlisting}

		\item Приведённый пример регистрирует новый класс
			\texttt{MyVendor\textbackslash
				MyExtension\textbackslash
				Form\textbackslash
				Element\textbackslash
				T3editorElement}
			в качестве класса, формирующего тип TCA \texttt{input}, который должен реализовывать интерфейс
			\texttt{TYPO3\textbackslash
				CMS\textbackslash
				Backend\textbackslash
				Form\textbackslash
				NodeInterface}

		\item Ключом массива служит Unix timestamp даты добавления регистрируемого элемента

	\end{itemize}

\end{frame}

% ------------------------------------------------------------------------------
% LTXE-SLIDE-START
% LTXE-SLIDE-UID:		d6f2eddf-6a817d94-81412de5-0d6638fd
% LTXE-SLIDE-ORIGIN:	94bf7d97-b60d2c04-a1693909-e7397a81 English
% LTXE-SLIDE-ORIGIN:	5cd56c65-d7266a27-401dd233-d9657b5d German
% LTXE-SLIDE-TITLE:		Feature #67229: FormEngine NodeFactory API
% LTXE-SLIDE-REFERENCE:	Feature-67229-FormEngineNodeFactoryApi.rst
% ------------------------------------------------------------------------------

\begin{frame}[fragile]
	\frametitle{Глубинные изменения}
	\framesubtitle{FormEngine NodeFactory API (2)}

	% decrease font size for code listing
	\lstset{basicstyle=\tiny\ttfamily}

	\begin{itemize}

		\item Если для обработки того же типа было зарегистрировано несколько элементов,
		используется имеющий высший приоритет (от 0 до 100)

		\item Новый тип TCA можно регистрировать следующим образом:

		\smaller\textbf{\texttt{TCA}}
		\begin{lstlisting}
			'columns' => array(
			  'bodytext' => array(
			    'config' => array(
			      'type' => 'text',
			      'renderType' => '3dCloud',
			    ),
			  ),
			)
		\end{lstlisting}

		\smaller\textbf{\texttt{ext\_localconf.php}}
		\begin{lstlisting}
			$GLOBALS['TYPO3_CONF_VARS']['SYS']['formEngine']['nodeRegistry'][1433197759] = array(
			  'nodeName' => '3dCloud',
			  'priority' => 40,
			  'class' => \MyVendor\MyExtension\Form\Element\ShowTextAs3dCloudElement::class
			);
		\end{lstlisting}

	\end{itemize}

\end{frame}

% ------------------------------------------------------------------------------
% LTXE-SLIDE-START
% LTXE-SLIDE-UID:		366c0b9f-cd51ea65-5e4c1acc-8a50ef88
% LTXE-SLIDE-ORIGIN:	818411e6-dd2a7aba-3a7afb94-4e970169 English
% LTXE-SLIDE-ORIGIN:	d7266a27-401dd233-d9657b5d-5cd56c65 German
% LTXE-SLIDE-TITLE:		Breaking #62983: postProcessMirrorUrl signal has moved
% LTXE-SLIDE-REFERENCE:	Breaking-62983-PostProcessMirrorUrlSignalHasMoved.rst
% ------------------------------------------------------------------------------

\begin{frame}[fragile]
	\frametitle{Глубинные изменения}
	\framesubtitle{Сигнал \texttt{postProcessMirrorUrl}}

	% decrease font size for code listing
	\lstset{basicstyle=\tiny\ttfamily}

	\begin{itemize}

		\item Сигнал \texttt{postProcessMirrorUrl} перемещён в новый класс

		\breakingchange

		\item В следующем примере кода принимается во внимание версия TYPO3:

		\begin{lstlisting}
			$signalSlotDispatcher->connect(
			  version_compare(TYPO3_version, '7.0', '<')
			    ? 'TYPO3\\CMS\\Lang\\Service\\UpdateTranslationService'
			    : 'TYPO3\\CMS\\Lang\\Service\\TranslationService',
			  'postProcessMirrorUrl',
			  'Vendor\\Extension\\Slots\\CustomMirror',
			  'postProcessMirrorUrl'
			);
		\end{lstlisting}

	\end{itemize}

\end{frame}

% ------------------------------------------------------------------------------
