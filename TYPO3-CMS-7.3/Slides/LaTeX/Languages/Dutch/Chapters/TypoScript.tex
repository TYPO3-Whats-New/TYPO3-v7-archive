% ------------------------------------------------------------------------------
% TYPO3 CMS 7.3 - What's New - Chapter "TypoScript" (English Version)
%
% @author	Michael Schams <schams.net>
% @license	Creative Commons BY-NC-SA 3.0
% @link		http://typo3.org/download/release-notes/whats-new/
% @language	English
% ------------------------------------------------------------------------------
% LTXE-CHAPTER-UID:		e8cc33e1-d7bdb74f-0421b75f-7be06b19
% LTXE-CHAPTER-NAME:	TypoScript
% ------------------------------------------------------------------------------

\section{TSconfig \& TypoScript}
\begin{frame}[fragile]
	\frametitle{TSconfig \& TypoScript}

	\begin{center}\huge{Hoofdstuk 2:}\end{center}
	\begin{center}\huge{\color{typo3darkgrey}\textbf{TSconfig \& TypoScript}}\end{center}

\end{frame}

% ------------------------------------------------------------------------------
% LTXE-SLIDE-START
% LTXE-SLIDE-UID:		ee7721fc-89fe826f-cb34fa39-6b8c0709
% LTXE-SLIDE-ORIGIN:	f8419191-d59094a8-01e156a1-908dac0f English
% LTXE-SLIDE-ORIGIN:	56b025e6-cf380b11-c9d1c009-0b548d6a German
% LTXE-SLIDE-TITLE:		Feature #63561: Add TypoScript stdWrap strtotime
% LTXE-SLIDE-REFERENCE:	Feature-63561-AddTypoScriptStdWrapStrtotime.rst
% ------------------------------------------------------------------------------
\begin{frame}[fragile]
	\frametitle{TSconfig \& TypoScript}
	\framesubtitle{Nieuwe stdWrap functie \texttt{strtotime}}

	\begin{itemize}

		\item De nieuwe stdWrap eigenschap \texttt{strtotime} zet geformatteerde datums om
			naar Unix timestamps, bijv. voor datum berekeningen

		\item Geldige waarden zijn \texttt{1} of elke tekst die gebruikt kan worden als eerste argument
			voor de PHP functie \texttt{strtotime()}

			\begin{lstlisting}
				date_as_timestamp = TEXT
				date_as_timestamp {
				  value = 2015-04-15
				  strtotime = 1
				}

				next_weekday = TEXT
				next_weekday {
				  data = GP:selected_date
				  strtotime = + 2 weekdays
				  strftime = %Y-%m-%d
				}
			\end{lstlisting}

	\end{itemize}

\end{frame}

% ------------------------------------------------------------------------------
% LTXE-SLIDE-START
% LTXE-SLIDE-UID:		25e4a837-c76e04cd-4841d46a-b6397f2a
% LTXE-SLIDE-ORIGIN:	687af5fc-b201bac6-e4ad678b-392fb9f8 English
% LTXE-SLIDE-ORIGIN:	c5dbf3c7-655fc6e4-227a3bf6-f5d89fb1 German
% LTXE-SLIDE-TITLE:		Feature #65250: TypoScript condition add GPmerged
% LTXE-SLIDE-REFERENCE:	Feature-65250-TypoScriptConditionAddGPmerged.rst
% ------------------------------------------------------------------------------

\begin{frame}[fragile]
	\frametitle{TSconfig \& TypoScript}
	\framesubtitle{\texttt{GPmerged} in Conditions}

	\begin{itemize}

		\item \text{GP} in TypoScript conditions geeft alleen de \text{POST}
			variabele, als \text{POST} \underline{en}
			\text{GET} variabelen aanwezig zijn

		\item De nieuwe optie \texttt{GPmerged} combineert \text{GET} en \text{POST}

			\begin{lstlisting}
				[globalVar = GPmerged:tx_demo|foo = 1]
				  page.90 = TEXT
				  page.90.value = DEMO
				[global]
			\end{lstlisting}

	\end{itemize}

\end{frame}

% ------------------------------------------------------------------------------
% LTXE-SLIDE-START
% LTXE-SLIDE-UID:		31bd0c82-68270bd6-6236394d-f8b4a226
% LTXE-SLIDE-ORIGIN:	2901d1f6-3d8e9a15-ad0268f1-4e30c04b English
% LTXE-SLIDE-ORIGIN:	05b2f338-7bcb651f-a4cebabd-2c677a20 German
% LTXE-SLIDE-TITLE:		Feature #66697: Add uppercamelcase and lowercamelcase to stdWrap.case
% LTXE-SLIDE-REFERENCE:	Feature-66697-AddUppercamelcaseAndLowercamelcaseToStdWrap.case.rst
% ------------------------------------------------------------------------------

\begin{frame}[fragile]
	\frametitle{TSconfig \& TypoScript}
	\framesubtitle{Nieuwe opties voor \texttt{stdWrap.case}}

	% decrease font size for code listing
	\lstset{basicstyle=\tiny\ttfamily}

	\begin{itemize}

		\item Opties \texttt{uppercamelcase} en \texttt{lowercamelcase} zijn toegevoegd aan
			\texttt{stdWrap.case}

		\item Example:

			\begin{lstlisting}
				tt_content = CASE
				tt_content {
				  key.field = CType
				  my_custom_ctype =< lib.userContent
				  my_custom_ctype {
				    file = EXT:site_base/Resources/Private/Templates/SomeOtherTemplate.html
				    settings.extraParam = 1
				  }
				  default =< lib.userContent
				  default {
				    file = TEXT
				    file.field = CType
				    file.stdWrap.case = uppercamelcase
				    file.wrap = EXT:site_base/Resources/Private/Templates/|.html
				  }
				}
			\end{lstlisting}

	\end{itemize}

\end{frame}

% ------------------------------------------------------------------------------
% LTXE-SLIDE-START
% LTXE-SLIDE-UID:		a20f1112-74ae6a6d-5961d2c6-a42a1239
% LTXE-SLIDE-ORIGIN:	72bacec0-2fa0303b-33c4c4d6-a3bc0857 English
% LTXE-SLIDE-ORIGIN:	cd745fc5-b8bd6483-ba1cf2a5-0ebf4310 German
% LTXE-SLIDE-TITLE:		Feature #66698: Add integrity property to JavaScript files (1)
% LTXE-SLIDE-REFERENCE:	Feature-66698-AddIntegrityPropertyToJavaScriptFiles.rst
% ------------------------------------------------------------------------------

\begin{frame}[fragile]
	\frametitle{TSconfig \& TypoScript}
	\framesubtitle{Eigenschap \texttt{integrity} toegevoegd voor JavaScript bestanden(1)}

	% decrease font size for code listing
	\lstset{basicstyle=\tiny\ttfamily}

	\begin{itemize}

		\item Nieuwe eigenschap \texttt{integrity} bij het invoegen van JS bestanden
			om een SRI hash in te stellen waarmee de bron geverifieerd wordt\newline
			(SRI: Sub-Resource Integrity, zie volgende pagina)

		\item Dit beïnvloedt de TypoScript PAGE eigenschappen \texttt{page.includeJSLibs},
			\texttt{page.includeJSFooterlibs}, \texttt{ includeJS} en
			\texttt{includeJSFooter}

		\item Voorbeeld:

			\begin{lstlisting}
				page {
				  includeJS {
				    jQuery = https://code.jquery.com/jquery-1.11.3.min.js
				    jquery.external = 1
				    jQuery.disableCompression = 1
				    jQuery.excludeFromConcatenation = 1
				    jQuery.integrity = sha256-7LkWEzqTdpEfELxcZZlS6wAx5Ff13zZ83lYO2/ujj7g=
				  }
				}
			\end{lstlisting}

	\end{itemize}

\end{frame}

% ------------------------------------------------------------------------------
% LTXE-SLIDE-START
% LTXE-SLIDE-UID:		cf439ef4-11873ee3-5500ce7e-af1b9474
% LTXE-SLIDE-ORIGIN:	bf190d17-80b961a2-a0c001d4-6ee48dfe English
% LTXE-SLIDE-ORIGIN:	b8bd6483-cd745fc5-ba1cf2a5-0ebf4310 German
% LTXE-SLIDE-TITLE:		Feature #66698: Add integrity property to JavaScript files (2)
% LTXE-SLIDE-REFERENCE:	Feature-66698-AddIntegrityPropertyToJavaScriptFiles.rst
% ------------------------------------------------------------------------------

\begin{frame}[fragile]
	\frametitle{TSconfig \& TypoScript}
	\framesubtitle{Eigenschap \texttt{integrity} toegevoegd voor JavaScript bestanden(2)}

	% decrease font size for code listing
	%\lstset{basicstyle=\tiny\ttfamily}

	\begin{itemize}

		\item SRI is een W3C specificatie waarmee vastgesteld kan worden dat bestanden op een server
		 	van derden niet gecompromiteerd zijn

		\item Aanmaken van integrity hashes:

			\begin{itemize}
				\item Optie 1: \url{https://srihash.org}
				\item Optie 2: gebruik het volgende shell commando
			\end{itemize}

			\begin{lstlisting}
				cat FILENAME.js | openssl dgst -sha256 -binary | openssl enc -base64 -A
			\end{lstlisting}

		\item Meer informatie:

			\begin{itemize}
				\item \url{http://www.w3.org/TR/SRI/}
			\end{itemize}

	\end{itemize}

\end{frame}

% ------------------------------------------------------------------------------
