% ------------------------------------------------------------------------------
% TYPO3 CMS 7.3 - What's New - Chapter "In-Depth Changes" (English Version)
%
% @author	Michael Schams <schams.net>
% @license	Creative Commons BY-NC-SA 3.0
% @link		http://typo3.org/download/release-notes/whats-new/
% @language	English
% ------------------------------------------------------------------------------
% LTXE-CHAPTER-UID:		e8cc33e1-d7bdb74f-0421b75f-7be06b19
% LTXE-CHAPTER-NAME:	In-Depth Changes
% ------------------------------------------------------------------------------

\section{Korenite promene}
\begin{frame}[fragile]
	\frametitle{Korenite promene}

	\begin{center}\huge{Poglavlje 3:}\end{center}
	\begin{center}\huge{\color{typo3darkgrey}\textbf{Korenite promene}}\end{center}

\end{frame}

% ------------------------------------------------------------------------------
% LTXE-SLIDE-START
% LTXE-SLIDE-UID:		925cf4cd-5017ec4c-b84f0153-cdc5a26d
% LTXE-SLIDE-ORIGIN:	17eaaa35-2a21b728-4c3488a0-30a7ac7b English
% LTXE-SLIDE-ORIGIN:	bd03e141-3842b75e-8a44920f-a8c139e5 German
% LTXE-SLIDE-TITLE:		Feature #59606: Integrate Symfony/Console into CommandController (1)
% LTXE-SLIDE-REFERENCE:	Feature-59606-IntegrateSymfonyConsoleIntoCommandController.rst
% ------------------------------------------------------------------------------

\begin{frame}[fragile]
	\frametitle{Korenite promene}\frametitle{Korenite promene}
	\framesubtitle{Symfony/Console integracija sa CommandController-om (1)}

	% decrease font size for code listing
	\lstset{basicstyle=\tiny\ttfamily}

	CommandController sada interno koristi Symfony/Console i omogucava razlicite metode:

	\begin{itemize}

		\item \smaller TableHelper
			\begin{itemize}
				\item\smaller\texttt{outputTable(\$rows, \$headers = NULL)}
			\end{itemize}

		\item DialogHelper
			\begin{itemize}
				\item\smaller\texttt{select(\$question, \$choices, \$default = NULL, \$multiSelect = false, \$attempts = FALSE)}
				\item\texttt{ask(\$question, \$default = NULL, array \$autocomplete = array())}
				\item\texttt{askConfirmation(\$question, \$default = TRUE)}
				\item\texttt{askHiddenResponse(\$question, \$fallback = TRUE)}
				\item\texttt{askAndValidate(\$question, \$validator, \$attempts = FALSE, \$default = NULL, array \$autocomplete = NULL)}
				\item\texttt{askHiddenResponseAndValidate(\$question, \$validator, \$attempts = FALSE, \$fallback = TRUE)}
			\end{itemize}

	\end{itemize}

\end{frame}

% ------------------------------------------------------------------------------
% LTXE-SLIDE-START
% LTXE-SLIDE-UID:		beb98625-e1cf5933-f541f8b9-dc9993e1
% LTXE-SLIDE-ORIGIN:	a24edba5-e325709b-54cabffa-5ba42de7 English
% LTXE-SLIDE-ORIGIN:	3842b75e-8a44920f-a8c139e5-bd03e141 German
% LTXE-SLIDE-TITLE:		Feature #59606: Integrate Symfony/Console into CommandController (2)
% LTXE-SLIDE-REFERENCE:	Feature-59606-IntegrateSymfonyConsoleIntoCommandController.rst
% ------------------------------------------------------------------------------

\begin{frame}[fragile]
	\frametitle{Korenite promene}
	\framesubtitle{Symfony/Console integracija sa CommandController-om (2)}

	% decrease font size for code listing
	\lstset{basicstyle=\tiny\ttfamily}

	\begin{itemize}
		\item \smaller ProgressHelper
			\begin{itemize}
				\item \smaller\texttt{progressStart(\$max = NULL)}
				\item \texttt{progressSet(\$current)}
				\item \texttt{progressAdvance(\$step = 1)}
				\item \texttt{progressFinish()}
			\end{itemize}
	\end{itemize}

	\smaller
		(pogledati naredne slajdove za primere koda)
	\normalsize

\end{frame}

% ------------------------------------------------------------------------------
% LTXE-SLIDE-START
% LTXE-SLIDE-UID:		5ac57243-827f2ff0-55031bcd-ec449250
% LTXE-SLIDE-ORIGIN:	26ee64a3-a8f3ef0d-247faf3b-7df91473 English
% LTXE-SLIDE-ORIGIN:	a8c139e5-8a44920f-3842b75e-bd03e141 German
% LTXE-SLIDE-TITLE:		Feature #59606: Integrate Symfony/Console into CommandController (3)
% LTXE-SLIDE-REFERENCE:	Feature-59606-IntegrateSymfonyConsoleIntoCommandController.rst
% ------------------------------------------------------------------------------

\begin{frame}[fragile]
	\frametitle{Korenite promene}
	\framesubtitle{Symfony/Console integracija sa CommandController-om (3)}

	% decrease font size for code listing
	\lstset{basicstyle=\tiny\ttfamily}

		\begin{lstlisting}
			<?php
			namespace Acme\Demo\Command;
			use TYPO3\CMS\Extbase\Mvc\Controller\CommandController;

			class MyCommandController extends CommandController {
			  public function myCommand() {

			    // render a table
			    $this->output->outputTable(array(
			      array('Bob', 34, 'm'),
			      array('Sally', 21, 'f'),
			      array('Blake', 56, 'm')
			    ),
			    array('Name', 'Age', 'Gender'));

			    // select
			    $colors = array('red', 'blue', 'yellow');
			    $selectedColorIndex = $this->output->select('Please select one color', $colors, 'red');
			    $this->outputLine('You choose the color %s.', array($colors[$selectedColorIndex]));

			    [...]
		\end{lstlisting}

\end{frame}

% ------------------------------------------------------------------------------
% LTXE-SLIDE-START
% LTXE-SLIDE-UID:		4db814ca-d4751984-93e822a4-2dc887ac
% LTXE-SLIDE-ORIGIN:	e7f1a1e9-cabd20ae-78d335fc-3868ca27 English
% LTXE-SLIDE-ORIGIN:	a8c139e5-8a44920f-3842b75e-bd03e141 German
% LTXE-SLIDE-TITLE:		Feature #59606: Integrate Symfony/Console into CommandController (4)
% LTXE-SLIDE-REFERENCE:	Feature-59606-IntegrateSymfonyConsoleIntoCommandController.rst
% ------------------------------------------------------------------------------

\begin{frame}[fragile]
	\frametitle{Korenite promene}
	\framesubtitle{Symfony/Console integracija sa CommandController-om (4)}

	% decrease font size for code listing
	\lstset{basicstyle=\tiny\ttfamily}

		\begin{lstlisting}
			    [...]
			    // ask
			    $name = $this->output->ask('What is your name?' . PHP_EOL, 'Bob', array('Bob', 'Sally', 'Blake'));
			    $this->outputLine('Hello %s.', array($name));

			    // prompt
			    $likesDogs = $this->output->askConfirmation('Do you like dogs?');
			    if ($likesDogs) {
			      $this->outputLine('You do like dogs!');
			    }

			    // progress
			    $this->output->progressStart(600);
			    for ($i = 0; $i < 300; $i ++) {
			      $this->output->progressAdvance();
			      usleep(5000);
			    }
			    $this->output->progressFinish();
			  }
			}
			?>
		\end{lstlisting}

\end{frame}

% ------------------------------------------------------------------------------
% LTXE-SLIDE-START
% LTXE-SLIDE-UID:		ca7bd199-a5643197-1fdb8572-181ca9a8
% LTXE-SLIDE-ORIGIN:	b9b53d17-7e2c0c83-dabec46a-1a756e39 English
% LTXE-SLIDE-ORIGIN:	9d56ca19-04a98896-ba0a973c-8a16e877 German
% LTXE-SLIDE-TITLE:		Feature #66669: BE login form API (1)
% LTXE-SLIDE-REFERENCE:	Feature-66669-BeLoginFormAPI.rst
% ------------------------------------------------------------------------------

\begin{frame}[fragile]
	\frametitle{Korenite promene}
	\framesubtitle{API za prijavljivanje na administratorski interfejs (1)}

	% decrease font size for code listing
	\lstset{basicstyle=\tiny\ttfamily}

	\begin{itemize}

		\item Prijavljivanje na administratorski interfejs je u potpunosti refaktorisano i uveden je sasvim novi API

		\item OpenID forma je izdvojena i sada koristi novi API
			(nezavisna je od klasa osnove sistema)

		\item Koncept novog logovanja na administratorski interfejs baziran je na "login providers"-ima 
			koji se mogu registrovati u fajlu \texttt{ext\_localconf.php} na sledeci nacin:

			\begin{lstlisting}
				$GLOBALS['TYPO3_CONF_VARS']['EXTCONF']['backend']['loginProviders'][1433416020] = [
				  'provider' => \TYPO3\CMS\Backend\LoginProvider\UsernamePasswordLoginProvider::class,
				  'sorting' => 50,
				  'icon-class' => 'fa-key',
				  'label' => 'LLL:EXT:backend/Resources/Private/Language/locallang.xlf:login.link'
				];
			\end{lstlisting}

	\end{itemize}

\end{frame}

% ------------------------------------------------------------------------------
% LTXE-SLIDE-START
% LTXE-SLIDE-UID:		fe1589f3-04b817b4-4076fa2d-fbdc566c
% LTXE-SLIDE-ORIGIN:	c443db74-ed3267f9-a113f486-d586081d English
% LTXE-SLIDE-ORIGIN:	9889604a-ca199d56-73cba0a9-e8778a16 German
% LTXE-SLIDE-TITLE:		Feature #66669: BE login form API (2)
% LTXE-SLIDE-REFERENCE:	Feature-66669-BeLoginFormAPI.rst
% ------------------------------------------------------------------------------

\begin{frame}[fragile]
	\frametitle{Korenite promene}
	\framesubtitle{API za prijavljivanje na administratorski interfejs (2)}

	\begin{itemize}

		\item Ocije su definisane na sledeci nacin:

			\begin{itemize}

				\item \texttt{provider}:\newline
					ime klase login provider-a, koja mora da implementira
						\texttt{TYPO3\textbackslash
							CMS\textbackslash
							Backend\textbackslash
							LoginProvider\textbackslash
							LoginProviderInterface}

				\item \texttt{sorting}:\newline
					redosled linkova do mogucih login provider-a na ekranu za logovanje

				\item \texttt{icon-class}:\newline
					ime font-awesome ikonice za link na ekranu za logovanje

				\item \texttt{label}:\newline
					labela za login provider link na ekranu za logovanje
			\end{itemize}

	\end{itemize}

\end{frame}

% ------------------------------------------------------------------------------
% LTXE-SLIDE-START
% LTXE-SLIDE-UID:		9caabc75-4cfc7c6d-1180c922-acbc31a0
% LTXE-SLIDE-ORIGIN:	66005bf1-6706e6ba-98feeb3f-ac972c3d English
% LTXE-SLIDE-ORIGIN:	04a98896-ba0a973c-8a16e877-9d56ca19 German
% LTXE-SLIDE-TITLE:		Feature #66669: BE login form API (3)
% LTXE-SLIDE-REFERENCE:	Feature-66669-BeLoginFormAPI.rst
% ------------------------------------------------------------------------------

\begin{frame}[fragile]
	\frametitle{Korenite promene}
	\framesubtitle{API za prijavljivanje na administratorski interfejs (3)}

	\begin{itemize}

		\item \texttt{LoginProviderInterface} sadrzi samo metod\newline
			\smaller\texttt{public function render(StandaloneView \$view, PageRenderer \$pageRenderer, LoginController \$loginController);}\normalsize

		\item Parametri se definisu na sledeci nacin:

			\begin{itemize}

				\item \texttt{\$view}:\newline
					Fluid StandaloneView koji rendera login ekran. Mora se podesiti sablon fajl
					i mogu se dodavati promenljive u zavisnosti od potreba.

				\item \texttt{\$pageRenderer}:\newline
					PageRenderer instanca obezbedjuje mogucnost dodavanja neophodnih JavaScript resursa.

				\item \texttt{\$loginController}:\newline
					LoginController instanca.

			\end{itemize}

	\end{itemize}

\end{frame}

% ------------------------------------------------------------------------------
% LTXE-SLIDE-START
% LTXE-SLIDE-UID:		2e49a894-431e3b12-895e7633-eb81170d
% LTXE-SLIDE-ORIGIN:	dd0c782c-161abebc-ba03f841-6d108444 English
% LTXE-SLIDE-ORIGIN:	04a98896-ba0a973c-8a16e877-9d56ca19 German
% LTXE-SLIDE-TITLE:		Feature #66669: BE login form API (4)
% LTXE-SLIDE-REFERENCE:	Feature-66669-BeLoginFormAPI.rst
% ------------------------------------------------------------------------------

\begin{frame}[fragile]
	\frametitle{Korenite promene}
	\framesubtitle{API za prijavljivanje na administratorski interfejs (4)}

	% decrease font size for code listing
	\lstset{basicstyle=\tiny\ttfamily}

	\begin{itemize}

		\item Sablon mora sadrzati \texttt{<f:layout name="Login">} i
			\texttt{<f:section name="loginFormFields">} (za polja forme):

			\begin{lstlisting}
				<f:layout name="Login" />
				<f:section name="loginFormFields">
				  <div class="form-group t3js-login-openid-section" id="t3-login-openid_url-section">
				    <div class="input-group">
				      <input type="text" id="openid_url"
				        name="openid_url"
				        value="{presetOpenId}"
				        autofocus="autofocus"
				        placeholder="{f:translate(key: 'openId', extensionName: 'openid')}"
				        class="form-control input-login t3js-clearable t3js-login-openid-field" />
				      <div class="input-group-addon">
				        <span class="fa fa-openid"></span>
				      </div>
				    </div>
				  </div>
				</f:section>
			\end{lstlisting}

	\end{itemize}

\end{frame}

% ------------------------------------------------------------------------------
% LTXE-SLIDE-START
% LTXE-SLIDE-UID:		6962a80e-c252850d-5bde9c3c-e2456824
% LTXE-SLIDE-ORIGIN:	2051c5f1-bd00622f-dfcbba05-c329dd60 English
% LTXE-SLIDE-ORIGIN:	cc63cd98-524ee6df-38430963-f7c7067c German
% LTXE-SLIDE-TITLE:		Feature #66681: CategoryRegistry: add options to set l10n_mode and l10n_display
% LTXE-SLIDE-REFERENCE:	Feature-66681-CategoryRegistryAddOptionsToSetL10n_modeAndL10n_display.rst
% ------------------------------------------------------------------------------

\begin{frame}[fragile]
	\frametitle{Korenite promene}
	\framesubtitle{\texttt{CategoryRegistry} sa novim opcijama}

	% decrease font size for code listing
	\lstset{basicstyle=\tiny\ttfamily}

	\begin{itemize}

		\item Metoda \texttt{CategoryRegistry->addTcaColumn} je dobila opcije da postavi
			\texttt{l10n\_mode} i \texttt{l10n\_display}:

			\begin{lstlisting}
				\TYPO3\CMS\Core\Utility\ExtensionManagementUtility::makeCategorizable(
				  $extensionKey,
				  $tableName,
				  'categories',
				  array(
				    'l10n_mode' => 'string (keyword)',
				    'l10n_display' => 'list of keywords'
				  )
				);
			\end{lstlisting}

	\end{itemize}

\end{frame}

% ------------------------------------------------------------------------------
% LTXE-SLIDE-START
% LTXE-SLIDE-UID:		e1f05d09-dd5c6794-ef3e1377-f1dc9eb2
% LTXE-SLIDE-ORIGIN:	52d9c9a1-37301644-b52c1658-372774a8 English
% LTXE-SLIDE-ORIGIN:	c9a5288c-f45aac1e-34606612-fdf6ec18 German
% LTXE-SLIDE-TITLE:		Feature #66822: Allow Sprites For Backend Modules
% LTXE-SLIDE-REFERENCE:	Feature-66822-SpriteIconsInBackendModules.rst
% ------------------------------------------------------------------------------

\begin{frame}[fragile]
	\frametitle{Korenite promene}
	\framesubtitle{Sprites u Administratorskim modulima}

	% decrease font size for code listing
	\lstset{basicstyle=\tiny\ttfamily}

	\begin{itemize}

		\item Administratorski moduli (glavni moduli kao sto je "Web" kao i podmoduli kao sto je "Filelist")
			sada mogu da koriste sprite za ikonice\newline
			\small
				(samo sprite ikonice poznate TYPO3-ju su dostupne!)
			\normalsize

		\item Primer:

			\begin{lstlisting}
				\TYPO3\CMS\Core\Utility\ExtensionManagementUtility::addModule(
				  'web',
				  'layout',
				  'top',
				  \TYPO3\CMS\Core\Utility\ExtensionManagementUtility::extPath($_EXTKEY) . 'Modules/Layout/',
				  array(
				    'script' => '_DISPATCH',
				    'access' => 'user,group',
				    'name' => 'web_layout',
				    'configuration' => array('icon' => 'module-web'),
				    'labels' => array(
				      'll_ref' => 'LLL:EXT:cms/layout/locallang_mod.xlf',
				    ),
				  )
				);
			\end{lstlisting}

	\end{itemize}

\end{frame}

% ------------------------------------------------------------------------------
% LTXE-SLIDE-START
% LTXE-SLIDE-UID:		c00b27c4-f3addc6d-f59214ac-92a91a0c
% LTXE-SLIDE-ORIGIN:	8d50a473-76b35cf0-4414be81-8efd94f3 English
% LTXE-SLIDE-ORIGIN:	d9657b5d-5cd56c65-d7266a27-401dd233 German
% LTXE-SLIDE-TITLE:		Feature #67229: FormEngine NodeFactory API
% LTXE-SLIDE-REFERENCE:	Feature-67229-FormEngineNodeFactoryApi.rst
% ------------------------------------------------------------------------------

\begin{frame}[fragile]
	\frametitle{Korenite promene}
	\framesubtitle{FormEngine NodeFactory API (1)}

	% decrease font size for code listing
	\lstset{basicstyle=\tiny\ttfamily}

	\begin{itemize}

		\item Sada je moguce registrovati nove nodove i pregaziti one vec postojece

		\begin{lstlisting}
			$GLOBALS['TYPO3_CONF_VARS']['SYS']['formEngine']['nodeRegistry'][1433196792] = array(
			  'nodeName' => 'input',
			  'priority' => 40,
			  'class' => \MyVendor\MyExtension\Form\Element\T3editorElement::class
			);
		\end{lstlisting}

		\item Primer iznad registruje novu klasu
			\texttt{MyVendor\textbackslash
				MyExtension\textbackslash
				Form\textbackslash
				Element\textbackslash
				T3editorElement}
			kao klasu za renderanje TCA tipa \texttt{input}, koji mora da implementira interfejs
			\texttt{TYPO3\textbackslash
				CMS\textbackslash
				Backend\textbackslash
				Form\textbackslash
				NodeInterface}

		\item Kljuc niza je Unix timestamp datuma kada je elemenat dodat

	\end{itemize}

\end{frame}

% ------------------------------------------------------------------------------
% LTXE-SLIDE-START
% LTXE-SLIDE-UID:		aa045b8b-51c009ae-b912308a-6949d398
% LTXE-SLIDE-ORIGIN:	94bf7d97-b60d2c04-a1693909-e7397a81 English
% LTXE-SLIDE-ORIGIN:	5cd56c65-d7266a27-401dd233-d9657b5d German
% LTXE-SLIDE-TITLE:		Feature #67229: FormEngine NodeFactory API
% LTXE-SLIDE-REFERENCE:	Feature-67229-FormEngineNodeFactoryApi.rst
% ------------------------------------------------------------------------------

\begin{frame}[fragile]
	\frametitle{Korenite promene}
	\framesubtitle{FormEngine NodeFactory API (2)}

	% decrease font size for code listing
	\lstset{basicstyle=\tiny\ttfamily}

	\begin{itemize}

		\item U slucaju da je vise elemenata dodato za isti tip, koristi se onaj sa najvecim prioritetom (od 0 do 100)

		\item Novi TCA tip moze se registrovati na sledeci nacin:

		\smaller\textbf{\texttt{TCA}}
		\begin{lstlisting}
			'columns' => array(
			  'bodytext' => array(
			    'config' => array(
			      'type' => 'text',
			      'renderType' => '3dCloud',
			    ),
			  ),
			)
		\end{lstlisting}

		\smaller\textbf{\texttt{ext\_localconf.php}}
		\begin{lstlisting}
			$GLOBALS['TYPO3_CONF_VARS']['SYS']['formEngine']['nodeRegistry'][1433197759] = array(
			  'nodeName' => '3dCloud',
			  'priority' => 40,
			  'class' => \MyVendor\MyExtension\Form\Element\ShowTextAs3dCloudElement::class
			);
		\end{lstlisting}

	\end{itemize}

\end{frame}

% ------------------------------------------------------------------------------
% LTXE-SLIDE-START
% LTXE-SLIDE-UID:		3f1f66b3-1710c26d-23182e11-c7c4449c
% LTXE-SLIDE-ORIGIN:	818411e6-dd2a7aba-3a7afb94-4e970169 English
% LTXE-SLIDE-ORIGIN:	d7266a27-401dd233-d9657b5d-5cd56c65 German
% LTXE-SLIDE-TITLE:		Breaking #62983: postProcessMirrorUrl signal has moved
% LTXE-SLIDE-REFERENCE:	Breaking-62983-PostProcessMirrorUrlSignalHasMoved.rst
% ------------------------------------------------------------------------------

\begin{frame}[fragile]
	\frametitle{Korenite promene}
	\framesubtitle{Signal \texttt{postProcessMirrorUrl}}

	% decrease font size for code listing
	\lstset{basicstyle=\tiny\ttfamily}

	\begin{itemize}

		\item Signal \texttt{postProcessMirrorUrl} prebacen je u novu klasu

		\breakingchange

		\item Sledeci primer koda uzima u obzir verziju TYPO3-ja:

		\begin{lstlisting}
			$signalSlotDispatcher->connect(
			  version_compare(TYPO3_version, '7.0', '<')
			    ? 'TYPO3\\CMS\\Lang\\Service\\UpdateTranslationService'
			    : 'TYPO3\\CMS\\Lang\\Service\\TranslationService',
			  'postProcessMirrorUrl',
			  'Vendor\\Extension\\Slots\\CustomMirror',
			  'postProcessMirrorUrl'
			);
		\end{lstlisting}

	\end{itemize}

\end{frame}

% ------------------------------------------------------------------------------
