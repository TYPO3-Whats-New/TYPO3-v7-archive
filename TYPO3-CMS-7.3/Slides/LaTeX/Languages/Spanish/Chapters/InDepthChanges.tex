% ------------------------------------------------------------------------------
% TYPO3 CMS 7.3 - What's New - Chapter "In-Depth Changes" (English Version)
%
% @author	Michael Schams <schams.net>
% @license	Creative Commons BY-NC-SA 3.0
% @link		http://typo3.org/download/release-notes/whats-new/
% @language	English
% ------------------------------------------------------------------------------
% LTXE-CHAPTER-UID:		e8cc33e1-d7bdb74f-0421b75f-7be06b19
% LTXE-CHAPTER-NAME:	In-Depth Changes
% ------------------------------------------------------------------------------

\section{Cambios en Profundidad}
\begin{frame}[fragile]
	\frametitle{Cambios en Profundidad}

	\begin{center}\huge{Capítulo 3:}\end{center}
	\begin{center}\huge{\color{typo3darkgrey}\textbf{Cambios en Profundidad}}\end{center}

\end{frame}

% ------------------------------------------------------------------------------
% LTXE-SLIDE-START
% LTXE-SLIDE-UID:		a2b9d9e5-3dab145f-b292b6ba-73ddb517
% LTXE-SLIDE-ORIGIN:	17eaaa35-2a21b728-4c3488a0-30a7ac7b English
% LTXE-SLIDE-ORIGIN:	bd03e141-3842b75e-8a44920f-a8c139e5 German
% LTXE-SLIDE-TITLE:		Feature #59606: Integrate Symfony/Console into CommandController (1)
% LTXE-SLIDE-REFERENCE:	Feature-59606-IntegrateSymfonyConsoleIntoCommandController.rst
% ------------------------------------------------------------------------------

\begin{frame}[fragile]
	\frametitle{Cambios en Profundidad}
	\framesubtitle{Integración Symfony/Console en CommandController (1)}

	% decrease font size for code listing
	\lstset{basicstyle=\tiny\ttfamily}

	El CommandController hace uso ahora de Symfony/Console internamente y proporciona varios métodos:

	\begin{itemize}

		\item \smaller TableHelper
			\begin{itemize}
				\item\smaller\texttt{outputTable(\$rows, \$headers = NULL)}
			\end{itemize}

		\item DialogHelper
			\begin{itemize}
				\item\smaller\texttt{select(\$question, \$choices, \$default = NULL, \$multiSelect = false, \$attempts = FALSE)}
				\item\texttt{ask(\$question, \$default = NULL, array \$autocomplete = array())}
				\item\texttt{askConfirmation(\$question, \$default = TRUE)}
				\item\texttt{askHiddenResponse(\$question, \$fallback = TRUE)}
				\item\texttt{askAndValidate(\$question, \$validator, \$attempts = FALSE, \$default = NULL, array \$autocomplete = NULL)}
				\item\texttt{askHiddenResponseAndValidate(\$question, \$validator, \$attempts = FALSE, \$fallback = TRUE)}
			\end{itemize}

	\end{itemize}

\end{frame}

% ------------------------------------------------------------------------------
% LTXE-SLIDE-START
% LTXE-SLIDE-UID:		da5a99c6-9a90189d-3d421409-373a0cc7
% LTXE-SLIDE-ORIGIN:	a24edba5-e325709b-54cabffa-5ba42de7 English
% LTXE-SLIDE-ORIGIN:	3842b75e-8a44920f-a8c139e5-bd03e141 German
% LTXE-SLIDE-TITLE:		Feature #59606: Integrate Symfony/Console into CommandController (2)
% LTXE-SLIDE-REFERENCE:	Feature-59606-IntegrateSymfonyConsoleIntoCommandController.rst
% ------------------------------------------------------------------------------

\begin{frame}[fragile]
	\frametitle{Cambios en Profundidad}
	\framesubtitle{Integración Symfony/Console en CommandController (2)}

	% decrease font size for code listing
	\lstset{basicstyle=\tiny\ttfamily}

	\begin{itemize}
		\item \smaller ProgressHelper
			\begin{itemize}
				\item \smaller\texttt{progressStart(\$max = NULL)}
				\item \texttt{progressSet(\$current)}
				\item \texttt{progressAdvance(\$step = 1)}
				\item \texttt{progressFinish()}
			\end{itemize}
	\end{itemize}

	\smaller
		(ver siguientes diapositivas para ejemplos de código)
	\normalsize

\end{frame}

% ------------------------------------------------------------------------------
% LTXE-SLIDE-START
% LTXE-SLIDE-UID:		48e1d3a5-9b26ec38-43c05096-c3443e35
% LTXE-SLIDE-ORIGIN:	26ee64a3-a8f3ef0d-247faf3b-7df91473 English
% LTXE-SLIDE-ORIGIN:	a8c139e5-8a44920f-3842b75e-bd03e141 German
% LTXE-SLIDE-TITLE:		Feature #59606: Integrate Symfony/Console into CommandController (3)
% LTXE-SLIDE-REFERENCE:	Feature-59606-IntegrateSymfonyConsoleIntoCommandController.rst
% ------------------------------------------------------------------------------

\begin{frame}[fragile]
	\frametitle{Cambios en Profundidad}
	\framesubtitle{Integración Symfony/Console en CommandController (3)}

	% decrease font size for code listing
	\lstset{basicstyle=\tiny\ttfamily}

		\begin{lstlisting}
			<?php
			namespace Acme\Demo\Command;
			use TYPO3\CMS\Extbase\Mvc\Controller\CommandController;

			class MyCommandController extends CommandController {
			  public function myCommand() {

			    // render a table
			    $this->output->outputTable(array(
			      array('Bob', 34, 'm'),
			      array('Sally', 21, 'f'),
			      array('Blake', 56, 'm')
			    ),
			    array('Name', 'Age', 'Gender'));

			    // select
			    $colors = array('red', 'blue', 'yellow');
			    $selectedColorIndex = $this->output->select('Please select one color', $colors, 'red');
			    $this->outputLine('You choose the color %s.', array($colors[$selectedColorIndex]));

			    [...]
		\end{lstlisting}

\end{frame}

% ------------------------------------------------------------------------------
% LTXE-SLIDE-START
% LTXE-SLIDE-UID:		179cba9f-19a1e6ea-04ade714-bc19bcb7
% LTXE-SLIDE-ORIGIN:	e7f1a1e9-cabd20ae-78d335fc-3868ca27 English
% LTXE-SLIDE-ORIGIN:	a8c139e5-8a44920f-3842b75e-bd03e141 German
% LTXE-SLIDE-TITLE:		Feature #59606: Integrate Symfony/Console into CommandController (4)
% LTXE-SLIDE-REFERENCE:	Feature-59606-IntegrateSymfonyConsoleIntoCommandController.rst
% ------------------------------------------------------------------------------

\begin{frame}[fragile]
	\frametitle{Cambios en Profundidad}
	\framesubtitle{Integración Symfony/Console en CommandController (4)}

	% decrease font size for code listing
	\lstset{basicstyle=\tiny\ttfamily}

		\begin{lstlisting}
			    [...]
			    // ask
			    $name = $this->output->ask('What is your name?' . PHP_EOL, 'Bob', array('Bob', 'Sally', 'Blake'));
			    $this->outputLine('Hello %s.', array($name));

			    // prompt
			    $likesDogs = $this->output->askConfirmation('Do you like dogs?');
			    if ($likesDogs) {
			      $this->outputLine('You do like dogs!');
			    }

			    // progress
			    $this->output->progressStart(600);
			    for ($i = 0; $i < 300; $i ++) {
			      $this->output->progressAdvance();
			      usleep(5000);
			    }
			    $this->output->progressFinish();
			  }
			}
			?>
		\end{lstlisting}

\end{frame}

% ------------------------------------------------------------------------------
% LTXE-SLIDE-START
% LTXE-SLIDE-UID:		4c7af1ea-a9761cea-2bf66770-6fe7776e
% LTXE-SLIDE-ORIGIN:	b9b53d17-7e2c0c83-dabec46a-1a756e39 English
% LTXE-SLIDE-ORIGIN:	9d56ca19-04a98896-ba0a973c-8a16e877 German
% LTXE-SLIDE-TITLE:		Feature #66669: BE login form API (1)
% LTXE-SLIDE-REFERENCE:	Feature-66669-BeLoginFormAPI.rst
% ------------------------------------------------------------------------------

\begin{frame}[fragile]
	\frametitle{Cambios en Profundidad}
	\framesubtitle{Backend Login API (1)}

	% decrease font size for code listing
	\lstset{basicstyle=\tiny\ttfamily}

	\begin{itemize}

		\item Se ha refactorizado completamente el login del backend y se ha introducido
		una nueva API

		\item Se ha extraído el formulario OpenID y ahora usa la nueva API
			(lo hace independiente de las clases centrales del Núcleo)

		\item El concepto del nuevo login del backend está basado en "login
		providers", que
			pueden ser registrados en el fichero \texttt{ext\_localconf.php} como sigue:

			\begin{lstlisting}
				$GLOBALS['TYPO3_CONF_VARS']['EXTCONF']['backend']['loginProviders'][1433416020] = [
				  'provider' => \TYPO3\CMS\Backend\LoginProvider\UsernamePasswordLoginProvider::class,
				  'sorting' => 50,
				  'icon-class' => 'fa-key',
				  'label' => 'LLL:EXT:backend/Resources/Private/Language/locallang.xlf:login.link'
				];
			\end{lstlisting}

	\end{itemize}

\end{frame}

% ------------------------------------------------------------------------------
% LTXE-SLIDE-START
% LTXE-SLIDE-UID:		626410b2-b0caaa0e-1462ccf5-242770ee
% LTXE-SLIDE-ORIGIN:	c443db74-ed3267f9-a113f486-d586081d English
% LTXE-SLIDE-ORIGIN:	9889604a-ca199d56-73cba0a9-e8778a16 German
% LTXE-SLIDE-TITLE:		Feature #66669: BE login form API (2)
% LTXE-SLIDE-REFERENCE:	Feature-66669-BeLoginFormAPI.rst
% ------------------------------------------------------------------------------

\begin{frame}[fragile]
	\frametitle{Cambios en Profundidad}
	\framesubtitle{Backend Login API (2)}

	\begin{itemize}

		\item Las opciones se definen a continuación:

			\begin{itemize}

				\item \texttt{provider}:\newline
					nombre de la clase provider del login, que debe implementar
						\texttt{TYPO3\textbackslash
							CMS\textbackslash
							Backend\textbackslash
							LoginProvider\textbackslash
							LoginProviderInterface}

				\item \texttt{sorting}:\newline
					orden de los enlaces para los posibles providers de login en la pantalla de login

				\item \texttt{icon-class}:\newline
					nombre del icono font-awesome para el enlace de la pantalla de login

				\item \texttt{label}:\newline
					etiqueta para el enlace del proveedor de login en la pantalla de login
			\end{itemize}

	\end{itemize}

\end{frame}

% ------------------------------------------------------------------------------
% LTXE-SLIDE-START
% LTXE-SLIDE-UID:		9d29bdce-390f6daf-00e17cc0-3d8cbfd5
% LTXE-SLIDE-ORIGIN:	66005bf1-6706e6ba-98feeb3f-ac972c3d English
% LTXE-SLIDE-ORIGIN:	04a98896-ba0a973c-8a16e877-9d56ca19 German
% LTXE-SLIDE-TITLE:		Feature #66669: BE login form API (3)
% LTXE-SLIDE-REFERENCE:	Feature-66669-BeLoginFormAPI.rst
% ------------------------------------------------------------------------------

\begin{frame}[fragile]
	\frametitle{Cambios en Profundidad}
	\framesubtitle{Backend Login API (3)}

	\begin{itemize}

		\item La \texttt{LoginProviderInterface} sólo contiene el método\newline
			\smaller\texttt{public function render(StandaloneView \$view, PageRenderer \$pageRenderer, LoginController \$loginController);}\normalsize

		\item Los parámetros se definen a continuación:

			\begin{itemize}

				\item \texttt{\$view}:\newline
					Fluid StandaloneView que renderiza la pantalla de inicio. Tiene que configurar
					el fichero template y puede añadir variables a la vista
					según sus necesidades.

				\item \texttt{\$pageRenderer}:\newline
					La instancia PageRenderer provee la posibilidad de añadir recursos JavaScript necesarios.

				\item \texttt{\$loginController}:\newline
					Instancia LoginController.

			\end{itemize}

	\end{itemize}

\end{frame}

% ------------------------------------------------------------------------------
% LTXE-SLIDE-START
% LTXE-SLIDE-UID:		22770c33-52460e94-6aeca9b8-d232132d
% LTXE-SLIDE-ORIGIN:	dd0c782c-161abebc-ba03f841-6d108444 English
% LTXE-SLIDE-ORIGIN:	04a98896-ba0a973c-8a16e877-9d56ca19 German
% LTXE-SLIDE-TITLE:		Feature #66669: BE login form API (4)
% LTXE-SLIDE-REFERENCE:	Feature-66669-BeLoginFormAPI.rst
% ------------------------------------------------------------------------------

\begin{frame}[fragile]
	\frametitle{Cambios en Profundidad}
	\framesubtitle{Backend Login API (4)}

	% decrease font size for code listing
	\lstset{basicstyle=\tiny\ttfamily}

	\begin{itemize}

		\item El template debe contener \texttt{<f:layout name="Login">} y
			\texttt{<f:section name="loginFormFields">} (para campos del formulario):

			\begin{lstlisting}
				<f:layout name="Login" />
				<f:section name="loginFormFields">
				  <div class="form-group t3js-login-openid-section" id="t3-login-openid_url-section">
				    <div class="input-group">
				      <input type="text" id="openid_url"
				        name="openid_url"
				        value="{presetOpenId}"
				        autofocus="autofocus"
				        placeholder="{f:translate(key: 'openId', extensionName: 'openid')}"
				        class="form-control input-login t3js-clearable t3js-login-openid-field" />
				      <div class="input-group-addon">
				        <span class="fa fa-openid"></span>
				      </div>
				    </div>
				  </div>
				</f:section>
			\end{lstlisting}

	\end{itemize}

\end{frame}

% ------------------------------------------------------------------------------
% LTXE-SLIDE-START
% LTXE-SLIDE-UID:		3586ee60-a0f9ff22-408929a0-e5a4aca0
% LTXE-SLIDE-ORIGIN:	2051c5f1-bd00622f-dfcbba05-c329dd60 English
% LTXE-SLIDE-ORIGIN:	cc63cd98-524ee6df-38430963-f7c7067c German
% LTXE-SLIDE-TITLE:		Feature #66681: CategoryRegistry: add options to set l10n_mode and l10n_display
% LTXE-SLIDE-REFERENCE:	Feature-66681-CategoryRegistryAddOptionsToSetL10n_modeAndL10n_display.rst
% ------------------------------------------------------------------------------

\begin{frame}[fragile]
	\frametitle{Cambios en Profundidad}
	\framesubtitle{\texttt{CategoryRegistry} con Nuevas Opciones}

	% decrease font size for code listing
	\lstset{basicstyle=\tiny\ttfamily}

	\begin{itemize}

		\item El método \texttt{CategoryRegistry->addTcaColumn} acepta opciones para configurar
			\texttt{l10n\_mode} y \texttt{l10n\_display}:

			\begin{lstlisting}
				\TYPO3\CMS\Core\Utility\ExtensionManagementUtility::makeCategorizable(
				  $extensionKey,
				  $tableName,
				  'categories',
				  array(
				    'l10n_mode' => 'string (keyword)',
				    'l10n_display' => 'list of keywords'
				  )
				);
			\end{lstlisting}

	\end{itemize}

\end{frame}

% ------------------------------------------------------------------------------
% LTXE-SLIDE-START
% LTXE-SLIDE-UID:		d00eb8fa-5797e731-f2c6e89f-4cba093c
% LTXE-SLIDE-ORIGIN:	52d9c9a1-37301644-b52c1658-372774a8 English
% LTXE-SLIDE-ORIGIN:	c9a5288c-f45aac1e-34606612-fdf6ec18 German
% LTXE-SLIDE-TITLE:		Feature #66822: Allow Sprites For Backend Modules
% LTXE-SLIDE-REFERENCE:	Feature-66822-SpriteIconsInBackendModules.rst
% ------------------------------------------------------------------------------

\begin{frame}[fragile]
	\frametitle{Cambios en Profundidad}
	\framesubtitle{Sprites en Módulos de Backend}

	% decrease font size for code listing
	\lstset{basicstyle=\tiny\ttfamily}

	\begin{itemize}

		\item Los módulos del Backend (módulos principales tales como "Web" así como submódulos tales como
			"Filelist") pueden ahora usar sprites como iconos \small(¡sólo están disponibles iconos sprite conocidos para TYPO3!)
			\normalsize

		\item Ejemplo:

			\begin{lstlisting}
				\TYPO3\CMS\Core\Utility\ExtensionManagementUtility::addModule(
				  'web',
				  'layout',
				  'top',
				  \TYPO3\CMS\Core\Utility\ExtensionManagementUtility::extPath($_EXTKEY) . 'Modules/Layout/',
				  array(
				    'script' => '_DISPATCH',
				    'access' => 'user,group',
				    'name' => 'web_layout',
				    'configuration' => array('icon' => 'module-web'),
				    'labels' => array(
				      'll_ref' => 'LLL:EXT:cms/layout/locallang_mod.xlf',
				    ),
				  )
				);
			\end{lstlisting}

	\end{itemize}

\end{frame}

% ------------------------------------------------------------------------------
% LTXE-SLIDE-START
% LTXE-SLIDE-UID:		cf2663a4-400e21b2-c4af75dc-60c1bb5d
% LTXE-SLIDE-ORIGIN:	8d50a473-76b35cf0-4414be81-8efd94f3 English
% LTXE-SLIDE-ORIGIN:	d9657b5d-5cd56c65-d7266a27-401dd233 German
% LTXE-SLIDE-TITLE:		Feature #67229: FormEngine NodeFactory API
% LTXE-SLIDE-REFERENCE:	Feature-67229-FormEngineNodeFactoryApi.rst
% ------------------------------------------------------------------------------

\begin{frame}[fragile]
	\frametitle{Cambios en Profundidad}
	\framesubtitle{FormEngine NodeFactory API (1)}

	% decrease font size for code listing
	\lstset{basicstyle=\tiny\ttfamily}

	\begin{itemize}

		\item Ahora es posible registrar nuevos nodos y sobreescribir nodos existentes

		\begin{lstlisting}
			$GLOBALS['TYPO3_CONF_VARS']['SYS']['formEngine']['nodeRegistry'][1433196792] = array(
			  'nodeName' => 'input',
			  'priority' => 40,
			  'class' => \MyVendor\MyExtension\Form\Element\T3editorElement::class
			);
		\end{lstlisting}

		\item El ejemplo de arriba registra una nueva clase
			\texttt{MyVendor\textbackslash
				MyExtension\textbackslash
				Form\textbackslash
				Element\textbackslash
				T3editorElement}
			como clase de renderizado para tipo TCA \texttt{input}, que debe implementar la interfaz
			\texttt{TYPO3\textbackslash
				CMS\textbackslash
				Backend\textbackslash
				Form\textbackslash
				NodeInterface}

		\item La clave del vector es la marca de tiempo Unix de la fecha cuando se añade un elemento de registro

	\end{itemize}

\end{frame}

% ------------------------------------------------------------------------------
% LTXE-SLIDE-START
% LTXE-SLIDE-UID:		3ff2eeae-85ff4f11-d86b4373-205f9284
% LTXE-SLIDE-ORIGIN:	94bf7d97-b60d2c04-a1693909-e7397a81 English
% LTXE-SLIDE-ORIGIN:	5cd56c65-d7266a27-401dd233-d9657b5d German
% LTXE-SLIDE-TITLE:		Feature #67229: FormEngine NodeFactory API
% LTXE-SLIDE-REFERENCE:	Feature-67229-FormEngineNodeFactoryApi.rst
% ------------------------------------------------------------------------------

\begin{frame}[fragile]
	\frametitle{Cambios en Profundidad}
	\framesubtitle{FormEngine NodeFactory API (2)}

	% decrease font size for code listing
	\lstset{basicstyle=\tiny\ttfamily}

	\begin{itemize}

		\item En aquellos casos donde elementos de registro múltiples hayan sido registrados para el mismo
			tipo, se usa el resolvedor con la máxima prioridad (de 0 a 100)

		\item Un nuevo tipo TCA puede ser registrado como sigue:

		\smaller\textbf{\texttt{TCA}}
		\begin{lstlisting}
			'columns' => array(
			  'bodytext' => array(
			    'config' => array(
			      'type' => 'text',
			      'renderType' => '3dCloud',
			    ),
			  ),
			)
		\end{lstlisting}

		\smaller\textbf{\texttt{ext\_localconf.php}}
		\begin{lstlisting}
			$GLOBALS['TYPO3_CONF_VARS']['SYS']['formEngine']['nodeRegistry'][1433197759] = array(
			  'nodeName' => '3dCloud',
			  'priority' => 40,
			  'class' => \MyVendor\MyExtension\Form\Element\ShowTextAs3dCloudElement::class
			);
		\end{lstlisting}

	\end{itemize}

\end{frame}

% ------------------------------------------------------------------------------
% LTXE-SLIDE-START
% LTXE-SLIDE-UID:		d23e0ed3-b0283ff7-dfcbe8c5-0c6277b1
% LTXE-SLIDE-ORIGIN:	818411e6-dd2a7aba-3a7afb94-4e970169 English
% LTXE-SLIDE-ORIGIN:	d7266a27-401dd233-d9657b5d-5cd56c65 German
% LTXE-SLIDE-TITLE:		Breaking #62983: postProcessMirrorUrl signal has moved
% LTXE-SLIDE-REFERENCE:	Breaking-62983-PostProcessMirrorUrlSignalHasMoved.rst
% ------------------------------------------------------------------------------

\begin{frame}[fragile]
	\frametitle{Cambios en Profundidad}
	\framesubtitle{Señal \texttt{postProcessMirrorUrl}}

	% decrease font size for code listing
	\lstset{basicstyle=\tiny\ttfamily}

	\begin{itemize}

		\item Se ha movido la señal \texttt{postProcessMirrorUrl} a una nueva clase

		\breakingchange

		\item El siguiente ejemplo de código tiene en cuenta la versión TYPO3:

		\begin{lstlisting}
			$signalSlotDispatcher->connect(
			  version_compare(TYPO3_version, '7.0', '<')
			    ? 'TYPO3\\CMS\\Lang\\Service\\UpdateTranslationService'
			    : 'TYPO3\\CMS\\Lang\\Service\\TranslationService',
			  'postProcessMirrorUrl',
			  'Vendor\\Extension\\Slots\\CustomMirror',
			  'postProcessMirrorUrl'
			);
		\end{lstlisting}

	\end{itemize}

\end{frame}

% ------------------------------------------------------------------------------
