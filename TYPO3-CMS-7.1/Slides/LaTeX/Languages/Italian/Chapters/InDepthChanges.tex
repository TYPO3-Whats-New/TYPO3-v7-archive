% ------------------------------------------------------------------------------
% TYPO3 CMS 7.1 - What's New - Chapter "In-Depth Changes" (English Version)
%
% @author	Michael Schams <schams.net>
% @license	Creative Commons BY-NC-SA 3.0
% @link		http://typo3.org/download/release-notes/whats-new/
% @language	English
% ------------------------------------------------------------------------------
% LTXE-CHAPTER-UID:		7cb2d402-105890ae-a1632623-719adbb6
% LTXE-CHAPTER-NAME:	In-Depth Changes
% ------------------------------------------------------------------------------

\section{Modifiche rilevanti}
\begin{frame}[fragile]
	\frametitle{Modifiche rilevanti}

	\begin{center}\huge{Capitolo 3:}\end{center}
	\begin{center}\huge{\color{typo3darkgrey}\textbf{Modifiche rilevanti}}\end{center}

\end{frame}

% ------------------------------------------------------------------------------
% LTXE-SLIDE-START
% LTXE-SLIDE-UID:		85455ee7-ee90e44c-286267af-35baed4b
% LTXE-SLIDE-ORIGIN:	505c827e-aab9a39a-612d0288-bb873878 English
% LTXE-SLIDE-TITLE:		Maximum chars in text element
% LTXE-SLIDE-REFERENCE:	Feature-24906-MaxForTextElement.rst
% ------------------------------------------------------------------------------

\begin{frame}[fragile]
	\frametitle{Modifiche rilevanti}
	\framesubtitle{TCA: Numero massimo di caratteri nell'elemento testo}

	\begin{itemize}
		\item Il tipo \texttt{text} in TCA supporta ora l'attributo HTML5 \texttt{maxlength}
			per limitare la lunghezza di un testo (nota: il ritorno a capo è solitamente contato come due
			caratteri)

			\begin{lstlisting}
				'teaser' => array(
				  'label' => 'Teaser',
				  'config' => array(
				    'type' => 'text',
				    'cols' => 60,
				    'rows' => 2,
				    'max' => '30' // <-- maxlength
				  )
				),
			\end{lstlisting}

			Da notare, che non tutti i browser supportano questo attributo.\newline
			Vedi \href{http://www.w3schools.com/tags/att_textarea_maxlength.asp}{Browser Support List} per dettagli.

	\end{itemize}

\end{frame}

% ------------------------------------------------------------------------------
% LTXE-SLIDE-START
% LTXE-SLIDE-UID:		89f4545f-b9f25876-32d4ace7-34c7aad6
% LTXE-SLIDE-ORIGIN:	a3d28e8d-6d40bc8f-f60bb05f-d9e31bba English
% LTXE-SLIDE-TITLE:		New SplFileInfo implementation
% LTXE-SLIDE-REFERENCE:	Feature-60019-SplFileInfo-MimeTypeGuesser-hook.rst
% ------------------------------------------------------------------------------

\begin{frame}[fragile]
	\frametitle{Modifiche rilevanti}
	\framesubtitle{Nuova implementazione SplFileInfo}

	% decrease font size for code listing
	\lstset{basicstyle=\smaller\ttfamily}

	\begin{itemize}

		\item Nuova classe:
			\texttt{TYPO3\textbackslash
				CMS\textbackslash
				Core\textbackslash
				Type\textbackslash
				File\textbackslash
				FileInfo}

		\item Questa classe estende la classe \texttt{SplFileInfo}, che consente il recupero delle informazioni meta dai file

			\begin{lstlisting}
				$fileIdentifier = '/tmp/foo.html';
				$fileInfo = GeneralUtility::makeInstance(
				  \TYPO3\CMS\Core\Type\File\FileInfo::class,
				  $fileIdentifier
				);
				echo $fileInfo->getMimeType(); // output: text/html
			\end{lstlisting}

		\item Implementazioni personalizzate possono utilizzare il seguente hook:

			\begin{lstlisting}
				$GLOBALS['TYPO3_CONF_VARS']['SC_OPTIONS']
				  [\TYPO3\CMS\Core\Type\File\FileInfo::class]['mimeTypeGuessers']
			\end{lstlisting}

	\end{itemize}

\end{frame}

% ------------------------------------------------------------------------------
% LTXE-SLIDE-START
% LTXE-SLIDE-UID:		4786fa8e-df242ca9-f2808f4e-093394cb
% LTXE-SLIDE-ORIGIN:	5351c635-145cc752-9c930534-3a8b4e50 English
% LTXE-SLIDE-TITLE:		UserFunc in TCA Display Condition
% LTXE-SLIDE-REFERENCE:	Feature-62944-UserFuncAsDisplayCond.rst
% ------------------------------------------------------------------------------

\begin{frame}[fragile]
	\frametitle{Modifiche rilevanti}
	\framesubtitle{UserFunc nelle condizioni di visualizzazione di TCA}

	\begin{itemize}
		\item userFunc \texttt{displayCondition} permette di controllare in qualsiasi condizione o stato immaginabile
		\item Se una situazione non può essere controllata con nessuno dei controlli disponibili, gli sviluppatori
			possono creare le proprie user function\newline
			(ritorna \texttt{TRUE/FALSE} per visualizzare/nascondere appropriati campi TCA)

			\begin{lstlisting}
				$GLOBALS['TCA']['tt_content']['columns']['bodytext']['displayCond'] =
				  'USER:Vendor\\Example\\User\\ElementConditionMatcher->
				    checkHeaderGiven:any:more:information';
			\end{lstlisting}

	\end{itemize}

\end{frame}

% ------------------------------------------------------------------------------
% LTXE-SLIDE-START
% LTXE-SLIDE-UID:		7ca59513-8433a215-40118e45-0f6917e8
% LTXE-SLIDE-ORIGIN:	3ba78d86-ca95152a-5b7e29c9-7e6593da English
% LTXE-SLIDE-TITLE:		API for Twitter Bootstrap modals (1)
% LTXE-SLIDE-REFERENCE:	Feature-63729-ApiForBootstrapModals.rst
% ------------------------------------------------------------------------------

\begin{frame}[fragile]
	\frametitle{Modifiche rilevanti}
	\framesubtitle{API per il modulo Twitter Bootstrap (1)}

	% decrease font size for code listing
	\lstset{basicstyle=\smaller\ttfamily}

	\begin{itemize}

		\item Due nuove API per creare/rimuovere popup modali:
			\begin{itemize}
				\item \texttt{TYPO3.Modal.confirm(title, content, severity, buttons)}
				\item \texttt{TYPO3.Modal.dismiss()}
			\end{itemize}

		\item Le opzioni \texttt{title} e \texttt{content} sono richieste
		\item Le opzioni \texttt{buttons.text} e \texttt{buttons.trigger} sono anche richieste, se è usato \texttt{buttons}

		\item Esempio 1:

			\begin{lstlisting}
				TYPO3.Modal.confirm(
				  'The title of the modal',         // title
				  'This the body of the modal', // content
				  TYPO3.Severity.warning            // severity
				);
			\end{lstlisting}

	\end{itemize}

\end{frame}

% ------------------------------------------------------------------------------
% LTXE-SLIDE-START
% LTXE-SLIDE-UID:		7f19ed6c-81a57077-e0a83ae0-54255c7d
% LTXE-SLIDE-ORIGIN:	ca95152a-5b7e29c9-7e6593da-3ba78d86 English
% LTXE-SLIDE-TITLE:		API for Twitter Bootstrap modals (2)
% LTXE-SLIDE-REFERENCE:	Feature-63729-ApiForBootstrapModals.rst
% ------------------------------------------------------------------------------

\begin{frame}[fragile]
	\frametitle{Modifiche rilevanti}
	\framesubtitle{API per il modulo Twitter Bootstrap (2)}

	\begin{itemize}

		\item Esempio 2:

		\begin{lstlisting}
			TYPO3.Modal.confirm('Warning', 'You may break the internet!',
			  TYPO3.Severity.warning,
			  [
			    {
			      text: 'Break it',
			      active: true,
			      trigger: function() { ... }
			    },
			    {
			      text: 'Abort!',
			      trigger: function() {
			        TYPO3.Modal.dismiss();
			      }
			    }
			  ]
			);
		\end{lstlisting}

	\end{itemize}

\end{frame}

% ------------------------------------------------------------------------------
% LTXE-SLIDE-START
% LTXE-SLIDE-UID:		167572ce-9a037d0d-d16e8ccc-6148e4a6
% LTXE-SLIDE-ORIGIN:	4e7489f8-bced2d21-6e6b87d3-2700d161 English
% LTXE-SLIDE-TITLE:		JavaScript Storage API (1)
% LTXE-SLIDE-REFERENCE:	Feature-64031-JavaScript-Storage-API.rst
% ------------------------------------------------------------------------------

\begin{frame}[fragile]
	\frametitle{Modifiche rilevanti}
	\framesubtitle{Archivio API Javascript (1)}

	\begin{itemize}
		\item L'accesso alla configurazione utente del BE (\texttt{\$BE\_USER->uc}) può essere gestito
			in JavaScript usando semplici coppie chiave-valore
		\item Inoltre, il \href{http://www.w3.org/TR/webstorage/}{localStorage} di HTML5
			può essere utilizzato per memorizzare i dati nel browser dell'utente (client-side)

		\item Due nuovi oggetti globali TYPO3:
			\begin{itemize}
				\item \texttt{top.TYPO3.Storage.Client}
				\item \texttt{top.TYPO3.Storage.Persistent}
			\end{itemize}

		\item Ogni oggetto ha le API seguenti:
			\begin{itemize}
				\item \texttt{get(key)}: recupera il dato
				\item \texttt{set(key,value)}: scrive il dato
				\item \texttt{isset(key)}: verifica se la chiave viene usata
				\item \texttt{clear()}: svuota tutto l'archivio dei dati
			\end{itemize}

	\end{itemize}

\end{frame}

% ------------------------------------------------------------------------------
% LTXE-SLIDE-START
% LTXE-SLIDE-UID:		72a394f2-d505ff25-75ee947a-393521ad
% LTXE-SLIDE-ORIGIN:	bced2d21-6e6b87d3-2700d161-4e7489f8 English
% LTXE-SLIDE-TITLE:		JavaScript Storage API (2)
% LTXE-SLIDE-REFERENCE:	Feature-64031-JavaScript-Storage-API.rst
% ------------------------------------------------------------------------------

\begin{frame}[fragile]
	\frametitle{Modifiche rilevanti}
	\framesubtitle{Archivio API Javascript (2)}

	\begin{itemize}
		\item Esempio:
			\begin{lstlisting}
				// leggi il valore della chiave 'startModule'
				var value = top.TYPO3.Storage.Persistent.get('startModule');

				// scrivi il valore 'web_info' come chiave di 'start_module'
				top.TYPO3.Storage.Persistent.set('startModule', 'web_info');
			\end{lstlisting}

	\end{itemize}

\end{frame}

% ------------------------------------------------------------------------------
% LTXE-SLIDE-START
% LTXE-SLIDE-UID:		2ec79ecc-3f3581e2-abe4f41c-b20f869b
% LTXE-SLIDE-ORIGIN:	62879554-fac9feac-89f11d0e-afeefc65 English
% LTXE-SLIDE-TITLE:		Inline Rendering of Checkboxes
% LTXE-SLIDE-REFERENCE:	Feature-64190-FormEngineCheckboxElement.rst
% ------------------------------------------------------------------------------

\begin{frame}[fragile]
	\frametitle{Modifiche rilevanti}
	\framesubtitle{Inline Rendering dei Checkbox}

	% decrease font size for code listing
	\lstset{basicstyle=\tiny\ttfamily}

	\begin{itemize}

		\item L'impostazione \texttt{inline} dei checkbox per "cols" può essere usata per visualizzare i checkbox
			direttamente uno vicino all'altro per ridurre lo spazio utilizzato 

		\begin{lstlisting}
			'weekdays' => array(
			  'label' => 'Weekdays',
			  'config' => array(
			    'type' => 'check',
			    'items' => array(
			      array('Mo', ''),
			      array('Tu', ''),
			      array('We', ''),
			      array('Th', ''),
			      array('Fr', ''),
			      array('Sa', ''),
			      array('Su', '')
			    ),
			    'cols' => 'inline'
			  )
			),
			...
		\end{lstlisting}

	\end{itemize}

\end{frame}

% ------------------------------------------------------------------------------
% LTXE-SLIDE-START
% LTXE-SLIDE-UID:		5f66bbce-323a1a7f-04758955-c4fa92c6
% LTXE-SLIDE-ORIGIN:	75519e56-b1721347-754a017e-ac5d5e03 English
% LTXE-SLIDE-TITLE:		Content Object Registration
% LTXE-SLIDE-REFERENCE:	Feature-64386-ContentObjectRegistration.rst
% ------------------------------------------------------------------------------

\begin{frame}[fragile]
	\frametitle{Modifiche rilevanti}
	\framesubtitle{Registrazione dei Content Object}

	\begin{itemize}

		\item Una nuova opzione globale per registrare e/o estendere/sovrascrivere cObjects come ad esempio
			\texttt{TEXT} è stata introdotta

		\item Una lista di tutti i cObjects presenti è disponibile come:

			\begin{lstlisting}
				$GLOBALS['TYPO3_CONF_VARS']['FE']['ContentObjects']
			\end{lstlisting}

		\item Esempio: registra un nuovo cObject \texttt{EXAMPLE}

			\begin{lstlisting}
				$GLOBALS['TYPO3_CONF_VARS']['FE']['ContentObjects']['EXAMPLE'] =
				  Vendor\MyExtension\ContentObject\ExampleContentObject::class;
			\end{lstlisting}

		\item La classe registrata deve essere una sottoclasse di
			\small
				\texttt{TYPO3\textbackslash
					CMS\textbackslash
					Frontend\textbackslash
					ContentObject\textbackslash
					AbstractContentObject}
			\normalsize

		\item Registra la tua classe nella directory\newline
			\small
				\texttt{typo3conf/myextension/Classes/ContentObject/}
			\normalsize\newline
			per essere disponibile per un futuro meccanismo di autocaricamento

	\end{itemize}

\end{frame}

% ------------------------------------------------------------------------------
% LTXE-SLIDE-START
% LTXE-SLIDE-UID:		afda3f43-68613834-974efb25-e531cc8b
% LTXE-SLIDE-ORIGIN:	2a4acfc9-da01c536-7ff147af-7cd94754 English
% LTXE-SLIDE-TITLE:		Hooks and Signals (1)
% LTXE-SLIDE-REFERENCE:	Feature-52131-HookForPageRepositoryInit.rst
% ------------------------------------------------------------------------------

\begin{frame}[fragile]
	\frametitle{Modifiche rilevanti}
	\framesubtitle{Hooks e Signals (1)}

	\begin{itemize}

		\item Un nuovo hook è stato aggiunto alla fine di \texttt{PageRepository->init()},
			per permettere un intervento sulla visibilità delle pagine

		\item Registra l'hook come di seguito:
			\begin{lstlisting}
				$GLOBALS['TYPO3_CONF_VARS']['SC_OPTIONS']
				   [\TYPO3\CMS\Frontend\Page\PageRepository::class]['init']
			\end{lstlisting}

		\item La classe dell'hook deve implementare la seguente interfaccia:
			\begin{lstlisting}
				\TYPO3\CMS\Frontend\Page\PageRepositoryInitHookInterface
			\end{lstlisting}

	\end{itemize}

\end{frame}

% ------------------------------------------------------------------------------
% LTXE-SLIDE-START
% LTXE-SLIDE-UID:		3778fc55-a8bfd88b-7b2bfd2f-e2e164ff
% LTXE-SLIDE-ORIGIN:	da01c536-7ff147af-7cd94754-2a4acfc9 English
% LTXE-SLIDE-TITLE:		Hooks and Signals (2)
% LTXE-SLIDE-REFERENCE: Feature-58929-FooterHookInPageLayoutView.rst
% ------------------------------------------------------------------------------

\begin{frame}[fragile]
	\frametitle{Modifiche rilevanti}
	\framesubtitle{Hooks e Signals (2)}

	\begin{itemize}

		\item Un nuovo hook è stato aggiunto a PageLayoutView per modificare il rendering
			del footer di un elemento di contenuto.

		\item Esempio:
			\begin{lstlisting}
				$GLOBALS['TYPO3_CONF_VARS']['SC_OPTIONS']
				  ['cms/layout/class.tx_cms_layout.php']['tt_content_drawFooter'];
			\end{lstlisting}

		\item La classe dell'hook deve implementare la seguente interfaccia:
			\begin{lstlisting}
				\TYPO3\CMS\Backend\View\PageLayoutViewDrawFooterHookInterface
			\end{lstlisting}

	\end{itemize}

\end{frame}

% ------------------------------------------------------------------------------
% LTXE-SLIDE-START
% LTXE-SLIDE-UID:		245d47fe-b0f0485d-83fb7f44-fec0b781
% LTXE-SLIDE-ORIGIN:	babf7cab-2216fff5-9bf5848c-f76386df English
% LTXE-SLIDE-TITLE:		Hooks and Signals (3)
% LTXE-SLIDE-REFERENCE: Feature-61725-AddHookToBackendUtilityCountVersionsOfRecordsOnPage.rst
% ------------------------------------------------------------------------------

\begin{frame}[fragile]
	\frametitle{Modifiche rilevanti}
	\framesubtitle{Hooks e Signals (3)}

	\begin{itemize}

		\item Un nuovo hook è stato aggiunto come post processor di
			\small
				\texttt{BackendUtility::countVersionsOfRecordsOnPage}
			\normalsize

		\item Questo può essere utilizzato per visualizzare lo stato dei workspace in un albero di pagine per esempio
		\item Registra l'hook come di seguito:

			\begin{lstlisting}
				$GLOBALS['TYPO3_CONF_VARS']['SC_OPTIONS']
				  ['t3lib/class.t3lib_befunc.php']['countVersionsOfRecordsOnPage'][] =
				  'My\Package\HookClass->hookMethod';
			\end{lstlisting}

	\end{itemize}

\end{frame}

% ------------------------------------------------------------------------------
% LTXE-SLIDE-START
% LTXE-SLIDE-UID:		b8abbabd-b42ac0a5-ee144b3a-effbd5b3
% LTXE-SLIDE-ORIGIN:	2216fff5-9bf5848c-f76386df-babf7cab English
% LTXE-SLIDE-TITLE:		Hooks and Signals (4)
% LTXE-SLIDE-REFERENCE:	Feature-61711-SignalAtVeryEndOfDataPreprocessorFetchRecord.rst
% ------------------------------------------------------------------------------

\begin{frame}[fragile]
	\frametitle{Modifiche rilevanti}
	\framesubtitle{Hooks e Signals (4)}

	\begin{itemize}

		\item Un nuovo signal è stato aggiunto alla fine del metodo \texttt{DataPreprocessor::fetchRecord()}
		\item Questo può essere utilizzato per modificare l'array \texttt{regTableItems\_data} per esempio,
			al fine di visualizzare i dati modificati in TCEForms

			\begin{lstlisting}
				$this->getSignalSlotDispatcher()->dispatch(
				  \TYPO3\CMS\Backend\Form\DataPreprocessor::class,
				  'fetchRecordPostProcessing',
				  array($this)
				);
			\end{lstlisting}

	\end{itemize}

\end{frame}

% ------------------------------------------------------------------------------
% LTXE-SLIDE-START
% LTXE-SLIDE-UID:		27f6e972-ea65786b-7d27d9c3-96caf4eb
% LTXE-SLIDE-ORIGIN:	016bf36a-e935e00f-54bfa8f7-d0d2c16d English
% LTXE-SLIDE-TITLE:		Hooks and Signals (5)
% LTXE-SLIDE-REFERENCE:	Feature-62960-SignalForMailerInitialization.rst
% ------------------------------------------------------------------------------

\begin{frame}[fragile]
	\frametitle{Modifiche rilevanti}
	\framesubtitle{Hooks e Signals (5)}

	% decrease font size for code listing
	\lstset{basicstyle=\tiny\ttfamily}

	\begin{itemize}

		\item Un nuovo signal è stato aggiunto, che permette ulteriori elaborazioni sull'inizializzazione di un
			oggetto mailer, es. registrare un Swift Mailer plugin

			\begin{lstlisting}
				$signalSlotDispatcher = \TYPO3\CMS\Core\Utility\GeneralUtility::makeInstance(
				  \TYPO3\CMS\Extbase\SignalSlot\Dispatcher::class
				);

				$signalSlotDispatcher->connect(
				  \TYPO3\CMS\Core\Mail\Mailer::class,
				  'postInitializeMailer',
				  \Vendor\Package\Slots\MailerSlot::class,
				  'registerPlugin'
				);
		\end{lstlisting}

	\end{itemize}

\end{frame}

% ------------------------------------------------------------------------------
% LTXE-SLIDE-START
% LTXE-SLIDE-UID:		98905dab-1c788345-ca9ca706-99cc7876
% LTXE-SLIDE-ORIGIN:	e935e00f-54bfa8f7-d0d2c16d-016bf36a English
% LTXE-SLIDE-TITLE:		Multiple UID in PageRepository::getMenu()
% LTXE-SLIDE-REFERENCE: Feature-64257-MultipleUidInPageRepositoryGetMenu.rst
% ------------------------------------------------------------------------------

\begin{frame}[fragile]
	\frametitle{Modifiche rilevanti}
	\framesubtitle{Multipli UID in \texttt{PageRepository::getMenu()}}

	\begin{itemize}


		\item Il metodo \texttt{PageRepository::getMenu()} ora accetta un array, per poter
			definire più pagine root.

			\begin{lstlisting}
				$pageRepository = new \TYPO3\CMS\Frontend\Page\PageRepository();
				$pageRepository->init(FALSE);
				$rows = $pageRepository->getMenu(array(2, 3));
			\end{lstlisting}

	\end{itemize}

\end{frame}

% ------------------------------------------------------------------------------
