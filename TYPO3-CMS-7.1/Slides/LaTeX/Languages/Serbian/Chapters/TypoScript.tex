% ------------------------------------------------------------------------------
% TYPO3 CMS 7.1 - What's New - Chapter "TypoScript" (Serbian Version)
%
% @author	Michael Schams <schams.net>
% @license	Creative Commons BY-NC-SA 3.0
% @link		http://typo3.org/download/release-notes/whats-new/
% @language	Serbian
% ------------------------------------------------------------------------------
% LTXE-CHAPTER-UID:		d92ab5b9-9ffa4925-30a2e34f-0d6939e5
% LTXE-CHAPTER-NAME:	TypoScript
% ------------------------------------------------------------------------------

\section{TSconfig i TypoScript}
\begin{frame}[fragile]
	\frametitle{TSconfig i TypoScript}

	\begin{center}\huge{Poglavlje 2:}\end{center}
	\begin{center}\huge{\color{typo3darkgrey}\textbf{TSconfig i TypoScript}}\end{center}

\end{frame}

% ------------------------------------------------------------------------------
% LTXE-SLIDE-START
% LTXE-SLIDE-UID:		b3cf7e1c-946b1f75-cfefde4a-cc701095
% LTXE-SLIDE-ORIGIN:	a55771e2-74f3a0fc-5be51f93-8b40a5d8 English
% LTXE-SLIDE-TITLE:		StdWrap for page.headTag
% LTXE-SLIDE-REFERENCE:	Feature-22086-StdWrapForHeadTag.rst
% ------------------------------------------------------------------------------

\begin{frame}[fragile]
	\frametitle{TSconfig i TypoScript}
	\framesubtitle{StdWrap za page.headTag}

	\begin{itemize}
		\item TypoScript setting \texttt{page.headTag} sada ima \texttt{stdWrap} funkcionalnost

			\begin{lstlisting}
				page = PAGE
				page.headTag = <head>
				page.headTag.override = <head class="special">
				page.headTag.override.if {
		  		  isInList.field = uid
		  		  value = 24
				}
			\end{lstlisting}	

	\end{itemize}

\end{frame}

% ------------------------------------------------------------------------------
% LTXE-SLIDE-START
% LTXE-SLIDE-UID:		2b66714b-389db605-9f1456ca-e2271c88
% LTXE-SLIDE-ORIGIN:	618eb9ea-56d5a4eb-d5d69d66-cc62f953 English
% LTXE-SLIDE-TITLE:		Include JavaScript files asynchronously
% LTXE-SLIDE-REFERENCE:	Feature-28382-AddAsyncPropertyToJavaScriptFiles.rst
% ------------------------------------------------------------------------------

\begin{frame}[fragile]
	\frametitle{TSconfig i TypoScript}
	\framesubtitle{Asinhrono ukljucivanje JavaScript fajlova}

	\begin{itemize}
		\item JavaScript fajlovi se mogu ucitavati asinhrono

			\begin{lstlisting}
				page {
				  includeJS {
				    jsFile = /path/to/file.js
				    jsFile.async = 1
				  }
				}
			\end{lstlisting}

		\item Ovo utice na:

			\begin{itemize}
				\item \texttt{includeJSlibs} / \texttt{includeJSLibs}
				\item \texttt{includeJSFooterlibs}
				\item \texttt{includeJS}
				\item \texttt{includeJSFooter}
			\end{itemize}

	\end{itemize}

\end{frame}

% ------------------------------------------------------------------------------
% LTXE-SLIDE-START
% LTXE-SLIDE-UID:		b03f5a72-aee032bc-9de1a9a3-2cd84a2f
% LTXE-SLIDE-ORIGIN:	8a6ca08e-cf9ba481-18fd9411-30dd0a0f English
% LTXE-SLIDE-TITLE:		HMENU item selection via additionalWhere
% LTXE-SLIDE-REFERENCE:	Feature-46624-AdditionalWhereForMenu.rst
% ------------------------------------------------------------------------------

\begin{frame}[fragile]
	\frametitle{TSconfig i TypoScript}
	\framesubtitle{HMENU izbor stavki uz pomoc \texttt{additionalWhere}}

	\begin{itemize}

		\item TypoScript cObject \texttt{HMENU} ima novu osobinu \texttt{additionalWhere}
		\item Ovo dozvoljava precizniji upit u bazu (npr. filtriranje)

		\item Primer:

			\begin{lstlisting}
				lib.authormenu = HMENU
				lib.authormenu.1 = TMENU
				lib.authormenu.1.additionalWhere = AND author!=""
			\end{lstlisting}

	\end{itemize}

\end{frame}

% ------------------------------------------------------------------------------
% LTXE-SLIDE-START
% LTXE-SLIDE-UID:		4aa23f02-1983f345-a37b6f56-939dd75c
% LTXE-SLIDE-ORIGIN:	e07d1984-117d1212-a1f7fab8-8ea21ebc English
% LTXE-SLIDE-TITLE:		Additional properties for HMENU browse menus
% LTXE-SLIDE-REFERENCE:	Feature-57178-SpecialHmenuExcludeSpecialParameters.rst
% ------------------------------------------------------------------------------

\begin{frame}[fragile]
	\frametitle{TSconfig i TypoScript}
	\framesubtitle{Dodatne opcije za \texttt{HMENU} menije}

	\begin{itemize}
		\item Dve nove opcije za cObject \texttt{HMENU} (opcija "\texttt{special=browse"})\newline
			kako bi se stavke u meniju finije izabrale:
		
			\begin{itemize}
				\item \texttt{excludeNoSearchPages}
				\item \texttt{includeNotInMenu}
			\end{itemize}

		\item Primer:

			\begin{lstlisting}
				lib.browsemenu = HMENU
				lib.browsemenu.special = browse
				lib.browsemenu.special.excludeNoSearchPages = 1
				lib.browsemenu.includeNotInMenu = 1
			\end{lstlisting}

	\end{itemize}

\end{frame}

% ------------------------------------------------------------------------------
% LTXE-SLIDE-START
% LTXE-SLIDE-UID:		899bc333-c22fda2a-8d57c762-478c1089
% LTXE-SLIDE-ORIGIN:	a286a87a-161365ed-5560e66f-e9f1a2ee English
% LTXE-SLIDE-TITLE:		Multiple HTTP headers
% LTXE-SLIDE-REFERENCE:	Feature-56236-Multiple-HTTP-Headers-In-Frontend.rst
% ------------------------------------------------------------------------------

\begin{frame}[fragile]
	\frametitle{TSconfig i TypoScript}
	\framesubtitle{Visestruki HTTP hederi}

	\begin{itemize}

		\item HTTP hederi mogu se smestiti kao niz  (\small\texttt{config.additionalHeaders}\normalsize)
		\item Ovo dozvoljava konfigurisanje vise hedera u isto vreme

			\begin{lstlisting}
				config.additionalHeaders {
				  10 {
				    # header string
				    header = WWW-Authenticate: Negotiate

				    # (optional) replace previous headers with the same name (default: 1)
				    replace = 0

				    # (optional) force HTTP response code
				    httpResponseCode = 401
				  }
				  # set second additional HTTP header
				  20.header = Cache-control: Private
				}
			\end{lstlisting}

	\end{itemize}

\end{frame}

% ------------------------------------------------------------------------------
% LTXE-SLIDE-START
% LTXE-SLIDE-UID:		76b0a0b5-2b8d8aa0-36bf8eda-c43cf05c
% LTXE-SLIDE-ORIGIN:	a296b9b8-9403ef17-30f80f28-2a8a94ce English
% LTXE-SLIDE-TITLE:		Option "auto" added for config.absRefPrefix
% LTXE-SLIDE-REFERENCE:	Feature-58366-AutomaticAbsRefPrefix.rst
% ------------------------------------------------------------------------------

\begin{frame}[fragile]
	\frametitle{TSconfig i TypoScript}
	\framesubtitle{Dodata opcija "auto" za config.absRefPrefix}

	\begin{itemize}
		\item TypoScript setting \texttt{config.absRefPrefix} se moze iskoristiti da dozvoli URL prepisivanje.
			Kao alternativa za \texttt{config.baseURL} (za konfigurisanje posebnog domena), \texttt{absRefPrefix} moze automatski detektovati osnovu (root) sajta:

			\begin{lstlisting}
				config.absRefPrefix = auto

				# ...instead of:
				[ApplicationContext = Production]
				config.absRefPrefix = /

				[ApplicationContext = Testing]
				config.absRefPrefix = /my_site_root/
			\end{lstlisting}

		\smaller
			Napomena: Ova opcija je takodje sigurna za okruzenja sa vise domena kako bi se izbeglo dupliciranje mehanizma kesiranja
		\normalsize

	\end{itemize}

\end{frame}

% ------------------------------------------------------------------------------
% LTXE-SLIDE-START
% LTXE-SLIDE-UID:		74536be0-e6eeb2cd-2720737a-00d7b1c6
% LTXE-SLIDE-ORIGIN:	8210d253-d630446f-ba747ba6-5568843d English
% LTXE-SLIDE-TITLE:		Two-letter ISO code for sys_language (1)
% LTXE-SLIDE-REFERENCE:	Feature-61542-AddIsoLanguageKeys.rst
% ------------------------------------------------------------------------------

\begin{frame}[fragile]
	\frametitle{TSconfig i TypoScript}
	\framesubtitle{Dvoslovni ISO kod za sys\_language (1)}

	\begin{itemize}
		\item Baratanje sa jezicima obavlja se preko rekorda smestenih u DB tabelu
			\texttt{sys\_language}, koji su najcesce povezani preko \texttt{sys\_language\_uid}
		\item U TYPO3 CMS 7.1, predstavljen je dvoslovni ISO 639-1:

			\begin{itemize}
				\item Novo DB polje: \texttt{sys\_language.language\_isocode}
				\item Nova TypoScript opcija: \texttt{sys\_language\_isocode}
			\end{itemize}


		\vspace{1cm}

		\small
			Napomena: \textbf{ISO 639} je set standarda Internacionalne Organizacije za Standarde.
			Lista ISO 639-1 kodova dostupna je na\newline
			\url{http://en.wikipedia.org/wiki/List_of_ISO_639-1_codes}
		\normalsize

	\end{itemize}

\end{frame}

% ------------------------------------------------------------------------------
% LTXE-SLIDE-START
% LTXE-SLIDE-UID:		70b6f77d-d575cdbd-27e5b287-8ef164d5
% LTXE-SLIDE-ORIGIN:	d630446f-ba747ba6-5568843d-8210d253 English
% LTXE-SLIDE-TITLE:		Two-letter ISO code for sys_language (2)
% LTXE-SLIDE-REFERENCE:	Feature-61542-AddIsoLanguageKeys.rst
% ------------------------------------------------------------------------------

\begin{frame}[fragile]
	\frametitle{TSconfig i TypoScript}
	\framesubtitle{Dvoslovni ISO kod za sys\_language (2)}

	\begin{itemize}
		\item Primer:

			\begin{lstlisting}
				# Danish by default
				config.sys_language_uid = 0
				config.sys_language_isocode_default = da

				[globalVar = GP:L = 1]
				  # ISO code stored in table sys_language (uid 1)
				  config.sys_language_uid = 1
				  # overwrite ISO code as required
				  config.sys_language_isocode = fr
				[GLOBAL]

				page.10 = TEXT
				page.10.data = TSFE:sys_language_isocode
				page.10.wrap = <div class="main" data-language="|">
			\end{lstlisting}

	\end{itemize}

\end{frame}

% ------------------------------------------------------------------------------
% LTXE-SLIDE-START
% LTXE-SLIDE-UID:		cd262dac-29d315ac-5d76e2b7-a7d6bca7
% LTXE-SLIDE-ORIGIN:	df2f7fec-fb8bc3b9-446fb779-c409ef27 English
% LTXE-SLIDE-TITLE:		Custom TypoScript Conditions in Backend
% LTXE-SLIDE-REFERENCE:	Feature-63600-CustomTypoScriptConditionsInBackend.rst
% ------------------------------------------------------------------------------

\begin{frame}[fragile]
	\frametitle{TSconfig i TypoScript}
	\framesubtitle{Prilagodjeni TypoScript uslovi u administratorskom interfejsu}

	% decrease font size for code listing
	\lstset{basicstyle=\tiny\ttfamily}

	\begin{itemize}
	
		\item Podrska za prilagodjene uslove za \textbf{korisnicki interfejs} je vec predstavljena u TYPO3 CMS 7.0
		\item Od TYPO3 CMS 7.1, takodje je moguce koristiti prilagodjene uslove i u \textbf{administratorskom interfejsu}
		\item Uslov mora biti izveden iz \texttt{AbstractCondition} i da implementira metod \texttt{matchCondition()}
		\item Primer koriscenja u TypoScript-u: 

			\begin{lstlisting}
				[BigCompanyName\TypoScriptLovePackage\MyCustomTypoScriptCondition]

				[BigCompanyName\TypoScriptLovePackage\MyCustomTypoScriptCondition = 7]

				[BigCompanyName\TypoScriptLovePackage\MyCustomTypoScriptCondition = 7, != 6]

				[BigCompanyName\TypoScriptLovePackage\MyCustomTypoScriptCondition = {$mysite.myconstant}]
			\end{lstlisting}

	\end{itemize}

\end{frame}

% ------------------------------------------------------------------------------
% LTXE-SLIDE-START
% LTXE-SLIDE-UID:		542bf4c4-83887e7a-5adc658d-ff70e6fe
% LTXE-SLIDE-ORIGIN:	b9687dbb-c0502234-7ca16b06-1879fe80 English
% LTXE-SLIDE-TITLE:		FormEngine: customize icons via PageTSconfig
% LTXE-SLIDE-REFERENCE:	Feature-35891-AddTCAItemsWithIconsViaPageTSConfig.rst
% ------------------------------------------------------------------------------

\begin{frame}[fragile]
	\frametitle{TSconfig i TypoScript}
	\framesubtitle{Prilagodjavanje ikonica preko PageTSconfig}

	\begin{itemize}
		\item Par vrednost/labela oznacenog polja vec se moze konfigurisati preko PageTSconfig opcije \texttt{addItems}
		\item Takodje je sada moguce uticati na izgled \textbf{ikonica} ovih polja 

			\begin{itemize}
				\item Opcija 1: koristeci \texttt{addItems} i podosobinom \texttt{.icon}
				\item Opcija 2: koristeci \texttt{altIcons} (sve stavke)
			\end{itemize}

		\item Primer:

			\begin{lstlisting}
				TCEFORM.pages.doktype.addItems {
				  10 = My Label
				  10.icon = EXT:t3skin/icons/gfx/i/pages.gif
				}
				TCEFORM.pages.doktype.altIcons {
				  10 = EXT:myext/icon.gif
				}
			\end{lstlisting}

	\end{itemize}

\end{frame}

% ------------------------------------------------------------------------------
% LTXE-SLIDE-START
% LTXE-SLIDE-UID:		e0977a03-f6b92a1d-44c3c213-b0e7367f
% LTXE-SLIDE-ORIGIN:	b8794774-4912764b-cd1cc91d-de4396fe English
% LTXE-SLIDE-TITLE:		Extend element browser with mount points
% LTXE-SLIDE-REFERENCE:	Feature-50780-AppendElementBrowserMountPoints.rst
% ------------------------------------------------------------------------------

\begin{frame}[fragile]
	\frametitle{TSconfig i TypoScript}
	\framesubtitle{Pretrazivac elemenata prosiren sa mount point-ima}

	\begin{itemize}
		\item Nova UserTSconfig opcija \texttt{.append} omogucava administratorima da \textbf{dodaju}
			mount point-e, umesto da zamenjuju konfigurisane DB mount point-e korisnika

		\item Primer:

			\begin{lstlisting}
				options.pageTree.altElementBrowserMountPoints = 20,31
				options.pageTree.altElementBrowserMountPoints.append = 1
			\end{lstlisting}

	\end{itemize}

\end{frame}

% ------------------------------------------------------------------------------
% LTXE-SLIDE-START
% LTXE-SLIDE-UID:		6a48bdfe-5bd72a48-b94f7413-ac764557
% LTXE-SLIDE-ORIGIN:	04b7f7c0-af900c25-5f16f178-8a94fcca English
% LTXE-SLIDE-TITLE:		Label override of checkboxes and radio buttons by TSconfig
% LTXE-SLIDE-REFERENCE:	Feature-58033-AltLabelsForFormEngineCheckboxAndRadioButtons.rst
% ------------------------------------------------------------------------------

\begin{frame}[fragile]
	\frametitle{TSconfig i TypoScript}
	\framesubtitle{Preimenovanje labela cekboksova i radio dugmica}

	% decrease font size for code listing
	\lstset{basicstyle=\tiny\ttfamily}

	\begin{itemize}

		\item Labele radio dugmica i cekboksova sada mogu da budu izmenjene
		\item Primer:

			\begin{lstlisting}
				// field with a single checkbox (use ".default")
				TCEFORM.pages.hidden.altLabels.default = new label
				TCEFORM.pages.hidden.altLabels.default = LLL:path/to/languagefile.xlf:individualLabel

				// field with multiple checkboxes (0, 1, 2, 3...)
				TCEFORM.pages.l18n_cfg.altLabels.0 = new label of first checkbox
				TCEFORM.pages.l18n_cfg.altLabels.1 = new label of second checkbox
				TCEFORM.pages.l18n_cfg.altLabels.2 = new label of third checkbox
				...
			\end{lstlisting}

	\end{itemize}

\end{frame}

% ------------------------------------------------------------------------------
% LTXE-SLIDE-START
% LTXE-SLIDE-UID:		0d6aae7f-0ff454a0-0474288e-15820f01
% LTXE-SLIDE-ORIGIN:	6cdd5010-f7d7d7bd-0a3c74ea-05424fb0 English
% LTXE-SLIDE-TITLE:		Miscellaneous (1)
% LTXE-SLIDE-REFERENCE: Feature-xxxxx ... UserTSconfig-width-and-height-of-element-browser
% LTXE-SLIDE-REFERENCE: Feature-59646-AddRteConfigurationPropertyButtonsLinkTypePropertiesTargetDefault.rst
% ------------------------------------------------------------------------------

\begin{frame}[fragile]
	\frametitle{TSconfig i TypoScript}
	\framesubtitle{Razno (1)}

	\begin{itemize}
		\item Sirina i visina pretrazivaca elemenata se moze konfigurisati pomocu UserTSconfig:

			\begin{lstlisting}
				options.popupWindowSize = 400x900
				options.RTE.popupWindowSize = 200x200
			\end{lstlisting}


		\item PageTSconfig: nova opcija za konfigurisanje RTE-a se moze iskoristit da se postave posrazumevani target-i za linkove odabranog tipa:

			\begin{lstlisting}
				buttons.link.[type].properties.target.default
			\end{lstlisting}

			Gde \texttt{[type]} moze bit \texttt{page}, \texttt{file}, \texttt{url}, \texttt{mail} ili \texttt{spec}\newline
			(prosirenja mogu omoguciti dodatne tipove)

	\end{itemize}

\end{frame}

% ------------------------------------------------------------------------------
% LTXE-SLIDE-START
% LTXE-SLIDE-UID:		e42006d1-4548d618-6a0ddecd-b1303f12
% LTXE-SLIDE-ORIGIN:	80595308-128c5961-8ae4cf89-8611ce66 English
% LTXE-SLIDE-TITLE:		Miscellaneous (2)
% LTXE-SLIDE-REFERENCE:	Feature-16794-MakeSectionLinkingForIndexedSearchResultsConfigurable.rst
% LTXE-SLIDE-REFERENCE: Feature-20767-getDataByNestedKey.rst
% LTXE-SLIDE-REFERENCE: Feature-33491-StdWrapForTitleTag.rst
% ------------------------------------------------------------------------------

\begin{frame}[fragile]
	\frametitle{TSconfig i TypoScript}
	\framesubtitle{Razno (2)}

	\begin{itemize}

		\item Naslovi rezultata indeksirane pretrage po podrazumevanim podesavanjima su linkovi. 
			Sada je moguce onesposobiti ove linkove i naslove prikazati kao obican tekst

			\begin{lstlisting}
				plugin.tx_indexedsearch.linkSectionTitles = 0
			\end{lstlisting}

		\item \texttt{getData} sada moze da pristupi podatcima \texttt{polja} (ne samo nizovi
			\texttt{GPVar} i \texttt{TSFE}):
		
			\begin{lstlisting}
				10 = TEXT
				10.data = field:fieldname|level1|level2
			\end{lstlisting}

		\item TypoScript podesavanje \texttt{config.pageTitle} sada ima \texttt{stdWrap} funkcionalnost

			\begin{lstlisting}
				# make value of <title> upper case
				page = PAGE
				page.config.pageTitle.case = upper
			\end{lstlisting}

	\end{itemize}

\end{frame}

% ------------------------------------------------------------------------------
