% ------------------------------------------------------------------------------
% TYPO3 CMS 7.1 - What's New - Chapter "In-Depth Changes" (Serbian Version)
%
% @author	Michael Schams <schams.net>
% @license	Creative Commons BY-NC-SA 3.0
% @link		http://typo3.org/download/release-notes/whats-new/
% @language	Serbian
% ------------------------------------------------------------------------------
% LTXE-CHAPTER-UID:		7cb2d402-105890ae-a1632623-719adbb6
% LTXE-CHAPTER-NAME:	In-Depth Changes
% ------------------------------------------------------------------------------

\section{Korenite promene}
\begin{frame}[fragile]
	\frametitle{Korenite promene}

	\begin{center}\huge{Poglavlje 3:}\end{center}
	\begin{center}\huge{\color{typo3darkgrey}\textbf{Korenite promene}}\end{center}

\end{frame}

% ------------------------------------------------------------------------------
% LTXE-SLIDE-START
% LTXE-SLIDE-UID:		1e0da623-1bc21bf9-89b52b3c-f1522604
% LTXE-SLIDE-ORIGIN:	505c827e-aab9a39a-612d0288-bb873878 English
% LTXE-SLIDE-TITLE:		Maximum chars in text element
% LTXE-SLIDE-REFERENCE:	Feature-24906-MaxForTextElement.rst
% ------------------------------------------------------------------------------

\begin{frame}[fragile]
	\frametitle{Korenite promene}
	\framesubtitle{TCA: Maksimum karaktera u tekstualnom elementu}

	\begin{itemize}
		\item TCA tip \texttt{text} sada podrzava HTML5 atribut \texttt{maxlength}
			 kako bi se ogranicila duzina teksta (napomena: novi red se obicno racuna kao dva karaktera)

			\begin{lstlisting}
				'teaser' => array(
				  'label' => 'Teaser',
				  'config' => array(
				    'type' => 'text',
				    'cols' => 60,
				    'rows' => 2,
				    'max' => '30' // <-- maxlength
				  )
				),
			\end{lstlisting}

			Imati na umu da ne pordzavaju svi pretrazivaci ovaj atribut.\newline
			Za detalje pogledati \href{http://www.w3schools.com/tags/att_textarea_maxlength.asp}{Browser Support List}.

	\end{itemize}

\end{frame}

% ------------------------------------------------------------------------------
% LTXE-SLIDE-START
% LTXE-SLIDE-UID:		676df9b8-0dc99a50-2617ce90-f30cded1
% LTXE-SLIDE-ORIGIN:	a3d28e8d-6d40bc8f-f60bb05f-d9e31bba English
% LTXE-SLIDE-TITLE:		New SplFileInfo implementation
% LTXE-SLIDE-REFERENCE:	Feature-60019-SplFileInfo-MimeTypeGuesser-hook.rst
% ------------------------------------------------------------------------------

\begin{frame}[fragile]
	\frametitle{Korenite promene}
	\framesubtitle{Nova SplFileInfo implementacija}

	% decrease font size for code listing
	\lstset{basicstyle=\smaller\ttfamily}

	\begin{itemize}

		\item Nova klasa:
			\texttt{TYPO3\textbackslash
				CMS\textbackslash
				Core\textbackslash
				Type\textbackslash
				File\textbackslash
				FileInfo}

		\item Ova klasa nasledjuje klasu \texttt{SplFileInfo}, koja dozvoljava dohvatanje meta informacija iz fajlova

			\begin{lstlisting}
				$fileIdentifier = '/tmp/foo.html';
				$fileInfo = GeneralUtility::makeInstance(
				  \TYPO3\CMS\Core\Type\File\FileInfo::class,
				  $fileIdentifier
				);
				echo $fileInfo->getMimeType(); // output: text/html
			\end{lstlisting}

		\item Za prilagodjene implementacije moze se koristit sledeci hook:

			\begin{lstlisting}
				$GLOBALS['TYPO3_CONF_VARS']['SC_OPTIONS']
				  [\TYPO3\CMS\Core\Type\File\FileInfo::class]['mimeTypeGuessers']
			\end{lstlisting}

	\end{itemize}

\end{frame}

% ------------------------------------------------------------------------------
% LTXE-SLIDE-START
% LTXE-SLIDE-UID:		2681a71b-e742e4ab-ef977cce-cde73f1a
% LTXE-SLIDE-ORIGIN:	5351c635-145cc752-9c930534-3a8b4e50 English
% LTXE-SLIDE-TITLE:		UserFunc in TCA Display Condition
% LTXE-SLIDE-REFERENCE:	Feature-62944-UserFuncAsDisplayCond.rst
% ------------------------------------------------------------------------------

\begin{frame}[fragile]
	\frametitle{Korenite promene}
	\framesubtitle{UserFunc u TCA Display Condition}

	\begin{itemize}
		\item userFunc \texttt{displayCondition} dozvoljava da se proveri bilo koji uslov ili stanje
		\item Ukoliko situacija ne moze biti procenjena nijednim postojecim proverama, developeri mogu napraviti svoje funkcije\newline
			(vratiti \texttt{TRUE/FALSE} da se prikaze/sakrije odredjeno TCA polje)

			\begin{lstlisting}
				$GLOBALS['TCA']['tt_content']['columns']['bodytext']['displayCond'] =
				  'USER:Vendor\\Example\\User\\ElementConditionMatcher->
				    checkHeaderGiven:any:more:information';
			\end{lstlisting}

	\end{itemize}

\end{frame}

% ------------------------------------------------------------------------------
% LTXE-SLIDE-START
% LTXE-SLIDE-UID:		e004a8af-b9effaff-946acdec-cfc6225e
% LTXE-SLIDE-ORIGIN:	3ba78d86-ca95152a-5b7e29c9-7e6593da English
% LTXE-SLIDE-TITLE:		API for Twitter Bootstrap modals (1)
% LTXE-SLIDE-REFERENCE:	Feature-63729-ApiForBootstrapModals.rst
% ------------------------------------------------------------------------------

\begin{frame}[fragile]
	\frametitle{Korenite promene}
	\framesubtitle{API za Twitter Bootstrap modals (1)}

	% decrease font size for code listing
	\lstset{basicstyle=\smaller\ttfamily}

	\begin{itemize}

		\item Dve nove API metode za pravljenje/uklanjanje modal popup-a:
			\begin{itemize}
				\item \texttt{TYPO3.Modal.confirm(title, content, severity, buttons)}
				\item \texttt{TYPO3.Modal.dismiss()}
			\end{itemize}

		\item Opcije \texttt{title} i \texttt{content} su obavezne
		\item Opcije \texttt{buttons.text} i \texttt{buttons.trigger} su takodje obavezne ako se koristi \texttt{buttons}

		\item Primer 1:

			\begin{lstlisting}
				TYPO3.Modal.confirm(
				  'The title of the modal',         // title
				  'This the the body of the modal', // content
				  TYPO3.Severity.warning            // severity
				);
			\end{lstlisting}

	\end{itemize}

\end{frame}

% ------------------------------------------------------------------------------
% LTXE-SLIDE-START
% LTXE-SLIDE-UID:		c55a8e05-fb601f3a-76ecbb9a-434a4a58
% LTXE-SLIDE-ORIGIN:	ca95152a-5b7e29c9-7e6593da-3ba78d86 English
% LTXE-SLIDE-TITLE:		API for Twitter Bootstrap modals (2)
% LTXE-SLIDE-REFERENCE:	Feature-63729-ApiForBootstrapModals.rst
% ------------------------------------------------------------------------------

\begin{frame}[fragile]
	\frametitle{Korenite promene}
	\framesubtitle{API za Twitter Bootstrap modals (2)}

	\begin{itemize}

		\item Primer 2:

		\begin{lstlisting}
			TYPO3.Modal.confirm('Warning', 'You may break the internet!',
			  TYPO3.Severity.warning,
			  [
			    {
			      text: 'Break it',
			      active: true,
			      trigger: function() { ... }
			    },
			    {
			      text: 'Abort!',
			      trigger: function() {
			        TYPO3.Modal.dismiss();
			      }
			    }
			  ]
			);
		\end{lstlisting}

	\end{itemize}

\end{frame}

% ------------------------------------------------------------------------------
% LTXE-SLIDE-START
% LTXE-SLIDE-UID:		64890c30-75ed381a-850e5a38-dbc1b701
% LTXE-SLIDE-ORIGIN:	4e7489f8-bced2d21-6e6b87d3-2700d161 English
% LTXE-SLIDE-TITLE:		JavaScript Storage API (1)
% LTXE-SLIDE-REFERENCE:	Feature-64031-JavaScript-Storage-API.rst
% ------------------------------------------------------------------------------

\begin{frame}[fragile]
	\frametitle{Korenite promene}
	\framesubtitle{JavaScript Storage API (1)}

	\begin{itemize}
		\item Pristup konfiguraciji administratora (\texttt{\$BE\_USER->uc}) moze se obavljati preko JavaScript-a koristeci jednostavne key-value parove
		\item Dodatno, HTML5 \href{http://www.w3.org/TR/webstorage/}{localStorage}
			se moze koristiti za skladistenje podataka u korisnikovom pretrazivacu (klijentska strana)

		\item Dva nova globalna TYPO3 objekta:
			\begin{itemize}
				\item \texttt{top.TYPO3.Storage.Client}
				\item \texttt{top.TYPO3.Storage.Persistent}
			\end{itemize}

		\item Svaki objekat ima sledece API metode:
			\begin{itemize}
				\item \texttt{get(key)}: hvata podatke
				\item \texttt{set(key,value)}: upisuje podatke
				\item \texttt{isset(key)}: proverava da li se kljuc vec koristi
				\item \texttt{clear()}: brise sve sacuvane podatke
			\end{itemize}

	\end{itemize}

\end{frame}

% ------------------------------------------------------------------------------
% LTXE-SLIDE-START
% LTXE-SLIDE-UID:		34ea249e-4cd7c19d-ffaa7ee0-bfa26516
% LTXE-SLIDE-ORIGIN:	bced2d21-6e6b87d3-2700d161-4e7489f8 English
% LTXE-SLIDE-TITLE:		JavaScript Storage API (2)
% LTXE-SLIDE-REFERENCE:	Feature-64031-JavaScript-Storage-API.rst
% ------------------------------------------------------------------------------

\begin{frame}[fragile]
	\frametitle{Korenite promene}
	\framesubtitle{JavaScript Storage API (2)}

	\begin{itemize}
		\item Primer:
			\begin{lstlisting}
				// get value of key 'startModule'
				var value = top.TYPO3.Storage.Persistent.get('startModule');

				// write value 'web_info' as key 'start_module'
				top.TYPO3.Storage.Persistent.set('startModule', 'web_info');
			\end{lstlisting}

	\end{itemize}

\end{frame}

% ------------------------------------------------------------------------------
% LTXE-SLIDE-START
% LTXE-SLIDE-UID:		0d5b3e9b-38486de3-d15bc704-d9010357
% LTXE-SLIDE-ORIGIN:	62879554-fac9feac-89f11d0e-afeefc65 English
% LTXE-SLIDE-TITLE:		Inline Rendering of Checkboxes
% LTXE-SLIDE-REFERENCE:	Feature-64190-FormEngineCheckboxElement.rst
% ------------------------------------------------------------------------------

\begin{frame}[fragile]
	\frametitle{Korenite promene}
	\framesubtitle{Ispisivanje cekboksova u nizu}

	% decrease font size for code listing
	\lstset{basicstyle=\tiny\ttfamily}

	\begin{itemize}

		\item Cekboks podesavanje \texttt{inline} za "cols", moze se koristiti kako bi se cekboksovi prikazivali jedan do drugog
			i time smanjilo nepotrebno oduzimanje prostora

		\begin{lstlisting}
			'weekdays' => array(
			  'label' => 'Weekdays',
			  'config' => array(
			    'type' => 'check',
			    'items' => array(
			      array('Mo', ''),
			      array('Tu', ''),
			      array('We', ''),
			      array('Th', ''),
			      array('Fr', ''),
			      array('Sa', ''),
			      array('Su', '')
			    ),
			    'cols' => 'inline'
			  )
			),
			...
		\end{lstlisting}

	\end{itemize}

\end{frame}

% ------------------------------------------------------------------------------
% LTXE-SLIDE-START
% LTXE-SLIDE-UID:		6e7983d6-b9a10381-2cc37beb-d432e627
% LTXE-SLIDE-ORIGIN:	75519e56-b1721347-754a017e-ac5d5e03 English
% LTXE-SLIDE-TITLE:		Content Object Registration
% LTXE-SLIDE-REFERENCE:	Feature-64386-ContentObjectRegistration.rst
% ------------------------------------------------------------------------------

\begin{frame}[fragile]
	\frametitle{Korenite promene}
	\framesubtitle{Registrovanje objekata sadrzaja}

	\begin{itemize}

		\item Uvedena je nova globalna opcija za registrovanje i/ili prosirivanje/prepisivanje cObject-a kao sto je \texttt{TEXT}

		\item Lista svih dostupnih cObject-a je dostupna kao:

			\begin{lstlisting}
				$GLOBALS['TYPO3_CONF_VARS']['FE']['ContentObjects']
			\end{lstlisting}

		\item Primer: registruj novi cObject \texttt{EXAMPLE}

			\begin{lstlisting}
				$GLOBALS['TYPO3_CONF_VARS']['FE']['ContentObjects']['EXAMPLE'] =
				  Vendor\MyExtension\ContentObject\ExampleContentObject::class;
			\end{lstlisting}

		\item Registrovana klasa mora biti podklasa
			\small
				\texttt{TYPO3\textbackslash
					CMS\textbackslash
					Frontend\textbackslash
					ContentObject\textbackslash
					AbstractContentObject}
			\normalsize

		\item Sacuvajte svoje klase u direktorijumu\newline
			\small
				\texttt{typo3conf/myextension/Classes/ContentObject/}
			\normalsize\newline
			kako bi bile spremne za automatsko ucitavanje

	\end{itemize}

\end{frame}

% ------------------------------------------------------------------------------
% LTXE-SLIDE-START
% LTXE-SLIDE-UID:		ddb7476f-34f5ef76-b4ab88d4-95c6cf5d
% LTXE-SLIDE-ORIGIN:	2a4acfc9-da01c536-7ff147af-7cd94754 English
% LTXE-SLIDE-TITLE:		Hooks and Signals (1)
% LTXE-SLIDE-REFERENCE:	Feature-52131-HookForPageRepositoryInit.rst
% ------------------------------------------------------------------------------

\begin{frame}[fragile]
	\frametitle{Korenite promene}
	\framesubtitle{Hooks and Signals (1)}

	\begin{itemize}

		\item Novi hook je dodat na kraju \texttt{PageRepository->init()}
			i dozvoljava uticanje na vidljivost strane

		\item Registrovati hook na sledeci nacin:
			\begin{lstlisting}
				$GLOBALS['TYPO3_CONF_VARS']['SC_OPTIONS']
				   [\TYPO3\CMS\Frontend\Page\PageRepository::class]['init']
			\end{lstlisting}

		\item Hook mora da implementira sledeci interfejs:
			\begin{lstlisting}
				\TYPO3\CMS\Frontend\Page\PageRepositoryInitHookInterface
			\end{lstlisting}

	\end{itemize}

\end{frame}

% ------------------------------------------------------------------------------
% LTXE-SLIDE-START
% LTXE-SLIDE-UID:		b04e2d06-e7d0722d-32792581-db040923
% LTXE-SLIDE-ORIGIN:	da01c536-7ff147af-7cd94754-2a4acfc9 English
% LTXE-SLIDE-TITLE:		Hooks and Signals (2)
% LTXE-SLIDE-REFERENCE: Feature-58929-FooterHookInPageLayoutView.rst
% ------------------------------------------------------------------------------

\begin{frame}[fragile]
	\frametitle{Korenite promene}
	\framesubtitle{Hooks and Signals (2)}

	\begin{itemize}

		\item PageLayoutView-u je pridodat novi hook kako bi se moglo manipulisati renderanjem futera kontent elemenata.

		\item Primer:
			\begin{lstlisting}
				$GLOBALS['TYPO3_CONF_VARS']['SC_OPTIONS']
				  ['cms/layout/class.tx_cms_layout.php']['tt_content_drawFooter'];
			\end{lstlisting}

		\item Hook mora da implementira sledeci interfejs:
			\begin{lstlisting}
				\TYPO3\CMS\Backend\View\PageLayoutViewDrawFooterHookInterface
			\end{lstlisting}

	\end{itemize}

\end{frame}

% ------------------------------------------------------------------------------
% LTXE-SLIDE-START
% LTXE-SLIDE-UID:		f421297a-925fecfc-79f7915d-f2829476
% LTXE-SLIDE-ORIGIN:	babf7cab-2216fff5-9bf5848c-f76386df English
% LTXE-SLIDE-TITLE:		Hooks and Signals (3)
% LTXE-SLIDE-REFERENCE: Feature-61725-AddHookToBackendUtilityCountVersionsOfRecordsOnPage.rst
% ------------------------------------------------------------------------------

\begin{frame}[fragile]
	\frametitle{Korenite promene}
	\framesubtitle{Hooks and Signals (3)}

	\begin{itemize}

		\item Novi hook je dodat kao post-procesor za
			\small
				\texttt{BackendUtility::countVersionsOfRecordsOnPage}
			\normalsize

		\item Ovo se na primer moze iskoristiti kako bi se vizuelizovali statusi workspace u stablu stranica
		\item Registorvati hook na sledeci nacin:

			\begin{lstlisting}
				$GLOBALS['TYPO3_CONF_VARS']['SC_OPTIONS']
				  ['t3lib/class.t3lib_befunc.php']['countVersionsOfRecordsOnPage'][] =
				  'My\Package\HookClass->hookMethod';
			\end{lstlisting}

	\end{itemize}

\end{frame}

% ------------------------------------------------------------------------------
% LTXE-SLIDE-START
% LTXE-SLIDE-UID:		c14de834-57089f81-5d15bdc3-bf210739
% LTXE-SLIDE-ORIGIN:	2216fff5-9bf5848c-f76386df-babf7cab English
% LTXE-SLIDE-TITLE:		Hooks and Signals (4)
% LTXE-SLIDE-REFERENCE:	Feature-61711-SignalAtVeryEndOfDataPreprocessorFetchRecord.rst
% ------------------------------------------------------------------------------

\begin{frame}[fragile]
	\frametitle{Korenite promene}
	\framesubtitle{Hooks and Signals (4)}

	\begin{itemize}

		\item Dodat je novi signal na kraju metode \texttt{DataPreprocessor::fetchRecord()}
		\item Ovo se na primer moze iskoristiti za manipulisanje nizom \texttt{regTableItems\_data},
			kako bi se dobili izmenjeni podaci u TCAForms

			\begin{lstlisting}
				$this->getSignalSlotDispatcher()->dispatch(
				  \TYPO3\CMS\Backend\Form\DataPreprocessor::class,
				  'fetchRecordPostProcessing',
				  array($this)
				);
			\end{lstlisting}

	\end{itemize}

\end{frame}

% ------------------------------------------------------------------------------
% LTXE-SLIDE-START
% LTXE-SLIDE-UID:		8f49c24e-ff766d63-6028615a-51918de6
% LTXE-SLIDE-ORIGIN:	016bf36a-e935e00f-54bfa8f7-d0d2c16d English
% LTXE-SLIDE-TITLE:		Hooks and Signals (5)
% LTXE-SLIDE-REFERENCE:	Feature-62960-SignalForMailerInitialization.rst
% ------------------------------------------------------------------------------

\begin{frame}[fragile]
	\frametitle{Korenite promene}
	\framesubtitle{Hooks and Signals (5)}

	% decrease font size for code listing
	\lstset{basicstyle=\tiny\ttfamily}

	\begin{itemize}

		\item Dodat je novi signal, koji dozvoljava dodatnu obradu podataka nakon inicijalizacije
			mailer objekta, npr. registrovanje Swift Mailer-a

			\begin{lstlisting}
				$signalSlotDispatcher = \TYPO3\CMS\Core\Utility\GeneralUtility::makeInstance(
				  \TYPO3\CMS\Extbase\SignalSlot\Dispatcher::class
				);

				$signalSlotDispatcher->connect(
				  \TYPO3\CMS\Core\Mail\Mailer::class,
				  'postInitializeMailer',
				  \Vendor\Package\Slots\MailerSlot::class,
				  'registerPlugin'
				);
		\end{lstlisting}

	\end{itemize}

\end{frame}

% ------------------------------------------------------------------------------
% LTXE-SLIDE-START
% LTXE-SLIDE-UID:		b88c8c57-da1d8f72-8c9c9643-e9071e46
% LTXE-SLIDE-ORIGIN:	e935e00f-54bfa8f7-d0d2c16d-016bf36a English
% LTXE-SLIDE-TITLE:		Multiple UID in PageRepository::getMenu()
% LTXE-SLIDE-REFERENCE: Feature-64257-MultipleUidInPageRepositoryGetMenu.rst
% ------------------------------------------------------------------------------

\begin{frame}[fragile]
	\frametitle{Korenite promene}
	\framesubtitle{Visestruki UID in \texttt{PageRepository::getMenu()}}

	\begin{itemize}


		\item Metod \texttt{PageRepository::getMenu()} sada prihvata nizove kako bi se moglo definisati vise pocetnih (root) strana

			\begin{lstlisting}
				$pageRepository = new \TYPO3\CMS\Frontend\Page\PageRepository();
				$pageRepository->init(FALSE);
				$rows = $pageRepository->getMenu(array(2, 3));
			\end{lstlisting}

	\end{itemize}

\end{frame}

% ------------------------------------------------------------------------------
