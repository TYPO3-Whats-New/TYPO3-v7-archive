% ------------------------------------------------------------------------------
% TYPO3 CMS 7.1 - What's New - Chapter "TypoScript" (English Version)
%
% @author	Michael Schams <schams.net>
% @license	Creative Commons BY-NC-SA 3.0
% @link		http://typo3.org/download/release-notes/whats-new/
% @language	English
% ------------------------------------------------------------------------------
% LTXE-CHAPTER-UID:		d92ab5b9-9ffa4925-30a2e34f-0d6939e5
% LTXE-CHAPTER-NAME:	TypoScript
% ------------------------------------------------------------------------------

\section{TSconfig \& TypoScript}
\begin{frame}[fragile]
	\frametitle{TSconfig \& TypoScript}

	\begin{center}\huge{Chapter 2:}\end{center}
	\begin{center}\huge{\color{typo3darkgrey}\textbf{TSconfig \& TypoScript}}\end{center}

\end{frame}

% ------------------------------------------------------------------------------
% LTXE-SLIDE-START
% LTXE-SLIDE-UID:		5353aa5c-742568e4-29bd0b86-6224f97a
% LTXE-SLIDE-ORIGIN:	a55771e2-74f3a0fc-5be51f93-8b40a5d8 English
% LTXE-SLIDE-TITLE:		StdWrap for page.headTag
% LTXE-SLIDE-REFERENCE:	Feature-22086-StdWrapForHeadTag.rst
% ------------------------------------------------------------------------------

\begin{frame}[fragile]
	\frametitle{TSconfig \& TypoScript}
	\framesubtitle{StdWrap for page.headTag}

	\begin{itemize}
		\item TypoScript setting \texttt{page.headTag} has \texttt{stdWrap} functionality now

			\begin{lstlisting}
				page = PAGE
				page.headTag = <head>
				page.headTag.override = <head class="special">
				page.headTag.override.if {
		  		  isInList.field = uid
		  		  value = 24
				}
			\end{lstlisting}	

	\end{itemize}

\end{frame}

% ------------------------------------------------------------------------------
% LTXE-SLIDE-START
% LTXE-SLIDE-UID:		44ad7ab3-fcd069a5-a87b359d-5a500bf7
% LTXE-SLIDE-ORIGIN:	618eb9ea-56d5a4eb-d5d69d66-cc62f953 English
% LTXE-SLIDE-TITLE:		Include JavaScript files asynchronously
% LTXE-SLIDE-REFERENCE:	Feature-28382-AddAsyncPropertyToJavaScriptFiles.rst
% ------------------------------------------------------------------------------

\begin{frame}[fragile]
	\frametitle{TSconfig \& TypoScript}
	\framesubtitle{Include JavaScript files asynchronously}

	\begin{itemize}
		\item JavaScript files can be loaded asynchronously

			\begin{lstlisting}
				page {
				  includeJS {
				    jsFile = /path/to/file.js
				    jsFile.async = 1
				  }
				}
			\end{lstlisting}

		\item This affects:

			\begin{itemize}
				\item \texttt{includeJSlibs} / \texttt{includeJSLibs}
				\item \texttt{includeJSFooterlibs}
				\item \texttt{includeJS}
				\item \texttt{includeJSFooter}
			\end{itemize}

	\end{itemize}

\end{frame}

% ------------------------------------------------------------------------------
% LTXE-SLIDE-START
% LTXE-SLIDE-UID:		2d50e1bb-cd94a33c-eee51b19-2db79098
% LTXE-SLIDE-ORIGIN:	8a6ca08e-cf9ba481-18fd9411-30dd0a0f English
% LTXE-SLIDE-TITLE:		HMENU item selection via additionalWhere
% LTXE-SLIDE-REFERENCE:	Feature-46624-AdditionalWhereForMenu.rst
% ------------------------------------------------------------------------------

\begin{frame}[fragile]
	\frametitle{TSconfig \& TypoScript}
	\framesubtitle{HMENU item selection via \texttt{additionalWhere}}

	\begin{itemize}

		\item TypoScript cObject \texttt{HMENU} features a new property \texttt{additionalWhere}
		\item This allows for a more specific database query (e.g. filtering)

		\item Example:

			\begin{lstlisting}
				lib.authormenu = HMENU
				lib.authormenu.1 = TMENU
				lib.authormenu.1.additionalWhere = AND author!=""
			\end{lstlisting}

	\end{itemize}

\end{frame}

% ------------------------------------------------------------------------------
% LTXE-SLIDE-START
% LTXE-SLIDE-UID:		df424321-3fccac49-bf93538a-236903f5
% LTXE-SLIDE-ORIGIN:	e07d1984-117d1212-a1f7fab8-8ea21ebc English
% LTXE-SLIDE-TITLE:		Additional properties for HMENU browse menus
% LTXE-SLIDE-REFERENCE:	Feature-57178-SpecialHmenuExcludeSpecialParameters.rst
% ------------------------------------------------------------------------------

\begin{frame}[fragile]
	\frametitle{TSconfig \& TypoScript}
	\framesubtitle{Additional properties for \texttt{HMENU} browse menus}

	\begin{itemize}
		\item Two new properties for cObject \texttt{HMENU} (option "\texttt{special=browse"})\newline
			to select menu items more fine-grained:
		
			\begin{itemize}
				\item \texttt{excludeNoSearchPages}
				\item \texttt{includeNotInMenu}
			\end{itemize}

		\item Example:

			\begin{lstlisting}
				lib.browsemenu = HMENU
				lib.browsemenu.special = browse
				lib.browsemenu.special.excludeNoSearchPages = 1
				lib.browsemenu.includeNotInMenu = 1
			\end{lstlisting}

	\end{itemize}

\end{frame}

% ------------------------------------------------------------------------------
% LTXE-SLIDE-START
% LTXE-SLIDE-UID:		286ebcff-5de8eff3-b39eb145-cfdee637
% LTXE-SLIDE-ORIGIN:	a286a87a-161365ed-5560e66f-e9f1a2ee English
% LTXE-SLIDE-TITLE:		Multiple HTTP headers
% LTXE-SLIDE-REFERENCE:	Feature-56236-Multiple-HTTP-Headers-In-Frontend.rst
% ------------------------------------------------------------------------------

\begin{frame}[fragile]
	\frametitle{TSconfig \& TypoScript}
	\framesubtitle{Multiple HTTP headers}

	\begin{itemize}

		\item HTTP headers can be set as an array (\small\texttt{config.additionalHeaders}\normalsize)
		\item This allows for the configuration of multiple headers at the same time

			\begin{lstlisting}
				config.additionalHeaders {
				  10 {
				    # header string
				    header = WWW-Authenticate: Negotiate

				    # (optional) replace previous headers with the same name (default: 1)
				    replace = 0

				    # (optional) force HTTP response code
				    httpResponseCode = 401
				  }
				  # set second additional HTTP header
				  20.header = Cache-control: Private
				}
			\end{lstlisting}

	\end{itemize}

\end{frame}

% ------------------------------------------------------------------------------
% LTXE-SLIDE-START
% LTXE-SLIDE-UID:		59f56d78-befc32d2-a079e4f0-6e18d92a
% LTXE-SLIDE-ORIGIN:	a296b9b8-9403ef17-30f80f28-2a8a94ce English
% LTXE-SLIDE-TITLE:		Option "auto" added for config.absRefPrefix
% LTXE-SLIDE-REFERENCE:	Feature-58366-AutomaticAbsRefPrefix.rst
% ------------------------------------------------------------------------------

\begin{frame}[fragile]
	\frametitle{TSconfig \& TypoScript}
	\framesubtitle{Option "auto" added for config.absRefPrefix}

	\begin{itemize}
		\item TypoScript setting \texttt{config.absRefPrefix} can be used to allow URL
			rewriting. As an alternative to \texttt{config.baseURL} (to configure a specific
			domain), \texttt{absRefPrefix} can detect the site root automatically:

			\begin{lstlisting}
				config.absRefPrefix = auto

				# ...instead of:
				[ApplicationContext = Production]
				config.absRefPrefix = /

				[ApplicationContext = Testing]
				config.absRefPrefix = /my_site_root/
			\end{lstlisting}

		\smaller
			Note: The new option is also safe for multi-domain environments to avoid
			duplicate caching mechanism.
		\normalsize

	\end{itemize}

\end{frame}

% ------------------------------------------------------------------------------
% LTXE-SLIDE-START
% LTXE-SLIDE-UID:		42a1d949-9d52db67-7d7ee155-85fdc101
% LTXE-SLIDE-ORIGIN:	8210d253-d630446f-ba747ba6-5568843d English
% LTXE-SLIDE-TITLE:		Two-letter ISO code for sys_language (1)
% LTXE-SLIDE-REFERENCE:	Feature-61542-AddIsoLanguageKeys.rst
% ------------------------------------------------------------------------------

\begin{frame}[fragile]
	\frametitle{TSconfig \& TypoScript}
	\framesubtitle{Two-letter ISO code for sys\_language (1)}

	\begin{itemize}
		\item The handling of languages is done by records stored in DB table
			\texttt{sys\_language}, which are usually referenced via \texttt{sys\_language\_uid}
		\item In TYPO3 CMS 7.1, the ISO 639-1 two-letter has been introduced:

			\begin{itemize}
				\item New DB field: \texttt{sys\_language.language\_isocode}
				\item New TypoScript option: \texttt{sys\_language\_isocode}
			\end{itemize}


		\vspace{1cm}

		\small
			Note: \textbf{ISO 639} is a set of standards by the International Organization
			for Standardization. List of ISO 639-1 codes is available at:\newline
			\url{http://en.wikipedia.org/wiki/List_of_ISO_639-1_codes}
		\normalsize

	\end{itemize}

\end{frame}

% ------------------------------------------------------------------------------
% LTXE-SLIDE-START
% LTXE-SLIDE-UID:		07124186-d61994d5-a5b9fae2-a8675bbe
% LTXE-SLIDE-ORIGIN:	d630446f-ba747ba6-5568843d-8210d253 English
% LTXE-SLIDE-TITLE:		Two-letter ISO code for sys_language (2)
% LTXE-SLIDE-REFERENCE:	Feature-61542-AddIsoLanguageKeys.rst
% ------------------------------------------------------------------------------

\begin{frame}[fragile]
	\frametitle{TSconfig \& TypoScript}
	\framesubtitle{Two-letter ISO code for sys\_language (2)}

	\begin{itemize}
		\item Example:

			\begin{lstlisting}
				# Danish by default
				config.sys_language_uid = 0
				config.sys_language_isocode_default = da

				[globalVar = GP:L = 1]
				  # ISO code stored in table sys_language (uid 1)
				  config.sys_language_uid = 1
				  # overwrite ISO code as required
				  config.sys_language_isocode = fr
				[GLOBAL]

				page.10 = TEXT
				page.10.data = TSFE:sys_language_isocode
				page.10.wrap = <div class="main" data-language="|">
			\end{lstlisting}

	\end{itemize}

\end{frame}

% ------------------------------------------------------------------------------
% LTXE-SLIDE-START
% LTXE-SLIDE-UID:		842a08f2-489f456e-fe6da63d-d79010a6
% LTXE-SLIDE-ORIGIN:	df2f7fec-fb8bc3b9-446fb779-c409ef27 English
% LTXE-SLIDE-TITLE:		Custom TypoScript Conditions in Backend
% LTXE-SLIDE-REFERENCE:	Feature-63600-CustomTypoScriptConditionsInBackend.rst
% ------------------------------------------------------------------------------

\begin{frame}[fragile]
	\frametitle{TSconfig \& TypoScript}
	\framesubtitle{Custom TypoScript Conditions in Backend}

	% decrease font size for code listing
	\lstset{basicstyle=\tiny\ttfamily}

	\begin{itemize}
	
		\item Support of custom conditions for the \textbf{frontend} has been introduced in TYPO3 CMS 7.0 already
		\item Since TYPO3 CMS 7.1, it is also possible to use custom conditions in the \textbf{backend}
		\item The condition must be derived from \texttt{AbstractCondition} and implement method \texttt{matchCondition()}
		\item Example usage in TypoScript: 

			\begin{lstlisting}
				[BigCompanyName\TypoScriptLovePackage\MyCustomTypoScriptCondition]

				[BigCompanyName\TypoScriptLovePackage\MyCustomTypoScriptCondition = 7]

				[BigCompanyName\TypoScriptLovePackage\MyCustomTypoScriptCondition = 7, != 6]

				[BigCompanyName\TypoScriptLovePackage\MyCustomTypoScriptCondition = {$mysite.myconstant}]
			\end{lstlisting}

	\end{itemize}

\end{frame}

% ------------------------------------------------------------------------------
% LTXE-SLIDE-START
% LTXE-SLIDE-UID:		77ec115b-9855c4ec-03895c8f-b32f65c5
% LTXE-SLIDE-ORIGIN:	b9687dbb-c0502234-7ca16b06-1879fe80 English
% LTXE-SLIDE-TITLE:		FormEngine: customize icons via PageTSconfig
% LTXE-SLIDE-REFERENCE:	Feature-35891-AddTCAItemsWithIconsViaPageTSConfig.rst
% ------------------------------------------------------------------------------

\begin{frame}[fragile]
	\frametitle{TSconfig \& TypoScript}
	\framesubtitle{Customize icons via PageTSconfig}

	\begin{itemize}
		\item Value/label pairs of select fields can be configured by PageTSconfig option \texttt{addItems} already
		\item It is also possible to influence the \textbf{icon} of these fields now

			\begin{itemize}
				\item Option 1: by using \texttt{addItems} and sub-property \texttt{.icon}
				\item Option 2: by using \texttt{altIcons} (all items in general)
			\end{itemize}

		\item Example:

			\begin{lstlisting}
				TCEFORM.pages.doktype.addItems {
				  10 = My Label
				  10.icon = EXT:t3skin/icons/gfx/i/pages.gif
				}
				TCEFORM.pages.doktype.altIcons {
				  10 = EXT:myext/icon.gif
				}
			\end{lstlisting}

	\end{itemize}

\end{frame}

% ------------------------------------------------------------------------------
% LTXE-SLIDE-START
% LTXE-SLIDE-UID:		a0a54957-de893841-f8ae18e9-59175329
% LTXE-SLIDE-ORIGIN:	b8794774-4912764b-cd1cc91d-de4396fe English
% LTXE-SLIDE-TITLE:		Extend element browser with mount points
% LTXE-SLIDE-REFERENCE:	Feature-50780-AppendElementBrowserMountPoints.rst
% ------------------------------------------------------------------------------

\begin{frame}[fragile]
	\frametitle{TSconfig \& TypoScript}
	\framesubtitle{Extend element browser with mount points}

	\begin{itemize}
		\item New UserTSconfig option \texttt{.append} allows administrators to \textbf{add}
			mount points, rather than replacing the configured database mount points of the user 

		\item Example:

			\begin{lstlisting}
				options.pageTree.altElementBrowserMountPoints = 20,31
				options.pageTree.altElementBrowserMountPoints.append = 1
			\end{lstlisting}

	\end{itemize}

\end{frame}

% ------------------------------------------------------------------------------
% LTXE-SLIDE-START
% LTXE-SLIDE-UID:		fd2353f5-33f9e6d5-f1dad357-2cc4b5d4
% LTXE-SLIDE-ORIGIN:	04b7f7c0-af900c25-5f16f178-8a94fcca English
% LTXE-SLIDE-TITLE:		Label override of checkboxes and radio buttons by TSconfig
% LTXE-SLIDE-REFERENCE:	Feature-58033-AltLabelsForFormEngineCheckboxAndRadioButtons.rst
% ------------------------------------------------------------------------------

\begin{frame}[fragile]
	\frametitle{TSconfig \& TypoScript}
	\framesubtitle{Label override of checkboxes and radio buttons}

	% decrease font size for code listing
	\lstset{basicstyle=\tiny\ttfamily}

	\begin{itemize}

		\item Labels of radio buttons and checkboxes can be overwritten now
		\item Example:

			\begin{lstlisting}
				// field with a single checkbox (use ".default")
				TCEFORM.pages.hidden.altLabels.default = new label
				TCEFORM.pages.hidden.altLabels.default = LLL:path/to/languagefile.xlf:individualLabel

				// field with multiple checkboxes (0, 1, 2, 3...)
				TCEFORM.pages.l18n_cfg.altLabels.0 = new label of first checkbox
				TCEFORM.pages.l18n_cfg.altLabels.1 = new label of second checkbox
				TCEFORM.pages.l18n_cfg.altLabels.2 = new label of third checkbox
				...
			\end{lstlisting}

	\end{itemize}

\end{frame}

% ------------------------------------------------------------------------------
% LTXE-SLIDE-START
% LTXE-SLIDE-UID:		29173aa8-9d7e989b-7e60f71d-0473d9d8
% LTXE-SLIDE-ORIGIN:	6cdd5010-f7d7d7bd-0a3c74ea-05424fb0 English
% LTXE-SLIDE-TITLE:		Miscellaneous (1)
% LTXE-SLIDE-REFERENCE: Feature-xxxxx ... UserTSconfig-width-and-height-of-element-browser
% LTXE-SLIDE-REFERENCE: Feature-59646-AddRteConfigurationPropertyButtonsLinkTypePropertiesTargetDefault.rst
% ------------------------------------------------------------------------------

\begin{frame}[fragile]
	\frametitle{TSconfig \& TypoScript}
	\framesubtitle{Miscellaneous (1)}

	\begin{itemize}
		\item Width and height of the Element Browser can be configured using UserTSconfig:

			\begin{lstlisting}
				options.popupWindowSize = 400x900
				options.RTE.popupWindowSize = 200x200
			\end{lstlisting}


		\item PageTSconfig: new RTE configuration property can be used to configure a default target for links of a given type:

			\begin{lstlisting}
				buttons.link.[type].properties.target.default
			\end{lstlisting}

			Where \texttt{[type]} can be \texttt{page}, \texttt{file}, \texttt{url}, \texttt{mail} or \texttt{spec}\newline
			(extensions may provide further types)

	\end{itemize}

\end{frame}

% ------------------------------------------------------------------------------
% LTXE-SLIDE-START
% LTXE-SLIDE-UID:		60bf8fab-7f90d6aa-5d621d10-7a8e7ebe
% LTXE-SLIDE-ORIGIN:	80595308-128c5961-8ae4cf89-8611ce66 English
% LTXE-SLIDE-TITLE:		Miscellaneous (2)
% LTXE-SLIDE-REFERENCE:	Feature-16794-MakeSectionLinkingForIndexedSearchResultsConfigurable.rst
% LTXE-SLIDE-REFERENCE: Feature-20767-getDataByNestedKey.rst
% LTXE-SLIDE-REFERENCE: Feature-33491-StdWrapForTitleTag.rst
% ------------------------------------------------------------------------------

\begin{frame}[fragile]
	\frametitle{TSconfig \& TypoScript}
	\framesubtitle{Miscellaneous (2)}

	\begin{itemize}

		\item Section headlines of indexed search results are links by default.
			It is now possible to disable these links and display sections as simple texts

			\begin{lstlisting}
				plugin.tx_indexedsearch.linkSectionTitles = 0
			\end{lstlisting}

		\item \texttt{getData} can access \texttt{field} data now (not only arrays
			such as \texttt{GPVar} and \texttt{TSFE}):
		
			\begin{lstlisting}
				10 = TEXT
				10.data = field:fieldname|level1|level2
			\end{lstlisting}

		\item TypoScript setting \texttt{config.pageTitle} has \texttt{stdWrap} functionality now

			\begin{lstlisting}
				# make value of <title> upper case
				page = PAGE
				page.config.pageTitle.case = upper
			\end{lstlisting}

	\end{itemize}

\end{frame}

% ------------------------------------------------------------------------------
