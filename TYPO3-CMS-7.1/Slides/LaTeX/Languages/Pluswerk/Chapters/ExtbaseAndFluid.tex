% ------------------------------------------------------------------------------
% TYPO3 CMS 7.1 - What's New - Chapter "Extbase & Fluid" (Pluswerk Version)
%
% @author	Patrick Lobacher <patrick@lobacher.de>
% @license	Creative Commons BY-NC-SA 3.0
% @link		http://typo3.org/download/release-notes/whats-new/
% @language	German
% ------------------------------------------------------------------------------
% LTXE-CHAPTER-UID:		7b0b9a18-6df803cd-8b5e7778-87e587d9
% LTXE-CHAPTER-NAME:	Chapter: Extbase & Fluid
% ------------------------------------------------------------------------------

\section{Extbase \& Fluid}
\begin{frame}[fragile]
	\frametitle{Extbase \& Fluid}

	\begin{center}\huge{Kapitel 4:}\end{center}
	\begin{center}\huge{\color{typo3darkgrey}\textbf{Extbase \& Fluid}}\end{center}

\end{frame}

% ------------------------------------------------------------------------------
% LTXE-SLIDE-START
% LTXE-SLIDE-UID:		2171a8ad-23e47eea-3074ccf5-14480fc1
% LTXE-SLIDE-TITLE:		PaginateViewHelper mit Nicht-Query-Daten
% LTXE-SLIDE-REFERENCE:	Feature-34944-PaginateHandleNonQueryResultObjects.rst
% ------------------------------------------------------------------------------

\begin{frame}[fragile]
	\frametitle{Extbase \& Fluid}
	\framesubtitle{PaginateViewHelper mit Nicht-Query-Daten}

	\lstset{
		basicstyle=\tiny\ttfamily
	}

	\begin{itemize}

		\item Der Paginate-ViewHelper unterstützt nun folgende Input-Werte:
		\begin{itemize}
			\item QueryResultInterface
			\item ObjectStorage
			\item ArrayAccess
			\item array
		\end{itemize}

		\begin{lstlisting}
			<f:widget.paginate objects="{blogs}" as="paginatedBlogs">
				<f:for each="{paginatedBlogs}" as="blog">
					<h4>{blog.title}</h4>
				</f:for>
			</f:widget.paginate>
			\end{lstlisting}

	\end{itemize}

\end{frame}

% ------------------------------------------------------------------------------
% LTXE-SLIDE-START
% LTXE-SLIDE-UID:		6fe9e0e9-09053753-a70dca26-8687a067
% LTXE-SLIDE-TITLE:		ContainerViewHelper kann RequireJS Module laden
% LTXE-SLIDE-REFERENCE:	Feature-63913-AllowRequireJsModulesForContainerViewHelper.rst
% ------------------------------------------------------------------------------

\begin{frame}[fragile]
	\frametitle{Extbase \& Fluid}
	\framesubtitle{ContainerViewHelper kann RequireJS Module laden}

	\lstset{
		basicstyle=\tiny\ttfamily
	}

	\begin{itemize}
		\item Der ContainerViewHelper kann RequireJS Module via \texttt{includeRequireJsModules} Attribut laden

		\begin{lstlisting}
			<f:be.container pageTitle="Extension Module" loadJQuery="true"
				includeRequireJsModules="{
					0:'TYPO3/CMS/Extension/Module',
					1:'TYPO3/CMS/Extension/Module2',
					2:'TYPO3/CMS/Extension/Module3',
					3:'TYPO3/CMS/Extension/Module4'
			}">
		\end{lstlisting}

	\end{itemize}

\end{frame}

% ------------------------------------------------------------------------------
% LTXE-SLIDE-START
% LTXE-SLIDE-UID:		44965a25-badc7693-b249474b-26e425f0
% LTXE-SLIDE-TITLE:		PaginateViewHelper mit Nicht-Query-Daten
% LTXE-SLIDE-REFERENCE:	Feature-56529-SupportHasInArrayObject.rst
% ------------------------------------------------------------------------------

\begin{frame}[fragile]
	\frametitle{Extbase \& Fluid}
	\framesubtitle{Methode has() im Objekt-Zugriff}

	\lstset{
		basicstyle=\tiny\ttfamily
	}

	\begin{itemize}

		\item Es gibt bereits die Objekt-Zugriffe:
		\begin{itemize}
			\item isProperty()
			\item getProperty()
		\end{itemize}
		für die Verwendung in Fluid von \texttt{object.property} bzw. \texttt{object.isProperty}.

		\item Nun gibt es neu \texttt{hasProperty()} - hier wird die Methode \texttt{\$object->hasProperty()} aufgerufen, wenn man in Fluid \texttt{object.hasProperty} verwendet.
	\end{itemize}

\end{frame}

% ------------------------------------------------------------------------------
% LTXE-SLIDE-START
% LTXE-SLIDE-UID:		4ff520f0-65ac1ddd-35fd5be5-1a22f4d3
% LTXE-SLIDE-TITLE:		Diverse
% LTXE-SLIDE-REFERENCE:	Feature-47666-AttributeMulitpleForFormUploadViewhelper.rst
% ------------------------------------------------------------------------------

\begin{frame}[fragile]
	\frametitle{Extbase \& Fluid}
	\framesubtitle{Diverse}

	\lstset{
		basicstyle=\tiny\ttfamily
	}

	\begin{itemize}
		\item Der FormUpload-ViewHelper kann nun mehrere Dateien über die Eigenschaft \texttt{multiple} aufnehmen. Für das Property-Mapping muss ein eigener TypeConverter erstellt werden
		\begin{lstlisting}
			<f:form.upload property="files" multiple="multiple" />
		\end{lstlisting}

	\end{itemize}

\end{frame}



% ------------------------------------------------------------------------------
