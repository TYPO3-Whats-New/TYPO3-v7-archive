% ------------------------------------------------------------------------------
% TYPO3 CMS 7.1 - What's New - Chapter "Deprecated Functions" (Pluswerk Version)
%
% @author	Patrick Lobacher <patrick@lobacher.de>
% @license	Creative Commons BY-NC-SA 3.0
% @link		http://typo3.org/download/release-notes/whats-new/
% @language	German
% ------------------------------------------------------------------------------
% LTXE-CHAPTER-UID:		08661c9e-3f5aee5c-0b8f8af6-9681f2f1
% LTXE-CHAPTER-NAME:	Deprecated Functions
% ------------------------------------------------------------------------------

\section{Veraltete/Entfernte Funktionen}
\begin{frame}[fragile]
	\frametitle{Veraltete/Entfernte Funktionen}

	\begin{center}\huge{Kapitel 5:}\end{center}
	\begin{center}\huge{\color{typo3darkgrey}\textbf{Veraltete und entfernte Funktionen}}\end{center}

\end{frame}

% ------------------------------------------------------------------------------
% LTXE-SLIDE-START
% LTXE-SLIDE-UID:		c1970d4b-83f3d12a-4b99f535-5f763c24
% LTXE-SLIDE-TITLE:		$TYPO3_CONF_VARS[SYS][compat_version]
% LTXE-SLIDE-REFERENCE:	Breaking-24900-CompatVersion-Setting-Removed.rst
% ------------------------------------------------------------------------------

\begin{frame}[fragile]
	\frametitle{Veraltete/Entfernte Funktionen}
	\framesubtitle{\$TYPO3\_CONF\_VARS[SYS][compat\_version] entfernt}

	\begin{itemize}

		\item The option \texttt{\$TYPO3\_CONF\_VARS[SYS][compat\_version]} (gesetzt beim Update im Install Tool wizard) wurde entfernt

		\item Alle Prüfung gegen \texttt{GeneralUtility::compat\_version} werden nun gegen die Konstante \texttt{TYPO3\_branch} gemacht

		\item Inbesondere hat nun auch die zugehörige Condition in TypoScript keine Wirkung mehr

	\end{itemize}

\end{frame}

% ------------------------------------------------------------------------------
% LTXE-SLIDE-START
% LTXE-SLIDE-UID:		4991d739-1ccd5bde-f424f2fc-47d59881
% LTXE-SLIDE-ORIGIN:	9eaf5bf4-b7582d54-4938c777-c1b6568b English
% LTXE-SLIDE-TITLE:		TypoScript inline styles from blockquote tag removed
% LTXE-SLIDE-REFERENCE: Breaking-44879-CSSStyledContentTypoScriptBlockQuoteInlineStylesRemoved.rst
% ------------------------------------------------------------------------------

\begin{frame}[fragile]
	\frametitle{Veraltete/Entfernte Funktionen}
	\framesubtitle{Inline styles of \texttt{<blockquote>} tag}

	% decrease font size for code listing
	\lstset{basicstyle=\tiny\ttfamily}

	\begin{itemize}

		\item CSS Styled Content rendert \texttt{<blockquote>} über die TypoScript Option \texttt{lib.parseFunc\_RTE}
		\item Diese Zeilen wurden ersatzlos entfernt:

			\begin{lstlisting}
				lib.parseFunc_RTE.externalBlocks.blockquote.callRecursive.tagStdWrap.HTMLparser = 1
				lib.parseFunc_RTE.externalBlocks.blockquote.callRecursive.tagStdWrap.HTMLparser.tags.blockquote.overrideAttribs = style="margin-bottom:0;margin-top:0;"
			\end{lstlisting}

		\item Das bedeutet, die Inline-Styles "\texttt{margin-bottom:0;margin-top:0;}"\newline
			werden dem \texttt{<blockquote>}-Tag nicht mehr hinzugefügt

			\vspace{0.2cm}

			\begingroup
				\color{red}
					Hinweis: nach einem Update auf TYPO3 CMS 7.1 könnte sich das
					Styling von \texttt{<blockquote>} geändert haben
			\endgroup

	\end{itemize}

\end{frame}

% ------------------------------------------------------------------------------
% LTXE-SLIDE-START
% LTXE-SLIDE-UID:		50cbbb42-080ed12e-a65821cd-99b447a1
% LTXE-SLIDE-TITLE:		Feld disable_autocreate in Workspaces entfernt
% LTXE-SLIDE-REFERENCE:	Breaking-62415-DisableAutoCreateRemoved.rst
% ------------------------------------------------------------------------------

\begin{frame}[fragile]
	\frametitle{Veraltete/Entfernte Funktionen}
	\framesubtitle{Feld \texttt{disable\_autocreate} in Workspaces entfernt}

	\begin{itemize}
		\item In Workspaces wurde das Feld \texttt{disable\_autocreate} entfernt

		\item Sofern sich eine 3rd Party Extension darauf verlassen hat, wird nun ein MySQL-Fehler ausgegeben
	\end{itemize}

\end{frame}

% ------------------------------------------------------------------------------
% LTXE-SLIDE-START
% LTXE-SLIDE-UID:		3a833032-a97d7460-43829b9a-7121f3e7
% LTXE-SLIDE-TITLE:		include_once Funktionen in ModulFunktionen enfernt
% LTXE-SLIDE-REFERENCE:	Breaking-63464-IncludeOnceArraysRemoved.rst
% ------------------------------------------------------------------------------

\begin{frame}[fragile]
	\frametitle{Veraltete/Entfernte Funktionen}
	\framesubtitle{\texttt{include\_once} Funktionen in Modul-Funktionen enfernt}

	\begin{itemize}

		\item Die Funktionalität, um PHP-Dateien mittels \texttt{include\_once} innerhalb von Modul-Funktionen (wie z.B. dem Info-Modul) zu inkludieren, wurde entfernt

		\item Insbesondere wurde die Funktionalität für folgende Module entfernt: "Web => Page", "Web => Page - New Content Element Wizard", "Web => Functions", "Web => Info", "Web => Template", "Web => Recycler", "User => Task Center", "System => Scheduler"

	\end{itemize}

\end{frame}

% ------------------------------------------------------------------------------
% LTXE-SLIDE-START
% LTXE-SLIDE-UID:		fc170771-6dcd6af4-2f0be8e4-47f86f44
% LTXE-SLIDE-TITLE:		TS Setting config.meaningfulTempFilePrefix entfernt
% LTXE-SLIDE-REFERENCE:	Breaking-62886-RemoveMeaningfulTempFilePrefix.rst
% ------------------------------------------------------------------------------

\begin{frame}[fragile]
	\frametitle{Veraltete/Entfernte Funktionen}
	\framesubtitle{TS Setting \texttt{config.meaningfulTempFilePrefix} entfernt}

	\begin{itemize}
		\item Früher war es möglich, per TypoScript Teile des originalen Dateinamen zu den Dateinamen zuzufügen, die vom GIFBUILDER generiert wurden. Denn diese bestanden nur aus einem Hash.
		\item Dafür wurde die TypoScript Option \texttt{config.meaningfulTempFilePrefix} verwendet
		\item Dieses Option wurde entfernt, da ab sofort alle Dateien im Verzeichnis \textt{typo3temp/GB/} immer den komplett Dateinamen als ersten Bestandteil enthalten
	\end{itemize}

\end{frame}


% ------------------------------------------------------------------------------
% LTXE-SLIDE-START
% LTXE-SLIDE-UID:		e159b211-22d42f6b-6b1cb333-16d098b6
% LTXE-SLIDE-TITLE:		Entfernte Dateien
% LTXE-SLIDE-REFERENCE:	Breaking-63296-Removed-Files.rst

% ------------------------------------------------------------------------------

\begin{frame}[fragile]
	\frametitle{Veraltete/Entfernte Funktionen}
	\framesubtitle{Entfernte Dateien}

	Die folgenden \textbf{Dateien} wurden entfernt, da sie nicht mehr benötigt werden:

	\begin{itemize}
		\item \texttt{typo3/file\_edit.php}
		\item \texttt{typo3/file\_newfolder.php}
		\item \texttt{typo3/file\_rename.php}
		\item \texttt{typo3/file\_upload.php}
		\item \texttt{typo3/show\_rechis.php}
		\item \texttt{typo3/listframe\_loader.php}
	\end{itemize}

	Die Funktionalität, die in den Dateien enthalten war, wurde in Backend-Module integriert, z.B. \texttt{typo3/file\_edit.php} in \texttt{BackendUtility::getModuleUrl('file\_edit');}
\end{frame}

% ------------------------------------------------------------------------------
% LTXE-SLIDE-START
% LTXE-SLIDE-UID:		ad297b11-8bd78893-b256ff51-522c5d65
% LTXE-SLIDE-TITLE:		Entfernte Funktionalitäten
% LTXE-SLIDE-REFERENCE:	Breaking-62925-RemoveExtJsDateTimePicker.rst
% ------------------------------------------------------------------------------

\begin{frame}[fragile]
	\frametitle{Veraltete/Entfernte Funktionen}
	\framesubtitle{Entfernte Funktionalitäten}

	Die folgenden \textbf{Funktionalitäten} wurden entfernt:

	\begin{itemize}
		\item Die ExtJS Komponente \texttt{Ext.ux.DateTimePicker} wurde entfernt und gegen eine Bootstrap Alternative ausgetauscht (siehe \texttt{EXT:belog}, \texttt{EXT:scheduler})	
	\end{itemize}

\end{frame}

% ------------------------------------------------------------------------------
% LTXE-SLIDE-START
% LTXE-SLIDE-UID:		01d649f8-8038f67c-5e3af008-e8f1dffb
% LTXE-SLIDE-TITLE:		Entfernte Funktionalitäten
% LTXE-SLIDE-REFERENCE:	Breaking-64226-OptionAccessListRenderModeRemoved.rst
% ------------------------------------------------------------------------------

\begin{frame}[fragile]
	\frametitle{Veraltete/Entfernte Funktionen}
	\framesubtitle{Access List Render Mode Änderungen}

	Die Variable \texttt{\$GLOBALS[TYPO3\_CONF\_VARS][BE][accessListRenderMode} wurde entfernt. Alle korrespondierenden Felder im TCA von \texttt{be\_users} und \texttt{be\_groups} wurde auf den Default-Wert \texttt{default} gesetzt. In \texttt{typo3conf/extTables.php} kann dies geändert werden:

	\begin{lstlisting}
	$GLOBALS['TCA']['be_users']['columns']['file_permissions']['config']['renderMode'] = 'singlebox';
	$GLOBALS['TCA']['be_users']['columns']['userMods']['config']['renderMode'] = 'singlebox';
	$GLOBALS['TCA']['be_groups']['columns']['file_permissions']['config']['renderMode'] = 'singlebox';
	$GLOBALS['TCA']['be_groups']['columns']['pagetypes_select']['config']['renderMode'] = 'singlebox';
	$GLOBALS['TCA']['be_groups']['columns']['tables_select']['config']['renderMode'] = 'singlebox';
	$GLOBALS['TCA']['be_groups']['columns']['tables_modify']['config']['renderMode'] = 'singlebox';
	$GLOBALS['TCA']['be_groups']['columns']['non_exclude_fields']['config']['renderMode'] = 'singlebox';
	$GLOBALS['TCA']['be_groups']['columns']['userMods']['config']['renderMode'] = 'singlebox';
	\end{lstlisting}

\end{frame}

% ------------------------------------------------------------------------------
% LTXE-SLIDE-START
% LTXE-SLIDE-UID:		a9730cca-e3d11df7-59152b12-7750aa6a
% LTXE-SLIDE-TITLE:		Entfernte Funktionalitäten
% LTXE-SLIDE-REFERENCE:	Breaking-64668-MailformMovedToLegacyExtension.rst
% ------------------------------------------------------------------------------

\begin{frame}[fragile]
	\frametitle{Geänderte/Entfernte Funktionen}
	\framesubtitle{Mailform wurde entfernt}

	Die Mailform-Funktionalität, die das \texttt{FORM} cObject enthält, wurde in die Legacy-Extension \textt{EXT:compatibility6} verschoben und stehen damit in Kürze nicht mehr zur Verfügung. Insbesondere sind folgende Option depracated:

	\begin{lstlisting}
	$TYPO3_CONF_VARS][FE][secureFormmail]
	$TYPO3_CONF_VARS][FE][strictFormmail]
	$TYPO3_CONF_VARS][FE][formmailMaxAttachmentSize]
	\end{lstlisting}

	Folgende Methoden im TypoScriptFrontendController wurden entfernt:

	\begin{lstlisting}
	protected checkDataSubmission()
	protected sendFormmail()
	public extractRecipientCopy()
	public codeString()
	protected roundTripCryptString()
	\end{lstlisting}

\end{frame}

% ------------------------------------------------------------------------------
% LTXE-SLIDE-START
% LTXE-SLIDE-UID:		14f23772-489cfe59-f29156ea-9e68718d
% LTXE-SLIDE-TITLE:		Geänderte Funktionalitäten
% LTXE-SLIDE-REFERENCE:	Breaking-61510-IndexedSearch.rst
% LTXE-SLIDE-REFERENCE: Breaking-63310-Wizard-Modules-Moved.rst
% LTXE-SLIDE-REFERENCE: Breaking-63431-BackendToolbarRefactored.rst
% ------------------------------------------------------------------------------

\begin{frame}[fragile]
	\frametitle{Geänderte/Entfernte Funktionen}
	\framesubtitle{Änderungen 1}

	Die folgenden \textbf{Funktionalitäten} wurden geändert:

	\begin{itemize}

		\item Die Extension \texttt{indexed\_search} wird automatisch aktiviert, sobald sie installiert wurde. Damit werden auch automatisch die zugehörigen TypoScript-Optionen aktiviert: \texttt{config.index\_enable = 1} und \texttt{config.index\_externals = 1} 
		
		\item Der TSconfig Schlüssel \texttt{web\_func.menu.wiz} ändert sich zu \texttt{web\_func.menu.functions}

		\item Backend-Module, die sich in die Toolbar oben rechts einklinken, müssen das neue Interface \lstinline{\TYPO3\CMS\Backend\Toolbar\ToolbarItemInterface} implementieren und in \texttt{\$GLOBALS['TYPO3\_CONF\_VARS']['BE']['toolbarItems']} registriert werden
	\end{itemize}

\end{frame}

% ------------------------------------------------------------------------------
% LTXE-SLIDE-START
% LTXE-SLIDE-UID:		1f4122d8-bb8787ba-5cb204ee-ddbd58a9
% LTXE-SLIDE-TITLE:		Geänderte Funktionalitäten
% LTXE-SLIDE-REFERENCE:	Breaking-64059-Rewritten-JavaScript-Tree-Components.rst
% LTXE-SLIDE-REFERENCE: Breaking-64070-GlobalWebmountsRemoved.rst
% LTXE-SLIDE-REFERENCE: Breaking-64102-MoveT3TableAndT3ButtonToBootstrap.rst
% LTXE-SLIDE-REFERENCE: Breaking-64143-FlagFilesMoved.rst
% ------------------------------------------------------------------------------

\begin{frame}[fragile]
	\frametitle{Geänderte/Entfernte Funktionen}
	\framesubtitle{Entfernungen 2}

	\lstset{
		basicstyle=\tiny\ttfamily
	}

	Die folgenden \textbf{Funktionalitäten} wurden geändert:

	\begin{itemize}

		\item Die Datei \texttt{typo3/js/tree.js} wurde enfernt und durch die Datei \texttt{EXT:backend/Resources/Public/JavaScript/LegacyTree.js} (basierend auf jQuery) ersetzt

		\item Die Variable \texttt{\$GLOBALS['WEBMOUNTS']} wurde entfernt und durch \texttt{\$GLOBALS['BE\_USER']->returnWebmounts()} ersetzt

		\item Der Support für \texttt{.t3-table} und \texttt{.t3-button} im Backend wurde zugunsten der zugehörigen Bootstrap Klassen eingestellt

		\item Alle PNG-Flaggen unter \textt{typo3/gfx/flags/} und \texttt{typo3/sysext/t3skin/images/flags/} wurden nach \texttt{typo3/sysext/core/Resources/Public/Icons/flags/} verschoben

	\end{itemize}

\end{frame}

% ------------------------------------------------------------------------------

% LTXE-SLIDE-START
% LTXE-SLIDE-UID:		22fcde86-bc07585c-9d33e983-be1f4246
% LTXE-SLIDE-TITLE:		Geänderte Funktionalitäten
% LTXE-SLIDE-REFERENCE:	Breaking-64637-CSSStyledContentLegacyTypoScriptRemoved.rst
% LTXE-SLIDE-REFERENCE: Breaking-64639-RemovedContentObjects.rst
% LTXE-SLIDE-REFERENCE: Breaking-64671-ContentObjectImgTextMovedToLegacyExtension.rst
% LTXE-SLIDE-REFERENCE: Breaking-64696-MoveSearchCTypeToLegacyExtension.rst
% LTXE-SLIDE-REFERENCE: Breaking-64762-FormEngineWizards.rst
% ------------------------------------------------------------------------------

\begin{frame}[fragile]
	\frametitle{Geänderte/Entfernte Funktionen}
	\framesubtitle{Entfernungen 3}

	Die folgenden \textbf{Funktionalitäten} wurden geändert:

	\begin{itemize}
		\item Die CSS Styled Content TypoScript-Templates für die Versionen 4.5 - 6.1 wurden ersatzlos gestrichen

		\item Die TypoScript Content Objects (cObjects) \texttt{SEARCHRESULTS}, \texttt{IMGTEXT}, texttt{CLEARGIF}, \texttt{COLUMNS}, \texttt{CTABLE}, \texttt{OTABLE} and \texttt{HRULER} wurden in die Legacy-Extension \textt{EXT:compatibility6} verschoben und stehen damit in Kürze nicht mehr zur Verfügung

		\item Das Content-Element \texttt{search} wurden in die Legacy-Extension \textt{EXT:compatibility6} verschoben

		\item Die TCA-Wizard Optionen \texttt{\_PADDING}, \texttt{\_VALIGN} und \texttt{\_DISTANCE} wurden entfernt
	\end{itemize}

\end{frame}

% ------------------------------------------------------------------------------
% LTXE-SLIDE-START
% LTXE-SLIDE-UID:		a7145bf6-333515f1-1d940f89-b7669b94
% LTXE-SLIDE-TITLE:		Die TS Option andWhere ist veraltet
% LTXE-SLIDE-REFERENCE:	Deprecation-25112-andWhere.rst
% ------------------------------------------------------------------------------

\begin{frame}[fragile]
	\frametitle{Veraltete Funktionen}
	\framesubtitle{Die TS Option andWhere ist veraltet}

	\lstset{
		basicstyle=\tiny\ttfamily
	}
	Die TypoScript Option \texttt{andWhere} innerhalb von der Select-Eigenschaft wurde als veraltet deklariert und sollte durch \texttt{where} und/oder \texttt{markers} ersetzt werden
	\begin{lstlisting}
	page.30 = CONTENT
	page.30 {
		table = tt_content
		select {
			pidInList = this
			orderBy = sorting
			where {
				dataWrap = sorting>{field:sorting}
			}
		}
	}
	page.60 = CONTENT
	page.60 {
		table = tt_content
		select {
			pidInList = 73
			where = header != ###whatever###
			orderBy = ###sortfield###
			markers {
				whatever.data = GP:first
				sortfield.value = sor
				sortfield.wrap = |ting
			}
		}
	}
	\end{lstlisting}

\end{frame}

% ------------------------------------------------------------------------------
% LTXE-SLIDE-START
% LTXE-SLIDE-UID:		25dbf7aa-7ac24863-d703d3fc-96fe5080
% LTXE-SLIDE-TITLE:		Veraltete Entry Points
% LTXE-SLIDE-REFERENCE:	Deprecation-64922-DeprecatedEntryPoints.rst
% ------------------------------------------------------------------------------

\begin{frame}[fragile]
	\frametitle{Veraltete Funktionen}
	\framesubtitle{Veraltete Entry Points}

	\lstset{
		basicstyle=\tiny\ttfamily
	}
	
	\begin{itemize}
		\item Die Entry Points \texttt{typo3/tce\_file.php}, \texttt{typo3/move\_el.php}, \texttt{typo3/tce\_db.php}, \texttt{typo3/login\_frameset.php}, \texttt{typo3/sysext/cms/layout/db\_new\_content\_el.php}, \texttt{typo3/sysext/cms/layout/db\_layout.php} sind veraltet
		\item Stattdessen verwendet man \lstinline{\TYPO3\CMS\Backend\Utility\BackendUtility::getModuleUrl('parameter')} und als Parameter \texttt{tce\_file}, \texttt{move\_element}, \texttt{tce\_db}, \texttt{login\_frameset}, \texttt{new\_content\_element} und \texttt{web\_layout}
	\end{itemize}

\end{frame}

% ------------------------------------------------------------------------------
% LTXE-SLIDE-START
% LTXE-SLIDE-UID:		6c185317-36a5f813-3578e2e2-11aebe1d
% LTXE-SLIDE-TITLE:		Veraltete Funktionen
% LTXE-SLIDE-REFERENCE:	Deprecation-24387-Xhtml2.rst
% LTXE-SLIDE-REFERENCE: Deprecation-46523-BackendUtilityImplodeTSParams.rst
% LTXE-SLIDE-REFERENCE: Deprecation-46770-LocalImageProcessorGraphicalFunctions.rst
% LTXE-SLIDE-REFERENCE: Deprecation-49247-textStyleTableStyleAddParams.rst
% LTXE-SLIDE-REFERENCE: Deprecation-60559-MakeLoginBoxImage.rst
% ------------------------------------------------------------------------------

\begin{frame}[fragile]
	\frametitle{Veraltete Funktionen}
	\framesubtitle{Diverse 1}

	\begin{itemize}
		\item Die TypoScript Option \texttt{config.xhtmlDoctype = xhtml\_2} wird in CMS 8 enfernt

		\item Die Methode \lstinline{TYPO3\CMS\Backend\Utility\BackendUtility::implodeTSParams()} ist veraltet

		\item Die öffentliche Methode \texttt{LocalImageProcessor::getTemporaryImageWithText()} ist veraltet und wird erstezt durch \lstinline{\TYPO3\CMS\Core\Imaging\GraphicalFunctions::getTemporaryImageWithText()}

		\item Die stdWrap Eigenschaften \texttt{textStyle} und \texttt{tableStyle} sind veraltet

		\item Die Methode \lstinline{TYPO3\CMS\Backend\Controller::makeLoginBoxImage()} ist veraltet
	\end{itemize}

\end{frame}

% ------------------------------------------------------------------------------
% LTXE-SLIDE-START
% LTXE-SLIDE-UID:		fd0d82ec-1326c70c-8fd313ee-97f544bc
% LTXE-SLIDE-TITLE:		Veraltete Funktionen
% LTXE-SLIDE-REFERENCE:	Deprecation-61605-ChangeNamingOfIncludeJSlibs.rst
% LTXE-SLIDE-REFERENCE: Deprecation-62329-DocumentTemplate-table.rst
% LTXE-SLIDE-REFERENCE: Deprecation-62855-XHTMLCleaningMovedToLegacyExtension.rst
% LTXE-SLIDE-REFERENCE: Deprecation-63522-ClientRelatedConditionDevice.rst 
% LTXE-SLIDE-REFERENCE: Deprecation-64109-Hook-softRefParserGL.rst
% ------------------------------------------------------------------------------

\begin{frame}[fragile]
	\frametitle{Veraltete Funktionen}
	\framesubtitle{Diverse 2}

	\begin{itemize}
		\item Die TypoScript-Option \texttt{page.includeJSlibs} wurde umbenannt in \texttt{page.includeJSLibs} (mit großem "L" - die alte Option ist deprecated)

		\item Die TypoScript Condition \texttt{device} ist veraltet

		\item Die Methode \texttt{DocumentTable::table()} ist veraltet - hierfür wird die Verwendung von Fluid empfohlen

		\item Die Methode \lstinline{TYPO3\CMS\Frontend\Controller\TypoScriptFrontendController::doXHTML_cleaning()} und die TypoScript-Option \texttt{config.xhtml\_cleaning} wurden als deprecated markiert.

		\item Der Hook \texttt{\$GLOBALS['TYPO3\_CONF\_VARS']['SC\_OPTIONS']['GLOBAL']['softRefParser\_GL']} ist veraltet
 
	\end{itemize}

\end{frame}

% ------------------------------------------------------------------------------
% LTXE-SLIDE-START
% LTXE-SLIDE-UID:		4ee3e8c6-03a50aba-45b32ae2-1c580e8a
% LTXE-SLIDE-TITLE:		Veraltete Funktionen
% LTXE-SLIDE-REFERENCE:	Deprecation-64134-TypoScriptTemplateObjectBrowserModuleFunctionController-verify_TSobjects.rst
% LTXE-SLIDE-REFERENCE: Deprecation-64147-ConstantEditorFunctions.rst
% LTXE-SLIDE-REFERENCE: Deprecation-64388-ContentObjectMethods.rst
% LTXE-SLIDE-REFERENCE: Deprecation-63847-FormEngine-renderReadonly.rst
% LTXE-SLIDE-REFERENCE: 
% ------------------------------------------------------------------------------

\begin{frame}[fragile]
	\frametitle{Veraltete Funktionen}
	\framesubtitle{Diverse 3}

	\begin{itemize}
		\item Die Methode \texttt{TypoScriptTemplateObjectBrowserModuleFunctionController::verify\_TSobjects()} ist veraltet
 		\item Die Methoden \texttt{ExtendedTemplateService::ext\_getKeyImage()} und \texttt{ConfigurationForm::ext\_getKeyImage()} sind veraltet
 		\item Der Aufruf von \texttt{contentObject->COBJECT()} (für alle cObjects wie \texttt{IMAGE} oder \texttt{SVG} wurde als veraltet gekennzeichnet. Stattdessen verwendet man: \texttt{\$cObj->cObjGetSingle('SVG', \$conf);}
 		\item Der Zugriff über \texttt{FormEngine::\$renderReadonly} gilt als veraltet - dafür sollte \texttt{AbstractFormElement::setRenderReadonly(TRUE)} verwendet werden

	\end{itemize}

\end{frame}

% ------------------------------------------------------------------------------
% LTXE-SLIDE-START
% LTXE-SLIDE-UID:		27445023-f959516d-3c34671a-bf2e0485
% LTXE-SLIDE-TITLE:		Veraltete Funktionen
% LTXE-SLIDE-REFERENCE:	Deprecation-63850-FormEngine-insertDefStyle.rst
% LTXE-SLIDE-REFERENCE: Deprecation-63852-FormEngine-getAvailableLanguages.rst
% LTXE-SLIDE-REFERENCE: Deprecation-63855-FormEngine-sL.rst
% LTXE-SLIDE-REFERENCE: Deprecation-63864-FormEngine-renderVDEFDiff.rst
% LTXE-SLIDE-REFERENCE: Deprecation-63878-FormEngine-getLL.rst
% LTXE-SLIDE-REFERENCE: Deprecation-63889-FormEngine-getTSCpid.rst
% LTXE-SLIDE-REFERENCE: Deprecation-63912-FormEngine-unusedMethods.rst
% ------------------------------------------------------------------------------

\begin{frame}[fragile]
	\frametitle{Veraltete Funktionen}
	\framesubtitle{Diverse 4}

	\begin{itemize}
		\item Die folgenden FormEngine-Methoden sind veraltet:
		\begin{itemize}
			\item \texttt{FormEngine::insertDefStyle}
			\item \texttt{FormEngine::getAvailableLanguages()}
			\item \texttt{FormEngine::sL()}
			\item \texttt{FormEngine::renderVDEFDiff()}
			\item \texttt{FormEngine::getLL()}
 			\item \texttt{FormEngine::getTSCpid()}
 			\item \texttt{FormEngine::getSingleField\_typeFlex\_langMenu()}
 			\item \texttt{FormEngine::getSingleField\_typeFlex\_sheetMenu()}
 			\item \texttt{FormEngine::getSpecConfFromString()}
 		\end{itemize}
	\end{itemize}

\end{frame}
