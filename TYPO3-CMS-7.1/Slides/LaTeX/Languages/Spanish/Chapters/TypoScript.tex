% ------------------------------------------------------------------------------
% TYPO3 CMS 7.1 - What's New - Chapter "TypoScript" (English Version)
%
% @author	Michael Schams <schams.net>
% @license	Creative Commons BY-NC-SA 3.0
% @link		http://typo3.org/download/release-notes/whats-new/
% @language	English
% ------------------------------------------------------------------------------
% LTXE-CHAPTER-UID:		d92ab5b9-9ffa4925-30a2e34f-0d6939e5
% LTXE-CHAPTER-NAME:	TypoScript
% ------------------------------------------------------------------------------

\section{TSconfig \& TypoScript}
\begin{frame}[fragile]
	\frametitle{TSconfig \& TypoScript}

	\begin{center}\huge{Capítulo 2:}\end{center}
	\begin{center}\huge{\color{typo3darkgrey}\textbf{TSconfig \& TypoScript}}\end{center}

\end{frame}

% ------------------------------------------------------------------------------
% LTXE-SLIDE-START
% LTXE-SLIDE-UID:		b4fc65b9-1c59e8f1-1f79434a-5d4a1c7f
% LTXE-SLIDE-ORIGIN:	a55771e2-74f3a0fc-5be51f93-8b40a5d8 English
% LTXE-SLIDE-TITLE:		StdWrap for page.headTag
% LTXE-SLIDE-REFERENCE:	Feature-22086-StdWrapForHeadTag.rst
% ------------------------------------------------------------------------------

\begin{frame}[fragile]
	\frametitle{TSconfig \& TypoScript}
	\framesubtitle{StdWrap para page.headTag}

	\begin{itemize}
		\item Ajuste TypoScript \texttt{page.headTag} tiene funcionalidad \texttt{stdWrap} ahora

			\begin{lstlisting}
				page = PAGE
				page.headTag = <head>
				page.headTag.override = <head class="special">
				page.headTag.override.if {
		  		  isInList.field = uid
		  		  value = 24
				}
			\end{lstlisting}	

	\end{itemize}

\end{frame}

% ------------------------------------------------------------------------------
% LTXE-SLIDE-START
% LTXE-SLIDE-UID:		651c5380-21eb01f8-2222e5d2-28ffb2d7
% LTXE-SLIDE-ORIGIN:	618eb9ea-56d5a4eb-d5d69d66-cc62f953 English
% LTXE-SLIDE-TITLE:		Include JavaScript files asynchronously
% LTXE-SLIDE-REFERENCE:	Feature-28382-AddAsyncPropertyToJavaScriptFiles.rst
% ------------------------------------------------------------------------------

\begin{frame}[fragile]
	\frametitle{TSconfig \& TypoScript}
	\framesubtitle{Incluir ficheros JavaScript asíncronamente}

	\begin{itemize}
		\item Se pueden cargar ficheros JavaScript asíncronamente

			\begin{lstlisting}
				page {
				  includeJS {
				    jsFile = /path/to/file.js
				    jsFile.async = 1
				  }
				}
			\end{lstlisting}

		\item Esto afecta a:

			\begin{itemize}
				\item \texttt{includeJSlibs} / \texttt{includeJSLibs}
				\item \texttt{includeJSFooterlibs}
				\item \texttt{includeJS}
				\item \texttt{includeJSFooter}
			\end{itemize}

	\end{itemize}

\end{frame}

% ------------------------------------------------------------------------------
% LTXE-SLIDE-START
% LTXE-SLIDE-UID:		a2d28b3a-9b402c71-c101449f-6edb0268
% LTXE-SLIDE-ORIGIN:	8a6ca08e-cf9ba481-18fd9411-30dd0a0f English
% LTXE-SLIDE-TITLE:		HMENU item selection via additionalWhere
% LTXE-SLIDE-REFERENCE:	Feature-46624-AdditionalWhereForMenu.rst
% ------------------------------------------------------------------------------

\begin{frame}[fragile]
	\frametitle{TSconfig \& TypoScript}
	\framesubtitle{Selección de item HMENU vía \texttt{additionalWhere}}

	\begin{itemize}

		\item TypoScript cObject \texttt{HMENU} cuenta con una nueva propiedad \texttt{additionalWhere}
		\item Esto permite una consulta a la base de datos más específica (p.e. filtrado)

		\item Ejemplo:

			\begin{lstlisting}
				lib.authormenu = HMENU
				lib.authormenu.1 = TMENU
				lib.authormenu.1.additionalWhere = AND author!=""
			\end{lstlisting}

	\end{itemize}

\end{frame}

% ------------------------------------------------------------------------------
% LTXE-SLIDE-START
% LTXE-SLIDE-UID:		6b1c724d-e70eca8a-a085c4bb-14ebfc4c
% LTXE-SLIDE-ORIGIN:	e07d1984-117d1212-a1f7fab8-8ea21ebc English
% LTXE-SLIDE-TITLE:		Additional properties for HMENU browse menus
% LTXE-SLIDE-REFERENCE:	Feature-57178-SpecialHmenuExcludeSpecialParameters.rst
% ------------------------------------------------------------------------------

\begin{frame}[fragile]
	\frametitle{TSconfig \& TypoScript}
	\framesubtitle{Propiedades adicionales para menús \texttt{HMENU} browse}

	\begin{itemize}
		\item Dos nuevas propiedades para el cObject \texttt{HMENU} (opción "\texttt{special=browse"})\newline
			para seleccionar ítems de menú más de grano fino:
		
			\begin{itemize}
				\item \texttt{excludeNoSearchPages}
				\item \texttt{includeNotInMenu}
			\end{itemize}

		\item Ejemplo:

			\begin{lstlisting}
				lib.browsemenu = HMENU
				lib.browsemenu.special = browse
				lib.browsemenu.special.excludeNoSearchPages = 1
				lib.browsemenu.includeNotInMenu = 1
			\end{lstlisting}

	\end{itemize}

\end{frame}

% ------------------------------------------------------------------------------
% LTXE-SLIDE-START
% LTXE-SLIDE-UID:		13b0e4ea-ea95b7f7-f9b4277a-16ca0168
% LTXE-SLIDE-ORIGIN:	a286a87a-161365ed-5560e66f-e9f1a2ee English
% LTXE-SLIDE-TITLE:		Multiple HTTP headers
% LTXE-SLIDE-REFERENCE:	Feature-56236-Multiple-HTTP-Headers-In-Frontend.rst
% ------------------------------------------------------------------------------

\begin{frame}[fragile]
	\frametitle{TSconfig \& TypoScript}
	\framesubtitle{Múltiples cabeceras HTTP}

	\begin{itemize}

		\item Se pueden establecer cabeceras HTTP como un array (\small\texttt{config.additionalHeaders}\normalsize)
		\item Esto permite la configuración de múltiples cabeceras a la vez

			\begin{lstlisting}
				config.additionalHeaders {
				  10 {
				    # header string
				    header = WWW-Authenticate: Negotiate

				    # (optional) replace previous headers with the same name (default: 1)
				    replace = 0

				    # (optional) force HTTP response code
				    httpResponseCode = 401
				  }
				  # set second additional HTTP header
				  20.header = Cache-control: Private
				}
			\end{lstlisting}

	\end{itemize}

\end{frame}

% ------------------------------------------------------------------------------
% LTXE-SLIDE-START
% LTXE-SLIDE-UID:		5e59a57e-49a88757-60cd02bd-258e613b
% LTXE-SLIDE-ORIGIN:	a296b9b8-9403ef17-30f80f28-2a8a94ce English
% LTXE-SLIDE-TITLE:		Option "auto" added for config.absRefPrefix
% LTXE-SLIDE-REFERENCE:	Feature-58366-AutomaticAbsRefPrefix.rst
% ------------------------------------------------------------------------------

\begin{frame}[fragile]
	\frametitle{TSconfig \& TypoScript}
	\framesubtitle{Opción "auto" añadida para config.absRefPrefix}

	\begin{itemize}
		\item Se puede usar el ajuste TypoScript \texttt{config.absRefPrefix} para permitir la reescritura URL.
			Como una alternativa a \texttt{config.baseURL} (para configurar un dominio específico), \texttt{absRefPrefix} puede detectar la raíz del sitio automáticamente:

			\begin{lstlisting}
				config.absRefPrefix = auto

				# ...instead of:
				[ApplicationContext = Production]
				config.absRefPrefix = /

				[ApplicationContext = Testing]
				config.absRefPrefix = /my_site_root/
			\end{lstlisting}

		\smaller
			Nota: La nueva opción es también segura para entornos multi-dominios para evitar
			mecanismo de cacheo duplicado.
		\normalsize

	\end{itemize}

\end{frame}

% ------------------------------------------------------------------------------
% LTXE-SLIDE-START
% LTXE-SLIDE-UID:		b3b27941-3f123b4b-a6bbb7d4-1114bfd3
% LTXE-SLIDE-ORIGIN:	8210d253-d630446f-ba747ba6-5568843d English
% LTXE-SLIDE-TITLE:		Two-letter ISO code for sys_language (1)
% LTXE-SLIDE-REFERENCE:	Feature-61542-AddIsoLanguageKeys.rst
% ------------------------------------------------------------------------------

\begin{frame}[fragile]
	\frametitle{TSconfig \& TypoScript}
	\framesubtitle{Código de dos letras ISO para sys\_language (1)}

	\begin{itemize}
		\item El manejo de idiomas se hace mediante registros almacenados en la tabla de BD
			\texttt{sys\_language}, que se referencian normalmente vía  \texttt{sys\_language\_uid}
		\item En TYPO3 CMS 7.1, se ha introducido el código de dos letras de la ISO 639-1:

			\begin{itemize}
				\item Nuevo campo DB: \texttt{sys\_language.language\_isocode}
				\item Nueva opción TypoScript: \texttt{sys\_language\_isocode}
			\end{itemize}


		\vspace{1cm}

		\small
			Nota: \textbf{ISO 639} es un conjunto de estándares de la Organización Internacional de Estandarización. La lista de códigos ISO 639-1 está disponible en :\newline
			\url{http://en.wikipedia.org/wiki/List_of_ISO_639-1_codes}
		\normalsize

	\end{itemize}

\end{frame}

% ------------------------------------------------------------------------------
% LTXE-SLIDE-START
% LTXE-SLIDE-UID:		5afd8378-f91b8b2a-2af27366-c2a353d8
% LTXE-SLIDE-ORIGIN:	d630446f-ba747ba6-5568843d-8210d253 English
% LTXE-SLIDE-TITLE:		Two-letter ISO code for sys_language (2)
% LTXE-SLIDE-REFERENCE:	Feature-61542-AddIsoLanguageKeys.rst
% ------------------------------------------------------------------------------

\begin{frame}[fragile]
	\frametitle{TSconfig \& TypoScript}
	\framesubtitle{Código de dos letras ISO para sys\_language (2)}

	\begin{itemize}
		\item Ejemplo:

			\begin{lstlisting}
				# Danish by default
				config.sys_language_uid = 0
				config.sys_language_isocode_default = da

				[globalVar = GP:L = 1]
				  # ISO code stored in table sys_language (uid 1)
				  config.sys_language_uid = 1
				  # overwrite ISO code as required
				  config.sys_language_isocode = fr
				[GLOBAL]

				page.10 = TEXT
				page.10.data = TSFE:sys_language_isocode
				page.10.wrap = <div class="main" data-language="|">
			\end{lstlisting}

	\end{itemize}

\end{frame}

% ------------------------------------------------------------------------------
% LTXE-SLIDE-START
% LTXE-SLIDE-UID:		dc7d84f3-566551df-5376bc11-d87f012a
% LTXE-SLIDE-ORIGIN:	df2f7fec-fb8bc3b9-446fb779-c409ef27 English
% LTXE-SLIDE-TITLE:		Custom TypoScript Conditions in Backend
% LTXE-SLIDE-REFERENCE:	Feature-63600-CustomTypoScriptConditionsInBackend.rst
% ------------------------------------------------------------------------------

\begin{frame}[fragile]
	\frametitle{TSconfig \& TypoScript}
	\framesubtitle{Condiciones TypoScript personalizadas en Backend}

	% decrease font size for code listing
	\lstset{basicstyle=\tiny\ttfamily}

	\begin{itemize}
	
		\item Se han introducido ya en TYPO3 CMS 7.0 soporte de condiciones personalizadas para el \textbf{frontend}
		\item Desde TYPO3 CMS 7.1, es también posible usar condiciones personalizadas en el \textbf{backend}
		\item La condición debe derivarse de \texttt{AbstractCondition} e implementar el método \texttt{matchCondition()}
		\item Uso de ejemplo en TypoScript:

			\begin{lstlisting}
				[BigCompanyName\TypoScriptLovePackage\MyCustomTypoScriptCondition]

				[BigCompanyName\TypoScriptLovePackage\MyCustomTypoScriptCondition = 7]

				[BigCompanyName\TypoScriptLovePackage\MyCustomTypoScriptCondition = 7, != 6]

				[BigCompanyName\TypoScriptLovePackage\MyCustomTypoScriptCondition = {$mysite.myconstant}]
			\end{lstlisting}

	\end{itemize}

\end{frame}

% ------------------------------------------------------------------------------
% LTXE-SLIDE-START
% LTXE-SLIDE-UID:		1fc6f6a1-7c9e8d4c-3f705baa-e586d554
% LTXE-SLIDE-ORIGIN:	b9687dbb-c0502234-7ca16b06-1879fe80 English
% LTXE-SLIDE-TITLE:		FormEngine: customize icons via PageTSconfig
% LTXE-SLIDE-REFERENCE:	Feature-35891-AddTCAItemsWithIconsViaPageTSConfig.rst
% ------------------------------------------------------------------------------

\begin{frame}[fragile]
	\frametitle{TSconfig \& TypoScript}
	\framesubtitle{Iconos personalizados vía PageTSconfig}

	\begin{itemize}
		\item Pueden configurarse ya pares valor/etiqueta de campos select a través de la opción PageTSconfig \texttt{addItems}
		\item Es también posible influenciar el \textbf{icono} de estos campos ahora

			\begin{itemize}
				\item Opción 1: usando \texttt{addItems} y la subpropiedad \texttt{.icon}
				\item Opción 2: usando \texttt{altIcons} (todos los items en general)
			\end{itemize}

		\item Ejemplo:

			\begin{lstlisting}
				TCEFORM.pages.doktype.addItems {
				  10 = My Label
				  10.icon = EXT:t3skin/icons/gfx/i/pages.gif
				}
				TCEFORM.pages.doktype.altIcons {
				  10 = EXT:myext/icon.gif
				}
			\end{lstlisting}

	\end{itemize}

\end{frame}

% ------------------------------------------------------------------------------
% LTXE-SLIDE-START
% LTXE-SLIDE-UID:		b2ed9eaf-2cc623e5-1535ca29-c949cf96
% LTXE-SLIDE-ORIGIN:	b8794774-4912764b-cd1cc91d-de4396fe English
% LTXE-SLIDE-TITLE:		Extend element browser with mount points
% LTXE-SLIDE-REFERENCE:	Feature-50780-AppendElementBrowserMountPoints.rst
% ------------------------------------------------------------------------------

\begin{frame}[fragile]
	\frametitle{TSconfig \& TypoScript}
	\framesubtitle{Extender elemento browser con puntos de montaje}

	\begin{itemize}
		\item Nueva opción UserTSconfig \texttt{.append} permite a los administradores \textbf{añadir}
			puntos de montaje, en lugar de reemplazar los puntos de montaje de base de datos del usuario

		\item Ejemplo:

			\begin{lstlisting}
				options.pageTree.altElementBrowserMountPoints = 20,31
				options.pageTree.altElementBrowserMountPoints.append = 1
			\end{lstlisting}

	\end{itemize}

\end{frame}

% ------------------------------------------------------------------------------
% LTXE-SLIDE-START
% LTXE-SLIDE-UID:		d8cd3186-b4e587ce-93bb1bae-7d49d02f
% LTXE-SLIDE-ORIGIN:	04b7f7c0-af900c25-5f16f178-8a94fcca English
% LTXE-SLIDE-TITLE:		Label override of checkboxes and radio buttons by TSconfig
% LTXE-SLIDE-REFERENCE:	Feature-58033-AltLabelsForFormEngineCheckboxAndRadioButtons.rst
% ------------------------------------------------------------------------------

\begin{frame}[fragile]
	\frametitle{TSconfig \& TypoScript}
	\framesubtitle{Sobreescritura de etiqueta de checkboxes y radio buttons}

	% decrease font size for code listing
	\lstset{basicstyle=\tiny\ttfamily}

	\begin{itemize}

		\item Ahora pueden sobreescribirse etiquetas de radio buttons y checkboxes
		\item Ejemplo:

			\begin{lstlisting}
				// field with a single checkbox (use ".default")
				TCEFORM.pages.hidden.altLabels.default = new label
				TCEFORM.pages.hidden.altLabels.default = LLL:path/to/languagefile.xlf:individualLabel

				// field with multiple checkboxes (0, 1, 2, 3...)
				TCEFORM.pages.l18n_cfg.altLabels.0 = new label of first checkbox
				TCEFORM.pages.l18n_cfg.altLabels.1 = new label of second checkbox
				TCEFORM.pages.l18n_cfg.altLabels.2 = new label of third checkbox
				...
			\end{lstlisting}

	\end{itemize}

\end{frame}

% ------------------------------------------------------------------------------
% LTXE-SLIDE-START
% LTXE-SLIDE-UID:		f2740502-fb85c1aa-cac3ace8-71d51614
% LTXE-SLIDE-ORIGIN:	6cdd5010-f7d7d7bd-0a3c74ea-05424fb0 English
% LTXE-SLIDE-TITLE:		Miscellaneous (1)
% LTXE-SLIDE-REFERENCE: Feature-xxxxx ... UserTSconfig-width-and-height-of-element-browser
% LTXE-SLIDE-REFERENCE: Feature-59646-AddRteConfigurationPropertyButtonsLinkTypePropertiesTargetDefault.rst
% ------------------------------------------------------------------------------

\begin{frame}[fragile]
	\frametitle{TSconfig \& TypoScript}
	\framesubtitle{Miscelánea (1)}

	\begin{itemize}
		\item Puede configurarse el ancho y alto del Elemento Browser usando UserTSconfig:

			\begin{lstlisting}
				options.popupWindowSize = 400x900
				options.RTE.popupWindowSize = 200x200
			\end{lstlisting}


		\item PageTSconfig: puede usarse la nueva propiedad de configuración RTE para configurar un destino por defecto para enlaces de un determinado tipo:

			\begin{lstlisting}
				buttons.link.[type].properties.target.default
			\end{lstlisting}

			Donde \texttt{[type]} puede ser \texttt{page}, \texttt{file}, \texttt{url}, \texttt{mail} o \texttt{spec}\newline
			(extensiones pueden proporcionar más tipos)

	\end{itemize}

\end{frame}

% ------------------------------------------------------------------------------
% LTXE-SLIDE-START
% LTXE-SLIDE-UID:		2b652f0e-ea3673ea-4bf0ef00-cd872579
% LTXE-SLIDE-ORIGIN:	80595308-128c5961-8ae4cf89-8611ce66 English
% LTXE-SLIDE-TITLE:		Miscellaneous (2)
% LTXE-SLIDE-REFERENCE:	Feature-16794-MakeSectionLinkingForIndexedSearchResultsConfigurable.rst
% LTXE-SLIDE-REFERENCE: Feature-20767-getDataByNestedKey.rst
% LTXE-SLIDE-REFERENCE: Feature-33491-StdWrapForTitleTag.rst
% ------------------------------------------------------------------------------

\begin{frame}[fragile]
	\frametitle{TSconfig \& TypoScript}
	\framesubtitle{Miscelánea (2)}

	\begin{itemize}

		\item Cabeceras de sección de resultados de búsqueda indexada son enlaces por defecto.
			Ahora es posible deshabilitar estos enlaces y desplegar secciones como textos simples

			\begin{lstlisting}
				plugin.tx_indexedsearch.linkSectionTitles = 0
			\end{lstlisting}

		\item \texttt{getData} puede acceder a datos de un \texttt{campo} ahora (no sólo arrays
			como \texttt{GPVar} y \texttt{TSFE}):
		
			\begin{lstlisting}
				10 = TEXT
				10.data = field:fieldname|level1|level2
			\end{lstlisting}

		\item Ajuste TypoScript \texttt{config.pageTitle} tiene funcionalidad \texttt{stdWrap} ahora

			\begin{lstlisting}
				# make value of <title> upper case
				page = PAGE
				page.config.pageTitle.case = upper
			\end{lstlisting}

	\end{itemize}

\end{frame}

% ------------------------------------------------------------------------------
