% ------------------------------------------------------------------------------
% TYPO3 CMS 7.0 - What's New - Chapter "In-Depth Changes" (German Version)
%
% @author	Patrick Lobacher <patrick@lobacher.de>
% @license	Creative Commons BY-NC-SA 3.0
% @link		http://typo3.org/download/release-notes/whats-new/
% @language	German
% ------------------------------------------------------------------------------
% LTXE-CHAPTER-UID:		7cb2d402-105890ae-a1632623-719adbb6
% LTXE-CHAPTER-NAME:	In-Depth Changes
% ------------------------------------------------------------------------------

\section{Änderungen im System}
\begin{frame}[fragile]
	\frametitle{Änderungen im System}

	\begin{center}\huge{Kapitel 3:}\end{center}
	\begin{center}\huge{\color{typo3darkgrey}\textbf{Änderungen im System}}\end{center}

\end{frame}

% ------------------------------------------------------------------------------
% LTXE-SLIDE-START
% LTXE-SLIDE-UID:		8f333cd6-54360c45-7246680e-e14eddca
% LTXE-SLIDE-ORIGIN:	xxxxxxxx-xxxxxxxx-xxxxxxxx-xxxxxxxx
% LTXE-SLIDE-TITLE:		Integration von jQuery UI 1.11.2
% LTXE-SLIDE-REFERENCE:	https://forge.typo3.org/issues/62916
% ------------------------------------------------------------------------------

\begin{frame}[fragile]
	\frametitle{Änderungen im System}
	\framesubtitle{Integration von jQuery UI 1.11.2}

	\begin{itemize}
		\item jQuery UI 1.11 unterstützt AMD
			(\href{http://en.wikipedia.org/wiki/Asynchronous_module_definition}{Asynchronous Module Definition}),
			mit welcher JavaScript Dateien erst geladen werden, wenn sie benötigt werden (Geschwindigkeitsvorteil)

    	\item jQuery UI 1.11 ersetzt jQuery UI 1.10 + Scriptaculous in TYPO3 CMS 7.0

    	\item Es werden nur der Core und Interaction Components inkludiert,
    		die notwendig sind, um ExtJS und Scriptaculous zu ersetzen.

    	\item Widgets werden nicht inkludiert, sondern über Bootstrap realisiert\newline
    		(z.B. DatePicker, Spinner, Dialog, Buttons, Tabs, Tooltip)

	\end{itemize}

\end{frame}

% ------------------------------------------------------------------------------
% LTXE-SLIDE-START
% LTXE-SLIDE-UID:		85cbd279-a1d0fb43-1d4d431c-825727f7
% LTXE-SLIDE-ORIGIN:	xxxxxxxx-xxxxxxxx-xxxxxxxx-xxxxxxxx
% LTXE-SLIDE-TITLE:		Registry für die File Rendering Classes
% LTXE-SLIDE-REFERENCE:	https://forge.typo3.org/issues/61800
% ------------------------------------------------------------------------------

\begin{frame}[fragile]
	\frametitle{Änderungen im System}
	\framesubtitle{Registry für die File Rendering Classes}

	\lstset{
		basicstyle=\tiny\ttfamily
	}

	\begin{itemize}
		\item Um verschiedene Dateiformate zu rendern, wird eine Registry benötigt,
			bei der sich FileRenderer registrieren können. Dies geschieht für "Name"
			(z.B. Video, MPEG, AVI, WAV, ...) wie folgt:

		\begin{lstlisting}
			<?php
			namespace ...;

			class NameTagRenderer implements FileRendererInterface {
			  protected $possibleMimeTypes = array('audio/mpeg', 'audio/wav', ...);
			  public function getPriority() {
			    return 1; // priority: the higher, the more important (max: 100)
			  }
			  public function canRender(FileInterface $file) {
			    return in_array($file->getMimeType(), $this->possibleMimeTypes, TRUE);
			  }
			  public function render(FileInterface $file, $width, $height, array $options = array(), $usedPathsRelativeToCurrentScript = FALSE) {
			    ...

			    return 'HTML code';
			  }
			}
		\end{lstlisting}

	\end{itemize}

\end{frame}

% ------------------------------------------------------------------------------
% LTXE-SLIDE-START
% LTXE-SLIDE-UID:		45fc629e-8b1c182e-772d79dd-92863e26
% LTXE-SLIDE-ORIGIN:	xxxxxxxx-xxxxxxxx-xxxxxxxx-xxxxxxxx
% LTXE-SLIDE-TITLE:		Evaluierungsfunktion "email" für TCA
% LTXE-SLIDE-REFERENCE:	https://forge.typo3.org/issues/62147
% ------------------------------------------------------------------------------

\begin{frame}[fragile]
	\frametitle{Änderungen im System}
	\framesubtitle{Evaluierungsfunktion "email" für das TCA}

	\begin{itemize}
		\item Für das TCA wurde eine neue Evaluierungsfunktion "email" hinzugefügt,
			welche serverseitig überprüft, ob ein eingegebener Wert eine gültige
			Email ist. Im Fehlerfall wird eine Flash-Message ausgegeben

		\item Example:

		\begin{lstlisting}
			'emailaddress' => array(
			  'exclude' => 1,
			  'label' => 'LLL:EXT:myextension/Resources/Private/Language/locallang_db.xlf:tx_myextension
		 	  'config' => array(
			      'type' => 'input',
			      'size' => 30,
			      'eval' => 'email,trim'
			  ),
			)
		\end{lstlisting}
	\end{itemize}

\end{frame}

% ------------------------------------------------------------------------------
% LTXE-SLIDE-START
% LTXE-SLIDE-UID:		021fbd7e-6a837a9c-68cf66fc-4ab5c95d
% LTXE-SLIDE-ORIGIN:	xxxxxxxx-xxxxxxxx-xxxxxxxx-xxxxxxxx
% LTXE-SLIDE-TITLE:		Einführung einer abstrakten Condition im TypoScript
% LTXE-SLIDE-REFERENCE:	https://forge.typo3.org/issues/61489
% ------------------------------------------------------------------------------

\begin{frame}[fragile]
	\frametitle{Änderungen im System}
	\framesubtitle{Einführung einer abstrakten Condition in TypoScript}

	\lstset{
		basicstyle=\tiny\ttfamily
	}

	\begin{itemize}
		\item Es gibt nun eine AbstractCondition, von der eigene Conditions ableiten können

		\begin{lstlisting}
			class TestCondition
			  extends \TYPO3\CMS\Core\Configuration\TypoScript\ConditionMatching\AbstractCondition {

			  public function matchCondition(array $conditionParameters) {
 			    if ($conditionParameters[0] === '= 7' && $conditionParameters[1] === '!= 6') {
			      throw new TestConditionException('All Ok', 1411581139);
			    }
			  }
			}
		\end{lstlisting}

		\item Die Verwendung in TypoScript sieht wie folgt aus:

		\begin{lstlisting}
			[Vendor\Package\TestCondition]
			[Vendor\Package\TestCondition = 7]
			[Vendor\Package\TestCondition = 7, != 6]
		\end{lstlisting}

		\item Welche Operatoren in der Condition zur Verfügung stehen, wird in der Klasse selbst festgelegt

	\end{itemize}

\end{frame}

% ------------------------------------------------------------------------------
% LTXE-SLIDE-START
% LTXE-SLIDE-UID:		53843173-31dd11ff-e4bdb602-fa4621e6
% LTXE-SLIDE-ORIGIN:	xxxxxxxx-xxxxxxxx-xxxxxxxx-xxxxxxxx
% LTXE-SLIDE-TITLE:		Signal zur Manipulation des HTML-Tags in IconUtility
% LTXE-SLIDE-REFERENCE:	https://forge.typo3.org/issues/61289
% ------------------------------------------------------------------------------

\begin{frame}[fragile]
	\frametitle{Änderungen im System}
	\framesubtitle{Signal zur Manipulation des HTML-Tags in IconUtility}

	\begin{itemize}
		\item Signal für IconUtility zur HTML-Tag Manipulation
		\begin{lstlisting}
			dispatch(
			  'TYPO3\\CMS\\Backend\\Utility\\IconUtility',
			  'buildSpriteHtmlIconTag',
			  array($tagAttributes, $innerHtml, $tagName)
			);
		\end{lstlisting}

		\item Wird aufgerufen in:

		\begin{lstlisting}
			TYPO3\CMS\Backend\Utility\IconUtility\buildSpriteHtmlIconTag
		\end{lstlisting}

	\end{itemize}

\end{frame}

% ------------------------------------------------------------------------------
% LTXE-SLIDE-START
% LTXE-SLIDE-UID:		cc20a934-37d08b86-bb2732f9-986fa7d2
% LTXE-SLIDE-ORIGIN:	xxxxxxxx-xxxxxxxx-xxxxxxxx-xxxxxxxx
% LTXE-SLIDE-TITLE:		Signal Slots für SoftReferenceIndex
% LTXE-SLIDE-REFERENCE:	https://forge.typo3.org/issues/21396
% ------------------------------------------------------------------------------

\begin{frame}[fragile]
	\frametitle{Änderungen im System}
	\framesubtitle{Signal Slots für SoftReferenceIndex}
	\lstset{
		basicstyle=\tiny\ttfamily
	}
	\begin{itemize}
		\item Zwei neue Signal Slot Dispatch Calls innerhalb von SoftReferenceIndex:

			\begin{lstlisting}
				protected function emitGetTypoLinkParts(
				  $linkHandlerFound, $finalTagParts, $linkHandlerKeyword, $linkHandlerValue) {
				  return $this->getSignalSlotDispatcher()->dispatch(
				    get_class($this),
				    'getTypoLinkParts',
				    array($linkHandlerFound, $finalTagParts, $linkHandlerKeyword, $linkHandlerValue)
				  );
				}
				protected function emitSetTypoLinkPartsElement(
				  $linkHandlerFound, $tLP, $content, $elements, $idx, $tokenID) {
				  return $this->getSignalSlotDispatcher()->dispatch(
				    get_class($this),
				    'setTypoLinkPartsElement',
				    array($linkHandlerFound, $tLP, $content, $elements, $idx, $tokenID, $this)
				  );
				}
			\end{lstlisting}

		\item Wird augerufen in:

			\begin{lstlisting}
				TYPO3\CMS\Core\Database\SoftReferenceIndex->findRef_typolink
				TYPO3\CMS\Core\Database\SoftReferenceIndex->getTypoLinkParts
			\end{lstlisting}

	\end{itemize}

\end{frame}

% ------------------------------------------------------------------------------
% LTXE-SLIDE-START
% LTXE-SLIDE-UID:		ac32f68e-4c39f8db-462f6249-2e1b56ca
% LTXE-SLIDE-ORIGIN:	xxxxxxxx-xxxxxxxx-xxxxxxxx-xxxxxxxx
% LTXE-SLIDE-TITLE:		Signal Slots für afterPersistObjects
% LTXE-SLIDE-REFERENCE:	https://forge.typo3.org/issues/59986
% ------------------------------------------------------------------------------

\begin{frame}[fragile]
	\frametitle{Änderungen im System}
	\framesubtitle{Signal Slots für afterPersistObjects}

	\lstset{
		basicstyle=\tiny\ttfamily
	}

	\begin{itemize}

		\item Bislang gab es nur ein Signal, wenn ein Objekt im Repository (Extbase) aktualisiert wurde: \texttt{afterUpdate}

		\item Da aber ein Aggregate Root beispielsweise Subobjekte ebenfalls persistiert, benötigt es ein Signal,
			welches erst dann emmitiert wird, wenn alle Objekte eines Aggregate Roots persistiert wurden:

			\begin{lstlisting}
				protected function emitAfterPersistObjectSignal(DomainObjectInterface $object) {
				  $this->signalSlotDispatcher->dispatch(__CLASS__, 'afterPersistObject', array($object));
				}
			\end{lstlisting}

		\item Wird aufgerufen in:

			\begin{lstlisting}
				TYPO3\CMS\Extbase\Persistence\Generic\Backend->persistObject
			\end{lstlisting}

	\end{itemize}

\end{frame}                                                 

% ------------------------------------------------------------------------------
% LTXE-SLIDE-START
% LTXE-SLIDE-UID:		e47d1750-43554686-2db53fb9-836f6559
% LTXE-SLIDE-ORIGIN:	xxxxxxxx-xxxxxxxx-xxxxxxxx-xxxxxxxx
% LTXE-SLIDE-TITLE:		Signal Slots für loadBaseTca
% LTXE-SLIDE-REFERENCE:	https://forge.typo3.org/issues/57863
% ------------------------------------------------------------------------------

\begin{frame}[fragile]
	\frametitle{Änderungen im System}
	\framesubtitle{Signal Slots für loadBaseTca}

	\begin{itemize}
		\item Mit diesem Signal kann das gesamte TCA (anstelle von Teilen) gecacht werden

			\begin{lstlisting}
				protected function emitTcaIsBeingBuiltSignal(array $tca) {
				  list($tca) = static::getSignalSlotDispatcher()->dispatch(
				    __CLASS__,
				    'tcaIsBeingBuilt',
				    array($tca)
				  );
				  $GLOBALS['TCA'] = $tca;
				}
			\end{lstlisting}

		\item Wird audgerufen in:

			\begin{lstlisting}
				TYPO3\CMS\Core\Utility\ExtensionManagementUtility\Backend->buildBaseTcaFromSingleFiles
			\end{lstlisting}

	\end{itemize}

\end{frame}  

% ------------------------------------------------------------------------------
% LTXE-SLIDE-START
% LTXE-SLIDE-UID:		d4196e48-7d10b852-ab693761-58408451
% LTXE-SLIDE-ORIGIN:	xxxxxxxx-xxxxxxxx-xxxxxxxx-xxxxxxxx
% LTXE-SLIDE-TITLE:		API um gecachte TCA Änderungen zuzufügen
% LTXE-SLIDE-REFERENCE:	https://forge.typo3.org/issues/57942
% ------------------------------------------------------------------------------

\begin{frame}[fragile]
	\frametitle{Änderungen im System}
	\framesubtitle{API um gecachte TCA Änderungen zuzufügen}

	\begin{itemize}
		\item PHP Dateien in \texttt{extkey/Configuration/TCA/Overrides/}
			werden ausgeführt, direkt nachdem der TCA-Cache aufgebaut wurde

		\item Die Dateien dürfen nur Code enthalten, der das TCA manipuliert,\newline
			wie beispielsweise: \texttt{addTCAColumns} oder \texttt{addToAllTCATypes}

		\item Sobald Extensions dieses Features verwenden, ist mit einem deutlichen
			Geschwindigkeitsschub von Backend Requests zu rechnen

	\end{itemize}

\end{frame} 

% ------------------------------------------------------------------------------
% LTXE-SLIDE-START
% LTXE-SLIDE-UID:		89fbd845-3952a157-98d97fc4-4244a2d4
% LTXE-SLIDE-ORIGIN:	xxxxxxxx-xxxxxxxx-xxxxxxxx-xxxxxxxx
% LTXE-SLIDE-TITLE:		Nur-lesender Zugriff auf File Mounts
% LTXE-SLIDE-REFERENCE:	https://forge.typo3.org/issues/49391
% ------------------------------------------------------------------------------

\begin{frame}[fragile]
	\frametitle{In-Depth Changes}
	\framesubtitle{Nur-lesender Zugriff auf File Mounts}

	\begin{itemize}

		\item File Mounts können (wieder) mit nur-lesenden Zugriff konfiguriert werden ("read-only")
		\item Jenes war bereits in TYPO3 CMS 4.x möglich, wurde aber in\newline
			TYPO3 CMS 6.x entfernt
		\item Zum Beispiel: Ordner "test" vom storage UID 3 als Mount mit nur lesenden Zugriff in der Dateiliste und dem Element Browser:\newline

			\smaller\texttt{options.folderTree.altElementBrowserMountPoints = 3:/test}\normalsize\newline

			Falls kein Storage angegeben wird, geht man davon aus, dass sich der Ordner im Default Storage befindet.
	\end{itemize}

\end{frame}

% ------------------------------------------------------------------------------
% LTXE-SLIDE-START
% LTXE-SLIDE-UID:		51a89ae1-7a60241c-c07a83b0-664ebb67
% LTXE-SLIDE-ORIGIN:	xxxxxxxx-xxxxxxxx-xxxxxxxx-xxxxxxxx
% LTXE-SLIDE-TITLE:		Sonstiges
% LTXE-SLIDE-REFERENCE:	https://forge.typo3.org/issues/61185
% LTXE-SLIDE-REFERENCE:	https://forge.typo3.org/issues/59830
% ------------------------------------------------------------------------------

\begin{frame}[fragile]
	\frametitle{Änderungen im System}
	\framesubtitle{Sonstiges}

	\begin{itemize}
		\item jQuery wurde von Version 1.11.0 auf Version 1.11.1 aktualisiert
		\item Datatables wurde von Version 1.9.4 zu Version 1.10.2 aktualisiert
		\item Einige alte, nicht mehr benutzte Variablen wurden von \texttt{EM\_CONF} entfernt
		\item Extension Icons können nun auch im SVG Bildformat vorliegen (\texttt{ext\_icon.svg})
		\item Das Senden eines unbekannten eID Identifikators im Request resultiert nun in eine Exception
	\end{itemize}

\end{frame}

% ------------------------------------------------------------------------------
