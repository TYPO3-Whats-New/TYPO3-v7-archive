% ------------------------------------------------------------------------------
% TYPO3 CMS 7.0 - What's New - Chapter "In-Depth Changes" (English Version)
%
% @author	Michael Schams <schams.net>
% @license	Creative Commons BY-NC-SA 3.0
% @link		http://typo3.org/download/release-notes/whats-new/
% @language	English
% ------------------------------------------------------------------------------
% LTXE-CHAPTER-UID:		3b45f546-46c70de8-9e7f1eb5-28ba10b0
% LTXE-CHAPTER-NAME:	In-Depth Changes
% ------------------------------------------------------------------------------

\section{Modifiche rilevanti}
\begin{frame}[fragile]
	\frametitle{Modifiche rilevanti}

	\begin{center}\huge{Capitolo 3:}\end{center}
	\begin{center}\huge{\color{typo3darkgrey}\textbf{Modifiche rilevanti}}\end{center}

\end{frame}

% ------------------------------------------------------------------------------
% LTXE-SLIDE-START
% LTXE-SLIDE-UID:		db1ccce9-5220fde7-7a299509-2e45b8d7
% LTXE-SLIDE-ORIGIN:	b47dd2a2-952b3dd2-b9ceecce-ab8630bc English
% LTXE-SLIDE-TITLE:		Integration of jQuery UI version 1.11.2
% LTXE-SLIDE-REFERENCE:	https://forge.typo3.org/issues/62916
% ------------------------------------------------------------------------------

\begin{frame}[fragile]
	\frametitle{Modifiche rilevanti}
	\framesubtitle{Integrazione di jQuery UI versione 1.11.2}

	\begin{itemize}
		\item jQuery UI 1.11 supporta AMD
			(\href{http://en.wikipedia.org/wiki/Asynchronous_module_definition}{Asynchronous Module Definition}),
			che carica i file JavaScript solo quando essi sono necessari (miglioramento di performance)

    	\item jQuery UI 1.11 sostituisce jQuery UI 1.10 + Scriptaculous in TYPO3 CMS 7.0

    	\item Sono inclusi solo i componenti di base e per le interazioni, che sono necessari
    		per sostituire ExtJS e Scriptaculous

    	\item I widgets non sono inclusi (ma quelli di Twitter Bootstrap sono utilizzati,
    		ad esempio: DatePicker, Spinner, Dialog, Buttons, Tabs, Tooltip)

	\end{itemize}

\end{frame}

% ------------------------------------------------------------------------------
% LTXE-SLIDE-START
% LTXE-SLIDE-UID:		7d44c7ad-bc8960ef-e3f5f65e-9974917b
% LTXE-SLIDE-ORIGIN:	66a996e1-bcfe7442-d44f59c8-d3495d4e English
% LTXE-SLIDE-TITLE:		Registry for File Rendering Classes
% LTXE-SLIDE-REFERENCE:	https://forge.typo3.org/issues/61800
% ------------------------------------------------------------------------------

\begin{frame}[fragile]
	\frametitle{Modifiche rilevanti}
	\framesubtitle{Registro per classi di renderizzazione file}

	\lstset{
		basicstyle=\tiny\ttfamily
	}

	\begin{itemize}
		\item Per essere in grado di renderizzare tutti i tipi di file multimediali,
			è stato implementato un registro per i file.\newline
			Questo funziona come segue (es. Video, MPEG, AVI, WAV, ecc.):

		\begin{lstlisting}
			<?php
			namespace ...;

			class NameTagRenderer implements FileRendererInterface {
			  protected $possibleMimeTypes = array('audio/mpeg', 'audio/wav', ...);
			  public function getPriority() {
			    return 1; // priority: the higher, the more important (max: 100)
			  }
			  public function canRender(FileInterface $file) {
			    return in_array($file->getMimeType(), $this->possibleMimeTypes, TRUE);
			  }
			  public function render(FileInterface $file, $width, $height, array $options = array(), $usedPathsRelativeToCurrentScript = FALSE) {
			    ...

			    return 'HTML code';
			  }
			}
		\end{lstlisting}

	\end{itemize}

\end{frame}

% ------------------------------------------------------------------------------
% LTXE-SLIDE-START
% LTXE-SLIDE-UID:		780bc480-1f2ec854-df5a6957-06c002bf
% LTXE-SLIDE-ORIGIN:	2c97df27-58f83c0c-2a7532ce-73ba3bd5 English
% LTXE-SLIDE-TITLE:		TCA: Validate Email Addresses
% LTXE-SLIDE-REFERENCE:	https://forge.typo3.org/issues/62147
% ------------------------------------------------------------------------------

\begin{frame}[fragile]
	\frametitle{Modifiche rilevanti}
	\framesubtitle{TCA: Validatore di indirizzi email}

	\lstset{
		basicstyle=\tiny\ttfamily
	}

	\begin{itemize}

		\item Nuova funzione per verificare se il valore inserito è un indirizzo email valido
		\item Se non lo è, appare un messaggio "Flash"
		\item Esempio:

		\begin{lstlisting}
			'emailaddress' => array(
			  'exclude' => 1,
			  'label' => 'LLL:EXT:myextension/Resources/Private/Language/locallang_db.xlf:tx_myextension
		 	  'config' => array(
			      'type' => 'input',
			      'size' => 30,
			      'eval' => 'email,trim'
			  ),
			)
		\end{lstlisting}

	\end{itemize}

\end{frame}

% ------------------------------------------------------------------------------
% LTXE-SLIDE-START
% LTXE-SLIDE-UID:		17e86067-d2424e87-3a46fb2c-f7bf6ef3
% LTXE-SLIDE-ORIGIN:	08cf25ed-0cba243f-e0738567-598b1534 English
% LTXE-SLIDE-TITLE:		AbstractCondition For Custom TypoScript Conditions
% LTXE-SLIDE-REFERENCE:	https://forge.typo3.org/issues/61489
% ------------------------------------------------------------------------------

\begin{frame}[fragile]
	\frametitle{Modifiche rilevanti}
	\framesubtitle{AbstractCondition per condizioni TypoScript personalizzate}

	\lstset{
		basicstyle=\tiny\ttfamily
	}

	\begin{itemize}
		\item Condizioni TypoScript personalizzate possono derivare da AbstractCondition

		\begin{lstlisting}
			class TestCondition
			  extends \TYPO3\CMS\Core\Configuration\TypoScript\ConditionMatching\AbstractCondition {

			  public function matchCondition(array $conditionParameters) {
 			    if ($conditionParameters[0] === '= 7' && $conditionParameters[1] === '!= 6') {
			      throw new TestConditionException('All Ok', 1411581139);
			    }
			  }
			}
		\end{lstlisting}

		\item Il codice TypoScript appropriato come segue:

		\begin{lstlisting}
			[Vendor\Package\TestCondition]
			[Vendor\Package\TestCondition = 7]
			[Vendor\Package\TestCondition = 7, != 6]
		\end{lstlisting}

		\item Gli operatori, che dovrebbero essere disponibili, sono definiti nella classe

	\end{itemize}

\end{frame}

% ------------------------------------------------------------------------------
% LTXE-SLIDE-START
% LTXE-SLIDE-UID:		046bbb40-53cf50f1-33a3dc38-22c2e3f1
% LTXE-SLIDE-ORIGIN:	c137d1cf-7ce68dbe-8149deb5-19fe62bb English
% LTXE-SLIDE-TITLE:		Signal for IconUtility HTML tag manipulation
% LTXE-SLIDE-REFERENCE:	https://forge.typo3.org/issues/61289
% ------------------------------------------------------------------------------

\begin{frame}[fragile]
	\frametitle{Modifiche rilevanti}
	\framesubtitle{Segnale per IconUtility HTML Tag Manipulation}

	\lstset{
		basicstyle=\tiny\ttfamily
	}

	\begin{itemize}

		\item Nuovo segnale per modificare il tag HTML per le icone "IconUtility sprite":

		\begin{lstlisting}
			dispatch(
			  'TYPO3\\CMS\\Backend\\Utility\\IconUtility',
			  'buildSpriteHtmlIconTag',
			  array($tagAttributes, $innerHtml, $tagName)
			);
		\end{lstlisting}

		\item Metodi di chiamata:

		\begin{lstlisting}
			TYPO3\CMS\Backend\Utility\IconUtility\buildSpriteHtmlIconTag
		\end{lstlisting}

	\end{itemize}

\end{frame}

% ------------------------------------------------------------------------------
% LTXE-SLIDE-START
% LTXE-SLIDE-UID:		54e5b5e7-f23d44f7-8f6deaf1-883cd52e
% LTXE-SLIDE-ORIGIN:	6713b880-9410d0a5-ef04d276-7bb989ba English
% LTXE-SLIDE-TITLE:		Add Signal Slots to SoftReferenceIndex
% ------------------------------------------------------------------------------

\begin{frame}[fragile]
	\frametitle{Modifiche rilevanti}
	\framesubtitle{Aggiunti slot di segnali a SoftReferenceIndex}

	\lstset{
		basicstyle=\tiny\ttfamily
	}

	\begin{itemize}
		\item
			\smaller
				Two new signal slot dispatch calls in SoftReferenceIndex:
			\normalsize

		\begin{lstlisting}
			protected function emitGetTypoLinkParts(
			  $linkHandlerFound, $finalTagParts, $linkHandlerKeyword, $linkHandlerValue) {
			  return $this->getSignalSlotDispatcher()->dispatch(
			    get_class($this),
			    'getTypoLinkParts',
			    array($linkHandlerFound, $finalTagParts, $linkHandlerKeyword, $linkHandlerValue)
			  );
			}
			protected function emitSetTypoLinkPartsElement(
			  $linkHandlerFound, $tLP, $content, $elements, $idx, $tokenID) {
			  return $this->getSignalSlotDispatcher()->dispatch(
			    get_class($this),
			    'setTypoLinkPartsElement',
			    array($linkHandlerFound, $tLP, $content, $elements, $idx, $tokenID, $this)
			  );
			}
		\end{lstlisting}

		\item
			\smaller
				Chiamata in:
			\normalsize

			\begin{lstlisting}
				TYPO3\CMS\Core\Database\SoftReferenceIndex->findRef_typolink
				TYPO3\CMS\Core\Database\SoftReferenceIndex->getTypoLinkParts
			\end{lstlisting}

	\end{itemize}

\end{frame}

% ------------------------------------------------------------------------------
% LTXE-SLIDE-START
% LTXE-SLIDE-UID:		e22ae26d-81e375d4-bdb6e26b-1d6f0f82
% LTXE-SLIDE-ORIGIN:	d743b8cf-4924c9ca-7e1d330c-3b101461 English
% LTXE-SLIDE-TITLE:		afterPersistObjetct Signal Slot
% LTXE-SLIDE-REFERENCE:	https://forge.typo3.org/issues/59986
% ------------------------------------------------------------------------------

\begin{frame}[fragile]
	\frametitle{Modifiche rilevanti}
	\framesubtitle{afterPersistObjetct Signal Slot}

	\lstset{
		basicstyle=\tiny\ttfamily
	}

	\begin{itemize}

		\item Nuovo afterPersistObject signal slot emits for the aggregate root after persisting all other objects

			\begin{lstlisting}
				protected function emitAfterPersistObjectSignal(DomainObjectInterface $object) {
				  $this->signalSlotDispatcher->dispatch(__CLASS__, 'afterPersistObject', array($object));
				}
			\end{lstlisting}

		\item Chiamata in:

			\begin{lstlisting}
				TYPO3\CMS\Extbase\Persistence\Generic\Backend->persistObject
			\end{lstlisting}

		\item The same signal is emitted in the persistObject method in the AbstractBackend class in Flow

	\end{itemize}

\end{frame}                                                 

% ------------------------------------------------------------------------------
% LTXE-SLIDE-START
% LTXE-SLIDE-UID:		a098bcfb-7c4df384-bbb923be-2d287ab0
% LTXE-SLIDE-ORIGIN:	17ea6530-0ec1ad3e-8e7d9d24-1d5b64aa English
% LTXE-SLIDE-TITLE:		Signal in loadBaseTca
% LTXE-SLIDE-REFERENCE:	https://forge.typo3.org/issues/57863
% ------------------------------------------------------------------------------

\begin{frame}[fragile]
	\frametitle{Modifiche rilevanti}
	\framesubtitle{Signal in loadBaseTca}

	\lstset{
		basicstyle=\tiny\ttfamily
	}

	\begin{itemize}
		\item Per migliorare le performance nel contesto di backend,
			l'intero TCA può essere messo in cache (non solo alcune parti di esso)

		\begin{lstlisting}
			protected function emitTcaIsBeingBuiltSignal(array $tca) {
			  list($tca) = static::getSignalSlotDispatcher()->dispatch(
			    __CLASS__,
			    'tcaIsBeingBuilt',
			    array($tca)
			  );
			  $GLOBALS['TCA'] = $tca;
			}
		\end{lstlisting}

		\item Chiamato in:

			\begin{lstlisting}
				TYPO3\CMS\Core\Utility\ExtensionManagementUtility\Backend->buildBaseTcaFromSingleFiles
			\end{lstlisting}

	\end{itemize}

\end{frame}  

% ------------------------------------------------------------------------------
% LTXE-SLIDE-START
% LTXE-SLIDE-UID:		753476f6-ed105ebd-9e6257fb-37126ba4
% LTXE-SLIDE-ORIGIN:	2fd6c25c-5f7952da-1406b74c-e0ff5e67 English
% LTXE-SLIDE-TITLE:		API to Add Cached TCA Changes
% LTXE-SLIDE-REFERENCE:	https://forge.typo3.org/issues/57942
% ------------------------------------------------------------------------------

\begin{frame}[fragile]
	\frametitle{Modifiche rilevanti}
	\framesubtitle{API per aggiungere modifiche TCA in cache}

	\begin{itemize}
		\item I file PHP in \texttt{extkey/Configuration/TCA/Overrides/}
			sono eseguiti direttamente dopo che la cache TCA è stata creata

		\item Questi file devono contenere solo codice che interviene sul TCA,\newline
			ad esempio: \texttt{addTCAColumns} o \texttt{addToAllTCATypes}

		\item Questa caratteristica dà al backend un miglioramento delle performance una volta che le estensioni sono abilitate ad usare questi file

	\end{itemize}

\end{frame} 

% ------------------------------------------------------------------------------
% LTXE-SLIDE-START
% LTXE-SLIDE-UID:		2b41b861-58caaeb7-787ea44d-d07c37db
% LTXE-SLIDE-ORIGIN:	89fbd845-3952a157-98d97fc4-4244a2d4 English
% LTXE-SLIDE-TITLE:		Read-only File Mounts
% LTXE-SLIDE-REFERENCE:	https://forge.typo3.org/issues/49391
% ------------------------------------------------------------------------------

\begin{frame}[fragile]
	\frametitle{Modifiche rilevanti}
	\framesubtitle{File Mounts in sola lettura}

	\begin{itemize}

		\item I File mounts possono essere configurati in sola lettura (nuovamente)
		\item Questo era già possibile in TYPO3 CMS 4.x, ma silenziosamente tolto in 6.x
		\item Esempio: aggiungi una directory "test" in storage UID 3 come elemento in sola lettura nella lista file e browser di elementi.\newline

			\smaller\texttt{options.folderTree.altElementBrowserMountPoints = 3:/test}\normalsize\newline

			Se nessun storage è configurato, si assume che la directory è nello storage di default.
	\end{itemize}

\end{frame}

% ------------------------------------------------------------------------------
% LTXE-SLIDE-START
% LTXE-SLIDE-UID:		414a0fc0-7891f615-fbddeb43-a9cac2d2
% LTXE-SLIDE-ORIGIN:	b964bc6b-27cd32e7-bda8cc3e-8bea33ca English
% LTXE-SLIDE-TITLE:		Miscellaneous
% LTXE-SLIDE-REFERENCE:	https://forge.typo3.org/issues/62993
% LTXE-SLIDE-REFERENCE:	https://forge.typo3.org/issues/61185
% LTXE-SLIDE-REFERENCE:	https://forge.typo3.org/issues/58365
% LTXE-SLIDE-REFERENCE:	https://forge.typo3.org/issues/59830
% LTXE-SLIDE-REFERENCE:	https://forge.typo3.org/issues/62857
% ------------------------------------------------------------------------------

\begin{frame}[fragile]
	\frametitle{Modifiche rilevanti}
	\framesubtitle{Varie}

	\begin{itemize}
		\item jQuery è stato aggiornato dalla versione 1.11.0 alla versione 1.11.1
		\item Datatables è stato aggiornato dalla versione 1.9.4 alla versione 1.10.2
		\item Le vecchie e inutilizzate variabili sono state rimosse da \texttt{EM\_CONF}
		\item Le icone delle estensioni sono nel formato SVG (\texttt{ext\_icon.svg})
		\item il passaggio di identificatori eID errati risulta ora in eccezioni
	\end{itemize}

\end{frame}

% ------------------------------------------------------------------------------
