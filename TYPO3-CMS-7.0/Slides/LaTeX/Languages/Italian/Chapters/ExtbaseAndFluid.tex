% ------------------------------------------------------------------------------
% TYPO3 CMS 7.0 - What's New - Chapter "Extbase & Fluid" (English Version)
%
% @author	Michael Schams <schams.net>
% @license	Creative Commons BY-NC-SA 3.0
% @link		http://typo3.org/download/release-notes/whats-new/
% @language	English
% ------------------------------------------------------------------------------
% LTXE-CHAPTER-UID:		e8f1e659-b27f81df-4152f77a-bef3ad99
% LTXE-CHAPTER-NAME:	Extbase & Fluid
% ------------------------------------------------------------------------------

\section{Extbase \& Fluid}
\begin{frame}[fragile]
	\frametitle{Extbase \& Fluid}

	\begin{center}\huge{Capitolo 4:}\end{center}
	\begin{center}\huge{\color{typo3darkgrey}\textbf{Extbase \& Fluid}}\end{center}

\end{frame}

% ------------------------------------------------------------------------------
% LTXE-SLIDE-START
% LTXE-SLIDE-UID:		09834c84-61564c06-7a01cd80-4b84c6ca
% LTXE-SLIDE-ORIGIN:	aba42e5a-136183fe-58e36ebd-bfab3106 English
% LTXE-SLIDE-TITLE:		Template Path Fallback
% LTXE-SLIDE-REFERENCE:	http://forge.typo3.org/issues/61361
% LTXE-SLIDE-REFERENCE:	http://forge.typo3.org/issues/39868
% ------------------------------------------------------------------------------

\begin{frame}[fragile]
	\frametitle{Extbase \& Fluid}
	\framesubtitle{Template Path Fallback}

	\lstset{
		basicstyle=\tiny\ttfamily
	}

	\begin{itemize}
		\item \texttt{Fluid Standalone View} così come l'oggetto TypoScript \texttt{FLUIDTEMPLATE} supporta ora il template fallback paths

			\begin{lstlisting}
				page.10 = FLUIDTEMPLATE
				page.10.file = EXT:myextension/Resources/Private/Templates/Main.html
				page.10.partialRootPaths {
				  10 = EXT:myextension/Resources/Private/Partials
				  20 = EXT:fallback/Resources/Private/Partials
				}
			\end{lstlisting}

		\item Se le nuove e le vecchie opzioni sono utilizzate (ad esempio \texttt{partialRootPaths} e \texttt{partialRootPath}),
			il percorso indicato dall'opzione è nella prima posizione (index = 0)

	\end{itemize}

\end{frame}

% ------------------------------------------------------------------------------
% LTXE-SLIDE-START
% LTXE-SLIDE-UID:		b4d5b68e-e585d214-d9be41cb-2ea3231d
% LTXE-SLIDE-ORIGIN:	8deb15a5-bff46e9f-ad676c73-427ca720 English
% LTXE-SLIDE-TITLE:		Typolink ViewHelper
% LTXE-SLIDE-REFERENCE:	https://forge.typo3.org/issues/59396
% ------------------------------------------------------------------------------

\begin{frame}[fragile]
	\frametitle{Extbase \& Fluid}
	\framesubtitle{Typolink ViewHelper}

	\lstset{
		basicstyle=\tiny\ttfamily
	}

	\begin{itemize}
		\item Un nuovo Typolink ViewHelper può analizzare le stringhe \texttt{typolink} create dal wizard dei link e da RTE

			\begin{lstlisting}
				<f:link.typolink parameter="{link}" target="_blank" class="ico-class" title="some title" additionalAttributes="{type:'button'}">
			\end{lstlisting}

			\texttt{link} può contenere:
			\begin{lstlisting}
				42 _blank - "Questo il link del titolo" &foo=bar
			\end{lstlisting}

			Output:
			\begin{lstlisting}
				<a href="index.php?id=42&foo=bar" title="Questo il link del titolo" target="_blank" class="ico-class" type="button">
			\end{lstlisting}

			Note: solo \texttt{parameter} è obbligatorio, gli altri sono opzionali

	\end{itemize}

\end{frame}

% ------------------------------------------------------------------------------
% LTXE-SLIDE-START
% LTXE-SLIDE-UID:		a939dc54-6802f59d-46ce9706-2c1c0740
% LTXE-SLIDE-ORIGIN:	143bb339-be7e8e0b-e354de35-3591d05a English
% LTXE-SLIDE-TITLE:		Generic data-* Attribute
% LTXE-SLIDE-REFERENCE:	https://forge.typo3.org/issues/61351
% ------------------------------------------------------------------------------


\begin{frame}[fragile]
	\frametitle{Extbase \& Fluid}
	\framesubtitle{Attributo generico data-*}

%	\lstset{
%		basicstyle=\tiny\ttfamily
%	}

	\begin{itemize}
		\item Tutti i ViewHelpers, che espongono tag HTML, supportano l'attributo HTML5 \texttt{data-*}
		\item Un array passato come \texttt{data} è elaborato e le coppie key/value costruiscono gli attributi:
			\texttt{data-\begingroup\color{typo3orange}key\endgroup
				="\begingroup\color{typo3orange}value\endgroup
				"}\newline

			Esempio:
			\begin{lstlisting}
				<f:form.textfield data="{foo: 'bar', baz: 'foos'}" />
			\end{lstlisting}

			Output:
			\begin{lstlisting}
				<input data-foo="bar" data-baz="foos" ... />
			\end{lstlisting}
		
	\end{itemize}

\end{frame}

% ------------------------------------------------------------------------------
% LTXE-SLIDE-START
% LTXE-SLIDE-UID:		a117667f-751a8144-32e1e534-150fa36f
% LTXE-SLIDE-ORIGIN:	c5e5ddfc-f68f277a-3e0bb2b1-623f3e49 English
% LTXE-SLIDE-TITLE:		Class Tag Values Via Reflection
% LTXE-SLIDE-REFERENCE:	https://forge.typo3.org/issues/60822
% ------------------------------------------------------------------------------

\begin{frame}[fragile]
	\frametitle{Extbase \& Fluid}
	\framesubtitle{Class Tag Values Via Reflection}

	\lstset{
		basicstyle=\tiny\ttfamily
	}

	\begin{itemize}
		\item Extbase Reflection Service può restituire i tag e le annotazioni aggiunte alla classe\newline

			Example:
			\begin{lstlisting}
				/**
				 * @SomeClassAnnotation A value
				 */
				class MyClass {
				}
			\end{lstlisting}

			L'annotazione può essere chiamata:
			\begin{lstlisting}
				$service = new \TYPO3\CMS\Extbase\Reflection\ReflectionService();

				// Returns all tags and their values the specified class is tagged with
				$classValues = $service->getClassTagsValues('MyClass');

				// Returns the values of the specified class tag
				$classValue = $service->getClassTagValue('MyClass', 'SomeClassAnnotation');
			\end{lstlisting}

	\end{itemize}

\end{frame}

% ------------------------------------------------------------------------------
