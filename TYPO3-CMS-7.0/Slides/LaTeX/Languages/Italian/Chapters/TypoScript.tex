% ------------------------------------------------------------------------------
% TYPO3 CMS 7.0 - What's New - Chapter "TypoScript" (English Version)
%
% @author	Michael Schams <schams.net>
% @license	Creative Commons BY-NC-SA 3.0
% @link		http://typo3.org/download/release-notes/whats-new/
% @language	English
% ------------------------------------------------------------------------------
% LTXE-CHAPTER-UID:		6d166a6c-4c3fdbb3-a19cfd06-2173720c
% LTXE-CHAPTER-NAME:	TypoScript
% ------------------------------------------------------------------------------

\section{TSconfig \& TypoScript}
\begin{frame}[fragile]
	\frametitle{TSconfig \& TypoScript}

	\begin{center}\huge{Capitolo 2:}\end{center}
	\begin{center}\huge{\color{typo3darkgrey}\textbf{TSconfig \& TypoScript}}\end{center}

\end{frame}

% ------------------------------------------------------------------------------
% LTXE-SLIDE-START
% LTXE-SLIDE-UID:		d49a34a9-434319d6-b145c7f8-fceb2506
% LTXE-SLIDE-ORIGIN:	c65cc12d-2f8c9d53-37d6577f-8a2f8357 English
% LTXE-SLIDE-TITLE:		TSconfig Available to Link Checkers
% LTXE-SLIDE-REFERENCE:	https://forge.typo3.org/issues/54518
% ------------------------------------------------------------------------------

\begin{frame}[fragile]
	\frametitle{TSconfig \& TypoScript}
	\framesubtitle{Disponibile in TSConfig un validatore di link}

	\begin{itemize}
		\item La configurazione di TSconfig viene letta

			\begin{itemize}
				\item sia dal backend (se è utilizzato Linkvalidator) 
				\item o dalla configurazione dello scheduler dei task 
			\end{itemize}

		\item Esempio: TSconfig, che può essere letto da Linkchecker:

			\lstinline!mod.linkvalidator.mychecker.myvar = 1!

		\item TSconfig è ora disponibile come \texttt{\$this->tsConfig}
	\end{itemize}

\end{frame}

% ------------------------------------------------------------------------------
% LTXE-SLIDE-START
% LTXE-SLIDE-UID:		5c608fb0-af5a8638-82cc332d-2bcdadea
% LTXE-SLIDE-ORIGIN:	1d38328d-a7245bbe-1b380ab3-88ea4885 English
% LTXE-SLIDE-TITLE:		Linkcheck: Report Deleted Records
% LTXE-SLIDE-REFERENCE:	https://forge.typo3.org/issues/54519
% ------------------------------------------------------------------------------

\begin{frame}[fragile]
	\frametitle{TSconfig \& TypoScript}
	\framesubtitle{Linkcheck: Rapporto record eliminati}

	\begin{itemize}
		\item In TYPO3 CMS < 7.0, linkhandler avvertiva solamente di link non esistenti o di record cancellati
		\item Da TYPO3 CMS >=  7.0, la seguente impostazione di TSconfig abilita un avviso anche se i link puntano a record disabilitati:

			\lstinline!mod.linkvalidator.linkhandler.reportHiddenRecords = 1!

	\end{itemize}

\end{frame}

% ------------------------------------------------------------------------------
% LTXE-SLIDE-START
% LTXE-SLIDE-UID:		2346ce77-d5f07a6a-a81c0449-200f372c
% LTXE-SLIDE-ORIGIN:	a27d2c0a-410cbfc8-50f176e2-027ce416 English
% LTXE-SLIDE-TITLE:		RTE: Multiple CSS Classes Per Style
% LTXE-SLIDE-REFERENCE:	https://forge.typo3.org/issues/51905
% ------------------------------------------------------------------------------

\begin{frame}[fragile]
	\frametitle{TSconfig \& TypoScript}
	\framesubtitle{RTE: Classi CSS multiple per stile}

	\begin{itemize}
		\item I framework moderni come Twitter Bootstrap richiedono classi CSS multiple per i tag HTML\newline
			\small Ad esempio: \texttt{<a class="btn btn-danger">Alert</a>}\normalsize
		\item Classi CSS multiple sono ora supportate, questo significa che gli editor possono selezionare un solo stile

			\begin{lstlisting}
				RTE.classes.[ *classname* ] {
				  .requires = elenco delle classi CSS
				}
			\end{lstlisting}

	\end{itemize}

\end{frame}

% ------------------------------------------------------------------------------
% LTXE-SLIDE-START
% LTXE-SLIDE-UID:		58c9a6fd-5d53b3af-feab9ee4-c2220f2a
% LTXE-SLIDE-ORIGIN:	e11a6c09-1eb3eaf7-6cebbee9-febef6ba English
% LTXE-SLIDE-TITLE:		RTE: Configure CSS Class As Not-Selectable
% LTXE-SLIDE-REFERENCE:	https://forge.typo3.org/issues/58122
% LTXE-SLIDE-REFERENCE:	https://forge.typo3.org/issues/51905
% ------------------------------------------------------------------------------

\begin{frame}[fragile]
	\frametitle{TSconfig \& TypoScript}
	\framesubtitle{RTE: Configurare classi CSS Class come "non selezionabili"}

	\begin{itemize}
		\item E' possibile configurare le classi CSS come "non-selezionabili"

			\begin{lstlisting}
				// valore "1" indica una classe selezionabile
				// valore "0" indica una classe non selezionabile
				RTE.classes.[ *classname* ] {
				  .selectable = 1
				}
			\end{lstlisting}

	\end{itemize}

\end{frame}

% ------------------------------------------------------------------------------
% LTXE-SLIDE-START
% LTXE-SLIDE-UID:		ce28e02b-d35e7322-8ef65017-7b3cfc07
% LTXE-SLIDE-ORIGIN:	e607c36f-e1be31da-05ace0c5-991e6e42 English
% LTXE-SLIDE-TITLE:		RTE: Include Multiple CSS Files
% LTXE-SLIDE-REFERENCE:	https://forge.typo3.org/issues/50039
% ------------------------------------------------------------------------------

\begin{frame}[fragile]
	\frametitle{TSconfig \& TypoScript}
	\framesubtitle{RTE: Inclusione multipla di file CSS}

	\begin{itemize}
		\item E' possibile includere più file CSS

			\begin{lstlisting}
				RTE.default.contentCSS {
				  file1 = fileadmin/rte_stylesheet1.css
				  file2 = fileadmin/rte_stylesheet2.css
				}
			\end{lstlisting}

		\item Senza definizione di file di stile CSS, il defalut è:\newline
			\texttt{typo3/sysext/rtehtmlarea/res/contentcss/default.css}

	\end{itemize}

\end{frame}

% ------------------------------------------------------------------------------
% LTXE-SLIDE-START
% LTXE-SLIDE-UID:		1cb5fe22-f494fda4-27f325d4-a6421bbc
% LTXE-SLIDE-ORIGIN:	c9d0ad7d-2c5aee11-79878250-9e8ce110 English
% LTXE-SLIDE-TITLE:		Exception Handling When cObjects Are Rendered (1)
% LTXE-SLIDE-REFERENCE:	https://forge.typo3.org/issues/47919
% ------------------------------------------------------------------------------

\begin{frame}[fragile]
	\frametitle{TSconfig \& TypoScript}
	\framesubtitle{Gestione delle eccezioni quando un cObjects è renderizzato (1)}

	\begin{itemize}
		\item In TYPO3 CMS < 7.0, se avveniva un errore durante la renderizzazione di un oggetto di contenuto (es. \texttt{USER}), l'errore bloccava l'intero frontend
		\item Da TYPO3 CMS >= 7.0, è stata implementata una gestione delle eccezioni, la quale permette la visualizzazione di un messaggio al posto del cObject errato
	\end{itemize}

%	*TODO* it would be so nice, if someone could create a screenshot of this scenario!
%
%	\begin{figure}\vspace*{-0.3cm}
%		\includegraphics[width=0.90\linewidth]{TypoScript/cObjectExceptionHandling.png}
%	\end{figure}

\end{frame}

% ------------------------------------------------------------------------------
% LTXE-SLIDE-START
% LTXE-SLIDE-UID:		43d9449e-d9501a50-a05e37c9-65770eeb
% LTXE-SLIDE-ORIGIN:	1132ffa6-a06b7448-0b59e149-a28c2d43 English
% LTXE-SLIDE-TITLE:		Exception Handling When cObjects Are Rendered (2)
% LTXE-SLIDE-REFERENCE:	https://forge.typo3.org/issues/47919
% ------------------------------------------------------------------------------

\begin{frame}[fragile]
	\frametitle{TSconfig \& TypoScript}
	\framesubtitle{Gestione delle eccezioni quando un cObjects è renderizzato (2)}
	
	\lstset{
		basicstyle=\tiny\ttfamily
	}
	
	\begin{lstlisting}
		# gestore predefinito di eccezioni (attivato nel contesto di "production")
		config.contentObjectExceptionHandler = 1

		# configurazione di una classe per la gestione delle eccezioni
		config.contentObjectExceptionHandler =
		  TYPO3\CMS\Frontend\ContentObject\Exception\ProductionExceptionHandler

		# personalizzazione del messaggio di errore (visualizza il codice di errore casuale)
		config.contentObjectExceptionHandler.errorMessage = Oops an error occurred. Code: %s

		# configurazione dei codici di eccezione, che non saranno trattati
		tt_content.login.20.exceptionHandler.ignoreCodes.10 = 1414512813

		# disattivazione della gestione delle eccezioni per uno specifico plugin o oggetto di contenuti
		tt_content.login.20.exceptionHandler = 0

		# ignoreCodes e errorMessage possono essere configurati in modo globale...
		config.contentObjectExceptionHandler.errorMessage = Oops an error occurred. Code: %s
		config.contentObjectExceptionHandler.ignoreCodes.10 = 1414512813

		# ...o in modo locale per singoli oggetti di contenuti
		tt_content.login.20.exceptionHandler.errorMessage = Oops an error occurred. Code: %s
		tt_content.login.20.exceptionHandler.ignoreCodes.10 = 1414512813
	\end{lstlisting}

\end{frame}

% ------------------------------------------------------------------------------
