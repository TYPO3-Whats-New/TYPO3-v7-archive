% ------------------------------------------------------------------------------
% TYPO3 CMS 7.0 - What's New - Chapter "In-Depth Changes" (English Version)
%
% @author	Michael Schams <schams.net>
% @license	Creative Commons BY-NC-SA 3.0
% @link		http://typo3.org/download/release-notes/whats-new/
% @language	English
% ------------------------------------------------------------------------------
% LTXE-CHAPTER-UID:		212ecbb8-5986dfff-21b3e733-edfb8d99
% LTXE-CHAPTER-NAME:	In-Depth Changes
% ------------------------------------------------------------------------------

\section{Systeemwijzigingen}
\begin{frame}[fragile]
	\frametitle{Systeemwijzigingen}

	\begin{center}\huge{Hoofdstuk 3:}\end{center}
	\begin{center}\huge{\color{typo3darkgrey}\textbf{Systeemwijzigingen}}\end{center}

\end{frame}

% ------------------------------------------------------------------------------
% LTXE-SLIDE-START
% LTXE-SLIDE-UID:		d3d9a8f5-982489ce-1fdd24b0-d49e1b32
% LTXE-SLIDE-ORIGIN:	b47dd2a2-952b3dd2-b9ceecce-ab8630bc English
% LTXE-SLIDE-TITLE:		Integration of jQuery UI version 1.11.2
% LTXE-SLIDE-REFERENCE:	https://forge.typo3.org/issues/62916
% ------------------------------------------------------------------------------

\begin{frame}[fragile]
	\frametitle{Systeemwijzigingen}
	\framesubtitle{Integratie van jQuery UI versie 1.11.2}

	\begin{itemize}
		\item jQuery UI 1.11 ondersteunt AMD
			(\href{http://en.wikipedia.org/wiki/Asynchronous_module_definition}{Asynchronous Module Definition}),
			die JavaScript-bestanden alleen inlaadt als ze nodig zijn (extra performance)

    	\item jQuery UI 1.11 vervangt jQuery UI 1.10 + Scriptaculous in TYPO3 CMS 7.0

    	\item Alleen de core- en interactiecomponenten die nodig zijn
    		om ExtJS en Scriptaculous te vervangen zijn opgenomen 

    	\item Widgets zijn niet opgenomen (die van Twitter Bootstrap worden gebruikt,
    		zoals: DatePicker, Spinner, Dialog, Buttons, Tabs, Tooltip)

	\end{itemize}

\end{frame}

% ------------------------------------------------------------------------------
% LTXE-SLIDE-START
% LTXE-SLIDE-UID:		68f25a3b-4df59cf1-b70546d5-e32dcb0c
% LTXE-SLIDE-ORIGIN:	66a996e1-bcfe7442-d44f59c8-d3495d4e English
% LTXE-SLIDE-TITLE:		Registry for File Rendering Classes
% LTXE-SLIDE-REFERENCE:	https://forge.typo3.org/issues/61800
% ------------------------------------------------------------------------------

\begin{frame}[fragile]
	\frametitle{Systeemwijzigingen}
	\framesubtitle{Register voor klassen om bestanden te renderen}

	\lstset{
		basicstyle=\tiny\ttfamily
	}

	\begin{itemize}
		\item Om alle soorten mediabestanden te kunnen renderen is een register
			voor het renderen van bestanden gemaakt.\newline
			Dit gebeurt als volgt (bijv. Video, MPEG, AVI, WAV, etc.):

		\begin{lstlisting}
			<?php
			namespace ...;

			class NameTagRenderer implements FileRendererInterface {
			  protected $possibleMimeTypes = array('audio/mpeg', 'audio/wav', ...);
			  public function getPriority() {
			    return 1; // prioriteit: hoe hoger, des te belangrijker (max: 100)
			  }
			  public function canRender(FileInterface $file) {
			    return in_array($file->getMimeType(), $this->possibleMimeTypes, TRUE);
			  }
			  public function render(FileInterface $file, $width, $height, array $options = array(), $usedPathsRelativeToCurrentScript = FALSE) {
			    ...

			    return 'HTML code';
			  }
			}
		\end{lstlisting}

	\end{itemize}

\end{frame}

% ------------------------------------------------------------------------------
% LTXE-SLIDE-START
% LTXE-SLIDE-UID:		4f933dcb-fbe9e005-93213680-67f8aa48
% LTXE-SLIDE-ORIGIN:	2c97df27-58f83c0c-2a7532ce-73ba3bd5 English
% LTXE-SLIDE-TITLE:		TCA: Validate Email Addresses
% LTXE-SLIDE-REFERENCE:	https://forge.typo3.org/issues/62147
% ------------------------------------------------------------------------------

\begin{frame}[fragile]
	\frametitle{Systeemwijzigingen}
	\framesubtitle{TCA: E-mailadressen valideren}

	\lstset{
		basicstyle=\tiny\ttfamily
	}

	\begin{itemize}

		\item Nieuwe functie "email" checkt of de waarde een geldig e-mailadres is
		\item Als controle niet slaagt, verschijnt een statusmelding
		\item Voorbeeld:

		\begin{lstlisting}
			'emailaddress' => array(
			  'exclude' => 1,
			  'label' => 'LLL:EXT:myextension/Resources/Private/Language/locallang_db.xlf:tx_myextension
		 	  'config' => array(
			      'type' => 'input',
			      'size' => 30,
			      'eval' => 'email,trim'
			  ),
			)
		\end{lstlisting}

	\end{itemize}

\end{frame}

% ------------------------------------------------------------------------------
% LTXE-SLIDE-START
% LTXE-SLIDE-UID:		7bd8930b-8ecb50d5-00dbc4d5-15509842
% LTXE-SLIDE-ORIGIN:	08cf25ed-0cba243f-e0738567-598b1534 English
% LTXE-SLIDE-TITLE:		AbstractCondition For Custom TypoScript Conditions
% LTXE-SLIDE-REFERENCE:	https://forge.typo3.org/issues/61489
% ------------------------------------------------------------------------------

\begin{frame}[fragile]
	\frametitle{Systeemwijzigingen}
	\framesubtitle{AbstractCondition voor eigen TypoScript-condities}

	\lstset{
		basicstyle=\tiny\ttfamily
	}

	\begin{itemize}
		\item Eigen TypoScript-condities kunnen afgeleid worden van AbstractCondition

		\begin{lstlisting}
			class TestCondition
			  extends \TYPO3\CMS\Core\Configuration\TypoScript\ConditionMatching\AbstractCondition {

			  public function matchCondition(array $conditionParameters) {
 			    if ($conditionParameters[0] === '= 7' && $conditionParameters[1] === '!= 6') {
			      throw new TestConditionException('Alles Ok', 1411581139);
			    }
			  }
			}
		\end{lstlisting}

		\item De bijgehorende TypoScript-code is:

		\begin{lstlisting}
			[Vendor\Package\TestCondition]
			[Vendor\Package\TestCondition = 7]
			[Vendor\Package\TestCondition = 7, != 6]
		\end{lstlisting}

		\item Operatoren die beschikbaar moeten zijn, worden in de klasse gedefinieerd

	\end{itemize}

\end{frame}

% ------------------------------------------------------------------------------
% LTXE-SLIDE-START
% LTXE-SLIDE-UID:		043d574c-4d8a45c2-4e6ae4b2-75101883
% LTXE-SLIDE-ORIGIN:	c137d1cf-7ce68dbe-8149deb5-19fe62bb English
% LTXE-SLIDE-TITLE:		Signal for IconUtility HTML tag manipulation
% LTXE-SLIDE-REFERENCE:	https://forge.typo3.org/issues/61289
% ------------------------------------------------------------------------------

\begin{frame}[fragile]
	\frametitle{Systeemwijzigingen}
	\framesubtitle{Signal voor manipulatie van IconUtility HTML tag}

	\lstset{
		basicstyle=\tiny\ttfamily
	}

	\begin{itemize}

		\item Nieuw signal voor manipulatie van de IconUtility HTML tag voor sprite-iconen:

		\begin{lstlisting}
			dispatch(
			  'TYPO3\\CMS\\Backend\\Utility\\IconUtility',
			  'buildSpriteHtmlIconTag',
			  array($tagAttributes, $innerHtml, $tagName)
			);
		\end{lstlisting}

		\item Aanroep:

		\begin{lstlisting}
			TYPO3\CMS\Backend\Utility\IconUtility\buildSpriteHtmlIconTag
		\end{lstlisting}

	\end{itemize}

\end{frame}

% ------------------------------------------------------------------------------
% LTXE-SLIDE-START
% LTXE-SLIDE-UID:		d1af14e6-dbc4c18b-f211315a-413b9cae
% LTXE-SLIDE-ORIGIN:	6713b880-9410d0a5-ef04d276-7bb989ba English
% LTXE-SLIDE-TITLE:		Add Signal Slots to SoftReferenceIndex
% ------------------------------------------------------------------------------

\begin{frame}[fragile]
	\frametitle{Systeemwijzigingen}
	\framesubtitle{Signal Slots voor SoftReferenceIndex}

	\lstset{
		basicstyle=\tiny\ttfamily
	}

	\begin{itemize}
		\item
			\smaller
				Twee nieuwe signal-slot aanroepen in SoftReferenceIndex:
			\normalsize

		\begin{lstlisting}
			protected function emitGetTypoLinkParts(
			  $linkHandlerFound, $finalTagParts, $linkHandlerKeyword, $linkHandlerValue) {
			  return $this->getSignalSlotDispatcher()->dispatch(
			    get_class($this),
			    'getTypoLinkParts',
			    array($linkHandlerFound, $finalTagParts, $linkHandlerKeyword, $linkHandlerValue)
			  );
			}
			protected function emitSetTypoLinkPartsElement(
			  $linkHandlerFound, $tLP, $content, $elements, $idx, $tokenID) {
			  return $this->getSignalSlotDispatcher()->dispatch(
			    get_class($this),
			    'setTypoLinkPartsElement',
			    array($linkHandlerFound, $tLP, $content, $elements, $idx, $tokenID, $this)
			  );
			}
		\end{lstlisting}

		\item
			\smaller
				Aangeroepen in:
			\normalsize

			\begin{lstlisting}
				TYPO3\CMS\Core\Database\SoftReferenceIndex->findRef_typolink
				TYPO3\CMS\Core\Database\SoftReferenceIndex->getTypoLinkParts
			\end{lstlisting}

	\end{itemize}

\end{frame}

% ------------------------------------------------------------------------------
% LTXE-SLIDE-START
% LTXE-SLIDE-UID:		4a351366-26376e7d-ab85f815-1b2810aa
% LTXE-SLIDE-ORIGIN:	d743b8cf-4924c9ca-7e1d330c-3b101461 English
% LTXE-SLIDE-TITLE:		afterPersistObjetct Signal Slot
% LTXE-SLIDE-REFERENCE:	https://forge.typo3.org/issues/59986
% ------------------------------------------------------------------------------

\begin{frame}[fragile]
	\frametitle{Systeemwijzigingen}
	\framesubtitle{afterPersistObjetct Signal Slot}

	\lstset{
		basicstyle=\tiny\ttfamily
	}

	\begin{itemize}

		\item Nieuwe afterPersistObject signal-slot zendt uit voor de aggregate root na het persisteren van alle andere objecten

			\begin{lstlisting}
				protected function emitAfterPersistObjectSignal(DomainObjectInterface $object) {
				  $this->signalSlotDispatcher->dispatch(__CLASS__, 'afterPersistObject', array($object));
				}
			\end{lstlisting}

		\item Aangeroepen in:

			\begin{lstlisting}
				TYPO3\CMS\Extbase\Persistence\Generic\Backend->persistObject
			\end{lstlisting}

		\item Hetzelfde signal is uitgezonden in de methode persistObject in de AbstractBackend-klasse in Flow

	\end{itemize}

\end{frame}                                                 

% ------------------------------------------------------------------------------
% LTXE-SLIDE-START
% LTXE-SLIDE-UID:		4cde04d3-f19c0c84-f1a221f7-5bc73e53
% LTXE-SLIDE-ORIGIN:	17ea6530-0ec1ad3e-8e7d9d24-1d5b64aa English
% LTXE-SLIDE-TITLE:		Signal in loadBaseTca
% LTXE-SLIDE-REFERENCE:	https://forge.typo3.org/issues/57863
% ------------------------------------------------------------------------------

\begin{frame}[fragile]
	\frametitle{Systeemwijzigingen}
	\framesubtitle{Signal in loadBaseTca}

	\lstset{
		basicstyle=\tiny\ttfamily
	}

	\begin{itemize}
		\item Om de prestaties te verbeteren in de backend, wordt de hele
			TCA gecachet (niet alleen delen ervan)

		\begin{lstlisting}
			protected function emitTcaIsBeingBuiltSignal(array $tca) {
			  list($tca) = static::getSignalSlotDispatcher()->dispatch(
			    __CLASS__,
			    'tcaIsBeingBuilt',
			    array($tca)
			  );
			  $GLOBALS['TCA'] = $tca;
			}
		\end{lstlisting}

		\item Aangeroepen in:

			\begin{lstlisting}
				TYPO3\CMS\Core\Utility\ExtensionManagementUtility\Backend->buildBaseTcaFromSingleFiles
			\end{lstlisting}

	\end{itemize}

\end{frame}  

% ------------------------------------------------------------------------------
% LTXE-SLIDE-START
% LTXE-SLIDE-UID:		13e6c16d-59e17e53-c6d6705d-2ccf822c
% LTXE-SLIDE-ORIGIN:	2fd6c25c-5f7952da-1406b74c-e0ff5e67 English
% LTXE-SLIDE-TITLE:		API to Add Cached TCA Changes
% LTXE-SLIDE-REFERENCE:	https://forge.typo3.org/issues/57942
% ------------------------------------------------------------------------------

\begin{frame}[fragile]
	\frametitle{Systeemwijzigingen}
	\framesubtitle{API om gecachete TCA-wijzigingen toe te voegen}

	\begin{itemize}
		\item PHP bestanden in \texttt{extkey/Configuration/TCA/Overrides/}
			worden direct uitgevoerd nadat de TCA cache is gebouwd

		\item Deze bestanden mogen alleen code bevatten die de TCA bewerkt,\newline
			zoals: \texttt{addTCAColumns} of \texttt{addToAllTCATypes}

		\item Dit geeft de backend een prestatieverhoging zodra extensies hiervan gebruik maken

	\end{itemize}

\end{frame} 

% ------------------------------------------------------------------------------
% LTXE-SLIDE-START
% LTXE-SLIDE-UID:		599b1914-e57ca89a-4df6b4b3-3921cd5a
% LTXE-SLIDE-ORIGIN:	89fbd845-3952a157-98d97fc4-4244a2d4 English
% LTXE-SLIDE-TITLE:		Read-only File Mounts
% LTXE-SLIDE-REFERENCE:	https://forge.typo3.org/issues/49391
% ------------------------------------------------------------------------------

\begin{frame}[fragile]
	\frametitle{Systeemwijzigingen}
	\framesubtitle{Alleen-lezen startpunten voor bestanden}

	\begin{itemize}

		\item Startpunten voor bestanden kunnen (weer) ingesteld worden als "alleen-lezen"
		\item Dit was al mogelijk in TYPO3 CMS 4.x, maar stilletjes verwijderd in 6.x
		\item Voorbeeld: voeg map "test" van opslag UID 3 als een alleen-lezen startpunt in de bestandslijst en elementbladerscherm.\newline

			\smaller\texttt{options.folderTree.altElementBrowserMountPoints = 3:/test}\normalsize\newline

			Als geen opslag is ingesteld wordt aangenomen dat de map in de standaardopslag zit.
	\end{itemize}

\end{frame}

% ------------------------------------------------------------------------------
% LTXE-SLIDE-START
% LTXE-SLIDE-UID:		6e87474c-2f903f40-bfcd09a0-03b03cea
% LTXE-SLIDE-ORIGIN:	b964bc6b-27cd32e7-bda8cc3e-8bea33ca English
% LTXE-SLIDE-TITLE:		Miscellaneous
% LTXE-SLIDE-REFERENCE:	https://forge.typo3.org/issues/62993
% LTXE-SLIDE-REFERENCE:	https://forge.typo3.org/issues/61185
% LTXE-SLIDE-REFERENCE:	https://forge.typo3.org/issues/58365
% LTXE-SLIDE-REFERENCE:	https://forge.typo3.org/issues/59830
% LTXE-SLIDE-REFERENCE:	https://forge.typo3.org/issues/62857
% ------------------------------------------------------------------------------

\begin{frame}[fragile]
	\frametitle{Systeemwijzigingen}
	\framesubtitle{Diversen}

	\begin{itemize}
		\item jQuery is bijgewerkt van versie 1.11.0 naar versie 1.11.1
		\item Datatabellen zijn bijgewerkt van versie 1.9.4 naar versie 1.10.2
		\item Enige oude, ongebruikte variabelen zijn verwijderd uit \texttt{EM\_CONF}
		\item Extensie-iconen kunnen ook in SVG-formaat zijn (\texttt{ext\_icon.svg})
		\item Een verkeerde eID-identifier geeft nu een exceptie
	\end{itemize}

\end{frame}

% ------------------------------------------------------------------------------
