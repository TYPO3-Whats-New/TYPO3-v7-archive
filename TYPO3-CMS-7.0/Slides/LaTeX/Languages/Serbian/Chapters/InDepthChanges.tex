% ------------------------------------------------------------------------------
% TYPO3 CMS 7.0 - What's New - Chapter "In-Depth Changes" (English Version)
%
% @author	Michael Schams <schams.net>
% @license	Creative Commons BY-NC-SA 3.0
% @link		http://typo3.org/download/release-notes/whats-new/
% @language	English
% ------------------------------------------------------------------------------
% LTXE-CHAPTER-UID:		83d831ac-d1e8ba24-a86bb957-c8cef9bb
% LTXE-CHAPTER-NAME:	In-Depth Changes
% ------------------------------------------------------------------------------

\section{Korenite promene}
\begin{frame}[fragile]
	\frametitle{Korenite promene}

	\begin{center}\huge{Poglavlje 3:}\end{center}
	\begin{center}\huge{\color{typo3darkgrey}\textbf{Korenite promene}}\end{center}

\end{frame}

% ------------------------------------------------------------------------------
% LTXE-SLIDE-START
% LTXE-SLIDE-UID:		11aa91f1-53e861a9-4fae5678-b95ba1bd
% LTXE-SLIDE-ORIGIN:	b47dd2a2-952b3dd2-b9ceecce-ab8630bc English
% LTXE-SLIDE-TITLE:		Integration of jQuery UI version 1.11.2
% LTXE-SLIDE-REFERENCE:	https://forge.typo3.org/issues/62916
% ------------------------------------------------------------------------------

\begin{frame}[fragile]
	\frametitle{Korenite promene}
	\framesubtitle{Integracija jQuery UI verzija 1.11.2}

	\begin{itemize}
		\item jQuery UI 1.11 podrzava AMD
			(\href{http://en.wikipedia.org/wiki/Asynchronous_module_definition}{Asynchronous Module Definition}),
			koji ucitava JavaScript fajlove samo kada su potrebni (poboljsanje performansi)

    	\item jQuery UI 1.11 zamenjuje jQuery UI 1.10 + Scriptaculous u TYPO3 CMS 7.0

    	\item Ukljucene su samo osnovne komponente i komponente za interakciju, koje su potrebne kako bi se zamenili ExtJS i Scriptaculous

    	\item Dodaci (widget-i) nisu ukljuceni (osim Twitter Bootstrap dodataka koji se koriste, kao sto su: DatePicker, Spinner, Dialog, Buttons, Tabs, Tooltip)

	\end{itemize}

\end{frame}

% ------------------------------------------------------------------------------
% LTXE-SLIDE-START
% LTXE-SLIDE-UID:		c2a28cc1-6fce4c6e-abd6c8c0-f5dfd819
% LTXE-SLIDE-ORIGIN:	66a996e1-bcfe7442-d44f59c8-d3495d4e English
% LTXE-SLIDE-TITLE:		Registry for File Rendering Classes
% LTXE-SLIDE-REFERENCE:	https://forge.typo3.org/issues/61800
% ------------------------------------------------------------------------------

\begin{frame}[fragile]
	\frametitle{Korenite promene}
	\framesubtitle{Registar klasa za renderanje fajlova}

	\lstset{
		basicstyle=\tiny\ttfamily
	}

	\begin{itemize}
		\item Kako bi se mogli renderati svi tipovi fajlova implementiran je registar za renderanje fajlova.\newline
			Ovo se desava na sledevi nacin (na primer Video, MPEG, AVI, WAV, itd.):

		\begin{lstlisting}
			<?php
			namespace ...;

			class NameTagRenderer implements FileRendererInterface {
			  protected $possibleMimeTypes = array('audio/mpeg', 'audio/wav', ...);
			  public function getPriority() {
			    return 1; // priority: the higher, the more important (max: 100)
			  }
			  public function canRender(FileInterface $file) {
			    return in_array($file->getMimeType(), $this->possibleMimeTypes, TRUE);
			  }
			  public function render(FileInterface $file, $width, $height, array $options = array(), $usedPathsRelativeToCurrentScript = FALSE) {
			    ...

			    return 'HTML code';
			  }
			}
		\end{lstlisting}

	\end{itemize}

\end{frame}

% ------------------------------------------------------------------------------
% LTXE-SLIDE-START
% LTXE-SLIDE-UID:		3631fc00-b2f959f2-3955c594-52978fab
% LTXE-SLIDE-ORIGIN:	2c97df27-58f83c0c-2a7532ce-73ba3bd5 English
% LTXE-SLIDE-TITLE:		TCA: Validate Email Addresses
% LTXE-SLIDE-REFERENCE:	https://forge.typo3.org/issues/62147
% ------------------------------------------------------------------------------

\begin{frame}[fragile]
	\frametitle{Korenite promene}
	\framesubtitle{TCA: Validacija email adresa}

	\lstset{
		basicstyle=\tiny\ttfamily
	}

	\begin{itemize}

		\item Nova funkcija “email” proverava da li je uneta vrednost validna email adresa
		\item Ukoliko provera ne prodje pojavljuje se poruka
		\item Primer:

		\begin{lstlisting}
			'emailaddress' => array(
			  'exclude' => 1,
			  'label' => 'LLL:EXT:myextension/Resources/Private/Language/locallang_db.xlf:tx_myextension
		 	  'config' => array(
			      'type' => 'input',
			      'size' => 30,
			      'eval' => 'email,trim'
			  ),
			)
		\end{lstlisting}

	\end{itemize}

\end{frame}

% ------------------------------------------------------------------------------
% LTXE-SLIDE-START
% LTXE-SLIDE-UID:		7f7a7693-263e5ca7-9a9a9381-a93a17f3
% LTXE-SLIDE-ORIGIN:	08cf25ed-0cba243f-e0738567-598b1534 English
% LTXE-SLIDE-TITLE:		AbstractCondition For Custom TypoScript Conditions
% LTXE-SLIDE-REFERENCE:	https://forge.typo3.org/issues/61489
% ------------------------------------------------------------------------------

\begin{frame}[fragile]
	\frametitle{Korenite promene}
	\framesubtitle{AbstractCondition za posebno napisane TypoScript uslove}

	\lstset{
		basicstyle=\tiny\ttfamily
	}

	\begin{itemize}
		\item Prilagodjeni TypoScript uslovi mogu se izvesti iz AbstractCondition-a

		\begin{lstlisting}
			class TestCondition
			  extends \TYPO3\CMS\Core\Configuration\TypoScript\ConditionMatching\AbstractCondition {

			  public function matchCondition(array $conditionParameters) {
 			    if ($conditionParameters[0] === '= 7' && $conditionParameters[1] === '!= 6') {
			      throw new TestConditionException('All Ok', 1411581139);
			    }
			  }
			}
		\end{lstlisting}

		\item Odgovarajuci TypoScript kod:

		\begin{lstlisting}
			[Vendor\Package\TestCondition]
			[Vendor\Package\TestCondition = 7]
			[Vendor\Package\TestCondition = 7, != 6]
		\end{lstlisting}

		\item Operatori koji bi trebalo da su dostupni, definisani su u klasi

	\end{itemize}

\end{frame}

% ------------------------------------------------------------------------------
% LTXE-SLIDE-START
% LTXE-SLIDE-UID:		2180d5d5-046de6e4-da50c870-a7bf6b8f
% LTXE-SLIDE-ORIGIN:	c137d1cf-7ce68dbe-8149deb5-19fe62bb English
% LTXE-SLIDE-TITLE:		Signal for IconUtility HTML tag manipulation
% LTXE-SLIDE-REFERENCE:	https://forge.typo3.org/issues/61289
% ------------------------------------------------------------------------------

\begin{frame}[fragile]
	\frametitle{Korenite promene}
	\framesubtitle{Signal za manipulaciju IconUtility HTML tagovima}

	\lstset{
		basicstyle=\tiny\ttfamily
	}

	\begin{itemize}

		\item Novi signal za manipulaciju IconUtility HTML tagovima za sprajt ikonice:

		\begin{lstlisting}
			dispatch(
			  'TYPO3\\CMS\\Backend\\Utility\\IconUtility',
			  'buildSpriteHtmlIconTag',
			  array($tagAttributes, $innerHtml, $tagName)
			);
		\end{lstlisting}

		\item Poziv metoda:

		\begin{lstlisting}
			TYPO3\CMS\Backend\Utility\IconUtility\buildSpriteHtmlIconTag
		\end{lstlisting}

	\end{itemize}

\end{frame}

% ------------------------------------------------------------------------------
% LTXE-SLIDE-START
% LTXE-SLIDE-UID:		e98a26a5-53aff64a-a157d9ec-0b6edeaf
% LTXE-SLIDE-ORIGIN:	6713b880-9410d0a5-ef04d276-7bb989ba English
% LTXE-SLIDE-TITLE:		Add Signal Slots to SoftReferenceIndex
% ------------------------------------------------------------------------------

\begin{frame}[fragile]
	\frametitle{Korenite promene}
	\framesubtitle{Dodati signal slotovi za SoftReferenceIndex}

	\lstset{
		basicstyle=\tiny\ttfamily
	}

	\begin{itemize}
		\item
			\smaller
				Dva nova signal slot poziva u SoftReferenceIndex-u:
			\normalsize

		\begin{lstlisting}
			protected function emitGetTypoLinkParts(
			  $linkHandlerFound, $finalTagParts, $linkHandlerKeyword, $linkHandlerValue) {
			  return $this->getSignalSlotDispatcher()->dispatch(
			    get_class($this),
			    'getTypoLinkParts',
			    array($linkHandlerFound, $finalTagParts, $linkHandlerKeyword, $linkHandlerValue)
			  );
			}
			protected function emitSetTypoLinkPartsElement(
			  $linkHandlerFound, $tLP, $content, $elements, $idx, $tokenID) {
			  return $this->getSignalSlotDispatcher()->dispatch(
			    get_class($this),
			    'setTypoLinkPartsElement',
			    array($linkHandlerFound, $tLP, $content, $elements, $idx, $tokenID, $this)
			  );
			}
		\end{lstlisting}

		\item
			\smaller
				Pozvani u:
			\normalsize

			\begin{lstlisting}
				TYPO3\CMS\Core\Database\SoftReferenceIndex->findRef_typolink
				TYPO3\CMS\Core\Database\SoftReferenceIndex->getTypoLinkParts
			\end{lstlisting}

	\end{itemize}

\end{frame}

% ------------------------------------------------------------------------------
% LTXE-SLIDE-START
% LTXE-SLIDE-UID:		74487c2e-4ae27506-6678fd7d-15b12d40
% LTXE-SLIDE-ORIGIN:	d743b8cf-4924c9ca-7e1d330c-3b101461 English
% LTXE-SLIDE-TITLE:		afterPersistObjetct Signal Slot
% LTXE-SLIDE-REFERENCE:	https://forge.typo3.org/issues/59986
% ------------------------------------------------------------------------------

\begin{frame}[fragile]
	\frametitle{Korenite promene}
	\framesubtitle{afterPersistObjetct Signal Slot}

	\lstset{
		basicstyle=\tiny\ttfamily
	}

	\begin{itemize}

		\item Novi afterPersistObject signal slot se emituje za aggregate root nakon perzistovanja svih ostalih objekata

			\begin{lstlisting}
				protected function emitAfterPersistObjectSignal(DomainObjectInterface $object) {
				  $this->signalSlotDispatcher->dispatch(__CLASS__, 'afterPersistObject', array($object));
				}
			\end{lstlisting}

		\item Pozvan u:

			\begin{lstlisting}
				TYPO3\CMS\Extbase\Persistence\Generic\Backend->persistObject
			\end{lstlisting}

		\item Isti signal se emituje u persistObject metodu u AbstractBackend klasi kod Flow-a

	\end{itemize}

\end{frame}                                                 

% ------------------------------------------------------------------------------
% LTXE-SLIDE-START
% LTXE-SLIDE-UID:		a66a8e21-3fbb8cf5-7c174668-da6c96ec
% LTXE-SLIDE-ORIGIN:	17ea6530-0ec1ad3e-8e7d9d24-1d5b64aa English
% LTXE-SLIDE-TITLE:		Signal in loadBaseTca
% LTXE-SLIDE-REFERENCE:	https://forge.typo3.org/issues/57863
% ------------------------------------------------------------------------------

\begin{frame}[fragile]
	\frametitle{Korenite promene}
	\framesubtitle{Signal u loadBaseTca}

	\lstset{
		basicstyle=\tiny\ttfamily
	}

	\begin{itemize}
		\item Kako bi se poboljsale performanse u kontekstu administratorskog interfejsa,
			sada kompletni TCA moze da se kesira (ne samo njegovi delovi)

		\begin{lstlisting}
			protected function emitTcaIsBeingBuiltSignal(array $tca) {
			  list($tca) = static::getSignalSlotDispatcher()->dispatch(
			    __CLASS__,
			    'tcaIsBeingBuilt',
			    array($tca)
			  );
			  $GLOBALS['TCA'] = $tca;
			}
		\end{lstlisting}

		\item Poziva se u:

			\begin{lstlisting}
				TYPO3\CMS\Core\Utility\ExtensionManagementUtility\Backend->buildBaseTcaFromSingleFiles
			\end{lstlisting}

	\end{itemize}

\end{frame}  

% ------------------------------------------------------------------------------
% LTXE-SLIDE-START
% LTXE-SLIDE-UID:		5a396561-3cd9fa55-45d801bf-4dd240e7
% LTXE-SLIDE-ORIGIN:	2fd6c25c-5f7952da-1406b74c-e0ff5e67 English
% LTXE-SLIDE-TITLE:		API to Add Cached TCA Changes
% LTXE-SLIDE-REFERENCE:	https://forge.typo3.org/issues/57942
% ------------------------------------------------------------------------------

\begin{frame}[fragile]
	\frametitle{Korenite promene}
	\framesubtitle{API za dodavanje kesiranih TCA promena}

	\begin{itemize}
		\item PHP fajlovi u \texttt{extkey/Configuration/TCA/Overrides/}
			se pozivaju odmah nakon sto je TCA kes izgradjen

		\item Ovi fajlovi mogu sadrzati samo kod koji manipulise TCA-om,\newline
			kao na primer: \texttt{addTCAColumns} ili \texttt{addToAllTCATypes}

		\item Nakon sto ekstenzije pocnu koristiti ove fajlove, ova opcija povecace performanse korisnickog interfejsa

	\end{itemize}

\end{frame} 

% ------------------------------------------------------------------------------
% LTXE-SLIDE-START
% LTXE-SLIDE-UID:		ba0e0ea5-32a747c1-5cf69cea-7e7c0f0b
% LTXE-SLIDE-ORIGIN:	89fbd845-3952a157-98d97fc4-4244a2d4 English
% LTXE-SLIDE-TITLE:		Read-only File Mounts
% LTXE-SLIDE-REFERENCE:	https://forge.typo3.org/issues/49391
% ------------------------------------------------------------------------------

\begin{frame}[fragile]
	\frametitle{Korenite promene}
	\framesubtitle{Read-only File Mounts}

	\begin{itemize}

		\item File mounts se mogu konfigurisati kao "read only" (ponovo)
		\item Ovo je vec bilo moguce u  TYPO3 CMS 4.x, ali je izbaceno u 6.x
		\item Primer: dodati folder "test" skladistu sa UID 3 kao read-only File Mount u File List i Element Browser.\newline

			\smaller\texttt{options.folderTree.altElementBrowserMountPoints = 3:/test}\normalsize\newline

			Ako ni jedno skladiste nije definisano, pretpostavlja se da je folder podrazumevano skladiste.
	\end{itemize}

\end{frame}

% ------------------------------------------------------------------------------
% LTXE-SLIDE-START
% LTXE-SLIDE-UID:		f315c37a-8fa6fc6f-1f5b84d1-66b8adba
% LTXE-SLIDE-ORIGIN:	b964bc6b-27cd32e7-bda8cc3e-8bea33ca English
% LTXE-SLIDE-TITLE:		Miscellaneous
% LTXE-SLIDE-REFERENCE:	https://forge.typo3.org/issues/62993
% LTXE-SLIDE-REFERENCE:	https://forge.typo3.org/issues/61185
% LTXE-SLIDE-REFERENCE:	https://forge.typo3.org/issues/58365
% LTXE-SLIDE-REFERENCE:	https://forge.typo3.org/issues/59830
% LTXE-SLIDE-REFERENCE:	https://forge.typo3.org/issues/62857
% ------------------------------------------------------------------------------

\begin{frame}[fragile]
	\frametitle{Korenite promene}
	\framesubtitle{Razno}

	\begin{itemize}
		\item jQuery je nadogradjen sa verzije 1.11.0 na verziju 1.11.1
		\item Datatables su nadogradjene sa verzije 1.9.4 na verziju 1.10.2
		\item Neke stare, beskorisne promenjive su uklonjene iz \texttt{EM\_CONF}
		\item Ikonice za prosirenja sada mogu biti u SVG formatu (\texttt{ext\_icon.svg})
		\item Prosledjivanje pogresnog eID identifikatora sada rezultira izuzetkom
	\end{itemize}

\end{frame}

% ------------------------------------------------------------------------------
