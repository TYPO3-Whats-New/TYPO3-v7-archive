% ------------------------------------------------------------------------------
% TYPO3 CMS 7.0 - What's New - Chapter "TypoScript" (English Version)
%
% @author	Michael Schams <schams.net>
% @license	Creative Commons BY-NC-SA 3.0
% @link		http://typo3.org/download/release-notes/whats-new/
% @language	English
% ------------------------------------------------------------------------------
% LTXE-CHAPTER-UID:		f1265349-a227bb4e-5735b3c3-332a7212
% LTXE-CHAPTER-NAME:	TypoScript
% ------------------------------------------------------------------------------

\section{TSconfig i TypoScript}
\begin{frame}[fragile]
	\frametitle{TSconfig i TypoScript}

	\begin{center}\huge{Poglavlje 2:}\end{center}
	\begin{center}\huge{\color{typo3darkgrey}\textbf{TSconfig i TypoScript}}\end{center}

\end{frame}

% ------------------------------------------------------------------------------
% LTXE-SLIDE-START
% LTXE-SLIDE-UID:		1b3ddc8d-1c1468ed-7e26130b-0d18904e
% LTXE-SLIDE-ORIGIN:	c65cc12d-2f8c9d53-37d6577f-8a2f8357 English
% LTXE-SLIDE-TITLE:		TSconfig Available to Link Checkers
% LTXE-SLIDE-REFERENCE:	https://forge.typo3.org/issues/54518
% ------------------------------------------------------------------------------

\begin{frame}[fragile]
	\frametitle{TSconfig i TypoScript}
	\framesubtitle{TSConfig dostupan za Link Checkers}

	\begin{itemize}
		\item TSconfig konfiguracija se cita

			\begin{itemize}
				\item ili iz administratorskog interfejsa (ako se koristi Linkvalidator)
				\item ili iz scheduler task konfiguracje 
			\end{itemize}

		\item Primer: TSconfig, koji moze biti procitan iz Linkchecker-a:

			\lstinline!mod.linkvalidator.mychecker.myvar = 1!

		\item TSconfig je dostupan kao \texttt{\$this->tsConfig}
	\end{itemize}

\end{frame}

% ------------------------------------------------------------------------------
% LTXE-SLIDE-START
% LTXE-SLIDE-UID:		b6964427-2c610c13-8d23b6cd-2fc7ad99
% LTXE-SLIDE-ORIGIN:	1d38328d-a7245bbe-1b380ab3-88ea4885 English
% LTXE-SLIDE-TITLE:		Linkcheck: Report Deleted Records
% LTXE-SLIDE-REFERENCE:	https://forge.typo3.org/issues/54519
% ------------------------------------------------------------------------------

\begin{frame}[fragile]
	\frametitle{TSconfig i TypoScript}
	\framesubtitle{Linkcheck: Prijavljuje obrisane zapise}

	\begin{itemize}
		\item U TYPO3 CMS < 7.0, linkhandler je upozoravao ukoliko postoje linkovi ka nepostojecim ili obrisanim zapisima
		\item Od TYPO3 CMS >=  7.0, sledece TSconfig podesavanje omogucava upozorenja ukoliko link pokazuje ka skrivenim zapisima:

			\lstinline!mod.linkvalidator.linkhandler.reportHiddenRecords = 1!

	\end{itemize}

\end{frame}

% ------------------------------------------------------------------------------
% LTXE-SLIDE-START
% LTXE-SLIDE-UID:		b24b5ef2-e7516a91-6b8593b6-dd0986b4
% LTXE-SLIDE-ORIGIN:	a27d2c0a-410cbfc8-50f176e2-027ce416 English
% LTXE-SLIDE-TITLE:		RTE: Multiple CSS Classes Per Style
% LTXE-SLIDE-REFERENCE:	https://forge.typo3.org/issues/51905
% ------------------------------------------------------------------------------

\begin{frame}[fragile]
	\frametitle{TSconfig i TypoScript}
	\framesubtitle{RTE: Mogusnost visestrukih CSS klasa na jednom stilu}

	\begin{itemize}
		\item Moderni framework-ci kao na primer Twitter Bootstrap zahtevaju visestruke CSS klase na jednom HTML tagu\newline
			\small Na primer: \texttt{<a class="btn btn-danger">Alert</a>}\normalsize
		\item Sada, kada je podrzana mogucnost visestrukih CSS klasa, to znaci da moderatori treba da izaberu samo jedan stil

			\begin{lstlisting}
				RTE.classes.[ *classname* ] {
				  .requires = list of CSS classes
				}
			\end{lstlisting}

	\end{itemize}

\end{frame}

% ------------------------------------------------------------------------------
% LTXE-SLIDE-START
% LTXE-SLIDE-UID:		47aee47b-97adc1d3-64c05e32-5fea60c6
% LTXE-SLIDE-ORIGIN:	e11a6c09-1eb3eaf7-6cebbee9-febef6ba English
% LTXE-SLIDE-TITLE:		RTE: Configure CSS Class As Not-Selectable
% LTXE-SLIDE-REFERENCE:	https://forge.typo3.org/issues/58122
% LTXE-SLIDE-REFERENCE:	https://forge.typo3.org/issues/51905
% ------------------------------------------------------------------------------

\begin{frame}[fragile]
	\frametitle{TSconfig i TypoScript}
	\framesubtitle{RTE: Konfigurisanje CSS klase kao Not-Selectable}

	\begin{itemize}
		\item Sada je moguce konfigurisati CSS klase kao "not-selectable"

			\begin{lstlisting}
				// value "1" means, class is selectable
				// value "0" makes it not-selectable
				RTE.classes.[ *classname* ] {
				  .selectable = 1
				}
			\end{lstlisting}

	\end{itemize}

\end{frame}

% ------------------------------------------------------------------------------
% LTXE-SLIDE-START
% LTXE-SLIDE-UID:		ee520435-b40b573f-b8490b2c-b5a72e4f
% LTXE-SLIDE-ORIGIN:	e607c36f-e1be31da-05ace0c5-991e6e42 English
% LTXE-SLIDE-TITLE:		RTE: Include Multiple CSS Files
% LTXE-SLIDE-REFERENCE:	https://forge.typo3.org/issues/50039
% ------------------------------------------------------------------------------

\begin{frame}[fragile]
	\frametitle{TSconfig i TypoScript}
	\framesubtitle{RTE: Ukljucivanje vise CSS fajlova}

	\begin{itemize}
		\item Sada je moguce ukljuciti vise CSS fajlova

			\begin{lstlisting}
				RTE.default.contentCSS {
				  file1 = fileadmin/rte_stylesheet1.css
				  file2 = fileadmin/rte_stylesheet2.css
				}
			\end{lstlisting}

		\item Ukoliko nijedan CSS fajl nije definisan, bice izabran standardni CSS fajl:\newline
			\texttt{typo3/sysext/rtehtmlarea/res/contentcss/default.css}

	\end{itemize}

\end{frame}

% ------------------------------------------------------------------------------
% LTXE-SLIDE-START
% LTXE-SLIDE-UID:		9cac2c08-48bf4778-2d895d76-057fc537
% LTXE-SLIDE-ORIGIN:	c9d0ad7d-2c5aee11-79878250-9e8ce110 English
% LTXE-SLIDE-TITLE:		Exception Handling When cObjects Are Rendered (1)
% LTXE-SLIDE-REFERENCE:	https://forge.typo3.org/issues/47919
% ------------------------------------------------------------------------------

\begin{frame}[fragile]
	\frametitle{TSconfig i TypoScript}
	\framesubtitle{Upravljanje izuzetcima kada se renderaju cObjects (1)}

	\begin{itemize}
		\item U TYPO3 CMS < 7.0, kada bi doslo do izuzetka prilikom procesa renderanja objekata sadrzaja (na primer \texttt{USER}), greska bi srusila ceo korisnicki interfejs
		\item Od TYPO3 CMS >= 7.0, implementiran je sistem za upravljanje izuzetcima, koji dozvoljava da se ispise poruka umesto neuspelog objekta sadrzaja
	\end{itemize}

%	*TODO* it would be so nice, if someone could create a screenshot of this scenario!
%
%	\begin{figure}\vspace*{-0.3cm}
%		\includegraphics[width=0.90\linewidth]{TypoScript/cObjectExceptionHandling.png}
%	\end{figure}

\end{frame}

% ------------------------------------------------------------------------------
% LTXE-SLIDE-START
% LTXE-SLIDE-UID:		e51b73da-6d93b716-4010786e-059dbe34
% LTXE-SLIDE-ORIGIN:	1132ffa6-a06b7448-0b59e149-a28c2d43 English
% LTXE-SLIDE-TITLE:		Exception Handling When cObjects Are Rendered (2)
% LTXE-SLIDE-REFERENCE:	https://forge.typo3.org/issues/47919
% ------------------------------------------------------------------------------

\begin{frame}[fragile]
	\frametitle{TSconfig i TypoScript}
	\framesubtitle{Upravljanje izuzetcima kada se renderaju cObjects (2)}
	
	\lstset{
		basicstyle=\tiny\ttfamily
	}
	
	\begin{lstlisting}
		# default exception handler (activated in context "production")
		config.contentObjectExceptionHandler = 1

		# configuration of a class for the exception handling
		config.contentObjectExceptionHandler =
		  TYPO3\CMS\Frontend\ContentObject\Exception\ProductionExceptionHandler

		# customised error message (show random error code)
		config.contentObjectExceptionHandler.errorMessage = Oops an error occurred. Code: %s

		# configuration of exception codes, which are not dealt with
		tt_content.login.20.exceptionHandler.ignoreCodes.10 = 1414512813

		# deactivation of exception handling for a specific plugins or content objects
		tt_content.login.20.exceptionHandler = 0

		# ignoreCodes and errorMessage can be configured globally...
		config.contentObjectExceptionHandler.errorMessage = Oops an error occurred. Code: %s
		config.contentObjectExceptionHandler.ignoreCodes.10 = 1414512813

		# ...or locally for individual content objects
		tt_content.login.20.exceptionHandler.errorMessage = Oops an error occurred. Code: %s
		tt_content.login.20.exceptionHandler.ignoreCodes.10 = 1414512813
	\end{lstlisting}

\end{frame}

% ------------------------------------------------------------------------------
