% ------------------------------------------------------------------------------
% TYPO3 CMS 7.0 - What's New - Chapter "In-Depth Changes" (English Version)
%
% @author	Michael Schams <schams.net>
% @license	Creative Commons BY-NC-SA 3.0
% @link		http://typo3.org/download/release-notes/whats-new/
% @language	English
% ------------------------------------------------------------------------------
% LTXE-CHAPTER-UID:		e60f2505-9b3d671c-d9efaba8-f3821775
% LTXE-CHAPTER-NAME:	In-Depth Changes
% ------------------------------------------------------------------------------

\section{In-Depth Changes}
\begin{frame}[fragile]
	\frametitle{In-Depth Changes}

	\begin{center}\huge{Chapter 3:}\end{center}
	\begin{center}\huge{\color{typo3darkgrey}\textbf{In-Depth Changes}}\end{center}

\end{frame}

% ------------------------------------------------------------------------------
% LTXE-SLIDE-START
% LTXE-SLIDE-UID:		fa4e3a64-44a6a3d1-2e850328-74b5b085
% LTXE-SLIDE-ORIGIN:	b47dd2a2-952b3dd2-b9ceecce-ab8630bc English
% LTXE-SLIDE-TITLE:		Integration of jQuery UI version 1.11.2
% LTXE-SLIDE-REFERENCE:	https://forge.typo3.org/issues/62916
% ------------------------------------------------------------------------------

\begin{frame}[fragile]
	\frametitle{In-Depth Changes}
	\framesubtitle{Integration of jQuery UI version 1.11.2}

	\begin{itemize}
		\item jQuery UI 1.11 supports AMD
			(\href{http://en.wikipedia.org/wiki/Asynchronous_module_definition}{Asynchronous Module Definition}),
			which loads JavaScript files only, when they are needed (performance boost)

    	\item jQuery UI 1.11 replaces jQuery UI 1.10 + Scriptaculous in TYPO3 CMS 7.0

    	\item Only e core and interaction components are included, which are required
    		to replace ExtJS und Scriptaculous

    	\item Widgets are not included (but those of Twitter Bootstrap are used,
    		such as: DatePicker, Spinner, Dialog, Buttons, Tabs, Tooltip)

	\end{itemize}

\end{frame}

% ------------------------------------------------------------------------------
% LTXE-SLIDE-START
% LTXE-SLIDE-UID:		3206bcaa-00224f47-dfb7703c-9c73509c
% LTXE-SLIDE-ORIGIN:	66a996e1-bcfe7442-d44f59c8-d3495d4e English
% LTXE-SLIDE-TITLE:		Registry for File Rendering Classes
% LTXE-SLIDE-REFERENCE:	https://forge.typo3.org/issues/61800
% ------------------------------------------------------------------------------

\begin{frame}[fragile]
	\frametitle{In-Depth Changes}
	\framesubtitle{Registry for File Rendering Classes}

	\lstset{
		basicstyle=\tiny\ttfamily
	}

	\begin{itemize}
		\item In order to be able to render all kinds of media files,
			a file rendering registry has been implemented.\newline
			This happens as follows (e.g. Video, MPEG, AVI, WAV, etc.):

		\begin{lstlisting}
			<?php
			namespace ...;

			class NameTagRenderer implements FileRendererInterface {
			  protected $possibleMimeTypes = array('audio/mpeg', 'audio/wav', ...);
			  public function getPriority() {
			    return 1; // priority: the higher, the more important (max: 100)
			  }
			  public function canRender(FileInterface $file) {
			    return in_array($file->getMimeType(), $this->possibleMimeTypes, TRUE);
			  }
			  public function render(FileInterface $file, $width, $height, array $options = array(), $usedPathsRelativeToCurrentScript = FALSE) {
			    ...

			    return 'HTML code';
			  }
			}
		\end{lstlisting}

	\end{itemize}

\end{frame}

% ------------------------------------------------------------------------------
% LTXE-SLIDE-START
% LTXE-SLIDE-UID:		160502c4-eb9ab444-22f4cd87-9623bf3b
% LTXE-SLIDE-ORIGIN:	2c97df27-58f83c0c-2a7532ce-73ba3bd5 English
% LTXE-SLIDE-TITLE:		TCA: Validate Email Addresses
% LTXE-SLIDE-REFERENCE:	https://forge.typo3.org/issues/62147
% ------------------------------------------------------------------------------

\begin{frame}[fragile]
	\frametitle{In-Depth Changes}
	\framesubtitle{TCA: Validate Email Addresses}

	\lstset{
		basicstyle=\tiny\ttfamily
	}

	\begin{itemize}

		\item New function "email" checks, if value entered is a valid email address
		\item If check fails, a Flash message appears
		\item Example:

		\begin{lstlisting}
			'emailaddress' => array(
			  'exclude' => 1,
			  'label' => 'LLL:EXT:myextension/Resources/Private/Language/locallang_db.xlf:tx_myextension
		 	  'config' => array(
			      'type' => 'input',
			      'size' => 30,
			      'eval' => 'email,trim'
			  ),
			)
		\end{lstlisting}

	\end{itemize}

\end{frame}

% ------------------------------------------------------------------------------
% LTXE-SLIDE-START
% LTXE-SLIDE-UID:		4b64a0ef-ed178858-a43d0a5e-bafcd6a8
% LTXE-SLIDE-ORIGIN:	08cf25ed-0cba243f-e0738567-598b1534 English
% LTXE-SLIDE-TITLE:		AbstractCondition For Custom TypoScript Conditions
% LTXE-SLIDE-REFERENCE:	https://forge.typo3.org/issues/61489
% ------------------------------------------------------------------------------

\begin{frame}[fragile]
	\frametitle{In-Depth Changes}
	\framesubtitle{AbstractCondition For Custom TypoScript Conditions}

	\lstset{
		basicstyle=\tiny\ttfamily
	}

	\begin{itemize}
		\item Custom TypoScript conditions can be derived from an AbstractCondition

		\begin{lstlisting}
			class TestCondition
			  extends \TYPO3\CMS\Core\Configuration\TypoScript\ConditionMatching\AbstractCondition {

			  public function matchCondition(array $conditionParameters) {
 			    if ($conditionParameters[0] === '= 7' && $conditionParameters[1] === '!= 6') {
			      throw new TestConditionException('All Ok', 1411581139);
			    }
			  }
			}
		\end{lstlisting}

		\item The appropriate TypoScript code as follows:

		\begin{lstlisting}
			[Vendor\Package\TestCondition]
			[Vendor\Package\TestCondition = 7]
			[Vendor\Package\TestCondition = 7, != 6]
		\end{lstlisting}

		\item Operators, which should be available, are defined in the class

	\end{itemize}

\end{frame}

% ------------------------------------------------------------------------------
% LTXE-SLIDE-START
% LTXE-SLIDE-UID:		50e85874-7619dd4c-25695727-da440495
% LTXE-SLIDE-ORIGIN:	c137d1cf-7ce68dbe-8149deb5-19fe62bb English
% LTXE-SLIDE-TITLE:		Signal for IconUtility HTML tag manipulation
% LTXE-SLIDE-REFERENCE:	https://forge.typo3.org/issues/61289
% ------------------------------------------------------------------------------

\begin{frame}[fragile]
	\frametitle{In-Depth Changes}
	\framesubtitle{Signal for IconUtility HTML Tag Manipulation}

	\lstset{
		basicstyle=\tiny\ttfamily
	}

	\begin{itemize}

		\item New signal to manipulate the IconUtility HTML tag for sprite icons:

		\begin{lstlisting}
			dispatch(
			  'TYPO3\\CMS\\Backend\\Utility\\IconUtility',
			  'buildSpriteHtmlIconTag',
			  array($tagAttributes, $innerHtml, $tagName)
			);
		\end{lstlisting}

		\item Method call:

		\begin{lstlisting}
			TYPO3\CMS\Backend\Utility\IconUtility\buildSpriteHtmlIconTag
		\end{lstlisting}

	\end{itemize}

\end{frame}

% ------------------------------------------------------------------------------
% LTXE-SLIDE-START
% LTXE-SLIDE-UID:		28a7bdce-01964ef9-2f6d70c5-2f5fd313
% LTXE-SLIDE-ORIGIN:	6713b880-9410d0a5-ef04d276-7bb989ba English
% LTXE-SLIDE-TITLE:		Add Signal Slots to SoftReferenceIndex
% ------------------------------------------------------------------------------

\begin{frame}[fragile]
	\frametitle{In-Depth Changes}
	\framesubtitle{Add Signal Slots to SoftReferenceIndex}

	\lstset{
		basicstyle=\tiny\ttfamily
	}

	\begin{itemize}
		\item
			\smaller
				Two new signal slot dispatch calls in SoftReferenceIndex:
			\normalsize

		\begin{lstlisting}
			protected function emitGetTypoLinkParts(
			  $linkHandlerFound, $finalTagParts, $linkHandlerKeyword, $linkHandlerValue) {
			  return $this->getSignalSlotDispatcher()->dispatch(
			    get_class($this),
			    'getTypoLinkParts',
			    array($linkHandlerFound, $finalTagParts, $linkHandlerKeyword, $linkHandlerValue)
			  );
			}
			protected function emitSetTypoLinkPartsElement(
			  $linkHandlerFound, $tLP, $content, $elements, $idx, $tokenID) {
			  return $this->getSignalSlotDispatcher()->dispatch(
			    get_class($this),
			    'setTypoLinkPartsElement',
			    array($linkHandlerFound, $tLP, $content, $elements, $idx, $tokenID, $this)
			  );
			}
		\end{lstlisting}

		\item
			\smaller
				Called in:
			\normalsize

			\begin{lstlisting}
				TYPO3\CMS\Core\Database\SoftReferenceIndex->findRef_typolink
				TYPO3\CMS\Core\Database\SoftReferenceIndex->getTypoLinkParts
			\end{lstlisting}

	\end{itemize}

\end{frame}

% ------------------------------------------------------------------------------
% LTXE-SLIDE-START
% LTXE-SLIDE-UID:		bd589086-a8f7abe3-7f5a1b6e-dd1f7a98
% LTXE-SLIDE-ORIGIN:	d743b8cf-4924c9ca-7e1d330c-3b101461 English
% LTXE-SLIDE-TITLE:		afterPersistObjetct Signal Slot
% LTXE-SLIDE-REFERENCE:	https://forge.typo3.org/issues/59986
% ------------------------------------------------------------------------------

\begin{frame}[fragile]
	\frametitle{In-Depth Changes}
	\framesubtitle{afterPersistObjetct Signal Slot}

	\lstset{
		basicstyle=\tiny\ttfamily
	}

	\begin{itemize}

		\item New afterPersistObject signal slot emits for the aggregate root after persisting all other objects

			\begin{lstlisting}
				protected function emitAfterPersistObjectSignal(DomainObjectInterface $object) {
				  $this->signalSlotDispatcher->dispatch(__CLASS__, 'afterPersistObject', array($object));
				}
			\end{lstlisting}

		\item Called in:

			\begin{lstlisting}
				TYPO3\CMS\Extbase\Persistence\Generic\Backend->persistObject
			\end{lstlisting}

		\item The same signal is emitted in the persistObject method in the AbstractBackend class in Flow

	\end{itemize}

\end{frame}                                                 

% ------------------------------------------------------------------------------
% LTXE-SLIDE-START
% LTXE-SLIDE-UID:		4454fd9a-4cf13eec-765b404c-c03827f1
% LTXE-SLIDE-ORIGIN:	17ea6530-0ec1ad3e-8e7d9d24-1d5b64aa English
% LTXE-SLIDE-TITLE:		Signal in loadBaseTca
% LTXE-SLIDE-REFERENCE:	https://forge.typo3.org/issues/57863
% ------------------------------------------------------------------------------

\begin{frame}[fragile]
	\frametitle{In-Depth Changes}
	\framesubtitle{Signal in loadBaseTca}

	\lstset{
		basicstyle=\tiny\ttfamily
	}

	\begin{itemize}
		\item To improve performance in the backend context,
			the complete TCA can be cached now (not only parts of it)

		\begin{lstlisting}
			protected function emitTcaIsBeingBuiltSignal(array $tca) {
			  list($tca) = static::getSignalSlotDispatcher()->dispatch(
			    __CLASS__,
			    'tcaIsBeingBuilt',
			    array($tca)
			  );
			  $GLOBALS['TCA'] = $tca;
			}
		\end{lstlisting}

		\item Called in:

			\begin{lstlisting}
				TYPO3\CMS\Core\Utility\ExtensionManagementUtility\Backend->buildBaseTcaFromSingleFiles
			\end{lstlisting}

	\end{itemize}

\end{frame}  

% ------------------------------------------------------------------------------
% LTXE-SLIDE-START
% LTXE-SLIDE-UID:		44013c05-5cca2bb9-707301b5-4738506c
% LTXE-SLIDE-ORIGIN:	2fd6c25c-5f7952da-1406b74c-e0ff5e67 English
% LTXE-SLIDE-TITLE:		API to Add Cached TCA Changes
% LTXE-SLIDE-REFERENCE:	https://forge.typo3.org/issues/57942
% ------------------------------------------------------------------------------

\begin{frame}[fragile]
	\frametitle{In-Depth Changes}
	\framesubtitle{API to Add Cached TCA Changes}

	\begin{itemize}
		\item PHP files in \texttt{extkey/Configuration/TCA/Overrides/}
			are executed directly after the TCA cache has been built

		\item These files may only include code, which manipulates the TCA,\newline
			such as: \texttt{addTCAColumns} or \texttt{addToAllTCATypes}

		\item This feature gives backend requests a performance boost, once extensions start using these files

	\end{itemize}

\end{frame} 

% ------------------------------------------------------------------------------
% LTXE-SLIDE-START
% LTXE-SLIDE-UID:		43166f24-138a2f8d-b0c71c3f-ee5cdd67
% LTXE-SLIDE-ORIGIN:	89fbd845-3952a157-98d97fc4-4244a2d4 English
% LTXE-SLIDE-TITLE:		Read-only File Mounts
% LTXE-SLIDE-REFERENCE:	https://forge.typo3.org/issues/49391
% ------------------------------------------------------------------------------

\begin{frame}[fragile]
	\frametitle{In-Depth Changes}
	\framesubtitle{Read-only File Mounts}

	\begin{itemize}

		\item File mounts can be configured as "read only" (again)
		\item This was already possible in TYPO3 CMS 4.x, but silently dropped in 6.x
		\item Example: add folder "test" of storage UID 3 as a read-only mount in the File List and Element Browser.\newline

			\smaller\texttt{options.folderTree.altElementBrowserMountPoints = 3:/test}\normalsize\newline

			If no storage is configured, it is assumed that the folder is in the default storage.
	\end{itemize}

\end{frame}

% ------------------------------------------------------------------------------
% LTXE-SLIDE-START
% LTXE-SLIDE-UID:		37d5c39f-a5754bb5-f885a9b1-2dfbdd19
% LTXE-SLIDE-ORIGIN:	b964bc6b-27cd32e7-bda8cc3e-8bea33ca English
% LTXE-SLIDE-TITLE:		Miscellaneous
% LTXE-SLIDE-REFERENCE:	https://forge.typo3.org/issues/62993
% LTXE-SLIDE-REFERENCE:	https://forge.typo3.org/issues/61185
% LTXE-SLIDE-REFERENCE:	https://forge.typo3.org/issues/58365
% LTXE-SLIDE-REFERENCE:	https://forge.typo3.org/issues/59830
% LTXE-SLIDE-REFERENCE:	https://forge.typo3.org/issues/62857
% ------------------------------------------------------------------------------

\begin{frame}[fragile]
	\frametitle{In-Depth Changes}
	\framesubtitle{Miscellaneous}

	\begin{itemize}
		\item jQuery has been updated from version 1.11.0 to version 1.11.1
		\item Datatables has been updated from version 1.9.4 to version 1.10.2
		\item Some old, unused variables have been removed from \texttt{EM\_CONF}
		\item Extension icons can be in SVG image format now (\texttt{ext\_icon.svg})
		\item Passing a wrong eID identifier results in an exception now
	\end{itemize}

\end{frame}

% ------------------------------------------------------------------------------
