% ------------------------------------------------------------------------------
% TYPO3 CMS 7.0 - What's New - Chapter "TypoScript" (Russian Version)
%
% @author	Michael Schams <schams.net>
% @license	Creative Commons BY-NC-SA 3.0
% @link		http://typo3.org/download/release-notes/whats-new/
% @language	Russian
% ------------------------------------------------------------------------------
% LTXE-CHAPTER-UID:		51377f37-5101c46b-3902bf34-2dabba0f
% LTXE-CHAPTER-NAME:	TypoScript
% ------------------------------------------------------------------------------

\section{TSconfig и TypoScript}
\begin{frame}[fragile]
	\frametitle{TSconfig и TypoScript}

	\begin{center}\huge{Глава 2:}\end{center}
	\begin{center}\huge{\color{typo3darkgrey}\textbf{TSconfig и TypoScript}}\end{center}

\end{frame}

% ------------------------------------------------------------------------------
% LTXE-SLIDE-START
% LTXE-SLIDE-UID:		f711228c-a8dbde27-e407b534-66490859
% LTXE-SLIDE-ORIGIN:	c65cc12d-2f8c9d53-37d6577f-8a2f8357 English
% LTXE-SLIDE-TITLE:		TSconfig Available to Link Checkers
% LTXE-SLIDE-REFERENCE:	https://forge.typo3.org/issues/54518
% ------------------------------------------------------------------------------

\begin{frame}[fragile]
	\frametitle{TSconfig и TypoScript}
	\framesubtitle{TSConfig доступен при проверке ссылое}

	\begin{itemize}
		\item Настройка TSconfig читается

			\begin{itemize}
				\item либо из внутреннего интерфейса (если используется Linkvalidator)
				\item либо из настроек задач планировщика
			\end{itemize}

		\item Пример: TSconfig, который может быть прочитан Linkchecker:

			\lstinline!mod.linkvalidator.mychecker.myvar = 1!

		\item TSconfig затем доступен в виде \texttt{\$this->tsConfig}
	\end{itemize}

\end{frame}

% ------------------------------------------------------------------------------
% LTXE-SLIDE-START
% LTXE-SLIDE-UID:		9cecfaac-4a21e8aa-6390133c-10064163
% LTXE-SLIDE-ORIGIN:	1d38328d-a7245bbe-1b380ab3-88ea4885 English
% LTXE-SLIDE-TITLE:		Linkcheck: Report Deleted Records
% LTXE-SLIDE-REFERENCE:	https://forge.typo3.org/issues/54519
% ------------------------------------------------------------------------------

\begin{frame}[fragile]
	\frametitle{TSconfig и TypoScript}
	\framesubtitle{Linkcheck: отчёт об удалённых записях}

	\begin{itemize}
		\item В TYPO3 CMS < 7.0, linkhandler только лишь предупреждал о не существующих или удалённых записях
		\item Начиная с TYPO3 CMS >=  7.0, следующие настройки TSconfig включают предупреждения, если ссылки указывают на
		несуществующие записи:

			\lstinline!mod.linkvalidator.linkhandler.reportHiddenRecords = 1!

	\end{itemize}

\end{frame}

% ------------------------------------------------------------------------------
% LTXE-SLIDE-START
% LTXE-SLIDE-UID:		78d1ef9d-6c648f07-ecbc74ec-1ebf6206
% LTXE-SLIDE-ORIGIN:	a27d2c0a-410cbfc8-50f176e2-027ce416 English
% LTXE-SLIDE-TITLE:		RTE: Multiple CSS Classes Per Style
% LTXE-SLIDE-REFERENCE:	https://forge.typo3.org/issues/51905
% ------------------------------------------------------------------------------

\begin{frame}[fragile]
	\frametitle{TSconfig и TypoScript}
	\framesubtitle{RTE: несколько классов CSS для стиля}

	\begin{itemize}
		\item Современным технологиям, вроде Twitter Bootstrap, требуется несколько классов CSS для тега HTML\newline
			\small Например: \texttt{<a class="btn btn-danger">Внимание</a>}\normalsize
		\item Теперь поддерживается несколько CSS классов, то есть редакторам нужно будет выбрать лишь один стиль

			\begin{lstlisting}
				RTE.classes.[ *classname* ] {
				  .requires = list of CSS classes
				}
			\end{lstlisting}

	\end{itemize}

\end{frame}

% ------------------------------------------------------------------------------
% LTXE-SLIDE-START
% LTXE-SLIDE-UID:		e1281447-104b6b59-f3cd57dc-e9e7bc38
% LTXE-SLIDE-ORIGIN:	e11a6c09-1eb3eaf7-6cebbee9-febef6ba English
% LTXE-SLIDE-TITLE:		RTE: Configure CSS Class As Not-Selectable
% LTXE-SLIDE-REFERENCE:	https://forge.typo3.org/issues/58122
% LTXE-SLIDE-REFERENCE:	https://forge.typo3.org/issues/51905
% ------------------------------------------------------------------------------

\begin{frame}[fragile]
	\frametitle{TSconfig и TypoScript}
	\framesubtitle{RTE: настройка CSS класса, как не выбираемого}

	\begin{itemize}
		\item Теперь возможно настроить классы CSS как "не выбираемые"

			\begin{lstlisting}
				// value "1" means, class is selectable
				// value "0" makes it not-selectable
				RTE.classes.[ *classname* ] {
				  .selectable = 1
				}
			\end{lstlisting}

	\end{itemize}

\end{frame}

% ------------------------------------------------------------------------------
% LTXE-SLIDE-START
% LTXE-SLIDE-UID:		c6ec59b8-a7747639-64b584ea-1a9d49ca
% LTXE-SLIDE-ORIGIN:	e607c36f-e1be31da-05ace0c5-991e6e42 English
% LTXE-SLIDE-TITLE:		RTE: Include Multiple CSS Files
% LTXE-SLIDE-REFERENCE:	https://forge.typo3.org/issues/50039
% ------------------------------------------------------------------------------

\begin{frame}[fragile]
	\frametitle{TSconfig и TypoScript}
	\framesubtitle{RTE: включение нескольких файлов CSS}

	\begin{itemize}
		\item Теперь возможно включить несколько файлов CSS

			\begin{lstlisting}
				RTE.default.contentCSS {
				  file1 = fileadmin/rte_stylesheet1.css
				  file2 = fileadmin/rte_stylesheet2.css
				}
			\end{lstlisting}

		\item Без указания на какой-либо файл CSS, будет подключаться файл по умолчанию:\newline
			\texttt{typo3/sysext/rtehtmlarea/res/contentcss/default.css}

	\end{itemize}

\end{frame}

% ------------------------------------------------------------------------------
% LTXE-SLIDE-START
% LTXE-SLIDE-UID:		08ea0088-f0bf545f-180a6adc-fba0f81c
% LTXE-SLIDE-ORIGIN:	c9d0ad7d-2c5aee11-79878250-9e8ce110 English
% LTXE-SLIDE-TITLE:		Exception Handling When cObjects Are Rendered (1)
% LTXE-SLIDE-REFERENCE:	https://forge.typo3.org/issues/47919
% ------------------------------------------------------------------------------

\begin{frame}[fragile]
	\frametitle{TSconfig и TypoScript}
	\framesubtitle{Обработка исключений при обработке cObjects (1)}

	\begin{itemize}
		\item В TYPO3 CMS < 7.0, при ошибке в процессе обработки объектов содержимого(например, \texttt{USER}), ошибка
		сказывалась на всём внешнем интерфейсе - но не работал
		\item Начиная с TYPO3 CMS >= 7.0, были разработаны обработки исключений, что позволяет вывести сообщение вместо объекта
		 cObject с ошибкой
	\end{itemize}

%	*TODO* it would be so nice, if someone could create a screenshot of this scenario!
%
%	\begin{figure}\vspace*{-0.3cm}
%		\includegraphics[width=0.90\linewidth]{TypoScript/cObjectExceptionHandling.png}
%	\end{figure}

\end{frame}

% ------------------------------------------------------------------------------
% LTXE-SLIDE-START
% LTXE-SLIDE-UID:		21364423-bb3a7783-c382e9f2-489f48bc
% LTXE-SLIDE-ORIGIN:	1132ffa6-a06b7448-0b59e149-a28c2d43 English
% LTXE-SLIDE-TITLE:		Exception Handling When cObjects Are Rendered (2)
% LTXE-SLIDE-REFERENCE:	https://forge.typo3.org/issues/47919
% ------------------------------------------------------------------------------

\begin{frame}[fragile]
	\frametitle{TSconfig и TypoScript}
	\framesubtitle{Обработка исключений при обработке cObjects (2)}
	
	\lstset{
		basicstyle=\tiny\ttfamily
	}
	
	\begin{lstlisting}
		# default exception handler (activated in context "production")
		config.contentObjectExceptionHandler = 1

		# configuration of a class for the exception handling
		config.contentObjectExceptionHandler =
		  TYPO3\CMS\Frontend\ContentObject\Exception\ProductionExceptionHandler

		# customised error message (show random error code)
		config.contentObjectExceptionHandler.errorMessage = Oops an error occurred. Code: %s

		# configuration of exception codes, which are not dealt with
		tt_content.login.20.exceptionHandler.ignoreCodes.10 = 1414512813

		# deactivation of exception handling for a specific plugins or content objects
		tt_content.login.20.exceptionHandler = 0

		# ignoreCodes and errorMessage can be configured globally...
		config.contentObjectExceptionHandler.errorMessage = Oops an error occurred. Code: %s
		config.contentObjectExceptionHandler.ignoreCodes.10 = 1414512813

		# ...or locally for individual content objects
		tt_content.login.20.exceptionHandler.errorMessage = Oops an error occurred. Code: %s
		tt_content.login.20.exceptionHandler.ignoreCodes.10 = 1414512813
	\end{lstlisting}

\end{frame}

% ------------------------------------------------------------------------------
