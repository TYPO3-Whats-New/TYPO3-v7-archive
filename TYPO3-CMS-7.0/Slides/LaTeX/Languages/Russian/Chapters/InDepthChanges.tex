% ------------------------------------------------------------------------------
% TYPO3 CMS 7.0 - What's New - Chapter "In-Depth Changes" (Russian Version)
%
% @author	Michael Schams <schams.net>
% @license	Creative Commons BY-NC-SA 3.0
% @link		http://typo3.org/download/release-notes/whats-new/
% @language	Russian
% ------------------------------------------------------------------------------
% LTXE-CHAPTER-UID:		83efd72e-726e3f4a-bdc483ba-29fc93cc
% LTXE-CHAPTER-NAME:	In-Depth Changes
% ------------------------------------------------------------------------------

\section{Глубинные изменения}
\begin{frame}[fragile]
	\frametitle{Глубинные изменения}

	\begin{center}\huge{Глава 3:}\end{center}
	\begin{center}\huge{\color{typo3darkgrey}\textbf{Глубинные изменения}}\end{center}

\end{frame}

% ------------------------------------------------------------------------------
% LTXE-SLIDE-START
% LTXE-SLIDE-UID:		8339f399-e5b1a627-e261574e-67268c7f
% LTXE-SLIDE-ORIGIN:	b47dd2a2-952b3dd2-b9ceecce-ab8630bc English
% LTXE-SLIDE-TITLE:		Integration of jQuery UI version 1.11.2
% LTXE-SLIDE-REFERENCE:	https://forge.typo3.org/issues/62916
% ------------------------------------------------------------------------------

\begin{frame}[fragile]
	\frametitle{Глубинные изменения}
	\framesubtitle{Интеграция jQuery UI версии 1.11.2}

	\begin{itemize}
		\item jQuery UI 1.11 поддерживает AMD
			(\href{http://en.wikipedia.org/wiki/Asynchronous_module_definition}{Asynchronous Module Definition - асинхронное
			определение модулей}), при этом файлы JavaScript подгружаются по необходимосит (прирост производительности)

    	\item jQuery UI 1.11 заменила jQuery UI 1.10 + Scriptaculous в TYPO3 CMS 7.0

    	\item Включены лишь компоненты core и interaction, что необходимо для замены ExtJS и Scriptaculous

    	\item Виджеты не включаются (кроме тех, что использует Twitter Bootstrap, вроде: DatePicker, Spinner, Dialog, Buttons, Tabs, Tooltip)

	\end{itemize}

\end{frame}

% ------------------------------------------------------------------------------
% LTXE-SLIDE-START
% LTXE-SLIDE-UID:		57e99dba-9806396e-3d1aed40-d74388ec
% LTXE-SLIDE-ORIGIN:	66a996e1-bcfe7442-d44f59c8-d3495d4e English
% LTXE-SLIDE-TITLE:		Registry for File Rendering Classes
% LTXE-SLIDE-REFERENCE:	https://forge.typo3.org/issues/61800
% ------------------------------------------------------------------------------

\begin{frame}[fragile]
	\frametitle{Глубинные изменения}
	\framesubtitle{Реестр для классов вывода файлов / File Rendering Classes}

	\lstset{
		basicstyle=\tiny\ttfamily
	}

	\begin{itemize}
		\item Для возможности обработки всех видов медиа файлов, был введён реестр вывода файлов.\newline
			Это работает следующим образом (например, Video, MPEG, AVI, WAV и т. п.):

		\begin{lstlisting}
			<?php
			namespace ...;

			class NameTagRenderer implements FileRendererInterface {
			  protected $possibleMimeTypes = array('audio/mpeg', 'audio/wav', ...);
			  public function getPriority() {
			    return 1; // priority: the higher, the more important (max: 100)
			  }
			  public function canRender(FileInterface $file) {
			    return in_array($file->getMimeType(), $this->possibleMimeTypes, TRUE);
			  }
			  public function render(FileInterface $file, $width, $height, array $options = array(), $usedPathsRelativeToCurrentScript = FALSE) {
			    ...

			    return 'HTML code';
			  }
			}
		\end{lstlisting}

	\end{itemize}

\end{frame}

% ------------------------------------------------------------------------------
% LTXE-SLIDE-START
% LTXE-SLIDE-UID:		bb909873-735523b9-400ae2be-ae1029fc
% LTXE-SLIDE-ORIGIN:	2c97df27-58f83c0c-2a7532ce-73ba3bd5 English
% LTXE-SLIDE-TITLE:		TCA: Validate Email Addresses
% LTXE-SLIDE-REFERENCE:	https://forge.typo3.org/issues/62147
% ------------------------------------------------------------------------------

\begin{frame}[fragile]
	\frametitle{Глубинные изменения}
	\framesubtitle{TCA: проверка адресов Email}

	\lstset{
		basicstyle=\tiny\ttfamily
	}

	\begin{itemize}

		\item Новая функция "email" проверяет, является ли введённое значение адресом email
		\item Если проверка не прошла, выводится Flash сообщение
		\item Example:

		\begin{lstlisting}
			'emailaddress' => array(
			  'exclude' => 1,
			  'label' => 'LLL:EXT:myextension/Resources/Private/Language/locallang_db.xlf:tx_myextension
		 	  'config' => array(
			      'type' => 'input',
			      'size' => 30,
			      'eval' => 'email,trim'
			  ),
			)
		\end{lstlisting}

	\end{itemize}

\end{frame}

% ------------------------------------------------------------------------------
% LTXE-SLIDE-START
% LTXE-SLIDE-UID:		e79898f4-6ca8cbcc-0a6e5a9d-1db8f7db
% LTXE-SLIDE-ORIGIN:	08cf25ed-0cba243f-e0738567-598b1534 English
% LTXE-SLIDE-TITLE:		AbstractCondition For Custom TypoScript Conditions
% LTXE-SLIDE-REFERENCE:	https://forge.typo3.org/issues/61489
% ------------------------------------------------------------------------------

\begin{frame}[fragile]
	\frametitle{Глубинные изменения}
	\framesubtitle{AbstractCondition для пользовательские условий TypoScript}

	\lstset{
		basicstyle=\tiny\ttfamily
	}

	\begin{itemize}
		\item Пользовательские TypoScript условия могут быть получены из AbstractCondition

		\begin{lstlisting}
			class TestCondition
			  extends \TYPO3\CMS\Core\Configuration\TypoScript\ConditionMatching\AbstractCondition {

			  public function matchCondition(array $conditionParameters) {
 			    if ($conditionParameters[0] === '= 7' && $conditionParameters[1] === '!= 6') {
			      throw new TestConditionException('All Ok', 1411581139);
			    }
			  }
			}
		\end{lstlisting}

		\item Соответствующий код TypoScript:

		\begin{lstlisting}
			[Vendor\Package\TestCondition]
			[Vendor\Package\TestCondition = 7]
			[Vendor\Package\TestCondition = 7, != 6]
		\end{lstlisting}

		\item Доступные операторы определяются в классе

	\end{itemize}

\end{frame}

% ------------------------------------------------------------------------------
% LTXE-SLIDE-START
% LTXE-SLIDE-UID:		cb76234f-9a49c523-0e5a6fb5-359a3be3
% LTXE-SLIDE-ORIGIN:	c137d1cf-7ce68dbe-8149deb5-19fe62bb English
% LTXE-SLIDE-TITLE:		Signal for IconUtility HTML tag manipulation
% LTXE-SLIDE-REFERENCE:	https://forge.typo3.org/issues/61289
% ------------------------------------------------------------------------------

\begin{frame}[fragile]
	\frametitle{Глубинные изменения}
	\framesubtitle{Сигнал для манипуляций с тегами HTML для IconUtility}

	\lstset{
		basicstyle=\tiny\ttfamily
	}

	\begin{itemize}

		\item Новый сигнал для манипуляций с тегами HTML IconUtility HTML для спрайта значков:

		\begin{lstlisting}
			dispatch(
			  'TYPO3\\CMS\\Backend\\Utility\\IconUtility',
			  'buildSpriteHtmlIconTag',
			  array($tagAttributes, $innerHtml, $tagName)
			);
		\end{lstlisting}

		\item Вызов метода:

		\begin{lstlisting}
			TYPO3\CMS\Backend\Utility\IconUtility\buildSpriteHtmlIconTag
		\end{lstlisting}

	\end{itemize}

\end{frame}

% ------------------------------------------------------------------------------
% LTXE-SLIDE-START
% LTXE-SLIDE-UID:		266fd108-dd944a04-f1eaf9f1-4927721c
% LTXE-SLIDE-ORIGIN:	6713b880-9410d0a5-ef04d276-7bb989ba English
% LTXE-SLIDE-TITLE:		Add Signal Slots to SoftReferenceIndex
% ------------------------------------------------------------------------------

\begin{frame}[fragile]
	\frametitle{Глубинные изменения}
	\framesubtitle{Добавление слота сигналов к SoftReferenceIndex}

	\lstset{
		basicstyle=\tiny\ttfamily
	}

	\begin{itemize}
		\item
			\smaller
				Два новых слота сигналов обслуживают вызовы в SoftReferenceIndex:
			\normalsize

		\begin{lstlisting}
			protected function emitGetTypoLinkParts(
			  $linkHandlerFound, $finalTagParts, $linkHandlerKeyword, $linkHandlerValue) {
			  return $this->getSignalSlotDispatcher()->dispatch(
			    get_class($this),
			    'getTypoLinkParts',
			    array($linkHandlerFound, $finalTagParts, $linkHandlerKeyword, $linkHandlerValue)
			  );
			}
			protected function emitSetTypoLinkPartsElement(
			  $linkHandlerFound, $tLP, $content, $elements, $idx, $tokenID) {
			  return $this->getSignalSlotDispatcher()->dispatch(
			    get_class($this),
			    'setTypoLinkPartsElement',
			    array($linkHandlerFound, $tLP, $content, $elements, $idx, $tokenID, $this)
			  );
			}
		\end{lstlisting}

		\item
			\smaller
				Вызывается в:
			\normalsize

			\begin{lstlisting}
				TYPO3\CMS\Core\Database\SoftReferenceIndex->findRef_typolink
				TYPO3\CMS\Core\Database\SoftReferenceIndex->getTypoLinkParts
			\end{lstlisting}

	\end{itemize}

\end{frame}

% ------------------------------------------------------------------------------
% LTXE-SLIDE-START
% LTXE-SLIDE-UID:		1c05ba9d-edde9c0c-16fc57ad-30f32e0c
% LTXE-SLIDE-ORIGIN:	d743b8cf-4924c9ca-7e1d330c-3b101461 English
% LTXE-SLIDE-TITLE:		afterPersistObjetct Signal Slot
% LTXE-SLIDE-REFERENCE:	https://forge.typo3.org/issues/59986
% ------------------------------------------------------------------------------

\begin{frame}[fragile]
	\frametitle{Глубинные изменения}
	\framesubtitle{afterPersistObjetct слот сигнала}

	\lstset{
		basicstyle=\tiny\ttfamily
	}

	\begin{itemize}

		\item Новый слот сигнала afterPersistObject маячит для корня совокупности после сохранения всех остальных объектов

			\begin{lstlisting}
				protected function emitAfterPersistObjectSignal(DomainObjectInterface $object) {
				  $this->signalSlotDispatcher->dispatch(__CLASS__, 'afterPersistObject', array($object));
				}
			\end{lstlisting}

		\item Вызывается в:

			\begin{lstlisting}
				TYPO3\CMS\Extbase\Persistence\Generic\Backend->persistObject
			\end{lstlisting}

		\item Тот же сигнал маячит в методе persistObject класса AbstractBackend во Flow

	\end{itemize}

\end{frame}                                                 

% ------------------------------------------------------------------------------
% LTXE-SLIDE-START
% LTXE-SLIDE-UID:		9ecb6873-845481c5-94e9905f-6876c391
% LTXE-SLIDE-ORIGIN:	17ea6530-0ec1ad3e-8e7d9d24-1d5b64aa English
% LTXE-SLIDE-TITLE:		Signal in loadBaseTca
% LTXE-SLIDE-REFERENCE:	https://forge.typo3.org/issues/57863
% ------------------------------------------------------------------------------

\begin{frame}[fragile]
	\frametitle{Глубинные изменения}
	\framesubtitle{Сигнал в loadBaseTca}

	\lstset{
		basicstyle=\tiny\ttfamily
	}

	\begin{itemize}
		\item Для улучшения производительности в контексте внутреннего интерфейса, сейчас возможно закешировать весь TCA (а не
		только его части)

		\begin{lstlisting}
			protected function emitTcaIsBeingBuiltSignal(array $tca) {
			  list($tca) = static::getSignalSlotDispatcher()->dispatch(
			    __CLASS__,
			    'tcaIsBeingBuilt',
			    array($tca)
			  );
			  $GLOBALS['TCA'] = $tca;
			}
		\end{lstlisting}

		\item Вызывается в:

			\begin{lstlisting}
				TYPO3\CMS\Core\Utility\ExtensionManagementUtility\Backend->buildBaseTcaFromSingleFiles
			\end{lstlisting}

	\end{itemize}

\end{frame}  

% ------------------------------------------------------------------------------
% LTXE-SLIDE-START
% LTXE-SLIDE-UID:		4a3723ca-bff3f723-c0799594-d5fb120f
% LTXE-SLIDE-ORIGIN:	2fd6c25c-5f7952da-1406b74c-e0ff5e67 English
% LTXE-SLIDE-TITLE:		API to Add Cached TCA Changes
% LTXE-SLIDE-REFERENCE:	https://forge.typo3.org/issues/57942
% ------------------------------------------------------------------------------

\begin{frame}[fragile]
	\frametitle{Глубинные изменения}
	\framesubtitle{API для добавления кешируемых изменений TCA}

	\begin{itemize}
		\item PHP файлы в \texttt{extkey/Configuration/TCA/Overrides/}
			выполняются сразу после построения кеша TCA

		\item Эти файлы предназначены лишь для кода управления TCA,\newline
			вроде: \texttt{addTCAColumns} или \texttt{addToAllTCATypes}

		\item Это даёт прирост производительности для внутреннего интерфейса, так как расширения стартуют используя эти файлы

	\end{itemize}

\end{frame} 

% ------------------------------------------------------------------------------
% LTXE-SLIDE-START
% LTXE-SLIDE-UID:		500e1bd7-3985b35f-e904c19b-1667d657
% LTXE-SLIDE-ORIGIN:	89fbd845-3952a157-98d97fc4-4244a2d4 English
% LTXE-SLIDE-TITLE:		Read-only File Mounts
% LTXE-SLIDE-REFERENCE:	https://forge.typo3.org/issues/49391
% ------------------------------------------------------------------------------

\begin{frame}[fragile]
	\frametitle{Глубинные изменения}
	\framesubtitle{Точки монтирования с правами исключительно на чтение}

	\begin{itemize}

		\item Точки монтирования можно настроить как "read only" (снова)
		\item Это было возможно в TYPO3 CMS 4.x, но незаметно исчезло в 6.x
		\item Пример: добавить папку "test" хранилища UID 3 в виде точки монтирования только для чтения в Список файлов и
		Проводник по элементам.\newline

			\smaller\texttt{options.folderTree.altElementBrowserMountPoints = 3:/test}\normalsize\newline

			Если хранишища не настроено, подразумевается что папка находится в хранилище по умолчанию.
	\end{itemize}

\end{frame}

% ------------------------------------------------------------------------------
% LTXE-SLIDE-START
% LTXE-SLIDE-UID:		a5ee284d-2094086d-3c0f05de-9aef8ac1
% LTXE-SLIDE-ORIGIN:	b964bc6b-27cd32e7-bda8cc3e-8bea33ca English
% LTXE-SLIDE-TITLE:		Miscellaneous
% LTXE-SLIDE-REFERENCE:	https://forge.typo3.org/issues/62993
% LTXE-SLIDE-REFERENCE:	https://forge.typo3.org/issues/61185
% LTXE-SLIDE-REFERENCE:	https://forge.typo3.org/issues/58365
% LTXE-SLIDE-REFERENCE:	https://forge.typo3.org/issues/59830
% LTXE-SLIDE-REFERENCE:	https://forge.typo3.org/issues/62857
% ------------------------------------------------------------------------------

\begin{frame}[fragile]
	\frametitle{Глубинные изменения}
	\framesubtitle{Разное}

	\begin{itemize}
		\item jQuery обновлена с версии 1.11.0 на 1.11.1
		\item Datatables было обновлено с версии 1.9.4 на 1.10.2
		\item Некоторые старые, не используемые переменные были удалены из \texttt{EM\_CONF}
		\item Значки расширений теперь имеют формат SVG (\texttt{ext\_icon.svg})
		\item Теперь передача неверного eID идентификатора приводит к исключению
	\end{itemize}

\end{frame}

% ------------------------------------------------------------------------------
