% ------------------------------------------------------------------------------
% TYPO3 CMS 7.0 - What's New - Chapter "TypoScript" (Spanish Version)
%
% @author	Michael Schams <schams.net>
% @author	Michel Mix <mmix@autistici.org>
% @license	Creative Commons BY-NC-SA 3.0
% @link		http://typo3.org/download/release-notes/whats-new/
% @language	Spanish
% ------------------------------------------------------------------------------
% LTXE-CHAPTER-UID:		66ac63f9-50c6feb9-6354535d-41991057
% LTXE-CHAPTER-NAME:	TypoScript
% ------------------------------------------------------------------------------

\section{TSconfig \& TypoScript}
\begin{frame}[fragile]
	\frametitle{TSconfig \& TypoScript}

	\begin{center}\huge{Capítulo 2:}\end{center}
	\begin{center}\huge{\color{typo3darkgrey}\textbf{TSconfig \& TypoScript}}\end{center}

\end{frame}

% ------------------------------------------------------------------------------
% LTXE-SLIDE-START
% LTXE-SLIDE-UID:		78195acb-51146039-e8aea68c-c4b79fee
% LTXE-SLIDE-ORIGIN:	c65cc12d-2f8c9d53-37d6577f-8a2f8357 English
% LTXE-SLIDE-TITLE:		TSconfig Available to Link Checkers
% LTXE-SLIDE-REFERENCE:	https://forge.typo3.org/issues/54518
% ------------------------------------------------------------------------------

\begin{frame}[fragile]
	\frametitle{TSconfig \& TypoScript}
	\framesubtitle{TSConfig disponible para verificadores de enlaces}

	\begin{itemize}
		\item Se lee configuración de TSconfig

			\begin{itemize}
				\item desde el backend (si Linkvalidator es usado)
				\item o desde la configuración del programador de tarea
			\end{itemize}

		\item Ejemplo: TSconfig, puede ser leído por Linkchecker:

			\lstinline!mod.linkvalidator.mychecker.myvar = 1!

		\item TSconfig está entonces disponible como \texttt{\$this->tsConfig}
	\end{itemize}

\end{frame}

% ------------------------------------------------------------------------------
% LTXE-SLIDE-START
% LTXE-SLIDE-UID:		286f459d-e79978c5-9a5e5b3c-0fbdd477
% LTXE-SLIDE-ORIGIN:	1d38328d-a7245bbe-1b380ab3-88ea4885 English
% LTXE-SLIDE-TITLE:		Linkcheck: Report Deleted Records
% LTXE-SLIDE-REFERENCE:	https://forge.typo3.org/issues/54519
% ------------------------------------------------------------------------------

\begin{frame}[fragile]
	\frametitle{TSconfig \& TypoScript}
	\framesubtitle{Linkcheck: Informe de registros eliminados}

	\begin{itemize}
		\item En TYPO3 CMS < 7.0, linkhandler sólo advierte sobre enlaces a registros eliminados o no existentes
		\item Desde TYPO3 CMS >=  7.0, el siguiente ajuste de TSconfig activa una advertencia si los enlaces apuntan a registros deshabilitados:

			\lstinline!mod.linkvalidator.linkhandler.reportHiddenRecords = 1!

	\end{itemize}

\end{frame}

% ------------------------------------------------------------------------------
% LTXE-SLIDE-START
% LTXE-SLIDE-UID:		36fe99d4-ad80c79a-5fab1850-dabee425
% LTXE-SLIDE-ORIGIN:	a27d2c0a-410cbfc8-50f176e2-027ce416 English
% LTXE-SLIDE-TITLE:		RTE: Multiple CSS Classes Per Style
% LTXE-SLIDE-REFERENCE:	https://forge.typo3.org/issues/51905
% ------------------------------------------------------------------------------

\begin{frame}[fragile]
	\frametitle{TSconfig \& TypoScript}
	\framesubtitle{RTE: Múltiples clases de CSS por estilo}

	\begin{itemize}
		\item Frameworks modernos como Twitter Bootstrap requieren múltiples clases de CSS por etiqueta HTML\newline
			\small Por ejemplo: \texttt{<a class="btn btn-danger">Alert</a>}\normalsize
		\item Ahora son soportadas múltiples clases de CSS, lo que significa que los editores necesitan seleccionar un único estilo

			\begin{lstlisting}
				RTE.classes.[ *classname* ] {
				  .requires = lista de clases de CSS
				}
			\end{lstlisting}

	\end{itemize}

\end{frame}

% ------------------------------------------------------------------------------
% LTXE-SLIDE-START
% LTXE-SLIDE-UID:		6edd0d57-867765a0-487c7375-ba730765
% LTXE-SLIDE-ORIGIN:	e11a6c09-1eb3eaf7-6cebbee9-febef6ba English
% LTXE-SLIDE-TITLE:		RTE: Configure CSS Class As Not-Selectable
% LTXE-SLIDE-REFERENCE:	https://forge.typo3.org/issues/58122
% LTXE-SLIDE-REFERENCE:	https://forge.typo3.org/issues/51905
% ------------------------------------------------------------------------------

\begin{frame}[fragile]
	\frametitle{TSconfig \& TypoScript}
	\framesubtitle{RTE: Configurar clase de CSS como no seleccionable}

	\begin{itemize}
		\item Ahora es posible Configurar clase de CSS como "no seleccionable"

			\begin{lstlisting}
				// valor "1" significa que la clase es seleccionable
				// valor "0" hace que no sea seleccionable
				RTE.classes.[ *classname* ] {
				  .selectable = 1
				}
			\end{lstlisting}

	\end{itemize}

\end{frame}

% ------------------------------------------------------------------------------
% LTXE-SLIDE-START
% LTXE-SLIDE-UID:		343fdeea-2912ff98-ead1d959-af3f98af
% LTXE-SLIDE-ORIGIN:	e607c36f-e1be31da-05ace0c5-991e6e42 English
% LTXE-SLIDE-TITLE:		RTE: Include Multiple CSS Files
% LTXE-SLIDE-REFERENCE:	https://forge.typo3.org/issues/50039
% ------------------------------------------------------------------------------

\begin{frame}[fragile]
	\frametitle{TSconfig \& TypoScript}
	\framesubtitle{RTE: Incluir múltiples archivos CSS}

	\begin{itemize}
		\item Ahora es posible incluir múltiples archivos CSS

			\begin{lstlisting}
				RTE.default.contentCSS {
				  file1 = fileadmin/rte_stylesheet1.css
				  file2 = fileadmin/rte_stylesheet2.css
				}
			\end{lstlisting}

		\item Sin definir ningún archivo de hoja de estilo CSS, el estándar es:\newline
			\texttt{typo3/sysext/rtehtmlarea/res/contentcss/default.css}

	\end{itemize}

\end{frame}

% ------------------------------------------------------------------------------
% LTXE-SLIDE-START
% LTXE-SLIDE-UID:		f4469b00-e470e39a-1a8b8d2f-e5b39cf5
% LTXE-SLIDE-ORIGIN:	c9d0ad7d-2c5aee11-79878250-9e8ce110 English
% LTXE-SLIDE-TITLE:		Exception Handling When cObjects Are Rendered (1)
% LTXE-SLIDE-REFERENCE:	https://forge.typo3.org/issues/47919
% ------------------------------------------------------------------------------

\begin{frame}[fragile]
	\frametitle{TSconfig \& TypoScript}
	\framesubtitle{Gestión de excepciones cuando cObjects son procesados (1)}

	\begin{itemize}
		\item En TYPO3 CMS < 7.0, si ocurre un error durante el proceso de renderización de los objetos de contenido (por ejemplo \texttt{USER}), el error ha roto todo el frontend
		\item Desde TYPO3 CMS >= 7.0, an exception handling has been implemented, se ha implementado una gestión de excepción, que permite mostrar un mensaje en vez de un cObject erróneo
	\end{itemize}

%	*TODO* it would be so nice, if someone could create a screenshot of this scenario!
%
%	\begin{figure}\vspace*{-0.3cm}
%		\includegraphics[width=0.90\linewidth]{TypoScript/cObjectExceptionHandling.png}
%	\end{figure}

\end{frame}

% ------------------------------------------------------------------------------
% LTXE-SLIDE-START
% LTXE-SLIDE-UID:		672c7343-d25772aa-b89facb8-2e3b1e49
% LTXE-SLIDE-ORIGIN:	1132ffa6-a06b7448-0b59e149-a28c2d43 English
% LTXE-SLIDE-TITLE:		Exception Handling When cObjects Are Rendered (2)
% LTXE-SLIDE-REFERENCE:	https://forge.typo3.org/issues/47919
% ------------------------------------------------------------------------------

\begin{frame}[fragile]
	\frametitle{TSconfig \& TypoScript}
	\framesubtitle{Gestión de excepciones cuando cObjects son procesados (2)}
	
	\lstset{
		basicstyle=\tiny\ttfamily
	}
	
	\begin{lstlisting}
		# default exception handler (activated in context "production")
		config.contentObjectExceptionHandler = 1

		# configuration of a class for the exception handling
		config.contentObjectExceptionHandler =
		  TYPO3\CMS\Frontend\ContentObject\Exception\ProductionExceptionHandler

		# customised error message (show random error code)
		config.contentObjectExceptionHandler.errorMessage = Oops an error occurred. Code: %s

		# configuration of exception codes, which are not dealt with
		tt_content.login.20.exceptionHandler.ignoreCodes.10 = 1414512813

		# deactivation of exception handling for a specific plugins or content objects
		tt_content.login.20.exceptionHandler = 0

		# ignoreCodes and errorMessage can be configured globally...
		config.contentObjectExceptionHandler.errorMessage = Oops an error occurred. Code: %s
		config.contentObjectExceptionHandler.ignoreCodes.10 = 1414512813

		# ...or locally for individual content objects
		tt_content.login.20.exceptionHandler.errorMessage = Oops an error occurred. Code: %s
		tt_content.login.20.exceptionHandler.ignoreCodes.10 = 1414512813
	\end{lstlisting}

\end{frame}

% ------------------------------------------------------------------------------
