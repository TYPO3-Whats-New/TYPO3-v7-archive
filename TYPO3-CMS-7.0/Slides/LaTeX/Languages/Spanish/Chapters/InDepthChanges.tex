% ------------------------------------------------------------------------------
% TYPO3 CMS 7.0 - What's New - Chapter "Cambios en Profundidad" (Spanish Version)
%
% @author	Michael Schams <schams.net>
% @author	Michel Mix <mmix@autistici.org>
% @license	Creative Commons BY-NC-SA 3.0
% @link		http://typo3.org/download/release-notes/whats-new/
% @language	Spanish
% ------------------------------------------------------------------------------
% LTXE-CHAPTER-UID:		bb864db3-444c6ac7-f46c503f-ac21f9df
% LTXE-CHAPTER-NAME:	Cambios en Profundidad
% ------------------------------------------------------------------------------

\section{Cambios en Profundidad}
\begin{frame}[fragile]
	\frametitle{Cambios en Profundidad}

	\begin{center}\huge{Capítulo 3:}\end{center}
	\begin{center}\huge{\color{typo3darkgrey}\textbf{Cambios en Profundidad}}\end{center}

\end{frame}

% ------------------------------------------------------------------------------
% LTXE-SLIDE-START
% LTXE-SLIDE-UID:		50e138c6-47fbf375-142b2a9c-64bae081
% LTXE-SLIDE-ORIGIN:	b47dd2a2-952b3dd2-b9ceecce-ab8630bc English
% LTXE-SLIDE-TITLE:		Integration of jQuery UI version 1.11.2
% LTXE-SLIDE-REFERENCE:	https://forge.typo3.org/issues/62916
% ------------------------------------------------------------------------------

\begin{frame}[fragile]
	\frametitle{Cambios en Profundidad}
	\framesubtitle{Integración de jQuery UI versión 1.11.2}

	\begin{itemize}
		\item jQuery UI 1.11 soporta AMD
			(\href{https://es.wikipedia.org/wiki/Asynchronous_module_definition}{Definición asíncrona de módulos}),
			que carga únicamente archivos JavaScript, cuando son necesarios (incremento de rendimiento)

    	\item jQuery UI 1.11 remplaza jQuery UI 1.10 + Scriptaculous en TYPO3 CMS 7.0

    	\item Sólo se incluyen componentes de interacción y centrales, que son requeridos para remplazar ExtJS y Scriptaculous

    	\item Widgets no están incluidos (pero se usan los de Twitter Bootstrap,
    		por ejemplo: DatePicker, Spinner, Dialog, Buttons, Tabs, Tooltip)

	\end{itemize}

\end{frame}

% ------------------------------------------------------------------------------
% LTXE-SLIDE-START
% LTXE-SLIDE-UID:		054a18fa-c1145ee5-70148def-7ff1c3c6
% LTXE-SLIDE-ORIGIN:	66a996e1-bcfe7442-d44f59c8-d3495d4e English
% LTXE-SLIDE-TITLE:		Registry for File Rendering Classes
% LTXE-SLIDE-REFERENCE:	https://forge.typo3.org/issues/61800
% ------------------------------------------------------------------------------

\begin{frame}[fragile]
	\frametitle{Cambios en Profundidad}
	\framesubtitle{Registro para clases de renderización de archivo}

	\lstset{
		basicstyle=\tiny\ttfamily
	}

	\begin{itemize}
		\item Con el fin de poder renderizar todos los tipos de archivos de medios, se ha implementado un registro de renderización de archivos.
			Esto pasa de la siguiente manera (por ejemplo Video, MPEG, AVI, WAV, etc.):

		\begin{lstlisting}
			<?php
			namespace ...;

			class NameTagRenderer implements FileRendererInterface {
			  protected $possibleMimeTypes = array('audio/mpeg', 'audio/wav', ...);
			  public function getPriority() {
			    return 1; // priority: the higher, the more important (max: 100)
			  }
			  public function canRender(FileInterface $file) {
			    return in_array($file->getMimeType(), $this->possibleMimeTypes, TRUE);
			  }
			  public function render(FileInterface $file, $width, $height, array $options = array(), $usedPathsRelativeToCurrentScript = FALSE) {
			    ...

			    return 'HTML code';
			  }
			}
		\end{lstlisting}

	\end{itemize}

\end{frame}

% ------------------------------------------------------------------------------
% LTXE-SLIDE-START
% LTXE-SLIDE-UID:		02e9b40f-0e590030-d8be3c8a-08f1e1d3
% LTXE-SLIDE-ORIGIN:	2c97df27-58f83c0c-2a7532ce-73ba3bd5 English
% LTXE-SLIDE-TITLE:		TCA: Validate Email Addresses
% LTXE-SLIDE-REFERENCE:	https://forge.typo3.org/issues/62147
% ------------------------------------------------------------------------------

\begin{frame}[fragile]
	\frametitle{Cambios en Profundidad}
	\framesubtitle{TCA: Validar dirección de correo electrónico}

	\lstset{
		basicstyle=\tiny\ttfamily
	}

	\begin{itemize}

		\item La nueva función "email" revisa si el valor ingresado es una dirección de email válida
		\item Si la revisión falla, un mensaje Flash aparece
		\item Ejemplo:

		\begin{lstlisting}
			'emailaddress' => array(
			  'exclude' => 1,
			  'label' => 'LLL:EXT:myextension/Resources/Private/Language/locallang_db.xlf:tx_myextension
		 	  'config' => array(
			      'type' => 'input',
			      'size' => 30,
			      'eval' => 'email,trim'
			  ),
			)
		\end{lstlisting}

	\end{itemize}

\end{frame}

% ------------------------------------------------------------------------------
% LTXE-SLIDE-START
% LTXE-SLIDE-UID:		5265d15e-c52d035b-cb366ed7-b267ff7c
% LTXE-SLIDE-ORIGIN:	08cf25ed-0cba243f-e0738567-598b1534 English
% LTXE-SLIDE-TITLE:		AbstractCondition For Custom TypoScript Conditions
% LTXE-SLIDE-REFERENCE:	https://forge.typo3.org/issues/61489
% ------------------------------------------------------------------------------

\begin{frame}[fragile]
	\frametitle{Cambios en Profundidad}
	\framesubtitle{AbstractCondition para condiciones TypoScript personalizadas}

	\lstset{
		basicstyle=\tiny\ttfamily
	}

	\begin{itemize}
		\item Las condiciones TypoScript personalizadas pueden derivarse de una AbstractCondition

		\begin{lstlisting}
			class TestCondition
			  extends \TYPO3\CMS\Core\Configuration\TypoScript\ConditionMatching\AbstractCondition {

			  public function matchCondition(array $conditionParameters) {
 			    if ($conditionParameters[0] === '= 7' && $conditionParameters[1] === '!= 6') {
			      throw new TestConditionException('All Ok', 1411581139);
			    }
			  }
			}
		\end{lstlisting}

		\item El código TypoScript apropiado sigue a continuación:

		\begin{lstlisting}
			[Vendor\Package\TestCondition]
			[Vendor\Package\TestCondition = 7]
			[Vendor\Package\TestCondition = 7, != 6]
		\end{lstlisting}

		\item Los operadores, que deben estar disponibles, están definidos en la clase

	\end{itemize}

\end{frame}

% ------------------------------------------------------------------------------
% LTXE-SLIDE-START
% LTXE-SLIDE-UID:		c0e234d8-5e0bf89c-b9a6b699-85e0a658
% LTXE-SLIDE-ORIGIN:	c137d1cf-7ce68dbe-8149deb5-19fe62bb English
% LTXE-SLIDE-TITLE:		Signal for IconUtility HTML tag manipulation
% LTXE-SLIDE-REFERENCE:	https://forge.typo3.org/issues/61289
% ------------------------------------------------------------------------------

\begin{frame}[fragile]
	\frametitle{Cambios en Profundidad}
	\framesubtitle{Señal para la manipulación de IconUtility HTML Tag}

	\lstset{
		basicstyle=\tiny\ttfamily
	}

	\begin{itemize}

		\item Nueva señal para manipular el IconUtility HTML tag para íconos sprite:

		\begin{lstlisting}
			dispatch(
			  'TYPO3\\CMS\\Backend\\Utility\\IconUtility',
			  'buildSpriteHtmlIconTag',
			  array($tagAttributes, $innerHtml, $tagName)
			);
		\end{lstlisting}

		\item Llamada al método:

		\begin{lstlisting}
			TYPO3\CMS\Backend\Utility\IconUtility\buildSpriteHtmlIconTag
		\end{lstlisting}

	\end{itemize}

\end{frame}

% ------------------------------------------------------------------------------
% LTXE-SLIDE-START
% LTXE-SLIDE-UID:		da9a7f63-c4fdebef-4617a382-1aacc550
% LTXE-SLIDE-ORIGIN:	6713b880-9410d0a5-ef04d276-7bb989ba English
% LTXE-SLIDE-TITLE:		Add Signal Slots to SoftReferenceIndex
% ------------------------------------------------------------------------------

\begin{frame}[fragile]
	\frametitle{Cambios en Profundidad}
	\framesubtitle{Signal Slot en SoftReferenceIndex}

	\lstset{
		basicstyle=\tiny\ttfamily
	}

	\begin{itemize}
		\item
			\smaller
				Dos nuevas llamadas de envío de Signal Slot en SoftReferenceIndex:
			\normalsize

		\begin{lstlisting}
			protected function emitGetTypoLinkParts(
			  $linkHandlerFound, $finalTagParts, $linkHandlerKeyword, $linkHandlerValue) {
			  return $this->getSignalSlotDispatcher()->dispatch(
			    get_class($this),
			    'getTypoLinkParts',
			    array($linkHandlerFound, $finalTagParts, $linkHandlerKeyword, $linkHandlerValue)
			  );
			}
			protected function emitSetTypoLinkPartsElement(
			  $linkHandlerFound, $tLP, $content, $elements, $idx, $tokenID) {
			  return $this->getSignalSlotDispatcher()->dispatch(
			    get_class($this),
			    'setTypoLinkPartsElement',
			    array($linkHandlerFound, $tLP, $content, $elements, $idx, $tokenID, $this)
			  );
			}
		\end{lstlisting}

		\item
			\smaller
				Llamada en:
			\normalsize

			\begin{lstlisting}
				TYPO3\CMS\Core\Database\SoftReferenceIndex->findRef_typolink
				TYPO3\CMS\Core\Database\SoftReferenceIndex->getTypoLinkParts
			\end{lstlisting}

	\end{itemize}

\end{frame}

% ------------------------------------------------------------------------------
% LTXE-SLIDE-START
% LTXE-SLIDE-UID:		0d4bbbe5-ea42eb6d-81977ddf-3e94b4ea
% LTXE-SLIDE-ORIGIN:	d743b8cf-4924c9ca-7e1d330c-3b101461 English
% LTXE-SLIDE-TITLE:		afterPersistObjetct Signal Slot
% LTXE-SLIDE-REFERENCE:	https://forge.typo3.org/issues/59986
% ------------------------------------------------------------------------------

\begin{frame}[fragile]
	\frametitle{Cambios en Profundidad}
	\framesubtitle{afterPersistObject Signal Slot}

	\lstset{
		basicstyle=\tiny\ttfamily
	}

	\begin{itemize}

		\item Nueva afterPersistObject signal slot emite para la raíz agregada después de que persisten todos los otros objetos

			\begin{lstlisting}
				protected function emitAfterPersistObjectSignal(DomainObjectInterface $object) {
				  $this->signalSlotDispatcher->dispatch(__CLASS__, 'afterPersistObject', array($object));
				}
			\end{lstlisting}

		\item Llamada en:

			\begin{lstlisting}
				TYPO3\CMS\Extbase\Persistence\Generic\Backend->persistObject
			\end{lstlisting}

		\item La misma señal es emitida en el método persistObject en la clase AbstractBackend en Flow

	\end{itemize}

\end{frame}                                                 

% ------------------------------------------------------------------------------
% LTXE-SLIDE-START
% LTXE-SLIDE-UID:		5d5c8a6a-fdcfc51b-4c0dbcbf-e5149337
% LTXE-SLIDE-ORIGIN:	17ea6530-0ec1ad3e-8e7d9d24-1d5b64aa English
% LTXE-SLIDE-TITLE:		Signal in loadBaseTca
% LTXE-SLIDE-REFERENCE:	https://forge.typo3.org/issues/57863
% ------------------------------------------------------------------------------

\begin{frame}[fragile]
	\frametitle{Cambios en Profundidad}
	\framesubtitle{Señal en loadBaseTca}

	\lstset{
		basicstyle=\tiny\ttfamily
	}

	\begin{itemize}
		\item Para mejorar el rendimiento en el contexto de backend, el TCA completo puede estar en caché ahora (no sólo en partes de él)

		\begin{lstlisting}
			protected function emitTcaIsBeingBuiltSignal(array $tca) {
			  list($tca) = static::getSignalSlotDispatcher()->dispatch(
			    __CLASS__,
			    'tcaIsBeingBuilt',
			    array($tca)
			  );
			  $GLOBALS['TCA'] = $tca;
			}
		\end{lstlisting}

		\item Llamada en:

			\begin{lstlisting}
				TYPO3\CMS\Core\Utility\ExtensionManagementUtility\Backend->buildBaseTcaFromSingleFiles
			\end{lstlisting}

	\end{itemize}

\end{frame}  

% ------------------------------------------------------------------------------
% LTXE-SLIDE-START
% LTXE-SLIDE-UID:		52ba2fd1-1f2ffcd0-c186765e-693b1d51
% LTXE-SLIDE-ORIGIN:	2fd6c25c-5f7952da-1406b74c-e0ff5e67 English
% LTXE-SLIDE-TITLE:		API to Add Cached TCA Changes
% LTXE-SLIDE-REFERENCE:	https://forge.typo3.org/issues/57942
% ------------------------------------------------------------------------------

\begin{frame}[fragile]
	\frametitle{Cambios en Profundidad}
	\framesubtitle{API para cambios de TCA en cache}

	\begin{itemize}
		\item Ficheros PHP en \texttt{extkey/Configuration/TCA/Overrides/}
			son ejecutados directamente después de que el caché TCA ha sido construido

		\item Estos ficheros pueden sólo incluir código, que manipula el TCA,\newline
			como: \texttt{addTCAColumns} o \texttt{addToAllTCATypes}

		\item Esta característica da a las solicitudes del backend un aumento en el rendimiento, una vez que las extensiones comienzan a usar estos archivos

	\end{itemize}

\end{frame} 

% ------------------------------------------------------------------------------
% LTXE-SLIDE-START
% LTXE-SLIDE-UID:		02c6dd5e-6cdf2da0-ef8faebd-ff6e7c73
% LTXE-SLIDE-ORIGIN:	89fbd845-3952a157-98d97fc4-4244a2d4 English
% LTXE-SLIDE-TITLE:		Read-only File Mounts
% LTXE-SLIDE-REFERENCE:	https://forge.typo3.org/issues/49391
% ------------------------------------------------------------------------------

\begin{frame}[fragile]
	\frametitle{Cambios en Profundidad}
	\framesubtitle{File Mounts de solo lectura}

	\begin{itemize}

		\item File Mounts pueden ser configurados de "solo lectura" (otra vez)
		\item Esto fue ya posible en TYPO3 CMS 4.x, pero silenciosamente removido en 6.x
		\item Ejemplo: añadir la carpeta de "prueba" de almacenamiento UID 3 como de solo lectura en la lista de archivo y en el explorador de elementos.\newline

			\smaller\texttt{options.folderTree.altElementBrowserMountPoints = 3:/test}\normalsize\newline

			Si no se configura el almacenamiento, este asume que la carpeta está en el almacenamiento por defecto.
	\end{itemize}

\end{frame}

% ------------------------------------------------------------------------------
% LTXE-SLIDE-START
% LTXE-SLIDE-UID:		4f7c19f9-cf86417f-fe550963-25856063
% LTXE-SLIDE-ORIGIN:	b964bc6b-27cd32e7-bda8cc3e-8bea33ca English
% LTXE-SLIDE-TITLE:		Miscellaneous
% LTXE-SLIDE-REFERENCE:	https://forge.typo3.org/issues/62993
% LTXE-SLIDE-REFERENCE:	https://forge.typo3.org/issues/61185
% LTXE-SLIDE-REFERENCE:	https://forge.typo3.org/issues/58365
% LTXE-SLIDE-REFERENCE:	https://forge.typo3.org/issues/59830
% LTXE-SLIDE-REFERENCE:	https://forge.typo3.org/issues/62857
% ------------------------------------------------------------------------------

\begin{frame}[fragile]
	\frametitle{Cambios en Profundidad}
	\framesubtitle{Varios}

	\begin{itemize}
		\item jQuery ha sido actualizado de la versión 1.11.0 a la versión 1.11.1 
		\item Datatables han sido actualizadas de la versión 1.9.4 a la versión 1.10.2
		\item Algunas variables viejas, inútiles, han sido removidas de \texttt{EM\_CONF}
		\item Los íconos de extensión pueden estar en formato SVG image ahora (\texttt{ext\_icon.svg})
		\item Pasar un identificador eID erróneo ahora da como resultado una excepción
	\end{itemize}

\end{frame}

% ------------------------------------------------------------------------------
